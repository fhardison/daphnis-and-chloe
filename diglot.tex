\documentclass{book}
\usepackage[osf,p]{libertinus}
\usepackage{microtype}
\usepackage[pdfusetitle,hidelinks]{hyperref}
\usepackage[series={},nocritical,noend,nofamiliar,noledgroup]{reledmac}
\usepackage{reledpar}

\usepackage{geometry}
\geometry{
  paperheight=8.5in,
  paperwidth=5.5in,
  margin=0.5in,
  heightrounded,}


\usepackage{graphicx}
\usepackage{polyglossia}
\setmainlanguage{english}
\setotherlanguage{greek}

\usepackage{metalogo}

\linenumincrement*{1}
\firstlinenum*{1}
\setlength{\Lcolwidth}{0.45\textwidth}
\setlength{\Rcolwidth}{0.45\textwidth} 

\begin{document}

\title{Daphnis and Chloe}
\date{}

\maketitle



\begin{pairs}

\begin{Rightside} 

\begin{english}
\beginnumbering
\pstart
1.praef  While hunting in a grove sacred to the Nymphs, in the island of Lesbos, I saw the most beautiful sight that I have ever seen: a picture representing a history of love.  The grove itself was pleasant to the eye, covered with trees, full of flowers, and well-watered: a single spring fed both trees and flowers.  But the picture itself was even more delightful: its subject was the fortunes of love, and the art displayed in it was marvellous: so that many, even strangers, who had heard it spoken of, visited the island, to pay their devotion to the Nymphs and examine the picture, on which were portrayed women in childbirth or wrapping children in swaddling clothes, poor babes exposed to the mercy of Fortune, beasts of the flock nurturing them, shepherds taking them up in token of adoption, young people binding one another by mutual vows, pirates over-running the seas, and enemies invading the land. Many other subjects, all of an amatory nature, were depicted, which I gazed upon with such admiration that I was seized with the desire to describe them in writing.  Accordingly, I diligently sought for someone to give me an explanation of the details: and, when I had thoroughly mastered them, I composed the four following books, as an offering to Love, the Nymphs, and Pan, and also as a work that will afford pleasure to many, in the hope that it may heal the sick, console the sorrowful, refresh the memory of him who once has loved, and instruct him who has never yet felt its flame.  For no one has yet escaped, or ever will escape, the attack of Love, as long as beauty exists and eyes can see.  May God grant that, unharmed ourselves, we may be able to describe the lot of others!
\pend


\pstart
1.1  There is in Lesbos a flourishing and beautiful city, named Mitylene.  It is intersected by numerous canals, formed by the waters of the sea, which flows in upon it, and adorned with several bridges of white polished stone: to look at it, you would say that it was not a single city, but a number of islands.  About two hundred stades distant from the city, a wealthy man possessed a very fine estate: mountains abounding in game, fruitful cornfields, hillocks covered with vine shoots, and ample pasturage for cattle; the sea washed a long stretch of soft sandy beach.  (3>)
\pend


\pstart
1.2  On this estate a goatherd named Lamon, while feeding his flock, found a child being suckled by a goat.  There was a thicket of shrubs and briars, over which the ivy straggled, and beneath, a couch of soft grass, whereon the infant lay.  Hither the goat often ran and wandered out of sight, and abandoning its own kid, remained by the side of the child.  Lamon, pitying the neglected kid, observed the direction in which the goat went: and, one day at noon, when the sun was at its height, he followed and saw it cautiously entering the thicket and walking round the child, so as not to tread on and hurt it, while the latter sucked vigorously at its teat as if it had been its mother's breast.  Astonished, as was natural, he approached closer, and found that it was a little boy, beautiful and well-grown, and wrapped in handsomer swaddling clothes than suited a child thus exposed: it had on a little purple tunic fastened with a golden clasp, and by its side was a little dagger with an ivory hilt.
\pend


\pstart
1.3  At first he was minded to take up the tokens, without troubling about the child: but afterwards, feeling ashamed at the idea of being outdone by the goat in humanity, he waited till night, and took everything to his wife Myrtale, the tokens, the child, and the goat.  When she expressed her astonishment that goats should bring forth little children, he told her everything: how he had found the child lying exposed, and being suckled by the goat, and how he had felt ashamed to leave it to die.  His wife agreed with him, and they resolved to hide the tokens, to bring up the child as their own, and to let the goat suckle him.  Further, they decided to call him Daphnis, that the name might have a more pastoral sound.
\pend


\pstart
1.4  When two years had passed, a shepherd belonging to the neighbourhood, named Dryas, while feeding his flocks, made a similar discovery and saw a similar sight.  In his district there was a cave sacred to the Nymphs: a large rock hollowed out within, and circular without.  Inside were statues of the Nymphs, carved in stone, with feet unshod, arms bared up to the shoulders, hair falling down over the neck, a girdle around the waist, and a smile on the face: to judge from their attitude, you would have said they were dancing.  The dome of the grotto was the centre of this mighty rock.  Water, gushing from a fountain, formed a running stream; a beautiful meadow extended in front of the cave, the soft and abundant herbage of which was nourished by the moisture of the stream.  Within were to be seen hanging up milk-pails, flutes, pipes, and reeds, the offerings of the older shepherds.
\pend


\pstart
1.5  A sheep, which had recently lambed, went so often to this grotto, that more than once she was thought to be lost.  Dryas, wishing to punish her and make her stay with the flock to feed, as before, twisted a bough of pliant osier into a (4>) collar in the form of a running noose, and went up to the rock, in order to snare her.  But when he drew near he beheld quite a different sight from what he had expected: he saw the sheep giving her teat, just like a human being, for a copious draught of milk, to a child, which, without a cry, eagerly shifted its clean and pretty mouth from one teat to the other, while the sheep licked its face, after it had had enough.  It was a female child, and by its side also lay swaddling clothes and tokens, a cap interwoven with gold, gilded shoes, and gold-embroidered anklets.
\pend


\pstart
1.6  Thinking that what he had found was sent from Heaven, and being moved to pity by the example of the sheep, he took the child up in his arms, put the tokens in his wallet, and prayed to the Nymphs that he might be permitted to bring up their suppliant happily.  Then when it was time to drive back his flock, he returned home, told his wife what he had seen, showed her what he had found, and bade her adopt and bring up the child as her own, without telling anyone what had happened.  Nape - that was his wife's name - immediately took up the child and caressed her, as if afraid of being outdone in kindliness by the sheep: and, that it might be more readily believed that the child was her own, she gave it the pastoral name of Chloe.
\pend


\pstart
1.7  The two children soon grew up, more beautiful than ordinary rustics.  When the boy was fifteen years of age, and the girl thirteen, Lamon and Dryas both dreamed the following dream on the same night.  They dreamed that the Nymphs of the grotto with the fountain, in which Dryas had found the little girl, delivered Daphnis and Chloe into the hands of a saucy and beautiful boy, who had wings on his shoulders and carried a little bow and arrow: and that this boy touched them both with the same arrow, and bade them tend, the one goats, the other sheep.
\pend


\pstart
1.8  When they saw this vision, they grieved to think that Daphnis and Chloe were destined to tend sheep and goats, since their swaddling clothes seemed to give promise of better fortune: for which reason they had brought them up more delicately than shepherds' children, had taught them to read, and given them all the instruction possible in a country place.  They resolved, however, to obey the gods in regard to those who had been saved by their providence.  Having communicated their dreams to each other, and offered sacrifice, in the cave of the Nymphs, to the winged boy (whose name they did not know), they sent the maiden and the lad into the fields, having instructed them in all that they had to do: how they ought to feed their flocks before midday, and when the heat had abated: when they should drive them to drink, and when drive them back to the fold: when they should use the shepherd's crook and when (5>) the voice alone.  They undertook this duty as joyfully as if they had been entrusted with some important office, and were fonder of their goats and sheep than shepherds usually are: for Chloe felt she owed her life to a ewe, while Daphnis remembered that when exposed, he had been nurtured by a goat.
\pend


\pstart
1.9  It was the beginning of spring, and all the flowers were blooming in the woods and meadows, and on the mountains.  The humming of bees, and the twittering of tuneful birds were already heard, and the new-born young were skipping through the fields: the lambs were gambolling on the mountains, the bees were buzzing through the meadows, the birds were singing in the bushes.  Under the influence of this beautiful season, Daphnis and Chloe, themselves tender and youthful, imitated what they saw and heard.  When they heard the birds sing, they sang: when they saw the lambs gambol, they nimbly skipped in rivalry: and, like the bees, they gathered flowers, some of which they placed in their bosoms, while they wove garlands of others, which they offered to the Nymphs.
\pend


\pstart
1.10  They did everything in common, and tended their flocks side by side.  Daphnis frequently gathered together Chloe's wandering sheep: while she often drove back his too venturesome goats from the precipices.  Sometimes one of them tended the two flocks alone, while the other was intent upon some amusement.  Their amusements were those of children or shepherds.  Chloe would pluck some stalks of asphodel from the marsh, to weave a locust-trap, without any thought for her flock: while Daphnis, having cut some slender reeds, and perforated the intervals between joints, joined them with soft wax, and practised himself in playing upon them until nightfall.  Sometimes they shared the food they had taken with them from home, their milk, or wine.  In short, it would have been easier to find sheep and goats feeding apart than Daphnis separated from Chloe.
\pend


\pstart
1.11  While they were thus engaged in their youthful sports, Love contrived the following trouble for them.  There was a wolf in the district, which, having recently brought forth young, frequently carried off lambs from the neighbouring fields to feed them.  The villagers accordingly assembled together by night, and dug some trenches, one fathom in depth and four in breadth: the greater part of the earth which they dug out they removed to a distance from the trenches: then, placing over the hole long pieces of dry wood, they covered them with the remainder of the earth, so that it looked level ground just as it had been before: this they did so cunningly that, if even a hare had run across, it would have broken the pieces of wood, which were more brittle than bits of straw; and then it would have been seen that it was not solid earth at all, but an imitation.  Although they dug several similar trenches (6>) on the mountains and plains, they could not succeed in catching the wolf, which perceived the snare, but were the cause of the loss of a number of sheep and goats, and Daphnis also nearly lost his life, in the following manner.
\pend


\pstart
1.12  Two goats, in a fit of jealousy, charged each other so violently that the horn of one was broken, and, mad with pain, he took to flight bellowing, closely and hotly pursued by his victorious adversary.  Daphnis, grieved at the sight of the mutilated horn, and annoyed at the insolence of the victor, seized his club and crook, and started in pursuit of the pursuer.  But, while the goat was trying to make his escape, and Daphnis was in angry pursuit, they could not see clearly what was in front of them, and both fell into one of these pits - the goat first, and Daphnis after him.  This saved Daphnis from injury, since he was able to hold onto the goat to break his fall.  In this situation he waited in tears to see if anyone would come to pull him up again.  Chloe, having seen what had happened, ran up, and, finding that he was still alive, called one of the herdsmen from the neighbouring fields to her assistance.  The herdsman came up, and looked for a long rope with which to haul him out, but found none.  Then Chloe unloosed the band which fastened her hair, and gave it to the herdsmen to let down.  Then they stood on the edge of the pit and pulled: and Daphnis, holding on to the band as it was being hauled up, at last succeeded in reaching the summit.  Then they drew up the wretched goat, whose horns were both broken - so fully was his vanquished adversary avenged – and made a present of him to the herdsman, in return for his assistance, having agreed to tell those at home that he had been carried off by a wolf, if anyone missed him.  Returning to their flocks, and finding them all feeding peacefully and in good order, they sat down on the trunk of an oak, to see whether Daphnis had been wounded in any part of his body by his fall.  But they found no trace of any injury or blood: only his hair and the rest of his person were covered with earth and mud.  Daphnis therefore resolved to wash himself, before Lamon and Myrtale found out what had happened.
\pend


\pstart
1.13  He went with Chloe to the grotto of the Nymphs, where the fountain was, and gave her his tunic and wallet.  And Daphnis, standing by the spring, began to wash his hair and his whole person.  His hair was dark and thick, and his body tanned by the sun; one would have thought that it was darkened by the reflection of his hair.  Chloe looked at him, and he seemed to her to be very handsome; and, because she had never thought him handsome before, she imagined that he owed his beauty to his bath.  She washed his back and shoulders, and, finding his skin soft and yielding beneath her hand, she more than once secretly touched (7>) herself, to see whether her own skin was more delicate.  Then, as it was near sunset, they drove back their flocks to the homestead: and, from that moment, Chloe had but one thought, one desire - to see Daphnis in the bath again.  The following day, when they returned to the pasture, Daphnis sat down under his favourite oak-tree and played on his pipe, looking awhile at his goats, which, lying at his feet, seemed to be listening to his strains. Chloe, seated near him, was also looking after her sheep, but her eyes were more frequently fixed upon Daphnis.  She again thought him handsome as he was playing on his pipe, and this time, imagining that he owed his beauty to the music, she took the pipe herself, to see whether she could make herself beautiful.  She persuaded him to take a bath again, saw him in the bath and touched him: then, on her way home, she again began to praise his beauty, and this praise was the beginning of love.  She did not know what was the matter with her, being a young girl brought up in the country, who had never even heard anyone mention the name of Love.  But her heart was a prey to langour, she no longer had control over her eyes, and she often uttered the name of Daphnis.  She ate little, could not sleep at night, and neglected her flock: by turns she laughed and cried, slept and started up: her face was pale one moment, and covered with blushes the next: a cow, stung by the gadfly, was not more uneasy than Chloe.  Sometimes, when she was quite alone, she talked to herself in the following strain:
\pend


\pstart
1.14  "I am ill, but I do not know the nature of my illness: I feel pain, but am not wounded: I am sad, but I have lost none of my sheep.  I am burning, although seated in the shade.  The brambles have often torn my flesh, but I did not weep: the bees have often stung me, but I ate my food.  The evil which now gnaws my heart must be sharper than all those.  Daphnis is beautiful, but so are the flowers: his pipe gives forth sweet notes, but so do the nightingales: but yet I care not for them.  Would that I were his pipe, that I might receive his breath!  Would that I were one of his goats, that I might be tended by him!  O cruel water, that hast made Daphnis so beautiful, while I have washed in thee in vain!  I perish, O beloved Nymphs, and you, too, refuse to save the girl who has been brought up in your midst.  When I am dead, who will crown you with garlands?  Who will feed my poor sheep?  Who will look after the noisy grasshopper, which I took so much trouble to catch, that it might send me to sleep, chirping in front of the grotto?  But now Daphnis has robbed me of sleep, and the grasshopper chirps in vain."
\pend


\pstart
1.15  Such were the words she spoke in her suffering, seeking in vain for the name of Love.  But Dorcon, the herdsman who had extricated Daphnis and the (8>) goat from the pit, a youth whose beard was just beginning to grow, who knew the name of Love and what it meant, had felt an affection for Chloe ever since that day, and, as time went on, his passion increased.  Thinking little of Daphnis, whom he looked upon as a mere child, he resolved to gain his object, either by bribery or violence.  He first made them presents: to Daphnis he gave a rustic pipe, the nine reeds of which were fastened together with brass instead of wax, and to Chloe a spotted fawn's skin, such as Bacchus was wont to wear.  Then, thinking that he was on sufficiently friendly terms with them, he gradually began to neglect Daphnis, while every day he brought Chloe a fresh cheese, a garland of flowers, or some ripe fruit; and once he presented her with a young calf, a gilt cup, and some young birds which he had caught on the mountains.  She, knowing nothing of the arts of lovers, was delighted to receive the presents, because she could pass them on to Daphnis.  One day - since Daphnis also was destined to learn what Love meant - a discussion arose between him and Dorcon as to which of them was the handsomer.  Chloe was appointed judge: and the victor's reward was to be a kiss.  Dorcon spoke first:
\pend


\pstart
1.16  "I am taller than Daphnis: I am a cowherd, while he is only a goatherd, as much superior to him as cows are superior to goats.  I am white as milk, ruddy as corn fit for the sickle: my mother reared me, not a wild beast.  He is short, beardless as a woman, black as a wolf.  He tends goats, and stinks like them.  He is so poor that he cannot even keep a clog: and if, as is reported, a goat has suckled him, he differs little from a kid." After Dorcon had spoken thus, Daphnis replied: "Yes, like Zeus, I was suckled by a goat: I tend goats that are larger than his cows, and I do not smell of them, any more than Pan, who is more like a goat than anything else.  I am content with cheese, hard bread, and sweet wine: if he have these, a man is rich in the country.  I am beardless, so was Dionysus: I am dark, so is the hyacinth: and yet Dionysus is superior to the Satyrs, the hyacinth to the lily.  He is as red as a fox, bearded like a goat, white as a woman from the city.  If you kiss me, you will kiss my mouth: but, if you kiss him, you will only kiss the hairs on his chin.  Lastly, O maiden, remember that you were suckled by a sheep: and yet how beautiful you are!"
\pend


\pstart
1.17  Chloe could wait no longer: delighted at such praise, and having long been eager to kiss Daphnis, she jumped up and kissed him, simply and artlessly, but yet her kiss had power to inflame his heart.  Dorcon, deeply annoyed, hastened away, to think of some other way of satisfying his desires.  Daphnis, on the other hand, seemed to have received a sting, rather than a kiss.  He (9>) immediately became sad and pensive: he was seized with a chill, and was unable to restrain his palpitating heart: he wanted to look at Chloe, and, when he did so, his face was covered with blushes.  Then, for the first time, he admired her fair hair, her eyes as large as those of a heifer, her face whiter than goats' milk: it seemed as if he then began for the first time to see, and had hitherto been blind.  He merely tasted his food, and hardly moistened his lips with drink.  He who was once more noisy than the locusts, remained silent: he who was formerly more active than his goats, sat idle: his flock was neglected, his pipe lay on the ground, his face was paler than the grass in summer.  He could only speak of Chloe: and, whenever he was away from her he would rave to himself like this.
\pend


\pstart
1.18  "What on earth has Chloe's kiss done to me?  Her lips are tenderer than roses, her mouth is sweeter than a honeycomb, but her kiss is sharper than the sting of a bee.  I have often kissed my kids: I have often kissed newly-born puppies, and the little calf which Dorcon gave me: but this kiss is something new.  My pulse beats high: my heart leaps: my soul melts: and yet I wish to kiss again.  O bitter victory!  O strange disease, the name of which I cannot even tell!  Can Chloe have tasted poison before she kissed me? why then did she not die?  How sweetly sing the nightingales; but my pipe is silent!  How wantonly leap the kids, but I sit still!  How sweetly bloom the flowers, but I weave no garlands!  The violets and hyacinths bloom, but Daphnis fades.  Shall even Dorcon appear more beautiful than Daphnis?"
\pend


\pstart
1.19  Such were the passionate outbursts of the worthy Daphnis, who then for the first time felt the influence of love.  But Dorcon, the herdsman, the lover of Chloe, seizing the opportunity when Dryas was planting a tree near a vine-shoot, went up to him with some cheeses and pipes.  He gave him the cheeses, since he had been an old friend of his, at the time when he himself pastured his flock.  Then he began to speak of marriage with Chloe, and promised him a number of valuable presents, if he should gain her hand: a yoke of oxen for ploughing, four swarms of bees, fifty young apple trees, an ox-hide for making shoes, and, every year, a calf that had been weaned.  Allured by the prospect of such presents, Dryas was on the point of giving his consent.  But afterwards, thinking that the maiden deserved to make a better marriage, and being afraid that, if he were found out, he might be punished and even put to death, he refused his consent, at the same time asking Dorcon not to be offended.
\pend


\pstart
1.20  Dorcon, thus for the second time cheated of his hopes, and having lost his fat cheeses for nothing, determined to lay violent hands on Chloe when he found her alone.  Having observed that Daphnis and Chloe took it in turns to (10>) drive their flocks to drink, he contrived a scheme worthy of a shepherd.  He took the skin of a large wolf, which one of his oxen, fighting in defence of the kine, had killed with his horns, and flung it over his shoulders, whence it hung down to his feet, so that the forefeet covered his hands, and the hinder his legs down to his heels, while the head with its gaping mouth enclosed his head like a soldier's helmet.  Having thus transformed himself into a wild beast as best he could, he proceeded to the spring where the goats and sheep used to drink after they came from pasture.  This fountain was in a hollow valley, and the whole spot around was full of wild brambles and thorns, low growing juniper bushes and thistles, so that even a real wolf could easily have concealed himself there.  Here Dorcon hid himself, waiting for the time when the animals came to drink, hoping to frighten Chloe under the guise of a wolf, and so easily lay hands upon her.
\pend


\pstart
1.21  After he had waited a little while, Chloe came driving the flocks to the spring, having left Daphnis cutting fresh foliage for the kids to eat after pasture.  The dogs who assisted them to guard the sheep and goats followed her: and, with the natural curiosity of keen-scented animals, they tracked and discovered Dorcon preparing to attack the maiden.  With a loud bark, they rushed upon him as if he had been a wolf, surrounded him, before he was able in his astonishment to rise upon his feet, and bit at him furiously.  At first, afraid of being recognised, and being for some time protected by the skin which covered him, he lay in the thicket without uttering a word: but when Chloe, terrified at the first sight of the supposed animal, shouted for Daphnis to help her, and the dogs, having torn off the skin, began to fix their teeth in his body, he cried out loudly and implored Chloe and Daphnis, who had just come up, to assist him.  They quickly calmed the dogs with their familiar shout; then taking Dorcon, who had been bitten in the legs and shoulders, to the fountain, they washed his wounds, where the dogs' teeth had entered the flesh, and chewed the green bark of an elm-tree and spread it over them.  In their ignorance of the audacity prompted by love, they thought that Dorcon had merely put on the wolf's skin for a joke: wherefore they felt no anger against him, but tried to console him, and, having helped him along a little distance, sent him on his way.
\pend


\pstart
1.22  Dorcon, having been in such deadly peril, after he had made good his escape from the mouth of a dog (not, as the proverb goes, of a wolf), devoted his attention to his wounds.  Daphnis and Chloe, however, found considerable difficulty in getting together their goats and sheep, which, alarmed by the sight (11>) of the wolf's skin, and thrown into confusion by the barking of the dogs, had fled to the tops of the mountains or down to the seashore.  Although they had been trained to obey their masters' voice and to be soothed by the sound of the pipe, and to gather together when they merely clapped their hands, fear had caused them to forget everything; and they could only get them back to the fold with difficulty, after tracking them like hares.  During that night alone they slept soundly, and weariness was a remedy for their amorous uneasiness: but, as soon as day came again, they felt the same passion as before.  They were glad when they saw each other, and sorrowful when they parted: they suffered, they wanted something, but they did not know what they wanted.  They only knew, the one that he had been undone by a kiss, the other that she had been destroyed by a bath.
\pend


\pstart
1.23  In addition to this, the season of the year still further inflamed their passion.  It was the end of spring and the commencement of summer: all Nature was in full vigour: the trees were full of fruit, the fields of corn.  The chirp of the grasshopper was sweet to hear, the fruit sweet to smell, and the bleating of the sheep pleasant to the ear.  The gently flowing rivers seemed to be singing a song: the winds, blowing softly through the pine branches, sounded like the notes of the pipe: even the apples seemed to fall to the ground smitten with love, stripped off by the sun that was enamoured of their beauty.  Daphnis, heated by all these surroundings, plunged into the river, sometimes to bathe, at other times to snare the fish that sported in the eddies of the stream: and he often drank, as if he could thereby quench the fire that consumed him.  Chloe, after having milked her sheep and most of Daphnis's goats, was for a long time busied in curdling the milk: for the flies annoyed her terribly and stung her, when she endeavoured to drive them away.  After this, she washed her face, and crowned with branches of pine, and girt with the skin of a fawn, filled a pail with wine and milk to share with Daphnis.
\pend


\pstart
1.24  When noon came on, they were more enamoured than ever.  For Chloe, having seen Daphnis quite naked, was struck by the bloom of his beauty, and her heart melted with love, for his whole person was too perfect for criticism: while Daphnis, seeing Chloe with her fawn skin and garland of pine, holding out the milk-pail for him to drink, thought that he was gazing upon one of the Nymphs of the grotto.  He snatched the garland from her head, kissed it, and placed it on his own: and Chloe took his clothes when he had stripped to bathe, kissed them, and in like manner put them on.  Sometimes they pelted each other with apples, and parted and decked each other's hair.  Chloe declared that Daphnis's hair, being dark, was like myrtle berries: while (12>) Daphnis compared Chloe's face to an apple, because it was fair and ruddy. He also taught her to play on the pipe: and, when she began to blow, he snatched it away and ran over the reeds with his lips: and, while he thus pretended to show her where she was wrong, he speciously kissed the pipe in the places where her mouth had been.
\pend


\pstart
1.25  While he was piping in the noonday heat, and the flocks were resting in the shade, Chloe unwittingly fell asleep.  When Daphnis perceived this, he put down his pipe, and gazed at her all over with greedy eyes, without any feeling of shame, and at the same time gently whispered to himself: "How lovely are her eyes in sleep!  How sweet the perfume from her mouth, sweeter than that of apples or the hawthorn!  Yet I dare not kiss it: her kiss pricks me to the heart, and maddens me like fresh honey.  Besides, if I kiss her, I am afraid of waking her.  O chattering grasshoppers!  You will prevent her from sleeping, if you chirp so loudly!  And on the other side, the he goats are butting each other with their horns: O wolves, more cowardly than foxes, why do you not carry them off?"
\pend


\pstart
1.26  While he was thus talking to himself, a grasshopper, pursued by a swallow, fell into Chloe's bosom: the swallow followed, but could not catch it: but, being unable to check its flight, touched Chloe's cheek with its wing.  Not knowing what the matter was, she cried out loudly, and woke up with a start: but, when she saw the swallow flying close to her, and Daphnis laughing at her alarm, she was reassured, and rubbed her still drowsy eyes.  The grasshopper, as if in gratitude for its safety, chirped its thanks from her bosom.  Then Chloe cried out again, and Daphnis laughed: and, seizing the opportunity, thrust his hand into her breast, and pulled out the grateful insect, which continued its song, even while held a prisoner in his hand.  Chloe was delighted, and having kissed the insect, took it and put it, still chirping, into her bosom.
\pend


\pstart
1.27  Another time, they were listening with delight to the cooing of a wood pigeon.  When Chloe asked what was the meaning of its song, Daphnis told her the popular story: "Once upon a time, dear maiden, there was a maiden, beautiful and blooming as you.  She tended cattle and sang beautifully: her cows were so enchanted by the music of her voice, that she never needed to strike them with her crook or to touch them with her goad: but, seated beneath a pine-tree, her head crowned with a garland, she sang of Pan and Pinus, and the cows stood near, enchanted by her song.  There was a young man who tended his flocks hard by, beautiful and a good singer himself, as she was, who entered into a rivalry of song with her: his voice was more powerful, (13>) since he was a man, and yet gentle, since he was but a youth.  He sang so sweetly that he charmed eight of her best cows and enticed them over to his own herd, and drove them away.  The maiden, grieved at the loss of her cattle, and at having been vanquished in singing, begged the Gods to transform her into a bird before she returned home.  The Gods listened to her prayer, and transformed her into a mountain bird, which loves to sing as she did.  Even now it tells in plaintive tones of her misadventure, and how that she is still seeking the cows that strayed away.
\pend


\pstart
1.28  Such were the enjoyments which the summer afforded them.  But, in mid-autumn, when the grapes grew ripe, some Tyrian pirates, having embarked on a light Carian vessel, that they might not be suspected of being barbarians, landed on the coast: and, armed with swords and corslets, carried off everything that came into their hands, fragrant wine, a great quantity of wheat, and honey in the honeycomb, besides some cows belonging to Dorcon.  They also seized Daphnis as he was wandering on the shore: for Chloe, being a simple girl, for fear of the insolence of the shepherds, did not drive out the flocks of Dryas so early.  When the robbers beheld the tall and handsome youth, a more valuable booty than any they could find in the fields, they paid no heed to the goats or the other fields, but carried him off to their ship, weeping and in great distress what to do, and calling the while for Chloe in a loud voice.  No sooner had they loosed the cable, and begun to ply their oars, and put out to sea, than Chloe drove down her flock, bringing with her a new pipe as a present to Daphnis.  But, seeing the goats scattered hither and thither, and hearing Daphnis calling to her ever louder and louder, thinking no more about her sheep, she flung away the pipe, and ran to Dorcon, to implore his aid.
\pend


\pstart
1.29  She found him lying prostrate on the ground, hacked by the swords of the robbers, and almost dead from loss of blood.  But, when he saw Chloe, revived by the smouldering fire of his former passion, he said: "Chloe, dear, I am at the point of death: when I tried to defend my cattle, the accursed brigands hewed me to pieces like an ox.  But do you save Daphnis for yourself: avenge me, and destroy them.  I have taught my cows to follow the sound of the pipe, and to come when they hear it, however far off they may be feeding.  Come, take this pipe, and play the same strain upon it which I once taught Daphnis, and he in turn taught you.  Leave the rest to my pipe and my cows that are on yonder ship.  I also make you a present of the pipe, with which I have gained the victory over many herdsmen and shepherds.  Kiss me once in return, and (14>) lament for me when I am dead: and, when you see another tending my cattle, then think of me."
\pend


\pstart
1.30  When Dorcon had thus spoken, and had kissed her for the last time, he breathed his last as he spoke and kissed her.  Chloe took the pipe, put it to her lips, and blew with all her might.  And the cows heard it, and, recognising the strain, began to low, and all with a bound sprang into the sea.  As they had leaped from the same side of the vessel, and caused the sea to part, it upset and sank under the waves that closed over it.  Those on board were flung into the sea, but with unequal prospect of safety.  For the pirates were encumbered with swords, and clad in scaly coats of mail, and greaves reaching halfway down the leg.  But Daphnis, who had been tending his flocks, was unshod, and only half clothed, owing to the burning heat.  The pirates had only swum a little way, when the weight of their armour dragged them down into the depths: Daphnis easily threw off the clothes he had on, yet it cost him some effort to swim, since he had hitherto only swum in rivers: but soon, under the impulse of necessity, he reached the cows by an effort, and, while with each hand he grasped one by the horns, he was carried along between them without difficulty, or danger, as if he had been driving a cart: for an ox swims far better than any man: it is only inferior to the water-fowl and fishes.  An ox would never sink, were it not that the horn falls off his hoofs when it gets wet through.  The truth of what I say is borne out by many places on the coast which are still found bearing the name of "Ox fords."
\pend


\pstart
1.31  Thus Daphnis, against all expectation, was saved from the double danger of the robbers and shipwreck.  When he came to land, and found Chloe weeping and smiling through her tears, he threw himself into her arms, and asked her what she had meant by playing on the pipe.  And she told him everything, how she had run to Dorcon for help, how his cows had been trained to obey the sound of the pipe, what strain she had been bidden to play, and how Dorcon had died: only, from a feeling of modesty, she said nothing about the kiss she had given him.  Then both resolved to honour the memory of their benefactor, and went with his relatives to bury the unhappy Dorcon.  They heaped earth over him in abundance, and planted a number of cultivated trees round about, and hung up as an offering to the deceased the first fruits of their labours: they poured libations of milk over his grave, crushed grapes, and broke several shepherds' pipes.  His cows lowed piteously, wandering hither and thither the while: and to the herdsmen and shepherds it seemed that they were mourning for the death of their master.  (15>)
\pend


\pstart
1.32  After the burial of Dorcon, Chloe led Daphnis to the grotto of the Nymphs, where she washed him, and then she herself, for the first time in Daphnis's presence, also washed her own fair and beautiful person, which needed no bath to set off its beauty: then, plucking the flowers that were in season, they crowned the statues of the Nymphs, and hung up Dorcon's pipe against the rock as an offering.  After this, they went to look after their sheep and goats, which were all lying on the ground, neither feeding nor bleating, but, I believe, pining for the absent Daphnis and Chloe.  But, as soon as they came in sight, and began to shout and pipe as usual, they jumped up and began to feed: the goats skipped wantonly, as if delighted at the safe return of their master.  Daphnis however could not bring himself to feel happy: for, since he had seen Chloe naked, in all her beauty formerly hidden and then revealed, he felt a pain in his heart, as if it was consumed by poison.  His breath now came rapidly, as if someone was pursuing him: and now failed him, as if exhausted in previous attacks.  Chloe's bath seemed to him more terrible than the sea.  He thought that his soul was still amongst the pirates, for he was merely a young rustic and as yet knew nothing of the thievish tricks of Love.
\pend


\pstart
2.1  It was now the middle of autumn, and the vintage was close at hand; everyone was in the fields, busily intent upon his work.  Some were repairing the wine-presses, others cleaning out the jars: some were weaving baskets of osier, and others sharpening short sickles for cutting the grapes: some were preparing stones to crush those full of wine, others preparing dry twigs which had been well beaten, to be used as torches to light the drawing off of the new wine by night.  Daphnis and Chloe, having abandoned the care of their flocks, assisted each other in these tasks.  Daphnis carried bunches of grapes in baskets, threw them into the press and trod them, and drew off the juice into jars: while Chloe prepared food for the vintagers, and poured some of the older wine for them to drink, while at the same time she picked some of the lowest bunches from the trees.  For all the vines in Lesbos grow low, and are not trained to trees: their branches hang down to the ground, spreading like ivy, so that even a child that is, so to speak, only just out of its swaddling clothes, could reach the grapes.
\pend


\pstart
2.2  As is customary at the festival of Bacchus, on the birthday of the wine, women had been summoned from the neighbouring fields to assist; and they cast amorous eyes on Daphnis, and extolled him as vying with Bacchus in (16>) beauty.  One of them, bolder than the rest, kissed him, which excited Daphnis, but annoyed Chloe.  On the other hand, the men who were treading the wine presses made all kinds of advances to Chloe, and leaped furiously, like Satyrs who had seen some Bacchante, declaring that they wished they were sheep, to be tended by her: this, again, pleased Chloe, while Daphnis felt annoyed.  Each wished that the vintage was over, and that they could return to the familiar fields, and, instead of uncouth shouts, hear the sound of the pipe and the bleating of their flocks.  In a few days the grapes were gathered in, the casks were full of new wine, and there was no need of so many hands: then they again began to drive their flocks down to the plain, and joyfully paid homage to the Nymphs, offering them grapes still hanging on the branches, the first fruits of  the vintage.  Even before that they had never neglected them as they passed by, but when they drove their flocks to pasture, as well as on their return, they reverently saluted them; never omitting to bring them a flower, some fruit, some green foliage, or a libation of milk.  And they afterwards reaped the reward of this piety from the Gods.  Then they gambolled like dogs loosed from their bonds, piped, sang to the goats, and wrestled sportively with the sheep.
\pend


\pstart
2.3  While they were thus amusing themselves, an old man appeared before them, clad in a goatskin, with shoes of undressed leather on his feet, and carrying a wallet, a very old one, round his neck.  Seating himself close by them, he addressed them as follows: "My children, I am old Philetas: I have sung many songs to these Nymphs, I have often played the pipe to Pan yonder, and guided a whole herd of oxen by my voice alone.  I am come to tell you what I have seen, and to declare to you what I have heard.  "I have a garden, which I have planted and cultivated myself, ever since I became too old to tend my flocks.  You will always find there everything that grows, in its proper season: in spring, roses, lilies, hyacinths, single and double violets: in summer, poppies, wild pears, and all kinds of apples: and, in the present autumn season, grapes, figs, pomegranates, and green myrtles.  Every morning flocks of birds assemble in the garden, some to seek food, others to sing: for it is thickly shaded by trees, and watered by three fountains.  If you were to remove the wall that surrounds it you would think it was a native forest.
\pend


\pstart
2.4  "When I went into my garden yesterday about mid-day, I saw a lad under the myrtles and pomegranate-trees, with some of their produce in his hands: he was white as milk and ruddy as fire, and his body shone as if he had just been bathing.  He was naked and alone, and he was amusing himself with plucking (17>) the fruit as if the garden had belonged to him.  I rushed at him to seize him, being afraid that, in his wantonness, he might break my trees: but he nimbly and easily escaped my hand, now running under the rose-bushes, now hiding himself under the poppies, like a young partridge.  I have often had trouble in chasing young kids, and tired myself with running after newly-born calves: but this was a wily creature, and could not be caught.  Being an old man, and obliged to support myself with a stick, I soon became tired: and, being afraid that he might escape, I asked him to which of my neighbours he belonged, and what he meant by plucking the fruit in a stranger's garden.  He made no answer, but, coming close to me, laughed quietly, flung some myrtle berries at me, and, somehow or other, appeased my anger.  I asked him to come to me without fear, and I swore by my myrtles, and, in addition, by my apples and pomegranates, that I would let him pluck the fruits of my trees and cull my flowers whenever he pleased, if he would only give me one kiss.
\pend


\pstart
2.5  "Then, laughing loudly, he began to speak in a voice sweeter than that of a swallow, or nightingale, or swan as old in years as myself: 'It would be easy for me to kiss you, Philetas: for my wish to be kissed is stronger than your desire to become young again: but look to it whether the gift is suitable to your age.  For, when you have once kissed me, your years will not exempt you from a desire to pursue me: but neither the hawk, nor eagle, nor other bird that is swift on the wing can catch me.  I am not a child, even though I seem to be: I am older than Kronos, more ancient than all time.  I knew you in the bloom of your first youth, when you tended your numerous flock in yonder marsh, and I was by your side when you played upon your pipe under the beech trees, when you were in love with Amaryllis, but you did not see me; and yet I was very close to her.  I gave her to you, and the fruit of your union has been stalwart sons, good herdsmen and labourers.  But now Daphnis and Chloe are my care: and, when I have brought them together in the morning, I come into your garden, to enjoy the sight of the plants and flowers, and to bathe in this spring.  This is why all the produce of your garden is fair to see, since it is watered by my bath.  Look whether any branch is broken, whether any fruit is plucked, whether any flower is trodden upon, or your springs disturbed.  Think yourself happy that you are the only man who has seen this child in your old age.'
\pend


\pstart
2.6  With these words, he sprang up, like a young nightingale, upon the myrtles, and, mounting from branch to branch, at length reached the top.  Then I saw that he had wings on his shoulders, and a bow and arrows between the wings and his shoulders, and after that I saw him no more.  But, unless my (18>) grey hairs count for nothing, unless I have grown more foolish with age, you are consecrated to Love, my children, and Love watches over you."
\pend


\pstart
2.7  Daphnis and Chloe were as delighted as if they had heard some fable, and not a true story, and asked what Love was; whether it was a bird or a child, and what it could do.  Philetas replied: "My children, Love is a winged God, young and beautiful.  Wherefore he takes delight in youth, pursues beauty, and furnishes the soul with wings: his power is greater than that of Zeus.  He has power over the elements and over the stars: and has greater control over the other Gods that are his equals than you have over your sheep and goats.  The flowers are all the work of Love; the plants are his creation.  He makes the rivers to run, and the winds to blow.  I have seen a bull smitten with love, and it bellowed as if stung by a gadfly: I have seen a he-goat kissing its mate, and following it everywhere.  I myself have been young, and was in love with Amaryllis: then I thought neither of eating nor drinking, and I took no rest.  My soul was troubled, my heart beat, my body was chilled: I shouted as if I were being beaten, I was as silent as a dead man, I plunged into the rivers as if I were consumed by fire: I called upon Pan, himself enamoured of Pitys, to help me: I thanked Echo, who repeated the name of Amaryllis after me: I broke my pipes, which, though they charmed my kine, could not bring Amaryllis to me.  For there is no remedy for Love, that can be eaten or drunk, or uttered in song, save kissing and embracing, and lying naked side by side."
\pend


\pstart
2.8  Philetas, having thus instructed them, departed, taking away with him a present of some cheeses and a horned goat.  When they were left alone, having then for the first time heard the name of Love, they were greatly distressed, and, on their return to their home at night, compared their feelings with what they had heard from the old man.  "Lovers suffer: so do we. They neglect their work: we have done the same.  They cannot sleep: it is the same with us.  They seem on fire: we are consumed by fire.  They are eager to see each other: it is for this that we wish the day to dawn more quickly.  This must be Love, and we are in love with each other without knowing it. If this be not love, and I am not beloved, why are we so distressed?  Why do we so eagerly seek each other?  All that Philetas has told us is true.  It was that boy in the garden who once appeared to our parents in a dream, and bade us tend the flocks.  How can we catch him?  He is small and will escape.  And how can we escape him?  He has wings and will overtake us.  We must appeal to the Nymphs for help.  But Pan could not help Philetas, when he was in love with Amaryllis.  Let us, therefore, try the remedies of which he told us: let us kiss and embrace each other, and (19>) lie naked on the ground.  It is cold: but we will endure it, after the example of Philetas."
\pend


\pstart
2.9  This was their nightly lesson.  At daybreak they drove out their flocks, kissed each other as soon as they met, which they had never done before, and embraced: but they were afraid to try the third remedy, to undress and lie down together: for it would have been too bold an act for a young shepherdess, even for a goatherd.  Then again they passed sleepless nights, thinking of what they had done, and regretting what they had left undone.  "We have kissed each other," they complained, "but it has profited us nothing.  We have embraced, but nothing has come of it.  The only remaining remedy is to lie down together: let us try it: surely there must be something in it more efficacious than in a kiss."
\pend


\pstart
2.10  With such thoughts as these their dreams were naturally of love and kisses and embraces: what they had not done in the day, they did in a dream: they lay naked together.  The next morning, they got up more inflamed with love than ever, and drove their flocks to pasture, whistling loudly, and hurried to embrace each other: and, when they saw each other from a distance, they ran up with a smile, kissed, and embraced: but the third remedy was slow to come: for Daphnis did not venture to speak of it, and Chloe was unwilling to lead the way, until chance brought them to it.
\pend


\pstart
2.11  They were sitting side by side on the trunk of an oak: and, having tasted the delights of kissing, they could not have enough: in their close embrace their lips met closely.  While Daphnis pulled Chloe somewhat roughly towards him, she somehow fell on her side, and Daphnis, following up his kiss, fell also on his side: then, recognising the likeness of the dream, they lay for a long time as if they had been bound together.  But, not knowing what to do next, and thinking that this was the consummation of love, they spent the greater part of the day in these idle embraces; then, cursing the night when it came on, they separated, and drove their flocks home.  Perhaps they would have found out the truth, had not a sudden disturbance occupied the attention of the whole district.
\pend


\pstart
2.12  Some wealthy young Methymnaeans, wishing to amuse themselves away from home during the vintage, launched a small vessel, manned with their servants as oarsmen, and coasted along the shore of Mitylene, which affords good harbourage, and is adorned with splendid houses, baths, parks, and groves, some natural, others artificial, but all pleasant to dwell in.  Coasting along and putting in to land from time to time, they did no damage, but amused themselves in various ways.  They fastened hooks to the end of a fine (20>) line attached to the end of a reed, and caught fish from a rock that jutted out into the sea: or, with dogs and nets, captured the hares which were scared by the noise of the labourers in the vineyard; or again, they set snares for ducks, wild geese, and bustards, which, besides affording them sport, provided them with an addition to their repast.  If they wanted anything else, they bought it from the villagers, at a higher price than it was worth.  They only needed bread, wine, and lodging, for, as it was late in the autumn, they did not think it was safe to pass the night on the water: they accordingly drew up the ship on land, being afraid of a storm by night.
\pend


\pstart
2.13  It chanced that a peasant, being in need of a rope to lift up the stone that was used for crushing the grapes after they had been trodden (his own [rope] being broken), went secretly down to the sea-shore, and, finding the ship unguarded, unfastened the cable, took it home, and used it for what he wanted.  In the morning the young Methymnaeans looked everywhere for the rope, and, as no one admitted the theft, after abusing their hosts, they put out to sea again.  Having sailed on about thirty stades, they put in at that part of the coast where was the estate on which Daphnis and Chloe dwelt: since it seemed to them to be a good country for coursing.  But, as they had no rope with which to moor their vessel, they twisted some long green osiers into a cable, and with them fastened it to land: then, having let loose their dogs to scent the game in the most likely spots, they spread their nets.  The dogs, running in all directions and barking, frightened the goats, which left the hills and fled hastily in the direction of the sea.  There, finding nothing to eat in the sand on the shore, some of them, bolder than the rest, went up to the boat, and gnawed off the osiers with which it was fastened.
\pend


\pstart
2.14  It so happened that the sea was rather rough, as there was a breeze blowing from the mountains: and, as soon as the boat was unfastened, the tide carried it away into the open sea.  When the young Methymnaeans saw what had occurred, some of them ran down to the shore, and others called their dogs together: and all raised such a shout that all the labourers hurried up from the neighbouring fields.  But it was all in vain: for, as the breeze freshened, it bore away the vessel down the current with irresistible force.  Then the Methymnaeans, having thus sustained a considerable loss, looked for the keeper of the goats, and, having found Daphnis, flogged him and stripped him of his clothes.  One of them, taking up a dog-leash, twisted Daphnis's hands behind his back, intending to bind him.  He shouted loudly as he was being beaten, and implored the countrymen to help him, above all Lamon and Dryas.  They, being vigorous old men, whose hands were hardened (21>) by their labours in the fields, assisted him stoutly, and demanded that a fair inquiry should be held into what had taken place.
\pend


\pstart
2.15  As the others who had come up pressed the same demand, the herdsman Philetas was chosen as umpire: for he was the oldest of those present, and he had the reputation amongst his fellow villagers of being perfectly impartial.  First the young Methymnaeans briefly and clearly made their complaint:  "We came to these fields to hunt.  We had fastened our boat to the shore with a green osier withy, and left it there: after which, we set out with our dogs to look for game.  Meanwhile, this man's goats went down to the shore, ate the osiers, and set loose the boat.  You yourself saw it being carried away out to sea: what do you think was the value of the property with which it was loaded? of the clothes and dog trappings, besides money enough to purchase this estate?  Wherefore, by way of recompense, we claim that we have a right to carry away this rascally goatherd, who pastures his flock at the sea-shore as if he were a sailor."
\pend


\pstart
2.16  Such was the charge brought by the Methymnaeans.  Daphnis, although suffering terribly from the blows which he had received, seeing Chloe amongst those present, made light of the pain, and spoke as follows:  "I tend my goats properly.  No one in the village has ever complained of a goat of mine browsing in his garden or breaking down his sprouting vines.  It is the fault of these sportsmen, who have dogs so badly broken that they keep running about and barking so loudly that, like so many wolves, they have driven my goats from the hills and plains to the seashore.  But they have eaten the osiers: could they find any grass, or wild arbutus, or thyme to eat on the sand?  Again, their boat had been destroyed by the winds and waves: the storm, not my goats, is to blame for this.  Again, there was a large store of clothes and money on board: who would be so foolish as to believe that a boat, carrying so valuable a freight, would have been fastened with nothing but a rope made of osier-withies?"
\pend


\pstart
2.17  Having thus spoken, Daphnis began to weep, and moved the villagers to great compassion: so that Philetas, who had to pronounce the verdict, swore by Pan and the Nymphs that neither Daphnis nor his goats were in the wrong, but the sea and the wind, which were under the jurisdiction of others.  However, Philetas could not convince the Methymnaeans, who, in the impulse of their rage, again seized Daphnis, and would have bound him, had not the villagers, roused at this, rushed upon them like a flock of starlings, or jackdaws, and speedily rescued Daphnis, who also was stoutly defending himself.  Then, with vigorous blows of their clubs, they routed the (22>) Methymnaeans, and did not cease from pursuing them, until they had driven them out of their territory.
\pend


\pstart
2.18  While they were thus engaged in the pursuit of the Methymnaeans, Chloe quietly led Daphnis to the grotto of the Nymphs, where she washed his face which was smeared with the blood from his nostrils; then, taking a slice of bread and some cheese from her wallet, she gave him to eat, and - what comforted him most of all - she imprinted upon his mouth a kiss sweeter than honey with her tender lips.
\pend


\pstart
2.19  Thus Daphnis had a narrow escape, but the matter did not rest there: for the Methymnaeans, having reached their home with great difficulty on foot, whereas they had come in a ship, full of wounds instead of in the enjoyment of luxury, called an assembly of their fellow citizens, and, holding out olive branches in sign of supplication, besought them to deign to avenge them: they did not, however, utter a word of truth, for fear that they might be laughed at, for having allowed themselves to be so maltreated by a few shepherds: but they accused the Mitylenaeans of having plundered them and seized their vessel and its contents, as if they had been at open war. The Methymnaeans believed what they said when they saw their wounds, and, thinking it their duty to avenge their wrongs, since the young men belonged to the highest families in the place they immediately decided to make war without the usual formalities, and ordered their chief captain to put to sea with ten galleys and ravage their coast: for, as the winter was close at hand, it was not safe to entrust a larger fleet to the mercy of the waves.
\pend


\pstart
2.20  On the following day, the captain put out to sea, using his soldiers as oarsmen, and directed his course towards the coastland of Mitylene.  He carried off a large number of cattle, and a quantity of corn and wine, since the vintage was only just over, and also took prisoner a considerable number of those who were working in the fields. He at last landed on the estate where Daphnis and Chloe were tending their flocks, and carried off everything that he could find.  At the time Daphnis was not with his flock: for he had gone up to the wood to cut some green branches to serve as fodder for the kids during the winter. Seeing the inroad from a distance, he hid himself in the hollow trunk of a dry beech-tree. Chloe, who was with her flocks, being pursued, fled to the grotto of the Nymphs as a suppliant, and besought her pursuers to spare herself and her flocks, out of respect for the goddesses. But it was all in vain: the Methymnaeans insulted the statues and drove off the flocks, and Chloe with them, as if she had been a sheep or a goat, whipping her with switches.
\pend


\pstart
2.21  Their ships being now loaded with all kinds of booty, they made up their minds to sail no further, but directed their course homewards, being afraid of the wintry season and hostile attacks.  Accordingly they rowed away as hard as they could, but they made slow progress, as there was no wind.  Daphnis, when all was quiet, went down to the plain where their flocks had been in the habit of feeding, and finding neither goats nor sheep nor Chloe, but everywhere desolation, and Chloe's pipe, with which she used to amuse herself, lying on the ground, he cried aloud and lamented piteously, now running to the beech under which they used to sit, and now to the seashore, to look for her, and then to the grotto of the Nymphs, where she had taken refuge when she was being carried off.  There he flung himself on the ground and reproached the Nymphs with having abandoned her:
\pend


\pstart
2.22  "Chloe has been carried off from you, O Nymphs, and you have had the heart to see and endure it - she who used to weave for you chaplets of flowers and offer you libations of fresh milk, whose pipe hangs suspended yonder as an offering to you.  No wolf has ever carried off a single goat of mine, but an enemy has carried off the flock and she who tended it with me.  They will flay the goats and sacrifice the sheep, and Chloe will have to dwell in some distant city.  How shall I dare to return to my father and mother without my goats and without Chloe, as if I had proved false to my charge?  For I have no longer anything to tend.  "Here I will lie and await death, or some other attack.  Are you suffering like myself, Chloe?  Do you still remember these fields, these Nymphs, and me?  Or do you find some consolation in the sheep and goats that are your fellow prisoners?"
\pend


\pstart
2.23  While he was thus lamenting, a deep sleep overcame him in the midst of his grief and tears.  The Nymphs appeared to him, three tall and beautiful women, half-naked, without sandals, with their hair floating down their backs, just like their statues.  At first they seemed to feel compassion for Daphnis: then the eldest addressed him in the following words of comfort:  "Do not reproach us, Daphnis: Chloe is more our care than yours.  We took pity on her when she was but a child, and adopted her when she was exposed in this cave and brought her up.  She has no more to do with the sheep and fields than you have to do with the goats of Lamon.  Besides, we have already thought of her future: she shall neither be carried off as a slave to Methymna, nor become part of the enemy's spoil.  We have begged the God Pan, whose statue is under yonder pine, to whom you have never offered so much as a chaplet of flowers in token of respect, to go to the assistance of Chloe: for he (24>) is more used to the ways of camps than we are, and he has often left the country to take part in battle.  He will set out, and the Methymnaeans will find him no contemptible foe.  Be not troubled: arise and show yourself to Lamon and Myrtale, who, like yourself, lie prostrate with sorrow, thinking that you also have been carried off.  Tomorrow Chloe will return with the sheep and goats; you shall tend them and play on the pipe together; leave the rest to the care of Love."
\pend


\pstart
2.24  At this sight and at these words Daphnis started up from sleep.  Weeping both for joy and grief, he did obeisance to the statues of the Nymphs and promised, if Chloe should be saved, that he would sacrifice to them the finest of his goats.  He next ran to the pine tree, beneath which stood the statue of Pan, with the legs of a goat, his head surmounted by horns, in one hand holding his pipe, in the other a bounding goat.  He did obeisance to him also, begged his assistance on behalf of Chloe, and promised to sacrifice a goat to him.  The sun was almost set before he ceased from his tears and entreaties; then, taking up the green branches which he had cut, he returned home, where his reappearance comforted Lamon and Myrtale and filled them with joy.  Having taken a little food, he went to bed: but even then his rest was disturbed by tears.  He prayed that the Nymphs might appear to him again in a dream, and prayed for the speedy coming of the day, on which they had promised him that Chloe should return.  Never had a night seemed so long to him.  Meanwhile, the following events had taken place.
\pend


\pstart
2.25  The Methymnaean captain, when he had proceeded about ten stades, was desirous of giving his men some rest, as they were greatly fatigued with rowing.  Accordingly, having reached a promontory which jutted out into the sea in the shape of a crescent, the bay of which afforded a quieter port than any harbour, he cast anchor, but at some distance from the shore, for fear that the inhabitants might annoy him; then he allowed his crew to enjoy themselves undisturbed.  Since they were abundantly supplied with everything, they drank and made merry, as if they had been celebrating a feast in honour of a victory.  But, when night began to fall and put an end to their enjoyment, suddenly the whole earth appeared in flames: the splash of oars was heard upon the waters, as if a numerous fleet were approaching.  They called upon the general to arm himself: they shouted to each other: some thought they were already wounded, others lay as if they were dead.  One would have thought that they were engaged in a battle by night, although there was no enemy.  (25>)
\pend


\pstart
2.26  After a night thus spent, a day followed even more terrible to them than the night.  They saw Daphnis's goats with ivy-branches, loaded with berries, on their horns: while Chloe's rams and ewes were heard howling like wolves: Chloe herself appeared, crowned with a garland of pine.  Many marvellous things also happened on the sea.  When they attempted to raise the anchors, they remained fast to the bottom: when the oars were dipped into the water to row, they snapped.  Dolphins, leaping from the waves, lashed the ships with their tails, and loosened the fastenings.  From the top of the steep rock overhanging the promontory was heard the sound of a pipe: but the sound did not soothe the hearers, but terrified them, like the blast of a trumpet.  Then, smitten with affright they ran to arms, and called upon their invisible enemies to appear: after which, they prayed for the return of night, hoping that it might afford them some relief.  All who possessed any intelligence clearly understood that all the marvellous things that they had seen and heard were the work of God Pan, who was angry with them for some offence they had committed against him: but they could not guess the cause of it, for they had not plundered any spot that was sacred to him.  At last, however, at mid-day, when their general had fallen asleep, not without the intervention of the Gods, Pan himself appeared to him and spoke as follows:
\pend


\pstart
2.27  "O most impious and sacrilegious of men!  What has driven your frenzied minds to such audacity?  You have filled with war the country that I love, and have carried off the herds of cattle and flocks of sheep and goats entrusted to my care: you have dragged away from my altars a young girl whom Love has reserved for himself, to adorn a tale.  Nay, you did not even respect the presence of the Nymphs, nor me, the great God Pan.  Wherefore you shall never again see Methymna with such booty on board, nor shall you escape this pipe, which has so smitten you with alarm: I will swamp you in the waves and give you as food to the fishes, unless you speedily restore Chloe and her flocks, sheep and goats, to the Nymphs.  Arise then, put ashore the young girl with all that I have mentioned: and then I will guide your course by sea, and Chloe's by land."
\pend


\pstart
2.28  Alarmed at this vision, Bryaxis - that was the captain's name – started up, summoned the commanders of the ships, and ordered them to search for Chloe with all speed amongst the captives.  They soon found her and brought her before him: for she was sitting down, with a pine garland on her head.  Recognising by this that it was she to whom his vision referred, he put her on board his own vessel, and conveyed her to land.  As soon as she had gone ashore, the sound of the pipe again made itself heard from the summit of the (26>) rock, not martial and awe-inspiring, as before, but playing a pastoral air such as shepherds play when driving out their flocks to feed.  Then immediately the sheep hurried down the gangway, without stumbling: while the goats descended with even greater confidence, being accustomed to climb steep places.
\pend


\pstart
2.29  Then the sheep and the goats danced, skipped, and bleated around Chloe, as if they rejoiced with her: but the herds and flocks of the other shepherds remained where they were in the hollow ship, as if the sound of the pipe had not summoned them.  While all were lost in admiration at this, and were singing the praises of Pan, stranger sights were seen on both elements.  For the vessels of the Methymnaeans unmoored themselves of their own accord, before the anchors were pulled up, and a dolphin, leaping out of the sea, piloted the commander's ship: on land the sweet sounds of a pipe guided the goats and sheep, although no one could be seen playing upon it.  Thus the two flocks went on, feeding the while, delighted to hear such strains.
\pend


\pstart
2.30  It was about the time when the flocks were being driven to the plains after mid-day, when Daphnis, perceiving from a lofty hill the approach of Chloe and the herds, with a loud cry of "O Nymphs! 0 Pan!" hastened down, ran towards Chloe, and, after embracing her, fainted from excess of joy.  Even the hot kisses of Chloe, as she clasped him in her arms, scarcely revived him; but at last, having regained consciousness, he made his way to the well-known beech, and, sitting on its trunk, inquired of her how she had managed to escape her numerous foes.  Then she told him everything: the ivy that grew on the horns of her goats, the roaring of the sheep, the garland of pine-leaves that sprouted upon her head, the fire that blazed forth upon the land, the noise of oars upon the sea, the two different sounds of the pipe, the martial and the peaceful, the horrors of the night, and how she had been guided on the road which she did not know by the sound of sweet music.  Then Daphnis, recognising the vision of the Nymphs and the influence of Pan, told her in turn all that he had seen and  heard, and how that, when he was on the point of death, his life had been restored by the Nymphs.  Then he sent her to fetch Dryas and Lamon, and all that was necessary for sacrifice: and, taking the choicest of his goats, he crowned it with ivy, just as the enemy had seen them, poured a libation of milk between its horns, sacrificed it to the Nymphs, hung up and flayed it, and consecrated its skin to them as a votive offering.
\pend


\pstart
2.31  When Chloe had returned, together with Dryas and Lamon and their wives, he roasted part of the flesh and boiled the rest, after having offered the (27>) firstlings to the Nymphs, and poured a libation from a full bowl of sweet wine.  Then, having spread couches of leaves on the ground for the use of the guests, he enjoyed himself eating and drinking; but at the same time he kept an eye upon his flocks, for fear that a wolf might attack them.  After this they sang some hymns in honour of the Nymphs, composed by some ancient shepherds.  When night came on, they lay down in the fields, and on the following day bethought them of Pan.  They crowned the goat that led the flock with branches of pine, and led him to the tree under which stood the image of the God: then, having poured a libation of wine over him, they sang praises to Pan, sacrificed, hung up, and flayed the goat.  They roasted part of the flesh and boiled the rest, and set it down close by in the meadow on green leaves.  The skin with the horns was hung up on the pine tree near the statue, an offering of shepherds to the shepherds' God.  They also gave him of the firstlings, and poured libations in his honour from a larger bowl, while Chloe sang, and Daphnis played the flute.
\pend


\pstart
2.32  After this they sat down and refreshed themselves.  While they were thus engaged, by chance the herdsman Philetas came up, bringing some garlands of flowers to Pan, and some vine-branches full of bunches of grapes.  He was accompanied by his youngest son Tityrus, a fair and impudent lad, with reddish hair and grey eyes, who ran and skipped along like a kid.  When they saw Philetas and his son, the others, jumping up, went with them to place the garlands on the statue of Pan, and hung the vine shoots on the branches of the pine: then, making Philetas sit down with them, they invited him to share their feast.  After the manner of old men who are somewhat heated with wine, they began to tell all sorts of tales: how they tended their flocks when they were young, and how they had escaped many attacks of robbers.  One boasted of having slain a wolf, and another (this was Philetas) of being inferior in his skill on the pipe to Pan alone.
\pend


\pstart
2.33  Daphnis and Chloe begged him to give them a specimen of his skill, and to play on his pipe at a feast in honour of the God who delighted in such music.  Philetas consented, although complaining that his years had left him but little breath, and took Daphnis's pipe.  But it was too small for the display of great skill, being only fit for a lad to play upon.  Philetas therefore sent Tityrus to his cottage, which was about ten stades distant, to fetch his own pipe.  The lad, throwing off his smock, ran off as swiftly as a fawn: meanwhile, Lamon offered to tell them the story of the pipe, which a Sicilian goatherd had related to him in return for the present of a goat and a pipe.  (28>)
\pend


\pstart
2.34  "This pipe in former times was not a musical instrument, but a beautiful maiden, who had a melodious voice.  She tended goats, sported with the Nymphs, and sang as now.  Pan, who saw her tending her goats, sporting, and singing, tried to persuade her to yield to his advances, promising that her goats should always bring forth twins.  But she scoffed at his love, and declared that she would never have anything to do with a lover who was neither a goat nor a perfect man.  Thereupon Pan was proceeding to violence, but Syrinx fled, until at last, weary of running, she flung herself into a swamp and disappeared amongst the reeds.  Pan, enraged, cut down the reeds, and, not finding the maiden, understood what had happened.  Then, cutting some reeds of unequal length, in token of an unequal love, he joined them together with wax and fashioned this instrument.  Thus she who was once a beautiful maiden is now an instrument of music-the pipe."
\pend


\pstart
2.35  Lamon had scarcely finished his story,- which was highly praised by Philetas, who declared that it was sweeter than any song - when Tityrus returned with his father's pipe, which was very large and made of larger reeds than usual, while the waxen fastenings were overlaid with brass.  One would have said that it was the very pipe which Pan had first made.  Then Philetas sat upright, tried all the reeds to see if there was a free current of air, and, finding that his breath passed through unchecked, blew so loud and lustily, that it seemed as if several pipes were being played at once: then, gradually blowing more gently, he changed his tune to a more pleasant strain, and, displaying to them the most perfect skill in pastoral music, he showed them what strains were best for a herd of oxen, or a flock of goats or sheep - sweet and gentle for sheep, loud and deep for oxen, sharp and clear for goats: and all these notes he imitated on a single pipe.
\pend


\pstart
2.36  While all, quietly reclining on the ground, listened in silence, charmed by the music, Dryas got up, begged Philetas to strike up a Bacchanalian air and then began the vintage dance. He seemed in turns to be plucking the fruit, carrying the baskets, treading the grapes, filling the jars, and drinking the new wine: so perfect was the imitation, and so naturally did the dance represent the vines, the wine-press, the jars, and Dryas drinking, to the life.
\pend


\pstart
2.37  The third old man, having thus danced amid the applause of all, embraced Daphnis and Chloe, who quickly started up and began to represent in the dance the story told by Lamon.  Daphnis took the part of Pan, and Chloe that of Syrinx. He tried to persuade her with his entreaties, while she rejected his advances with a smile. He pursued her, and ran on tiptoe, to represent the goat's cloven feet: while Chloe pretended to be weary in her flight and at last (29>) hid herself in the forest which served as a swamp.  Then Daphnis took Philetas's large pipe, drew from it a mournful strain, like the lamentations of a lover, then a passionate air, to touch her heart, and, lastly, a strain of recall, as if he had lost and was seeking her.  So well did he play that Philetas, overcome by admiration, jumped up and embraced him, and made him a present of his pipe, with a prayer that Daphnis in his turn might leave it to a successor like himself.
\pend


\pstart
2.38  Daphnis dedicated to the God Pan the small flute which he had hitherto used, embraced Chloe as if he had really lost and found her again, and drove back his flock, playing on his pipe the while.  As night was close at hand, Chloe also drove back her sheep to the sound of the same pipe: the goats went side by side with the sheep, while Daphnis walked close to Chloe. Thus they enjoyed each other's society until nightfall, when they separated, after promising to drive their flocks to pasture earlier than usual on the following day, which they did. At daybreak, they were in the fields.  Having first saluted the Nymphs, and next, the God Pan, they sat down beneath the oak, where they played upon the pipe, kissed and embraced each other, and lay down side by side, but that was all.  Then they got up and bethought themselves of food, and drank wine, mingled with milk.
\pend


\pstart
2.39  Warmed and further emboldened by what they had drunk, they commenced an amorous contest, and at last swore mutual fidelity.  Daphnis swore by Pan beneath the pine tree that he could not live without Chloe, even for a single day: while Chloe, having entered the grotto, swore by the Nymphs to live and die with Daphnis. So simple and innocent was she that, when she came out of the grotto, she demanded that Daphnis should take a second oath.  "Daphnis," said she, "Pan is an amorous and inconstant God: he was enamoured of Pitys and Syrinx, he never ceases to annoy the Dryads and the Epimelian Nymphs with his solicitations.  Wherefore, even if you forget the oath that you have sworn by him, he will forget to punish you, even though you should have more mistresses than there are reeds in your pipe.  Do you therefore swear by this herd of goats and by the she-goat that reared you, that you will never desert Chloe as long as she remains true to you: but if she breaks her vows to you and the Nymphs, flee from her, loathe her, and kill her like a wolf."  Daphnis, pleased at being thus mistrusted, stood upright in the midst of his flock, and, taking hold of a she-goat with one hand, and of a he-goat with the other, swore to love Chloe as long as she loved him: and that, if she ever preferred another, he would kill himself instead of her. Then Chloe was delighted, and no longer had any doubts: for she was young and a simple (30>) shepherdess, and saw in the sheep and goats the Gods of shepherds and goatherds.
\pend


\pstart
3.1  When the Mitylenaeans heard of the descent of the ten vessels, and were informed by certain persons who came from the country of the plundering of their territory, they considered such outrages on the part of the Methymnaeans unbearable, and resolved to take up arms against them without delay.  Collecting a force of three thousand heavy-armed infantry, and five hundred cavalry, they despatched them by land, under the command of Hippasus, being afraid of journeying by sea during the winter season.
\pend


\pstart
3.2  Hippasus accordingly set out, but was careful not to plunder the territory of the Methymnaeans: he carried off neither flocks nor any kind of booty from the husbandmen and shepherds, considering such conduct to be rather the act of a brigand than of a general.  He marched with all speed against the city itself, hoping to be able to attack it while the gates were left unguarded.  When he was about one hundred stades distant from the city, a herald met them to propose a truce.  The Methymnaeans, having learnt from the prisoners that the Mitylenaeans knew nothing of what had taken place, and that the whole affair was merely an attack of a few shepherds and labourers upon some insolent young men, regretted that they had behaved with greater violence than prudence towards a neighbouring city.  They were accordingly anxious to restore all the plunder that they had taken, and to re-establish friendly relations between the two cities, both by sea and land.  Hippasus sent the herald to the Mitylenaeans, although he had been appointed commander with unlimited power: at the same time he pitched his camp about ten stades from Methymna, to await instructions from his government.  At the end of two days, the messenger returned with orders to the commander to receive the booty, and to return home without committing any act of hostility.  Having the choice between peace and war, they were of opinion that peace would be more advantageous.
\pend


\pstart
3.3  Thus ended the war between Methymna and Mitylene, as suddenly as it had commenced.  Winter came on, a greater hardship than the war for Daphnis and Chloe: suddenly there was a heavy fall of snow, which blocked up all the roads and kept all the labourers indoors.  Torrents rushed down with violence from the mountains, the water was frozen hard, the trees seemed buried beneath the hoar frost: the earth was completely hidden, except around the (31>) fountains and the banks of the streams.  No herdsman led his flocks to pasture, or set foot outside his door: in the morning, at cockcrow, they lighted a large fire, round which they gathered, some twisting hemp, others weaving goats' hairs or making snares for birds.  The only thing they had to think about was to give the oxen in the stalls straw to eat, the sheep and goats in the cotes plenty of leaves, and the pigs in the sties acorns and beech nuts.
\pend


\pstart
3.4  The necessity of remaining at home gladdened the hearts of the other labourers and shepherds, who thus enjoyed some relaxation from their daily task, and, after they had breakfasted, had a long sleep.  In this respect the winter seemed to them more enjoyable than spring, summer, or autumn.  But Daphnis and Chloe had always in mind the pleasant pastimes which they were now forced to abandon - their kisses, embraces, and meals shared together: they passed sad and sleepless nights, and waited for the return of spring as a resurrection.  It grieved them sorely when they touched a wallet from which they had eaten, or saw a pail from which they had drunk together, or a pipe, carelessly thrown aside, that had been a gift of affection.  They prayed to Pan and the Nymphs to put an end to their sorrows, and to show the sun again to them and their flocks; at the same time, they endeavoured to find some means of seeing each other.  Chloe was terribly embarrassed, and did not know what to do: for her supposed mother never left her for a moment: she taught her to card wool, and turn the spindle, and talked to her of marriage.  Daphnis, however, since he had more time to himself, and was cleverer than the young girl, devised the following scheme for seeing her.
\pend


\pstart
3.5  In front of Dryas's cottage, close to the courtyard gate, grew two large myrtles and an ivy plant.  The myrtles almost touched, and the ivy had worked its way between them in such a manner that, spreading its branches on either side like a vine, it formed a kind of arbour shaded by its intertwining foliage: berries, large as grapes, hung down from the branches, upon which settled swarms of birds, which were unable to procure food outside - blackbirds, thrushes, doves, starlings, and all the birds that are fond of feeding on ivy.  Daphnis went out under pretence of catching some of these birds, taking with him a wallet full of honey-cakes, and some birdlime and snares, so as to allay all suspicion.  Although the distance was ten stades at the most, the snow, which was not yet melted, caused him great inconvenience: but Love can make its way through everything, through fire, water, and the snows of Scythia.
\pend


\pstart
3.6  He made all haste to the cottage, and, having shaken the snow from his feet, he set up his snares, and smeared some long sticks with birdlime: then he (32>) sat down waiting for the birds and thinking of Chloe.  The birds came in great numbers, and he caught so many that he had plenty to do to pick them up, kill, and pluck them.  But no one left the house, neither man, nor woman, nor fowl: for all had shut themselves up and were seated round the fire.  Daphnis was utterly at a loss what to do, and cursed his unlucky star: then he thought of venturing to knock at the door, but did not know what plausible excuse to make.  He discussed the matter with himself as follows: "If I say that I have come to fetch something to light a fire with, they will ask me if I have no nearer neighbours.  If I ask for some bread, they will tell me that my wallet is full of food.  If I say I want wine, they will answer that we have only just got in the vintage.  If I say I have been chased by a wolf, they will ask where his footprints are.  If I say that I came to catch birds, they will ask me why I do not return home, now that I have caught enough.  And, as for declaring openly that I want to see Chloe, who would make such a confession to a girl's mother and father?  All such excuses are open to suspicion: the best thing will be to hold my tongue.  I shall see Chloe again in the spring, since I am not destined to see her this winter."  After this soliloquy he picked up his birds and was preparing to go, when, as if Love had taken compassion upon him, the following incident occurred.
\pend


\pstart
3.7  Dryas was at table with his family: the meat had been cut up and distributed, the bread served, and the goblet mixed, when one of the sheep dogs, taking advantage of the moment when no one was watching him, seized a piece of meat, and ran out of doors.  Dryas, greatly enraged (for the piece of meat was his own portion), snatched up a cudgel, and ran after him like another dog.  In his pursuit, he passed close to the ivy, and saw Daphnis who had just flung his spoil over his shoulders, and had made up his mind to depart.  Then, immediately forgetting all about the meat and the dog, he shouted, "Good day, my lad," embraced him, and led him into the house.  When Daphnis and Chloe saw each other, they nearly fainted for joy: however, they managed to keep on their feet, and greeted and saluted each other: and this helped to prevent them from falling.
\pend


\pstart
3.8  Thus Daphnis, having, beyond all expectation, both seen and kissed Chloe, took a seat near the fire, and laid upon the table the doves and blackbirds with which his shoulders were burdened.  He told them how, weary of being obliged to stay at home, he had set out to catch birds, and how he had trapped them with snares and birdlime, owing to their greediness for myrtle and ivy-berries.  They praised his activity, and pressed him to eat some of what the dog had left.  Chloe was bidden to pour out wine for them to drink, which she gladly (33>) did.  She served all the rest first, reserving Daphnis for the last: for she pretended to be angry with him because, having come so far, he was on the point of going home without seeing her.  However, before she offered him the cup, she dipped her lips into it and then gave it to him: and he, although very thirsty, drank the contents slowly, in order to make the pleasure last longer.
\pend


\pstart
3.9  The bread and meat soon disappeared from the table: then, remaining seated, his hosts began to ask him about Myrtale and Lamon, at the same time congratulating them upon having such a support in their old age.  Daphnis was delighted at their commendation, since Chloe heard them: but when they invited him to stay until the following day, when they intended to offer sacrifice to Dionysus, he was ready to fall down and worship them in place of the God.  He immediately pulled out the honey-cakes from his wallet and all the birds which he had caught: and they got them ready for the evening meal.  A second goblet was prepared, and the fire re-lighted: and, when it was night, they sat down to another hearty meal.  After this they sang and told stories, and then went to bed.  Chloe with her mother, and Daphnis with Dryas.  Chloe thought of nothing but the happiness of seeing Daphnis on the following day; while Daphnis satisfied himself with an idle enjoyment: he thought it happiness even to sleep with Chloe's father, clasped him in his arms, and kissed him again and again, dreaming that he was kissing and embracing Chloe.
\pend


\pstart
3.10  At daybreak, it was bitterly cold, and a north wind was nipping everything. The family got up, and having sacrificed a year old ram to Dionysus, lighted a large fire, and made preparations for a meal.  While Nape was making the bread, and Dryas cooking the meat, Daphnis and Chloe, being left to themselves, retired to the ivy bower in front of the yard, where they again set up the nets and smeared the twigs with birdlime, and caught a large number of birds.  In the meantime, they continually kissed each other and held delightful converse. "It was for your sake that I came, dear Chloe."  "I know it, Daphnis."  "It is for your sake that I am destroying these poor birds.  What then am I to you?  Do not forget me."  "I do not forget you, I swear by the Nymphs whom I formerly invoked as the witnesses of my oath in the grotto, whither we will soon return, as soon as the snow melts."  "It lies very deep, Chloe: I am afraid that I myself shall melt first."  "Courage, Daphnis: the sun is hot."  "Would that it were as hot as the fire which consumes my heart."  "You are laughing at me and trying to deceive me."  "No, I swear it by the goats, by which you bade me swear."  (34>)
\pend


\pstart
3.11  While Chloe was thus answering Daphnis, like an echo, Nape called them.  They ran into the house with their catch, which was much larger than that of the previous day.  After they had poured libations to Dionysus, they ate, crowned with garlands of ivy.  Then, when the time came, after they had celebrated the praises of Bacchus and chanted Evoe, Dryas and Nape sent Daphnis on his way, having first filled his wallet with bread and meat.  They also gave him the wood-pigeons and thrushes to take to Lamon and Myrtale, since they knew that they would be able to catch as many as they wanted, as long as the winter and the ivy-berries lasted.  Then Daphnis departed, after kissing them all - Chloe last, that her kiss might remain pure and without alloy.  He afterwards found several fresh excuses for returning, so that they did not pass the winter entirely deprived of the joys of love.
\pend


\pstart
3.12  With the commencement of spring the snow began to melt, the earth again became visible, and the green grass sprouted.  The shepherds again drove their flocks into the fields, Daphnis and Chloe first of all, since they served a mightier shepherd.  They ran first to the grotto of the Nymphs, then to the pine tree and the image of Pan, and after that to the oak, under which they sat down, watching their flocks and kissing each other.  Then, to weave chaplets for the Gods, they went in search of some flowers, which were only just beginning to blossom under the fostering influence of Zephyr and the warmth of the sun: however, they found some violets, hyacinths, pimpernel, and other flowers of early spring.  After they had drunk some new milk drawn from the sheep and goats, they crowned the images, and poured libations.  Then they began to play upon their pipes, as if challenging to song the nightingales, which were warbling in the thickets and gradually perfecting their lamentation for Itys, as if anxious, after long silence, to recall their strains.
\pend


\pstart
3.13  The sheep began to bleat, the lambs gambolled, or stooped under their mothers' bellies to suck their teats.  The rams chased the sheep which had not yet borne young, and mounted them.  The he-goats also chased the she-goats with even greater heat, leaped amorously upon them, and fought for them.  Each had his own mate, and jealously guarded her against the attacks of a wanton rival.  At this sight even old men would have felt the fire of love rekindled within them: the more so Daphnis and Chloe, who were young and tortured by desire, and had long been in quest of the delights of love.  All that they heard inflamed them, all that they saw melted them and they longed for something more than mere embraces and kisses, but especially Daphnis, who, having spent the winter in the house doing nothing, kissed Chloe fiercely, (35>) pressed her wantonly in his arms, and showed himself in every respect more curious and audacious.
\pend


\pstart
3.14  He begged her to grant him all he desired, and to lie with him naked longer than they had been accustomed to do: "This," said he, "is the only one of Philetas's instructions that we have not yet followed, the only remedy that can appease Love."  When Chloe asked him what else there could be besides kisses, embraces, and lying together, and what he meant to do, if they both lay naked together, he replied: "The same as the rams and the he-goats do to their mates.  You see how, after this has been accomplished, the former no longer pursue the latter, nor do the latter flee from the former: but, from that moment, they feed quietly together, as if they had enjoyed the same pleasure in common.  This pastime, methinks, is something sweet, which can overcome the bitterness of love."  "But," answered Chloe, "do you not see that he-goats and she goats, rams and sheep, all satisfy their desire standing upright: the males leap upon the females, who receive them on their backs?  You ask me to lie down with you naked: but see how much thicker their fleece is than my garments."  Daphnis obeyed (!), lay down by her side, and held her for a long time clasped in his arms: but, not knowing how to do what he was burning to do, he made her get up, and embraced her behind, in imitation of the he-goats, but with even less success: then, utterly at a loss what to do, he sat down on the ground and began to weep at the idea of being more ignorant of the mysteries of love than the rams.
\pend


\pstart
3.15  In the neighbourhood there dwelt a labourer named Chromis, already advanced in years, who farmed his own estate.  He had a wife whom he had brought from the city, young, beautiful, and more refined than the countrywomen: her name was Lycaenium.  Every morning she saw Daphnis driving his goats to pasture, and back again at night.  She was seized with a desire of winning him for her lover by presents.  Having watched until he was alone, she gave him a pipe, a honeycomb, and a deerskin wallet, but she was afraid to say anything, suspecting his love for Chloe.  For she had observed that he was devoted to the girl, although hitherto she had only guessed his affection from having seen them exchange nods and smiles.  One day, in the morning, making the excuse to Chromis that she was going to visit a neighbour who had been brought to bed, she followed them, concealed herself in a thicket to avoid being seen, and heard all they said, and saw all they did.  Even Daphnis's tears did not escape her.  Pitying the poor young couple, and thinking that she had a two-fold opportunity - of getting them out of their (36>) trouble and, at the same time, satisfying her own desires - she had recourse to the following stratagem.
\pend


\pstart
3.16  The next day, having gone out again on pretence of visiting her sick neighbour, she proceeded straight to the oak under which Daphnis and Chloe were sitting, and, pretending to be in great distress, cried: "Help me, Daphnis: I am most unhappy.  An eagle has just carried off the finest of my twenty geese: but, as the burden was a heavy one, he could not carry it up to the top of the rock, his usual refuge, but has alighted with his prey at the end of the wood.  In the name of the Nymphs and Pan yonder, I beseech you, go with me into the forest, for I am afraid to go alone: save my goose, and do not leave the number of my flock imperfect.  Perhaps you will also be able to slay the eagle, and he will no longer carry off your kids and lambs.  Meanwhile, Chloe can look after your goats: they know her as well as you: for you always tend your flocks together."
\pend


\pstart
3.17  Daphnis, suspecting nothing of what was to come, immediately got up, took his crook and followed Lycaenium.  She took him as far from Chloe as possible, and, when they had come to the thickest part of the forest, she bade him sit down near a fountain, and said: "Daphnis, you are in love with Chloe: the Nymphs revealed this to me last night.  They told me in a dream of the tears you shed yesterday, and bade me relieve you of your trouble by teaching you the mysteries of love.  These consist not in kisses and embraces alone, or the practices of sheep and goats, but in connexion far more delightful than these: for the pleasure lasts longer.  If then you wish to be freed from your troubles and to try the delights of which you are in search, come, put yourself in my hands, a delightful pupil: out of gratitude to the Nymphs, I will be your instructress."
\pend


\pstart
3.18  Daphnis, at these words, could no longer contain himself for joy: but, being a simple countryman and goatherd, young and amorous, he threw himself at her feet and begged her to teach him without delay the art which would enable him to do to Chloe what he desired: and, as if it had been some profound and heaven-sent secret, he promised to give her a kid lately weaned, fresh cheeses made of new milk, and even the mother herself.  Lycaenium seeing, from his generous offer, that Daphnis was more simple than she had imagined, began to instruct him in the following manner.  She ordered him to sit down by her side just as he was, and to kiss her as he had been accustomed to kiss Chloe, and, while kissing, to embrace her and lie down by her side.  When he had done so, Lycaenium, finding that he was ready for action and inflamed with desire, lifted him up a little, and, cleverly slipping under him, (37>) set him on the road he had sought so long in vain: and, without more ado, Nature herself taught him the rest.
\pend


\pstart
3.19  When this lesson in the mysteries of Love was finished, Daphnis, still as simple as before, would have hastened at once to Chloe, to teach her all that he had learnt, for fear of forgetting it, if he delayed.  But Lycaenium stopped him, and said: "There is something else you must know, Daphnis: I am a woman, and you have not hurt me: for, long ago, another man taught me what I have just taught you, and took my maidenhead as his reward.  But Chloe, when she enters upon this struggle with you for the first time, will weep and cry out, and will bleed as if she had been wounded.  But you need not be afraid at the sight of the blood: when you have persuaded her to yield to your desire, bring her here, where, if she cries, no one can hear her; if she weeps, no one can see her; if she bleeds, she can wash herself in the spring.  And never forget that I made you a man before Chloe."
\pend


\pstart
3.20  After she had given him this advice, Lycaenium went off to another part of the wood, as if she was still looking for her goose.  Daphnis, thinking over what she had said, felt his passion somewhat cooled, and hesitated to press Chloe to grant him anything more than kisses and embraces.  He did not wish to make her cry out, as if she was being attacked by an enemy, or to make her weep, as if she were in pain, or to make her bleed, as if she had been wounded: for, being a novice in the art of love, he was afraid of this blood, thinking it impossible that it could proceed from anything but a wound.  He accordingly left the wood, resolved to enjoy himself with her in the usual way, and, when he reached the place where she was sitting weaving a chaplet of violets, he pretended that he had rescued the goose from the eagle's claws: then he embraced and kissed her, as he had kissed Lycaenium while they toyed together: for this at least he thought was free from danger.  Chloe crowned his head with the chaplet, and kissed his hair, which smelt sweeter to her than the violets: then she took out of her wallet a piece of fruit-cake and some bread and gave him to eat; and, while he was eating, she would snatch a morsel from his mouth, and eat it, just like a young bird pecking from its mother's beak.
\pend


\pstart
3.21  While they were eating, and were even more busily engaged in kissing each other, a fishing-boat came in sight proceeding along the coast.  There was no wind, and the sea was calm: wherefore the crew decided to use their oars, and rowed on vigorously, for they were taking some fish that they had just caught to one of the wealthy citizens.  After the custom of sailors, in order to lighten their toil, one of them sang a song of the sea, which regulated the (38>) movement of the oars, while the rest, like a chorus, joined in with the singer at intervals.  As long as they were in the open sea, their song was but faintly heard, since their voices were lost in the expanse of air: but when they ran under a promontory, or entered a deep crescent-shaped bay, their voices sounded louder, and the refrain of their song was heard more distinctly on the land: for the bottom of the bay terminated in a hollow valley, which received the sound like a musical instrument, and gave back an echo which represented separately the plash of the oars and the voice of the singers, delightful to hear: for, when one sound came from the sea, the answering echo from the land took it up, and lasted longer, since it had commenced later.
\pend


\pstart
3.22  Daphnis, knowing what it was, had eyes for nothing but the sea.  He was delighted at the sight of the boat gliding along the coast swifter than a bird on the wing, and endeavoured to catch some of the airs that he might play them on his pipe.  Chloe, who had never heard an echo before, looked first towards the sea, while the fishermen were singing, and then towards the wood, to see whose voices answered.  When the boat had passed, all was silent in the valley.  Then Chloe asked Daphnis whether there was another sea behind the promontory, or another boat with another crew singing the same strains, and whether they all ceased singing at once.  Then Daphnis smiled pleasantly, and kissed her more tenderly; and, placing upon her head the chaplet of violets, began to tell her the story of Echo, demanding as his reward ten kisses more.
\pend


\pstart
3.23  "There are several kinds of Nymphs, my dear Chloe, Nymphs of the forest, of the woods, and of the meadows: they are all beautiful, and all skilled in singing.  Echo was the daughter of one of these: she was mortal, since her father was a mortal, and beautiful, being born of a beautiful mother.  She was brought up by the Nymphs, and taught by the Muses to play on the flute and pipe, the lyre and the lute, and to sing all kinds of songs: when she grew up, she danced with the Nymphs and sang with the Muses: but, jealous of her virginity, she avoided all males, both Gods and men.  Pan was incensed against the maiden, being jealous of her singing, and vexed that he could not enjoy her beauty.  He inspired with frenzy the shepherds and goatherds, who, like dogs or wolves, tore the maiden to pieces, and flung her limbs here and there, still quivering with song.  Earth, out of respect for the Nymphs, received and hid them in her bosom, where they still preserve their gift of song, and, by the will of the Muses, speak and imitate all sounds, as the maiden did when alive - the voices of men and Gods, musical instruments, and the cries of wild beasts: they even imitate the notes of Pan when playing on his pipe.  And he, when he hears it, springs up and rushes down the mountains, with the sole desire of (39>) finding out who is the pupil who thus conceals himself."  When Daphnis had finished his story Chloe gave him, not ten, but ten times ten kisses: for Echo had repeated nearly all her words, as if to testify that he had spoken nothing but the truth.
\pend


\pstart
3.24  The sun grew daily hotter for spring was at its close and summer was beginning, and the delights of summer returned to them once more.  Daphnis swam in the rivers, Chloe bathed in the springs: he played on the pipe, in rivalry with the rustling of the pines, she emulated the nightingales in her song: they chased the noisy locusts, caught the chirping grasshoppers, plucked the flowers, shook the fruit from the trees and ate it: they even sometimes lay naked together side by side under the same goatskin.  Then Chloe would have soon become a woman, had not Daphnis been deterred by his horror of blood.  Often, being afraid that he might not be able to contain himself, he would not allow Chloe to remove her clothes: whereat she was astonished, but was too bashful to inquire the reason.
\pend


\pstart
3.25  During this summer, a number of suitors for the hand of Chloe presented themselves, coming from all parts to ask her of Dryas in marriage.  Some brought presents, others made lavish promises.  Nape, her hopes being thus excited, advised him to let Chloe marry, and not keep a girl of her age at home, who might, at any moment, while tending her flocks, lose her virginity and bestow herself upon some shepherd for a present of roses or apples: it would be better, said she, to make her mistress of a home and to keep the presents they had received for their own son lately born.  Sometimes Dryas felt tempted by these arguments: for each of the suitors made far handsomer offers than might have been expected in the case of a simple shepherdess; but at other times he came to the conclusion that the girl was too good for a rustic husband, and that, if she ever found her parents again, they might make him and Nape rich.  He accordingly put off answering from day to day, receiving in the meantime a considerable number of presents.  Chloe, seeing all this, was overcome with grief, which she for a long time concealed from Daphnis to avoid giving him pain: but at last, as he importuned her with questions, and was even more unhappy than if he knew all, she told him everything - her numerous and wealthy suitors, Nape's reasons for hastening on her marriage, and how Dryas, without absolutely refusing his consent, had deferred his answer to the next vintage.
\pend


\pstart
3.26  When Daphnis heard this, he nearly went out of his mind: he sat down and began to weep, declaring that he should die if Chloe no longer came to tend her flocks in the fields; and not he alone, but her sheep also, if they lost such a (40>) shepherdess.  Then, having recovered himself a little, he took courage and thought of asking her father for her hand himself.  He already reckoned himself one of her suitors, and hoped to be easily preferred before the rest.  One thing alone disturbed him: Lamon was not rich, and even though (?) he had been rich, he was not free: this alone made his chances slighter.  Nevertheless, he decided to prefer (?) his suit, and Chloe approved his resolution.  He did not, however, venture to speak directly to Lamon, but, feeling bolder with Myrtale, he told her of his love and spoke to her of his wish to marry Chloe.  At night, she told Lamon, who was greatly annoyed at the proposal: he sharply rebuked her for wanting to marry, to the daughter of a simple shepherd, a youth who, to judge from the tokens found with him when he lay exposed, might look forward to a higher destiny, and who, if he found his parents again, might not only grant them their freedom, but might bestow upon them a larger estate even than the one on which they worked.  Myrtale, fearing that Daphnis might do something desperate, or even take his own life, if he lost all hope of winning Chloe, gave him other reasons for Lamon's refusal.  "We are poor, my son," she said to him, "we rather want a bride who will bring a dowry with her: while they [Dryas and Nape] are wealthy, and seek wealthy suitors.  But, come, persuade Chloe, and let her try and persuade her father, not to ask for a large settlement, but to allow you to marry.  No doubt she loves you and would prefer for her bed fellow a handsome youth, though poor, to an ape, however wealthy."
\pend


\pstart
3.27  Myrtale, who never expected that Dryas would give his consent, since there were far wealthier suitors for the hand of Chloe, thought that she had very cleverly avoided the question of the marriage.  Daphnis, for his part, could find nothing to say against this: but, finding how little chance he had of getting what he wanted, he did what poor lovers usually do - he began to weep, and again implored the assistance of the Nymphs, who appeared to him at night, while he was asleep, in the same dress and form as on the first occasion.  The eldest of them again addressed him: "Chloe's marriage is the business of another God: but we will give you some presents which will soften the heart of Dryas.  The vessel which belonged to the young Methymnaeans, the osier cable of which your goats formerly ate, was carried far out to sea all that day by the winds.  But, during the night, when a violent breeze blew from the sea, it was driven ashore on the rocky promontory.  The vessel was shattered to pieces, and nearly all that was in it was lost: but a purse of three thousand drachmas was cast up by the waves, and it now lies upon the shore, hidden under some seaweed, close to a dead dolphin, the stench from which is so noisome that no passer-by will go near it.  Go, take the purse, and give it to (41>) Dryas.  It is enough for you now to show that you are not poor: but a day will come when you will be even wealthy."
\pend


\pstart
3.28  With these words, they disappeared, and night with them.  At daybreak, Daphnis jumped up full of joy, and eagerly drove his goats to pasture.  Having kissed Chloe and paid his respects to the Nymphs, he went down to the shore, saying he was going to bathe, and walked along the sand on the beach, looking for the three thousand drachmas.  He had not to trouble himself long: for the evil smell of the dolphin, which lay rotting on the shore, soon reached his nostrils.  Following the smell as a guide, he soon reached the spot, removed the seaweed, and found the purse full of money.  He took it, stowed it away in his wallet, and, before departing, gave thanks to the Nymphs and the sea itself: for, although he was a goatherd, he began to think that the sea was pleasanter than the earth, since it had assisted his marriage with Chloe.
\pend


\pstart
3.29  Having gained possession of the three thousand drachmas, he delayed no longer.  He thought himself the richest man, not only amongst the husbandmen in the neighbourhood, but of all men living, hastened to Chloe, told her of the dream, showed her the purse, told her to mind the flocks till he returned, and then ran with all speed to Dryas, whom he found with Nape, beating some wheat on a threshing-floor.  Then, quite confidently, he approached the subject of marriage: "Give Chloe to me to wife: I know how to play on the pipe, to prime vines, and to plant trees: I also know how to plough, and to winnow the corn in the breeze: how I can tend flocks, Chloe herself can testify.  I had fifty goats at first, I have doubled their number.  I have reared some fine large he-goats, whereas before I was obliged to borrow those belonging to others.  I am young and your neighbour, against whom no one has any complaint.  I was brought up by a goat, Chloe by a sheep: and, though I am so far superior to her other suitors, I will not be outdone by them even in presents.  They will give you some goats and sheep, a yoke of mangy oxen, or some corn, not enough to feed a few fowls: but I will give you these three thousand drachmas.  But let no one know of this, not even my father Lamon."  With these words, he offered Dryas the money, and embraced him.
\pend


\pstart
3.30  When Dryas and Nape saw so large a sum of money, they immediately promised him Chloe in marriage, and undertook to persuade Lamon to give his consent.  Daphnis and Nape remained, driving the oxen round, and beating out the ears with the threshing machines.  Dryas, having first stored away the money with the tokens, hastened to Lamon and Myrtale, to ask for the hand of Daphnis for their daughter, a most unusual proceeding.  He found them measuring out some barley that had lately been winnowed, and greatly (42>) disheartened, because the crop was disproportionate to the seed that had been sown.  He tried to console them, saying that the same complaint was to be heard everywhere: and then asked the hand of Daphnis for Chloe, saying: "Although others offer much for the honour, I will take nothing from you, but will rather give you something out of my own purse.  They have been brought up together, and while tending their flocks, have become so attached to each other, that it would be hard to separate them: and they are now both of marriageable age."  This and more said Dryas, as a man who was to have 3,000 drachmas for a reward, if he persuaded Lamon and Myrtale.  Lamon, being no longer able to allege his poverty as an excuse (since the parents of the girl did not reject the alliance), nor the age of Daphnis (for he was now a well-grown youth), nevertheless shrunk from stating the real reason of his hesitation, namely, that Daphnis was above such a connection.  He remained silent for a while, and then said:
\pend


\pstart
3.31  "You do right in preferring neighbours to strangers, and in esteeming riches above honourable poverty.  May Pan and the Nymphs reward you for it.  I myself am anxious for this marriage: for I should be mad, seeing that I am now an old man, and have need of more hands to help me, if I did not consider it a great honour to enter into an alliance with your family.  Chloe herself is much sought after, being a good and beautiful girl.  But, as I am a serf, I have nothing of which I can dispose: I must first inform my master and gain his consent.  Come then, let us put off the marriage until autumn, when, according to those who have visited us from the city, he will be here.  Then they shall become man and wife: in the meantime, let them love each other like brother and sister.  But let me tell you this, Dryas: you are asking for the hand of a youth whose station is superior to our own."  When he had thus spoken, Lamon kissed Dryas, and offered him wine to drink, for the sun was at its height: then he accompanied him part of the way home, with every mark of affection.
\pend


\pstart
3.32  Dryas, who had listened attentively to Lamon's last words, began to think, as he was walking along, who this Daphnis might be: "He was reared by a goat, as if the Gods watched over him: he is fair to look upon, and in no way resembles this snub-nosed old man or his bald-headed wife.  He has been able to lay his hands upon three thousand drachmas, a larger sum than a man in his position could make out of pears.  Was be exposed by some one, like Chloe? did Lamon find him, as I found her? were any tokens found with him, like those I found with Chloe?  If this be so, O Pan and you, dear Nymphs, perhaps (43>) Daphnis will one day find his parents and find out the mystery attached to Chloe."  Thus reflecting and dreaming, Dryas went on until he reached the threshing floor, where he found Daphnis eagerly waiting to hear what news he had brought.  He cheered him, called him his son-in-law, promised that the marriage should take place in the autumn, and pledged him his word that Chloe should never marry anyone but Daphnis.
\pend


\pstart
3.33  Daphnis, then, quicker than thought, without tasting food or drink, ran straight to Chloe, whom he found milking the cows and making cheese.  He told her the good news of their approaching marriage, and kissed her, openly and without concealment, as his betrothed, and assisted her in all her tasks.  He drew the milk into the pails, curdled the cheeses in the crates, and put the lambs and kids under their mothers.  When all this was done, they washed themselves, ate and drank, and went in search of ripe fruit, of which they found abundance, since it was the fruitful season of the year-wild and garden pears and apples, some fallen on the ground, and others still on the trees.  Those on the ground were more fragrant, and smelt like wine: those on the trees were fresher, and glittered like gold.  There was one apple-tree, the fruit of which had already been plucked, and which was stripped of its fruit and leaves.  All its branches were bare, and only a single apple remained on the topmost bough, fine and large, more fragrant than all the rest.  He who had plucked the others had not ventured to climb so high, or had forgotten to take it: or it may be that so fine an apple was reserved for a love-sick shepherd.
\pend


\pstart
3.34  When Daphnis saw this apple, he was eager to climb and pluck it, and, when Chloe tried to prevent him, he paid no heed to her, and she went off to her flocks.  Then Daphnis climbed the tree, reached and plucked the apple, and took it to Chloe.  Seeing that she was annoyed, he said: "Dear Chloe, the beautiful seasons have made this apple to grow, a beautiful tree has nourished it, the sun has ripened it, and chance has preserved it.  I should have been blind not to see it, and foolish to leave it there, to fall to the ground and be trodden under foot by a grazing herd or poisoned by some creeping serpent, or to be consumed by time, though admired by all who saw it.  Aphrodite was presented with an apple as the prize of beauty: I present this to you as the meed of victory.  You are as beautiful as Aphrodite: your judges are alike: Paris was a shepherd, I am a goatherd."  With these words, he placed the apple in Chloe's bosom, and, when he drew near, she kissed him, so that he did not regret that he had been bold enough to climb so high, for he was rewarded with a kiss that he valued above the golden apples of the Hesperides.  (44>)
\pend


\pstart
4.1  Meanwhile, one of Lamon's fellow servants arrived from Mitylene and informed him that their master would visit his estate a little before the vintage, to see whether the inroad of the Methymnaeans had done any damage.  As the summer was nearly over, and autumn was close at hand, Lamon made preparations to receive his master, and put his house and garden in order, that he might find everything pleasant to look upon. He cleaned the fountains, that the water might be bright and pure, removed the dung from the yard, that he might not be annoyed by the smell, and put the grounds in order, that they might look as pleasant as possible.
\pend


\pstart
4.2  These grounds were very beautiful, like royal parks.  They were about a stade in length, situated on high ground, about four plethra in breadth, so that they were rectangular in shape. All kinds of trees were to be found there, apple-trees, myrtles, pear trees, pomegranates, fig-trees, and olives: on one side was a lofty vine, which with its black grapes overspread the apple and pear trees, as if to contend with them in fruitfulness. These were the cultivated trees: but there were also cypresses, laurels, planes, and pines, over which, instead of the vine, spread branches of ivy, whose large berries turning black looked like ripe grapes. The fruit trees were in the centre of the garden, as if for better protection: those that did not bear fruit stood round them like an artificial fence, and the whole was shut in by a little wall. Everything was admirably arranged and distributed: the trunks of the trees were kept apart, but, overhead, the branches were so intertwined that what was due to Nature seemed to be the work of art.  There were also beds of flowers, some growing wild, others cultivated roses, hyacinths, and lilies that had been planted: violets, narcissuses, and pimpernel, which grew wild.  There was shade in summer, flowers in spring, grapes in autumn, and fruit in every season of the year.
\pend


\pstart
4.3  From this spot the plain could be seen, with the shepherds feeding their flocks; also the sea, and the vessels passing along, which added enjoyment to this delightful spot. In the very centre of the garden, there was a temple and altar of Dionysus, the latter covered with ivy, the former with vine branches.  Within the temple were pictures representing incidents in the life of the God: Semele brought to bed, Ariadne asleep, Lycurgus bound in chains, Pentheus being torn to pieces, conquered Indians, Tyrrhenians changed into dolphins, and everywhere Satyrs and Bacchantes leading the dance.  Nor was Pan (45>) forgotten: he was seated on a rock, playing upon his pipe, so that he seemed to be playing the same air both for those who were treading the wine-press and for the Bacchantes who were dancing.
\pend


\pstart
4.4  Such were the grounds to which Lamon devoted all his attention, lopping off the dry leaves and tying up the vine-branches.  He placed a garland of flowers upon the head of Dionysus, and conveyed water to the flower-beds from a spring which had been discovered by Daphnis, and was hence called "Daphnis's spring."  Lamon also advised Daphnis to get his goats into as good condition as possible, as his master would want to inspect them, since he had not visited his estate for so long a time. Daphnis had no fear of not being praised for the condition of his flock: he had doubled their number, not one of them had been carried off by wolves, and they were fatter than the sheep.  But, as he was eager to do everything to obtain his master's approval of his marriage, he spared no pains to make them sleek and fat, driving them out to pasture in the early morning, and not driving them home until late in the evening.  He took them twice to drink, and carefully sought for the places where there was the best pasturage.  He also took care that there were new drinking vessels, plenty of milk-pails, and larger cheese-vats.  He was so particular that he even anointed their horns, and combed their hair: you would have thought you were looking upon Pan's sacred flock.  Chloe also assisted him in his labours, and, neglecting her sheep, devoted the greater part of her time to the goats: so that Daphnis declared that it was owing to her that they looked in such good condition.
\pend


\pstart
4.5  While they were thus engaged, a second messenger arrived from the city, bidding them gather the grapes as speedily as possible; since he had been ordered to stay until the new wine was made, when he was to return to the city to fetch his master in time for the autumn vintage.  They gave Eudromus (so was the slave called, because he acted as his master's courier) a hearty reception, stripped the vines, pressed the grapes, put the new wine into casks, and cut off a number of branches with the grapes still unpicked, so that those who came from the city might have an idea of the delights of the vintage and might think that they had taken part in them.
\pend


\pstart
4.6  When Eudromus was ready to hurry back to the city, Daphnis gave him several presents, such as a goatherd might have been expected to give some well-made cheeses, a young kid, and the shaggy skin of a white goat, to wear during the winter when he was running messages. Eudromus was highly pleased, kissed Daphnis, and promised to say everything in his favour to his master.  Then he departed, full of kindly feelings: but Daphnis, full of anxiety, (46>) remained with Chloe in the fields.  She felt equally timid, when she remembered that Daphnis, a youth who had never seen anything but goats, mountains, husbandmen, and herself, was now for the first time to see his master, whom he had hitherto only known by name.  She was very anxious to know how Daphnis would address his master, and was greatly disturbed in mind regarding their marriage, for fear it might prove an idle dream.  They kissed each other over and over again, and embraced tenderly: but their kisses were mingled with apprehension and their embraces were tinged with sadness, as if their master were already present, and they were afraid of him or were obliged to keep their love a secret.  While they were in this distress, the following trouble came upon them.
\pend


\pstart
4.7  In the neighbourhood there lived a cowherd named Lampis, a man of insolent and overweening disposition.  He too sought Chloe's hand from Dryas, and had already given him several presents to further his suit.  Seeing that, if his master's consent were obtained, Daphnis would marry her, he cast about for the means of embittering the master against the young couple.  Knowing that he took great pride in his garden, he determined to spoil and rob it of its beauty.  Since, if he cut down the trees, he might be betrayed by the noise, he determined to devote his energies to destroying the flowers.  He waited until it was night, climbed over the low wall, pulled up, broke off, and trod down the flowers like a wild boar, and then withdrew without having been seen by anybody.  The next morning, Lamon went into the garden to water the flowers from his spring; and, when he saw the whole place thus ravaged, at the sight of this desolation, which was clearly the work of an enemy rather than of a robber, he immediately rent his cloak, and invoked the Gods with loud cries.  Myrtale at once threw down what she had in her hands and ran out: Daphnis, who was driving out his goats, turned back: and when they saw what had happened they cried aloud and burst into tears.
\pend


\pstart
4.8  It was idle (?) to lament the loss of the flowers, but the fear of their master made them weep.  Even a stranger would have wept at the sight: the whole place was in disorder, and nothing could be seen but upturned earth and mud.  If by chance some flower had escaped the general destruction, it still looked gay and bright, and retained its former beauty, although lying on the ground.  Swarms of bees hovered round, humming incessantly, as if they too lamented what had happened.  Lamon cried out in his consternation: "Alas! my rose trees, how they are broken!  Alas! my violets, how they are trodden under foot!  Alas! my hyacinths and narcissuses, which the hand of some wretch has uprooted!  The spring will return, but they will blossom no more: the summer (47>) will come, but they will not bloom: autumn will come again, but they will not deck anyone's head.  And you, my lord Dionysus, had you no pity for these unhappy flowers, near which you dwelt, with which I have often crowned your brows?  How can I show the garden to my master?  What will he think when he sees it?  He will hang the old man on a pine tree, like Marsyas: and perhaps Daphnis as well, thinking that his goats have done this damage."
\pend


\pstart
4.9  At these words, they wept even more bitterly, not so much on account of the flowers as for themselves.  Chloe was bitterly distressed, at the thought that Daphnis would be hung: she prayed that their master might not come, and passed her days in bitterness, thinking that she already saw Daphnis under the lash.  One evening, Eudromus came to inform them that the master himself would not arrive for three days, but that his son would be there on the morrow.  They accordingly thought over what had happened, and took Eudromus into their confidence.  He, being well disposed towards Daphnis, advised him to tell everything to their young master beforehand, promising to do his best for them, since he possessed some influence with him, being his foster-brother.  When the day came, they did as he had advised them.
\pend


\pstart
4.10  Astylus arrived on horseback, accompanied by his parasite, also on horseback.  Astylus's beard was only just beginning to grow, but Gnatho's (so was the parasite named) had long been familiar with the razor.  Lamon, together with Daphnis and Myrtale, fell at his feet and besought him to have compassion upon an unfortunate old man, and to save from his father's wrath one who had committed no offence: and at the same time he told him all that had occurred.  Astylus was moved to pity by his supplication: he went to the garden, inspected the damage done to the flowers, promised to make his father relent, and undertook to lay the blame upon his own horses, and to say that they had been fastened up in the garden, but, having become frisky, had broken loose, and trampled down, trodden under foot, and uprooted the flowers.  Lamon and Myrtale wished him all prosperity in return for his kindness: and Daphnis presented him with some kids, cheeses, birds with their young, grapes still on the vine-branches, and apples on the boughs: to these he added some fragrant Lesbian wine, most delightful to drink.
\pend


\pstart
4.11  Astylus, having expressed his satisfaction, went to hunt the hares, like a wealthy young man who had nothing to do but amuse himself, and was visiting the country in search of some fresh diversion.  Gnatho, who knew nothing except how to eat till he could eat no more, and to drink till he was drunk, and was all throat and belly and lust, had carefully observed Daphnis when he brought the presents to Astylus.  Being naturally fond of boys, and (48>) finding Daphnis handsomer than any of the youths in the city, he resolved to make advances to him, thinking that he would find no difficulty in seducing a simple goatherd.  Having made up his mind to this, instead of accompanying Astylus to the chase, he went down to the place where Daphnis was tending his flock, under pretence of looking at the goats, but in reality he had eyes for nothing but Daphnis.  In order to coax him, he praised his goats, and begged him to play a pastoral air upon his pipe: then he promised to obtain his freedom for him shortly, saying that he was all-powerful with his master.
\pend


\pstart
4.12  When Gnatho thought he had won Daphnis's affection, he lay in wait for him one evening as he was driving back his goats from pasture, ran up to him and kissed him.  Then he asked him to turn his back to him and let him do to him what the he-goats did to the she-goats.  Daphnis was slow to understand, but at last he said to himself that, while it was quite natural for he-goats to mount she-goats, no one had ever seen a he-goat mounting a he-goat, or a ram another ram instead of a sheep, or a cock treading a cock in place of a hen.  Meanwhile, Gnatho attempted to lay violent hands upon Daphnis, who dealt him a vigorous blow, which felled him to the ground, since he was already drunk and could hardly stand.  After this, Daphnis ran away as swiftly as a fawn, leaving Gnatho on the ground, more in need of the assistance of a man, than of a boy, to help him along.  From that time Daphnis shunned him altogether, changing the pasturage of his goats from one place to another, avoiding Gnatho as carefully as he sought Chloe.  Nor did Gnatho trouble him any more, when he found that he was not only handsome, but also strong and vigorous.  But he watched for ail opportunity to speak to Astylus about him, hoping that his young master would make him a present of Daphnis, since he knew that he was ready to grant almost every favour he asked.
\pend


\pstart
4.13  For the moment he could do nothing: for Dionysophanes had just arrived with Clearista, and nothing was heard but the noise of animals, slaves, men, and women.  In the meantime, Gnatho set about composing a long and amorous discourse upon Daphnis.  Dionysophanes, whose hairs were already beginning to turn grey, was a tall, handsome man, who need not have shrunk from rivalry with many a young man: in addition to this, he was richer than most men, and none were more virtuous.  On the first day of his arrival, he offered sacrifice to all the Gods who preside over husbandry, to Demeter, Dionysus, Pan, and the Nymphs, and gave a feast to all the household.  On the following days, he went to see how Lamon had done his work: and, at the sight of the ploughed fields, the well-kept vines, and the beautiful garden - for Astylus had taken the blame for the damage done to the flowers - he was (49>) delighted, congratulated Lamon, and promised him his freedom.  After this he went to see the goats and the goatherd.
\pend


\pstart
4.14  Chloe immediately ran away into the forest, feeling bashful and afraid of so many visitors: but Daphnis remained where he was, with a shaggy goat skin fastened round him, and a new wallet hanging from his shoulder, holding in one hand some fresh cheeses, and in the other some sucking kids.  If ever Apollo tended the flocks of Laomedon as a hired servant, he must have looked like Daphnis, who, without saying a word, his face covered with blushes, bowed and presented his gifts.  Then Lamon said: "O master, this is the goatherd: you gave me fifty goats and two he-goats to look after: he has doubled the number of the goats, and increased the he-goats to ten.  You see how fat and sleek they are, what long hair they have, and how sound their horns are.  He has also taught them to understand music: when they hear the sound of his pipe, they are ready to do anything."
\pend


\pstart
4.15  Clearista, who was present and heard what was said, was anxious to put it to the proof: she ordered Daphnis to play on his pipe to his goats as he was accustomed to do, and promised to give him a cloak, a tunic, and a pair of shoes for his trouble.  Daphnis made them sit down as if they were at the theatre, stood up under the beech tree, took his pipe out of his wallet, and, to commence with, drew from it merely a feeble strain.  The goats immediately stood up, and lifted their heads.  Then he piped to pasture and the goats began to browse, with their heads towards the ground.  He played a clear sweet strain, and they all lay down.  He played a shrill air, and they fled towards the forest, as if a wolf was approaching.  After a brief interval, he piped a recall, and they came out of the forest, and ran to his feet.  They obeyed the notes of the pipe more readily than servants obey their masters' orders.  The visitors were astonished, especially Clearista, who swore to give what she had promised to the gentle goatherd who played so well.  Then they returned to the homestead for dinner, and sent Daphnis something from their own table.  Daphnis shared the food with Chloe, highly pleased at tasting city cookery, and feeling sanguine of obtaining his master's consent to his marriage.
\pend


\pstart
4.16  Gnatho, inflamed still more by what he had seen of the goatherd, and considering that life would not be endurable if he did not get possession of Daphnis, waited his opportunity until Astylus was walking in the garden: then, leading him up to the temple of Dionysus, he kissed his hands and feet.  When Astylus asked what was the meaning of his behaviour, and bade him speak, swearing that he would grant whatever favour he asked, Gnatho replied: (50>) "Your poor Gnatho is lost, O master.  I who hitherto cared for nothing but the pleasures of the table, who used to swear that there was nothing more delightful than old wine, who considered your cooks far superior to all the youths of Mitylene - I now think that there is nothing beautiful in the world but Daphnis.  I do not so much as taste the most dainty dishes, although so many are prepared each day - meat, fish, and honey-cakes.  I should like to be a goat, I should like to eat grass and leaves, listening to his pipe and tended by him.  Save Gnatho, I beseech you, and remedy a love that is irremediable.  If you do not, I swear to you by my God that I will take a hearty meal, and then stab myself in front of Daphnis's door; and you will never again call me your dear little Gnatho, as you used to do in jest."
\pend


\pstart
4.17  When Gnatho began to kiss his feet again, Astylus could no longer resist his entreaties, for he was a generous youth, who had himself felt the pains of love.  He promised to ask his father for Daphnis and to take him to the city, nominally as his slave, but really as Gnatho's minion.  Then, wishing to cheer him up, he asked him with a smile if he were not ashamed of being in love with Lamon's son, and why he was so anxious to sleep with this young goatherd, at the same time pretending that the smell of goats disgusted him.  But Gnatho, like one who had gone through the whole course of erotic lore at the tables of debauchees, replied shrewdly enough in regard to himself and Daphnis:  "No lover troubles himself about such things: in whatever form he finds beauty, he is smitten with it.  Men have been known to become enamoured of a plant, a river, or a wild beast: and yet who would not pity a lover who has to fear what he loves?  No doubt the form that I love is that of a slave, but its beauty is free.  Do you see how like his hair is to the hyacinth, how his eyes glitter beneath his brows, like a jewel in a setting of gold?  His face is ruddy, his teeth are white as ivory.  Who would not long for a tender kiss from his lips?  In loving a goatherd, I am but following the example of the Gods.  Anchises was a cowherd, and Aphrodite possessed him: Branchius tended goats, and Apollo loved him: Ganymede was a shepherd, and Zeus carried him up to heaven.  Let us not despise a lad, whose goats we see obey him, as if even they were enamoured of him: let us rather thank the eagles of Zeus for allowing such beauty to remain upon the earth."
\pend


\pstart
4.18  Astylus, who was highly amused by this speech, laughed and told Gnatho that love produced very plausible orators: at the same time, he promised to watch for an opportunity to speak to his father about Daphnis.  But Eudromus had heard all that was said without being seen.  His friendship for Daphnis, whom he considered a worthy young man, and his indignation at the idea of (51>) such beauty being handed over to the insults of a drunken wretch like Gnatho, made him go and tell Daphnis and Lamon at once.  Daphnis, in great consternation, at first thought of flight in company with Chloe, or of dying together with her.  Then Lamon called Myrtale out and said to her: "We are lost, my dear wife: the moment is come to reveal what has long been hidden.  Although the goats and everything else be abandoned, I swear, by Pan and the Nymphs, even though I should be left like a worn-out ox in the stall, that I will no longer hold my tongue in regard to the history of Daphnis.  I will tell how I found him exposed: I will declare how he has been brought up: and I will show all the tokens that I found exposed with him.  That infamous wretch Gnatho shall know what manner of man he is, and who it is that he has the audacity to love.  Do you look after the tokens, and see that I have them ready to hand."
\pend


\pstart
4.19  Having settled this, they went indoors.  Meanwhile, Astylus, finding his father disengaged, hastened to him and asked permission to take Daphnis home with him to the city, declaring that he was a handsome lad and too superior to be left in the country, and that Gnatho would soon teach him city manners.  His father willingly gave his consent, and, having sent for Lamon and Myrtale, told them the good news that Daphnis would in future serve his son Astylus instead of tending goats, and promised to give them two goatherds to take his place.  Then, when all the other slaves had gathered together, delighted at the prospect of having so handsome a fellow-servant, Lamon asked leave to speak, and, on its being granted, began as follows:  "O master, hear a true story from an old man: I swear by Pan and the Nymphs that I will not utter a word that is false.  I am not the father of Daphnis, nor has Myrtale the good fortune to be his mother.  He was exposed when a child by other parents, who perhaps had enough children already.  I found him abandoned, and being suckled by one of my goats, which I buried in the garden when it died: for I loved it because it had performed the part of a mother towards the infant.  I also found some tokens lying by its side: this I confess, master, and also that I kept them: for they show that he belongs to a higher rank of life than our own.  I have no objection to his serving Astylus, for he will be a good servant to a good and honourable master: but I cannot endure that he should become the laughing-stock of the drunken Gnatho, who wants to take him to Mitylene and make him play the part of a woman."
\pend


\pstart
4.20  After this Lamon was silent and burst into tears.  But when Gnatho waxed bolder and threatened to chastise him, Dionysophanes, astounded at what Lamon had said, knitted his brows and ordered Gnatho to hold his tongue: (52>) then he again questioned the old man, exhorting him to speak the truth, and not to invent some story, in order that he might keep his son.  When Lamon persisted in his tale, swore by all the Gods that it was true, and offered to submit to the torture if he had lied, Dionysophanes, with Clearista sitting by his side, carefully considered what he had said. "What object could Lamon have in speaking falsely, seeing that he was to have two goatherds in place of one?  How could a rude peasant have invented such a story?  Again, was it not at the outset incredible that so handsome a youth should be the offspring of an old man like Lamon and a shabby old woman like Myrtale?"
\pend


\pstart
4.21  They determined not to trust any further to conjecture, but to examine the tokens at once, to see if they indicated that Daphnis belonged to a higher rank of life.  Myrtale immediately went to fetch them out of an old sack in which they had been stored away.  When they were brought, Dionysophanes looked at them first, and when he saw the little purple tunic with its golden clasp, and the dagger with the ivory handle, he cried aloud, "O Lord and master Zeus," and called his wife to look: and she, as soon as she saw them, in like manner cried aloud, "O kindly Fates: are not these the jewels which we gave to Sophrosyne to put by the side of our own son when she exposed him?  There is no doubt about it: they are the same.  Dear husband, the child is ours.  Daphnis is your son, and has fed his father's goats."
\pend


\pstart
4.22  While she was still speaking, Dionysophanes kissed the tokens, and wept from excess of joy.  Then Astylus, understanding that Daphnis was his brother, immediately threw off his cloak, and hastened to the garden, wishing to be the first to embrace him.  But when Daphnis saw him coming towards him, accompanied by a number of people, and shouting "Daphnis," thinking that he wanted to seize him, he threw away his wallet and his pipe, and fled towards the sea, intending to throw himself from the top of the rock: and perhaps, by a strange caprice of Fortune, Daphnis, who had just been found, would have been lost, had not Astylus, perceiving his intention, shouted to him: "Stop, Daphnis, fear nothing: I am your brother: your former master and mistress are your parents.  Lamon has told us all about the goat, and shown us the tokens: look, turn around and see how glad and cheerful they seem.  But kiss me first: I swear by the Nymphs that I am speaking the truth."
\pend


\pstart
4.23  Even when he heard this oath, Daphnis was loath to stop: however, he waited for Astylus, and kissed him when he came running up to him.  In the meantime, all the household, men and women servants, and his mother and father came and embraced and kissed him, with tears of joy.  Daphnis welcomed them all affectionately, but especially his father and mother, whom (53>) he clasped to his bosom as if he had already known them for a long time: so quickly does Nature make her claim felt.  For a while he even forgot Chloe: and when he reached the homestead, they gave him a handsome dress (?), and he sat down by the side of his father, who addressed him and Astylus as follows:
\pend


\pstart
4.24  "My sons, I married when I was a very young man, and, after a short time, I became a happy father, as I then imagined.  My first child was a son, the second a daughter, and the third, Astylus.  I thought that three children were enough, and, when another son was born, I exposed him together with these jewels and tokens, which I considered rather as funeral ornaments than as tokens by which he might be afterwards recognised.  But Fortune willed otherwise.  My eldest son and daughter died of the same complaint on the same day: but you, Daphnis, have been preserved to us by the providence of the Gods that we may have greater support in our old age.  Do not bear a grudge against me, my son, for having exposed you: for, though I did so, it was sorely against my will.  Nor do you, Astylus, be annoyed that you will have to share your inheritance, for to a wise man a brother is better than all possessions.  Love one another: as far as wealth is concerned, you need not envy even a king.  For I will leave to both [of you] large estates, a number of clever and industrious servants, gold, silver, and all other blessings that rich men enjoy.  But I specially wish that Daphnis should have this estate, and I make him a present of Lamon and Myrtale, and the goats which he has tended."
\pend


\pstart
4.25  While he was still speaking, Daphnis suddenly started up and said: "You have just reminded me, father: I will go and take my goats to drink: they are thirsty about this time, and are waiting for the sound of my pipe, while I am sitting here."  Hereupon all laughed, at the idea that Daphnis, who had just become a master, should still wish to perform the duties of a goatherd.  They sent someone else to look after his goats, offered sacrifice to Zeus Soter, and held high festival.  Gnatho alone was not present, but, seized with alarm, he remained day and night in the temple of Dionysus, as a suppliant.  The report soon spread that Dionysophanes had found his son, and that the goatherd Daphnis had become master of the estate: and, the next morning, the peasants gathered together from all parts to congratulate the young man, and offer presents to his father, the first to arrive being Dryas, who had brought up Chloe.
\pend


\pstart
4.26  Dionysophanes made them all stay for the festivities: for he had prepared abundance of bread and wine, waterfowl, sucking-pigs, honey-cakes of all kinds, and victims to be offered as a sacrifice to the Gods of the country.  Then Daphnis, having collected all his pastoral equipments, distributed them as (54>) offerings to the Gods.  To Dionysus he consecrated his wallet and goat-skin, to Pan his pipe and flute, to the Nymphs his crook and the milk-pails which he had made himself.  But - so much sweeter is that to which we are accustomed than strange and unexpected good fortune - Daphnis wept as he parted with each of these things.  He did not offer up his milk-pails before he had milked his goats once again, nor his goat-skin before he had put it on again, nor his pipe before he had played upon it: he kissed them all, spoke to his goats, and called his he-goats by name: he also went and drank at the fountain, because he had often done so before with Chloe.  But he did not yet venture to declare his love, since he was waiting for a better opportunity.
\pend


\pstart
4.27  While Daphnis was engaged in these ceremonies, this was what happened to Chloe.  She was sitting down, weeping, while she tended her flock, and lamenting, as indeed was only natural: "Daphnis has forgotten me: he is dreaming of a wealthy match.  Why did I make him swear by his goats instead of the Nymphs?  He has abandoned them as he has abandoned Chloe: even when he was sacrificing to the Nymphs and Pan, he felt no desire to come and see me.  Perhaps he has found some handmaids at his mother's house whom he prefers.  May he be happy: but I can live no longer."
\pend


\pstart
4.28  While she thus gave utterance to her thoughts, the herdsman Lampis came up with a band of peasants and carried her off, being persuaded that Daphnis would no longer care to marry her and that Dryas would accept Lampis as her husband.  As she was being carried off, uttering piercing cries, some one who had seen what had taken place went and told Nape, who informed Dryas, who in his turn told Daphnis.  The latter, almost beside himself, had neither the courage to confess everything to his father, nor the strength of mind to resign himself to this misfortune; he entered the garden-walk, and thus lamented:  "What a painful discovery!  How much better it would have been for me to remain a shepherd!  How much happier I was when I was a slave!  Then I used to see Chloe: but now Lampis has carried her off, and at night he will sleep with her.  But I am drinking and enjoying myself, and in vain have I taken an oath by Pan, my goats, and the Nymphs."
\pend


\pstart
4.29  Daphnis's lamentations were heard by Gnatho, who was concealed in the garden.  Thinking this a good opportunity for making peace with him, he went in search of Dryas, accompanied by some young men of Astylus's retinue, ordered him to conduct him to Lampis's house, and hastened thither with him.  He came upon the herdsman just as he was taking Chloe inside, snatched her away from him, and severely beat the peasants who were with him.  He was anxious to bind Lampis, and to take him away like a prisoner of war, but he got the start and (55>) ran away.  Having accomplished this exploit, Gnatho returned at nightfall.  He found Dionysophanes in bed, but Daphnis was still up, weeping in the garden.  Gnatho conducted Chloe to him, told him what had taken place, begged him not to bear him ill will any longer, but to keep him - for he would be a useful servant - and not to drive him away from his table, otherwise he would die of hunger.  When Daphnis saw Chloe, and clasped her in his arms, he pardoned Gnatho because of the service he had rendered him, and excused himself to Chloe for his own neglect.
\pend


\pstart
4.30  After taking counsel together, they resolved not to mention their marriage as yet: meanwhile, Daphnis would see Chloe secretly, and only tell her mother of his love.  Dryas, however, did not agree with this: he thought it best to tell Daphnis's father, and himself promised to obtain his consent.  At daybreak, he put the tokens which had been found with Chloe into his wallet, and presented himself before Dionysophanes and Clearista, whom he found seated in the garden, together with Astylus and Daphnis.  When all were silent, he addressed them as follows: "A necessity, similar to that which forced Lamon to speak, compels me to reveal what has hitherto been kept a secret.  Chloe is not my daughter, neither did I rear her.  She is the daughter of other parents who exposed her in the grotto of the Nymphs, where she was suckled by an ewe.  I saw this with my own eyes, and when I saw it, I wondered, and brought up the child as my own.  Her beauty is sufficient proof of this: she in no way resembles us.  The tokens also bear witness; for they are too valuable to belong to shepherds.  Look at them, try and discover the girl's parents, and see whether you consider her worthy of marriage with Daphnis."
\pend


\pstart
4.31  Dryas did not say this without a purpose, and it was not lost upon Dionysophanes, who, casting his eyes upon Daphnis, and seeing that he turned pale and was weeping silently, easily discovered the secret of his love.  He accordingly took the greatest pains to verify what Dryas had said, being more anxious about his own son than about a young girl who was a stranger to him.  When he saw the tokens - the gilt shoes, the anklets, and the head dress - he called Chloe to him, and bade her be of good cheer, since she already had a husband, and would soon find her father and mother.  Then Clearista took her and dressed her as became her son's intended wife: while Dionysophanes took Daphnis aside, and inquired of him whether Chloe was a virgin: and when he swore that nothing more had taken place between them than kisses and vows of fidelity, he expressed himself pleased at the oath they had taken, and made them sit down to table.  (56>)
\pend


\pstart
4.32  Then could be seen the power of beauty, when it is adorned: for Chloe, richly dressed, with her hair plaited and her face washed, appeared far handsomer to all who saw her, so that even Daphnis scarcely recognised her.  Leaving the tokens out of consideration, anyone would have been ready to swear that Dryas could not be the father of such a daughter.  However, he was present, and sat on the same couch with Nape, Lamon, and Myrtale.  On the next and following days, victims were sacrificed, goblets of wine were prepared, and Chloe also consecrated to the Gods everything that belonged to her - her pipe, wallet, goat-skin, and milk pails.  She poured some wine into the water of the fountain at the bottom of the grotto, because she had been suckled on its brink, and had often bathed in it: she also crowned with a garland of flowers the tomb of the sheep, which was pointed out to her by Dryas.  She also piped to her flocks, and, having sung a hymn to the Nymphs, she prayed to them that the parents who had exposed her might be found worthy to be allied by marriage with Daphnis.
\pend


\pstart
4.33  When they became tired of the rustic festivities, they resolved to return to the city, to try and find out who Chloe's parents were, and to hasten on the marriage.  Accordingly, in the morning, they packed up their things, and made ready for their journey: but, before they started, they gave Dryas another three thousand drachmas, and to Lamon the privilege of gathering the corn and grapes of half the estate, together with the goats and goatherds, four yoke of oxen, some winter garments, and freedom for himself and his wife.  After this, they set out for Mitylene, with a splendid equipage of horses and chariots.  As they reached the city at night, the inhabitants were not aware of their arrival: but, on the following day, a crowd of men and women assembled round the house.  The former congratulated Dionysophanes on having found a son, and all the more, when they saw how handsome Daphnis was: the latter shared Clearista's joy at having found, not only a son, but a wife for him.  They also were struck with astonishment at Chloe's incomparable beauty.  The whole city was in a state of excitement over the young man and the maiden: their union was already looked upon as a happy one, and hopes were expressed that Chloe's birth might be found to be worthy of her beauty.  More than one wealthy woman prayed to the Gods that she might be credited with being the mother of so beautiful a daughter.
\pend


\pstart
4.34  Dionysophanes, weary with constant thought, fell into a deep sleep, and dreamed a dream.  It seemed to him that the Nymphs were begging Love to give his consent to the marriage.  Then the God unbent his bow, placed it on the ground by the side of his quiver, and ordered Dionysophanes to invite all (57>) the nobles of Mitylene to a banquet, and, when the last cup was filled, to show the tokens to each guest, and to sing the song of Hymen.  Struck with this vision and the directions given by the God, when he rose in the morning, he ordered a sumptuous banquet to be prepared, furnished with every dainty that the land, the sea, the lakes, and rivers could produce, and invited all the nobles of Mitylene.  At evening, after the cup with which libations are offered to Hermes had been filled, one of the attendants brought in the tokens upon a silver vessel, and carried them round and showed them to each of the guests.
\pend


\pstart
4.35  All declared that they did not recognise them, with the exception of one Megacles, who, on account of his great age, had been placed at the end of the table.  As soon as he beheld them, he shouted out loudly:  "What is this I see?  My daughter, what has become of you?  Are you still alive?  Or did some shepherd find these tokens and pick them up?  Dionysophanes, I beseech you, tell me, where did you get these tokens of my child?  Now that you have found Daphnis, do not grudge me the happiness of finding something."  Dionysophanes at first desired him to state how she had been exposed: and Megacles, in as firm a tone and voice as before, replied:  "Formerly I was badly off, for I had spent what I possessed upon the public games and triremes.  While I was thus situated, a daughter was born to me.  Being afraid to bring her up in poverty, I decked her out with these tokens and exposed her, for I knew that there were many people who are ready to adopt the children of others.  She was exposed in the grotto of the Nymphs, and entrusted to the protection of the Goddesses.  In the meantime, Fortune favoured me: my wealth increased daily, but I had no heir, for I have not been fortunate to have even another daughter.  The Gods also, as if to mock me, send me visions at night, announcing that a ewe shall make me a father."
\pend


\pstart
4.36  Then Dionysophanes shouted even louder than Megacles: he started up, brought in Chloe richly attired, and said: "Here is the child you exposed: thanks to the providence of the Nymphs, a ewe nourished this maiden, as a goat suckled Daphnis for me.  Take the tokens and your daughter, and give her to Daphnis as his bride.  We exposed them both: we have found them both: both have been under the care of Pan, the Nymphs, and the God of Love."  Megacles approved, clasped Chloe in his arms, and sent for his wife Rhode.  They slept that night at the house of Dionysophanes: for Daphnis had sworn that he would not entrust Chloe to anyone, not even to her own father.
\pend


\pstart
4.37  At daybreak they agreed to return to the country, at the earnest request of Daphnis and Chloe, who could not get used to city life: besides, they had decided that the wedding should be a rustic one.  They returned to Lamon's (58>) house, where Dryas was presented to Megacles, and Nape to Rhode, and all preparations were made for a brilliant festival.  Megacles consecrated Chloe in presence of the Nymphs, and, amongst other offerings, dedicated the tokens to them, and made up to Dryas the sum of ten thousand drachmas.
\pend


\pstart
4.38  As it was a very fine day, Dionysophanes ordered couches of green leaves to be spread in front of the grotto, invited all the villagers to the festivities, and entertained them handsomely.  Lamon and Myrtale were there, together with Dryas and Nape, Dorcon's relations, Philetas and his sons, Chromis and Lycaenium: even Lampis was forgiven, and allowed to be present.  All the amusements were of a rustic and pastoral character, as was natural, considering the guests.  One sang a reaper's song, another repeated the jests of the vintage season: Philetas played the pipe, Lampis the flute, Dryas and Lamon danced: Daphnis and Chloe embraced each other.  The goats also were feeding close at hand, as if they desired to take part in the banquet.  This was not altogether to the taste of the city people: but Daphnis called some of them by name, gave them some green leaves to eat, took them by the horns and kissed them.
\pend


\pstart
4.39  And not only then, but as long as they lived, they devoted most of their time to a pastoral life.  They paid especial reverence to the Nymphs, Pan, and Love, acquired large flocks of goats and sheep, and considered fruit and milk superior to every other kind of food.  When a son was born to them, they put him to suck a goat: their daughter was suckled by a ewe: and they called the former Philopoemen, and the latter Agele.  Thus they lived to a good old age in the fields, decorated the grotto, set up statues, and erected an altar to Shepherd Love, and, in place of the pine, built a temple for Pan to dwell in, and dedicated it to Pan the Soldier.
\pend


\pstart
4.40  But this did not take place until later.  After the banquet, when night came, all the guests accompanied them to the nuptial chamber, playing on the pipe and flute, and carrying large blazing torches.  When they were near the door, they began to sing in a harsh and rough voice, as if they were breaking up the earth with forks, instead of singing the marriage hymn.  Daphnis and Chloe, lying naked side by side, embraced and kissed each other, more wakeful than the owl, the whole night long.  Daphnis put into practice the lessons of Lycaenium, and then for the first time Chloe learned that all that had taken place between them in the woods was nothing more than the childish amusement of shepherds.  (<58)
\pend

\endnumbering
\end{english}

\end{Rightside}




\begin{Leftside} 
\begin{greek}
\beginnumbering
\pstart
1.praef  Ἐν Λέσβῳ θηρῶν ἐν ἄλσει Νυμφῶν θέαμα εἶδον κάλλιστον ὧν εἶδον· εἰκόνα, γραφήν, ἱστορίαν ἔρωτος. Καλὸν μὲν καὶ τὸ ἄλσος, πολύδενδρον, ἀνθηρόν, κατάρρυτον· μία πηγὴ πάντα ἔτρεφε, καὶ τὰ ἄνθη καὶ τὰ δένδρα· ἀλλ’ ἡ γραφὴ τερπνοτέρα καὶ τέχνην ἔχουσα περιττὴν καὶ τύχην ἐρωτικήν· ὥστε πολλοὶ καὶ τῶν ξένων κατὰ φήμην ᾔεσαν, τῶν μὲν Νυμφῶν ἱκέται, τῆς δὲ εἰκόνος θεαταί.  Γυναῖκες ἐπʼ αὐτῆς τίκτουσαι καὶ ἄλλαι σπαργάνοις κοσμοῦσαι· παιδία ἐκκείμενα, ποίμνια τρέφοντα· ποιμένες ἀναιρούμενοι, νέοι συντιθέμενοι· λῃστῶν καταδρομή, πολεμίων ἐμβολή. Πολλὰ ἄλλα καὶ πάντα ἐρωτικὰ ἰδόντα με καὶ θαυμάσαντα πόθος ἔσχεν ἀντιγράψαι τῇ γραφῇ·  καὶ ἀναζητησάμενος ἐξηγητὴν τῆς εἰκόνος τέτταρας βίβλους ἐξεπονησάμην, ἀνάθημα μὲν Ἔρωτι καὶ Νύμφαις καὶ Πανί, κτῆμα δὲ τερπνὸν πᾶσιν ἀνθρώποις, ὃ καὶ νοσοῦντα ἰάσεται, καὶ λυπούμενον παραμυθήσεται, τὸν ἐρασθέντα ἀναμνήσει,  τὸν οὐκ ἐρασθέντα προπαιδεύσει. Πάντως γὰρ οὐδεὶς ἔρωτα ἔφυγεν ἢ φεύξεται, μέχρι ἂν κάλλος ᾖ καὶ ὀφθαλμοὶ βλέπωσιν. Ἡμῖν δʼ ὁ θεὸς παράσχοι σωφρονοῦσι τὰ τῶν ἄλλων γράφειν.
\pend


\pstart
1.1  Πόλις ἐστὶ τῆς Λέσβου Μυτιλήνη, μεγάλη καὶ καλή· διείληπται γὰρ εὐρίποις ὑπεισρεούσης τῆς θαλάττης, καὶ κεκόσμηται γεφύραις ξεστοῦ καὶ λευκοῦ λίθου. Νομίσειας οὐ πόλιν ὁρᾶν ἀλλὰ νῆσον.  Ταύτης τῆς πόλεως ὅσον ἀπὸ σταδίων διακοσίων ἀγρὸς ἦν ἀνδρὸς εὐδαίμονος, κτῆμα κάλλιστον· ὄρη θηροτρόφα, πεδία πυροφόρα· γήλοφοι κλημάτων, νομαὶ ποιμνίων· καὶ ἡ θάλαττα προσέκλυζεν ᾐόνι ἐκτεταμένῃ, ψάμμῳ μαλθακῇ.
\pend


\pstart
1.2  Ἐν τῷδε τῷ ἀγρῷ νέμων αἰπόλος, Λάμων τοὔνομα, παιδίον εὗρεν ὑπὸ μιᾶς τῶν αἰγῶν τρεφόμενον. Δρυμὸς ἦν καὶ λόχμη βάτων καὶ κιττὸς ἐπιπλανώμενος καὶ πόα μαλθακή, καθʼ ἧς ἔκειτο τὸ παιδίον. Ἐνταῦθα ἡ αἲξ θέουσα συνεχὲς ἀφανὴς ἐγίνετο πολλάκις καὶ τὸν ἔριφον ἀπολιποῦσα τῷ βρέφει παρέμενε.  Φυλάττει τὰς διαδρομὰς ὁ Λάμων οἰκτείρας ἀμελούμενον τὸν ἔριφον, καὶ μεσημβρίας ἀκμαζούσης κατʼ ἴχνος ἐλθὼν ὁρᾷ τὴν μὲν αἶγα πεφυλαγμένως περιβεβηκυῖαν  Θαυμάσας, ὥσπερ εἰκὸς ἦν, πρόσεισιν ἐγγὺς καὶ εὑρίσκει παιδίον ἄρρεν, μέγα καὶ καλὸν καὶ τῆς κατὰ τὴν ἔκθεσιν τύχης ἐν σπαργάνοις κρείττοσι· χλαμύδιόν τε γὰρ ἦν ἁλουργὲς καὶ πόρπη χρυσῆ καὶ ξιφίδιον ἐλεφαντόκωπον.
\pend


\pstart
1.3  Τὸ μὲν οὖν πρῶτον ἐβουλεύσατο μόνα τὰ γνωρίσματα βαστάσας ἀμελῆσαι τοῦ βρέφους· ἔπειτα αἰδεσθεὶς εἰ μηδὲ αἰγὸς φιλανθρωπίαν μιμήσεται, νύκτα φυλάξας κομίζει πάντα πρὸς τὴν γυναῖκα Μυρτάλην, καὶ τὰ γνωρίσματα καὶ τὸ παιδίον καὶ τὴν αἶγα αὐτήν.  Τῆς δὲ ἐκπλαγείσης εἰ παιδία τίκτουσιν αἶγες, ὁ δὲ πάντα αὐτῇ διηγεῖται, πῶς εὗρεν ἐκκείμενον, πῶς εἶδε τρεφόμενον, πῶς ᾐδέσθη καταλιπεῖν ἀποθανούμενον. Δόξαν δὴ κἀκείνῃ, τὰ μὲν συνεκτεθέντα κρύπτουσι, τὸ δὲ παιδίον αὑτῶν νομίζουσι, τῇ δὲ αἰγὶ τὴν τροφὴν ἐπιτρέπουσιν. Ὡς δʼ ἂν καὶ τοὔνομα τοῦ παιδίου ποιμενικὸν δοκοίη, Δάφνιν αὐτὸν ἔγνωσαν καλεῖν.
\pend


\pstart
1.4  Ἤδη δὲ διετοῦς χρόνου διικνουμένου, ποιμὴν ἐξ ἀγρῶν ὁμόρων νέμων, Δρύας τοὔνομα, καὶ αὐτὸς ὁμοίοις ἐπιτυγχάνει καὶ εὑρήμασι καὶ θέαμασι. Νυμφῶν ἄντρον ἦν, πέτρα μεγάλη, τὰ ἔνδοθεν κοίλη,  τὰ ἔξωθεν περιφερής. Τὰ ἀγάλματα τῶν Νυμφῶν αὐτῶν λίθοις ἐπεποίητο· πόδες ἀνυπόδητοι, χεῖρες εἰς ὤμους γυμναί, κόμαι μέχρι τῶν αὐχένων λελυμέναι, ζῶμα περὶ τὴν ἰξύν, μειδίαμα περὶ τὴν ὀφρύν· τὸ πᾶν σχῆμα χορεία ἦν ὀρχουμένων. Ἡ ὤα τοῦ ἄντρου τῆς μεγάλης πέτρας ἦν τὸ μεσαίτατον.  Ἐκ πηγῆς ἀναβλύζον ὕδωρ ῥεῖθρον ἐποίει χεόμενον, ὥστε καὶ λειμὼν πάνυ γλαφυρὸς ἐξετέτατο πρὸ τοῦ ἄντρου, πολλῆς καὶ μαλθακῆς πόας ὑπὸ τῆς νοτίδος τρεφομένης. Ἀνέκειντο δὲ καὶ γαυλοὶ καὶ αὐλοὶ πλάγιοι καὶ σύριγγες καὶ κάλαμοι, πρεσβυτέρων ποιμένων ἀναθήματα.
\pend


\pstart
1.5  Εἰς τοῦτο τὸ νυμφαῖον οἶς ἀρτιτόκος συχνὰ φοιτῶσα δόξαν πολλάκις ἀπωλείας παρεῖχε. Κολάσαι δὴ βουλόμενος αὐτὴν καὶ εἰς τὴν πρότερον εὐνομίαν καταστῆσαι, δεσμὸν ῥάβδου χλωρᾶς λυγίσας ὅμοιον βρόχῳ τῇ πέτρᾳ προσῆλθεν, ὡς ἐκεῖ συλληψόμενος αὐτήν.  Ἐπιστὰς δὲ οὐδὲν εἶδεν ὧν ἤλπισεν, ἀλλὰ τὴν μὲν διδοῦσαν πάνυ ἀνθρωπίνως τὴν θηλὴν εἰς ἄφθονον τοῦ γάλακτος ὁλκήν, τὸ δὲ παιδίον ἀκλαυτὶ λάβρως εἰς ἀμφοτέρας τὰς θηλὰς μεταφέρον τὸ στόμα καθαρὸν καὶ φαιδρόν, οἷα τῆς οἰὸς τῇ γλώττῃ τὸ πρόσωπον ἀπολιχμωμένης μετὰ τὸν κόρον τῆς τροφῆς.  Θῆλυ ἦν τοῦτο τὸ παιδίον, καὶ παρέκειτο καὶ τούτῳ γνωρίσματα· μίτρα διάχρυσος, ὑποδήματα ἐπίχρυσα, περισκελίδες χρυσαῖ.
\pend


\pstart
1.6  Θεῖον δή τι νομίσας τὸ εὕρημα καὶ διδασκόμενος παρὰ τῆς οἰὸς ἐλεεῖν τε τὸ παιδίον καὶ φιλεῖν ἀναιρεῖται μὲν τὸ βρέφος ἐπʼ ἀγκῶνος, ἀποτίθεται δὲ τὰ γνωρίσματα κατὰ τῆς πήρας, εὔχεται δὲ ταῖς Νύμφαις ἐπὶ χρηστῇ τύχῃ θρέψαι τὴν ἱκέτιν αὐτῶν.  Καὶ ἐπεὶ καιρὸς ἦν ἀπελαύνειν τὴν ποίμνην, ἐλθὼν εἰς τὴν ἔπαυλιν τῇ γυναικὶ διηγεῖται τὰ ὀφθέντα, δείκνυσι τὰ εὑρεθέντα, παρακελεύεται θυγάτριον νομίζειν καὶ λανθάνουσαν ὡς ἴδιον τρέφειν.  Ἡ μὲν δὴ Νάπη (τοῦτο γὰρ ἐκαλεῖτο) μήτηρ εὐθὺς ἦν καὶ ἐφίλει τὸ παιδίον, ὥσπερ ὑπὸ τῆς οἰὸς παρευδοκιμηθῆναι δεδοικυῖα, καὶ τίθεται καὶ αὐτὴ ποιμενικὸν ὄνομα πρὸς πίστιν αὐτῷ, Χλόην.
\pend


\pstart
1.7  Ταῦτα τὰ παιδία ταχὺ μάλα ηὔξησε, καὶ κάλλος αὐτοῖς ἐνεφαίνετο κρεῖττον ἀγροικίας. Ἤδη τε ἦν ὁ μὲν πέντε καὶ δέκα ἐτῶν ἀπὸ γενεᾶς, ἡ δὲ τοσούτων, δυοῖν ἀποδεόντων, καὶ ὁ Δρύας καὶ ὁ Λάμων ἐπὶ μιᾶς νυκτὸς ὁρῶσιν ὄναρ τοιόνδε τι.  Τὰς Νύμφας ἐδόκουν ἐκείνας, τὰς ἐν τῷ ἄντρῳ, ἐν ᾧ ἡ πηγή, ἐν ᾧ τὸ παιδίον εὗρεν ὁ Δρύας, τὸν Δάφνιν καὶ τὴν Χλόην παραδιδόναι παιδίῳ μάλα σοβαρῷ καὶ καλῷ, πτερὰ ἐκ τῶν ὤμων ἔχοντι, βέλη σμικρὰ ἅμα τοξαρίῳ φέροντι· τὸ δὲ ἐφαψάμενον ἀμφοτέρων ἑνὶ βέλει κελεῦσαι λοιπὸν ποιμαίνειν τὸν μὲν τὸ αἰπόλιον, τὴν δὲ τὸ ποίμνιον.
\pend


\pstart
1.8  Τοῦτο τὸ ὄναρ ἰδόντες ἤχθοντο μὲν εἰ ποιμένες ἔσονται οἱ τύχην ἐκ σπαργάνων ἐπαγγελλόμενοι κρείττονα (διὸ αὐτοὺς καὶ τροφαῖς ἔτρεφον ἁβροτέραις καὶ γράμματα ἐπαίδευον καὶ πάντα ὅσα καλὰ ἦν ἐπʼ ἀγροικίας), ἐδόκει δὲ πείθεσθαι θεοῖς περὶ τῶν σωθέντων προνοίᾳ θεῶν.  Καὶ κοινώσαντες ἀλλήλοις τὸ ὄναρ καὶ θύσαντες τῷ τὰ πτερὰ ἔχοντι παιδίῳ παρὰ ταῖς Νύμφαις (τὸ γὰρ ὄνομα λέγειν οὐκ εἶχον) ὡς ποιμένας ἐκπέμπουσιν αὐτοὺς ἅμα ταῖς ἀγέλαις, ἐκδιδάξαντες ἕκαστα· πῶς δεῖ νέμειν πρὸ μεσημβρίας,  πῶς κοπάσαντος τοῦ καύματος· πότε ἐξάγειν ἐπὶ πότον, πότε ἀπάγειν ἐπὶ κοῖτον· ἐπὶ τίσι καλαύροπι χρηστέον, ἐπὶ τίσι φωνῇ μόνῃ. Οἱ δὲ μάλα χαίροντες ὡς ἀρχὴν μεγάλην παρελάμβανον καὶ ἐφίλουν τὰς αἶγας καὶ τὰ πρόβατα μᾶλλον ἢ ποιμέσιν ἔθος, ἡ μὲν ἐς ποίμνιον ἀνάγουσα τῆς σωτηρίας τὴν αἰτίαν, ὁ δὲ μεμνημένος ὡς ἐκκείμενον αὐτὸν αἲξ ἀνέθρεψεν.
\pend


\pstart
1.9  Ἦρος ἦν ἀρχὴ καὶ πάντα ἤκμαζεν ἄνθη, τὰ ἐν δρυμοῖς, τὰ ἐν λειμῶσι καὶ ὅσα ὄρεια· βόμβος ἦν ἤδη μελιττῶν, ἦχος ὀρνίθων μουσικῶν, σκιρτήματα ποιμνίων ἀρτιγεννήτων· ἄρνες ἐσκίρτων ἐν τοῖς ὄρεσιν, ἐβόμβουν ἐν τοῖς λειμῶσι μέλιτται, ἐν ταῖς λόχμαις κατῇδον ὄρνιθες.  Τοσαύτης δὴ πάντα κατεχούσης εὐωρίας οἱ ἁπαλοὶ καὶ νέοι μιμηταὶ τῶν ἀκουομένων ἐγίνοντο καὶ βλεπομένων· ἀκούοντες μὲν τῶν ὀρνίθων ᾀδόντων ᾖδον, βλέποντες δὲ σκιρτῶντας τοὺς ἄρνας ἥλλοντο κοῦφα, τὰς μελίττας δὲ μιμούμενοι ἄνθη συνέλεγον· καὶ τὰ μὲν εἰς τοὺς κόλπους ἔβαλλον, τὰ δὲ στεφανίσκους πλέκοντες ταῖς Νύμφαις ἐπέφερον.
\pend


\pstart
1.10  Ἔπραττον δὲ κοινῇ πάντα, πλησίον ἀλλήλων νέμοντες. Καὶ πολλάκις μὲν ὁ Δάφνις τῶν προβάτων τὰ ἀποπλανώμενα συνέστελλε, πολλάκις δὲ ἡ Χλόη τὰς θρασυτέρας τῶν αἰγῶν ἀπὸ τῶν κρημνῶν κατήλαυνεν, ἤδη δέ τις καὶ τὰς ἀγέλας ἀμφοτέρας ἐφρούρησε θατέρου προσλιπαρήσαντος ἁθύρματι. Ἁθύρματα δὲ ἦν αὐτοῖς ποιμενικὰ καὶ παιδικά.  Ἡ μὲν ἀνθερίκους ἀνελομένη ποθὲν ἐξ ἕλους ἀκριδοθήραν ἔπλεκε καὶ περὶ τοῦτο πονουμένη τῶν ποιμνίων ἠμέλησεν· ὁ δὲ καλάμους ἐκτεμὼν λεπτοὺς καὶ τρήσας τὰς τῶν γονάτων διαφυὰς ἐπαλλήλους τε κηρῷ μαλθακῷ συναρτήσας μέχρι νυκτὸς συρίττειν ἐμελέτα·  ποτὲ δὲ ἐκοινώνουν γάλακτος καὶ οἴνου, καὶ τροφάς, ἃς οἴκοθεν ἔφερον, εἰς κοινὸν ἔφερον. Θᾶττον ἅν τις εἶδε τὰ ποίμνια καὶ τὰς αἶγας ἀπʼ ἀλλήλων μεμερισμένας ἢ Χλόην καὶ Δάφνιν.
\pend


\pstart
1.11  Τοιαῦτα δὲ αὐτῶν παιζόντων τοιάνδε σπουδὴν Ἔρως ἀνέπλασε. Λύκαινα τρέφουσα σκύμνους νέους ἐκ τῶν πλησίον ἀγρῶν ἐξ ἄλλων ποιμνίων πολλάκις ἥρπαζε, πολλῆς τροφῆς ἐς ἀνατροφὴν τῶν σκύμνων δεομένη.  Συνελθόντες οὖν οἱ κωμῆται νύκτωρ σιροὺς ὀρύττουσι τὸ εὖρος ὀργυιᾶς, τὸ βάθος τεττάρων. Τὸ μὲν δὴ χῶμα τὸ πολὺ σπείρουσι κομίσαντες μακράν, ξύλα δὲ ξηρὰ μακρὰ τείναντες ὑπὲρ τοῦ χάσματος τὸ περιττὸν τοῦ χώματος κατέπασαν τῆς πρότερον γῆς εἰκόνα, ὥστε κἂν λαγὼς ἐπιδράμῃ, κατακλᾷ τὰ ξύλα κάρφων ἀσθενέστερα ὄντα, καὶ τότε παρέχει μαθεῖν ὅτι γῆ οὐκ ἦν, ἀλλὰ ἐμεμίμητο γῆν. Τοιαῦτα πολλὰ ὀρύγματα κἀν τοῖς ὄρεσι κἀν τοῖς πεδίοις ὀρύξαντες τὴν μὲν λύκαιναν οὐκ εὐτύχησαν λαβεῖν· αἰσθάνεται γὰρ γῆς σεσοφισμένης· πολλὰς δὲ αἶγας καὶ ποίμνια διέφθειραν, καὶ Δάφνιν παρ’ ὀλίγον ὧδε.
\pend


\pstart
1.12  Τράγοι παροξυνθέντες ἐς μάχην συνέπεσον. Τῷ οὖν ἑτέρῳ τὸ ἕτερον κέρας βιαιοτέρας γενομένης συμβολῆς θραύεται, καὶ ἀλγήσας φριμαξάμενος ἐς φυγὴν ἐτράπετο· ὁ δὲ νικῶν κατʼ ἴχνος ἑπόμενος ἄπαυστον ἐποίει τὴν φυγήν. Ἀλγεῖ Δάφνις περὶ τῷ κέρατι καὶ τῇ θρασύτητι ἀχθεσθεὶς τὴν καλαύροπα λαβὼν ἐδίωκε τὸν διώκοντα.  Οἷα δὲ τοῦ μὲν ὑπεκφεύγοντος, τοῦ δὲ ὀργῇ διώκοντος οὐκ ἀκριβὴς ἦν τῶν ἐν ποσὶν ἡ πρόσοψις, ἀλλὰ κατὰ χάσματος ἄμφω πίπτουσιν, ὁ τράγος πρότερος, ὁ Δάφνις δεύτερος. Τοῦτο καὶ ἔσωσε Δάφνιν, χρήσασθαι τῆς καταφορᾶς ὀχήματι τῷ τράγῳ.  Ὁ μὲν δὴ τὸν ἀνιμησόμενον, εἴ τις ἄρα γένοιτο, δακρύων ἀνέμενεν· ἡ δὲ Χλόη θεασαμένη τὸ συμβὰν δρόμῳ παραγίνεται πρὸς τὸν σιρόν, καὶ μαθοῦσα ὅτι ζῇ, καλεῖ τινὰ βουκόλον ἐκ τῶν ἀγρῶν τῶν πλησίον πρὸς ἐπικουρίαν.  Ὁ δὲ ἐλθὼν σχοῖνον ἐζήτει μακράν, ἧς ἐχόμενος ἀνιμώμενος ἐκβήσεται. Καὶ σχοῖνος μὲν οὐκ ἦν, ἡ δὲ Χλόη λυσαμένη ταινίαν δίδωσι καθεῖναι τῷ βουκόλῳ· καὶ οὕτως οἱ μὲν ἐπὶ τοῦ χείλους ἑστῶτες εἷλκον, ὁ δὲ ἀνέβη ταῖς τῆς ταινίας ὁλκαῖς ταῖς χερσὶν ἀκολουθῶν.  Ἀνιμήσαντο δὲ καὶ τὸν ἄθλιον τράγον συντεθραυσμένον ἄμφω τὰ κέρατα· τοσοῦτον ἄρα ἡ δίκη μετῆλθε τοῦ νικηθέντος τράγου. Τοῦτον μὲν δὴ τυθησόμενον χαρίζονται σῶστρα τῷ βουκόλῳ, καὶ ἔμελλον ψεύδεσθαι πρὸς τοὺς οἴκοι λύκων ἐπιδρομήν, εἴ τις αὐτὸν ἐπόθησεν· αὐτοὶ δὲ ἐπανελθόντες ἐπεσκόπουν τὴν ποίμνην καὶ τὸ αἰπόλιον· καὶ ἐπεὶ κατέμαθον ἐν κόσμῳ νομῆς καὶ τὰς αἶγας καὶ τὰ πρόβατα, καθίσαντες ὑπὸ στελέχει δρυὸς ἐσκόπουν μή τι μέρος τοῦ σώματος ὁ Δάφνις ᾕμαξε καταπεσών.  Ἐτέτρωτο μὲν οὖν οὐδὲν οὐδὲ ᾕμακτο οὐδέν, χώματος δὲ καὶ πηλοῦ ἐπέπαστο καὶ τὰς κόμας καὶ τὸ ἄλλο σῶμα. Ἐδόκει δὴ λούσασθαι, πρὶν αἴσθησιν γενέσθαι τοῦ συμβάντος Λάμωνι καὶ Μυρτάλῃ.
\pend


\pstart
1.13  Καὶ ἐλθὼν ἅμα τῇ Χλόῃ πρὸς τὸ νυμφαῖον τῇ μὲν ἔδωκε καὶ τὸν χιτωνίσκον καὶ τὴν πήραν φυλάττειν, αὐτὸς δὲ τῇ πηγῇ παραστὰς τήν τε κόμην καὶ τὸ σῶμα πᾶν ἀπελούετο.  Ἦν δὲ ἡ μὲν κόμη μέλαινα καὶ πολλή, τὸ δὲ σῶμα ἐπίκαυτον ἡλίῳ· εἴκασεν ἄν τις αὐτὸ χρώζεσθαι τῇ σκιᾷ τῆς κόμης. Ἐδόκει δὲ τῇ Χλόῃ θεωμένῃ καλὸς ὁ Δάφνις, ὅτι πρῶτον αὐτῇ καλὸς ἐδόκει, τὸ λουτρὸν ἐνόμιζε τοῦ κάλλους αἴτιον. Τὰ νῶτα δὲ ἀπολουούσης ἡ σὰρξ ὑπέπιπτε μαλθακή, ὥστε λαθοῦσα ἑαυτῆς ἥψατο πολλάκις, εἰ τρυφερωτέρα εἴη πειρωμένη.  Καὶ (τότε γὰρ ἐπὶ δυσμαῖς ἦν ὁ ἥλιος) ἀπήλασαν τὰς ἀγέλας οἴκαδε, καὶ ἐπεπόνθει Χλόη περιττὸν οὐδέν, ὅτι μὴ Δάφνιν ἐπεθύμει λουόμενον ἰδέσθαι πάλιν.  Τῆς δὲ ἐπιούσης ὡς ἧκον εἰς τὴν νομήν, ὁ μὲν Δάφνις ὑπὸ τῇ δρυῒ τῇ συνήθει καθεζόμενος ἐσύριττε καὶ ἅμα τὰς αἶγας ἐπεσκόπει κατακειμένας καὶ ὥσπερ τῶν μελῶν ἀκροωμένας, ἡ δὲ Χλόη πλησίον καθημένη τὴν ἀγέλην μὲν τῶν προβάτων ἐπέβλεπε, τὸ δὲ πλέον εἰς Δάφνιν ἑώρα· καὶ ἐδόκει καλὸς αὐτῇ συρίττων πάλιν, καὶ αὖθις αἰτίαν ἐνόμιζε τὴν μουσικὴν τοῦ κάλλους, ὥστε μετʼ ἐκεῖνον καὶ αὐτὴ τὴν σύριγγα ἔλαβεν, εἴ πως γένοιτο καὶ αὐτὴ καλή.  Ἔπεισε δὲ αὐτὸν καὶ λούσασθαι πάλιν καὶ λουόμενον εἶδε καὶ ἰδοῦσα ἥψατο καὶ ἀπῆλθε πάλιν ἐπαινέσασα, καὶ ὁ ἔπαινος ἦν ἔρωτος ἀρχή. Ὅ τι μὲν οὖν ἔπασχεν οὐκ ᾔδει νέα κόρη καὶ ἐν ἀγροικίᾳ τεθραμμένη καὶ οὐδὲ ἄλλου λέγοντος ἀκούσασα τὸ τοῦ ἔρωτος ὄνομα· ἄση δὲ αὐτῆς εἶχε τὴν ψυχήν,  καὶ τῶν ὀφθαλμῶν οὐκ ἐκράτει καὶ πολλὰ ἐλάλει Δάφνιν· τροφῆς ἠμέλει, νύκτωρ ἠγρύπνει, τῆς ἀγέλης κατεφρόνει· νῦν ἐγέλα, νῦν ἔκλαεν· εἶτα ἐκάθευδεν, εἶτα ἀνεπήδα· ὠχρία τὸ πρόσωπον, ἐρυθήματι αὖθις ἐφλέγετο. Οὐδὲ βοὸς οἴστρῳ πληγείσης τοιαῦτα ἔργα. Ἐπῆλθόν ποτε αὐτῇ καὶ τοιοίδε λόγοι μόνῃ γενομένῃ.
\pend


\pstart
1.14  “Νῦν ἐγὼ νοσῶ μέν, τί δὲ ἡ νόσος ἀγνοῶ· ἀλγῶ, καὶ ἕλκος οὐκ ἔστι μοι· λυποῦμαι, καὶ οὐδὲν τῶν προβάτων ἀπόλωλέ μοι· κάομαι, καὶ ἐν σκιᾷ τοσαύτῃ κάθημαι.  Πόσοι βάτοι με ἤμυξαν, καὶ οὐκ ἔκλαυσα· πόσαι μέλιτται κέντρον ἐνῆκαν, ἀλλὰ ἔφαγον· τουτὶ δὲ τὸ νύττον μου τὴν καρδίαν πάντων ἐκείνων πικρότερον. Καλὸς ὁ Δάφνις, καὶ γὰρ τὰ ἄνθη· καλὸν ἡ σύριγξ αὐτοῦ φθέγγεται, καὶ γὰρ αἱ ἀηδόνες. Ἀλλʼ ἐκείνων οὐδείς μοι λόγος. Εἴθε αὐτοῦ σύριγξ ἐγενόμην, ἵνʼ ἐμπνέῃ μοι·  εἴθε αἴξ, ἵνʼ ὑπʼ ἐκείνου νέμωμαι. Ὦ πονηρὸν ὕδωρ, μόνον Δάφνιν καλὸν ἐποίησας, ἐγὼ δὲ μάτην ἀπελουσάμην. Οἴχομαι, Νύμφαι φίλαι· οὐδὲ ὑμεῖς σώζετε τὴν παρθένον τὴν ἐν ὑμῖν τραφεῖσαν; Τίς ὑμᾶς στεφανώσει μετʼ ἐμέ;  τίς τοὺς ἀθλίους ἄρνας ἀναθρέψει; τίς τὴν λάλον ἀκρίδα θεραπεύσει, ἣν πολλὰ καμοῦσα ἐθήρασα, ἵνα με κατακοιμίζῃ φθεγγομένη πρὸ τοῦ ἄντρου; νῦν δὲ ἐγὼ μὲν ἀγρυπνῶ διὰ Δάφνιν, ἡ δὲ μάτην λαλεῖ.”
\pend


\pstart
1.15  Τοιαῦτα ἔπασχε, τοιαῦτα ἔλεγεν, ἐπιζητοῦσα τὸ ἔρωτος ὄνομα. Δόρκων δὲ ὁ βουκόλος, ὁ τὸν Δάφνιν ἐκ τοῦ σιροῦ ἀνιμησάμενος, ἀρτιγένειος μειρακίσκος καὶ εἰδὼς ἔρωτος καὶ τὰ ἔργα καὶ τοὔνομα, εὐθὺς μὲν ἐπʼ ἐκείνης τῆς ἡμέρας ἐρωτικῶς τῆς Χλόης διετέθη, πλειόνων δὲ διαγενομένων μᾶλλον τὴν ψυχὴν ἐξεπυρσεύθη καὶ τοῦ Δάφνιδος ὡς παιδὸς καταφρονήσας ἔγνω κατεργάσασθαι δώροις ἢ βίᾳ.  Τὰ μὲν δὴ πρῶτα δῶρα αὐτοῖς ἐκόμισε τῷ μὲν σύριγγα βουκολικὴν καλάμων ἐννέα χαλκῷ δεδεμένων ἀντὶ κηροῦ, τῇ δὲ νεβρίδα βακχικήν· καὶ αὐτῇ τὸ χρῶμα ἦν ὥσπερ γεγραμμένον χρώμασιν.  Ἐντεῦθεν δὲ φίλος νομιζόμενος τοῦ μὲν Δάφνιδος ἠμέλει κατʼ ὀλίγον, τῇ Χλόῃ δὲ ἀνὰ πᾶσαν ἡμέραν ἐπέφερεν ἢ τυρὸν ἁπαλὸν ἢ στέφανον ἀνθηρὸν ἢ μῆλον ὡραῖον· ἐκόμισε δέ ποτε αὐτῇ καὶ μόσχον ἀρτιγέννητον καὶ κισσύβιον διάχρυσον καὶ ὀρνίθων ὀρείων νεοττούς. Ἡ δὲ ἄπειρος οὖσα τέχνης ἐραστοῦ, λαμβάνουσα τὰ δῶρα ἔχαιρεν, ὅτι Δάφνιδι εἶχεν αὐτὰ χαρίζεσθαι.  Καὶ (ἔδει γὰρ ἤδη καὶ Δάφνιν γνῶναι τὰ ἔρωτος ἔργα) γίνεταί ποτε τῷ Δόρκωνι πρὸς αὐτὸν ὑπὲρ κάλλους ἔρις, καὶ ἐδίκαζε μὲν Χλόη, ἔκειτο δὲ ἆθλον τῷ νικήσαντι φιλῆσαι Χλόην. Δόρκων δὲ πρότερος ὧδε ἔλεγεν.
\pend


\pstart
1.16  “Ἐγώ, παρθένε, μείζων εἰμὶ Δάφνιδος, κἀγὼ μὲν βουκόλος, ὁ δʼ αἰπόλος· τοσοῦτον κρείττων ὅσον αἰγῶν βόες· καὶ λευκός εἰμι ὡς γάλα, καὶ πυρρὸς ὡς θέρος μέλλον ἀμᾶσθαι, καὶ ἔθρεψε μήτηρ,  οὐ θηρίον. Οὗτος δέ ἐστι σμικρὸς καὶ ἀγένειος ὡς γυνή, καὶ μέλας ὡς λύκος· νέμει δὲ τράγους, ὀδωδὼς ἀπʼ αὐτῶν δεινόν, καὶ ἔστι πένης ὡς μηδὲ κύνα τρέφειν. Εἰ δʼ, ὡς λέγουσι, καὶ αἲξ αὐτῷ γάλα δέδωκεν, οὐδὲν ἐρίφου διαφέρει.”  Ταῦτα καὶ τοιαῦτα ὁ Δόρκων· καὶ μετὰ ταῦτα ὁ Δάφνις “Ἐμὲ αἲξ ἀνέθρεψεν ὥσπερ τὸν Δία· νέμω δὲ τράγους τῶν τούτου βοῶν μείζονας· ὄζω δὲ οὐδὲν ἀπʼ αὐτῶν, ὅτι μηδὲ ὁ Πάν, καίτοιγε ὢν τὸ πλέον τράγος.  Ἀρκεῖ δέ μοι τυρὸς καὶ ἄρτος ὀβελίας καὶ οἶνος λευκός, ὅσα ἀγροίκων πλουσίων κτήματα. Ἀγένειός εἰμι, καὶ γὰρ ὁ Διόνυσος· μέλας, καὶ γὰρ ὁ ὑάκινθος· ἀλλὰ κρείττων καὶ ὁ Διόνυσος Σατύρων ὁ ὑάκινθος κρίνων.  Οὗτος δὲ καὶ πυρρὸς ὡς ἀλώπηξ καὶ προγένειος ὡς τράγος καὶ λευκὸς ὡς ἐξ ἄστεος γυνή· κἂν δέῃ σε φιλεῖν, ἐμοῦ μὲν φιλήσεις τὸ στόμα, τούτου δὲ τὰς ἐπὶ τοῦ γενείου τρίχας. Μέμνησο δέ, ὦ παρθένε, ὅτι σὲ ποίμνιον ἔθρεψεν, ἀλλὰ καὶ εἶ καλή.”
\pend


\pstart
1.17  Οὐκέθʼ ἡ Χλόη περιέμεινεν, ἀλλὰ τὰ μὲν ἡσθεῖσα τῷ ἐγκωμίῳ, τὰ δὲ πάλαι ποθοῦσα φιλῆσαι Δάφνιν, ἀναπηδήσασα αὐτὸν ἐφίλησεν, ἀδίδακτον μὲν καὶ ἄτεχνον, πάνυ δὲ ψυχὴν θερμᾶναι δυνάμενον.  Δόρκων μὲν οὖν ἀλγήσας ἀπέδραμε ζητῶν ἄλλην ὁδὸν ἔρωτος· Δάφνις δέ, ὥσπερ οὐ φιληθείς, ἀλλὰ δηχθείς, σκυθρωπός τις εὐθὺς ἦν καὶ πολλάκις ἐψύχετο καὶ τὴν καρδίαν παλλομένην εἶχε, καὶ βλέπειν μὲν ἤθελε τὴν Χλόην, βλέπων δʼ ἐρυθήματος ἐνεπίμπλατο.  Τότε πρῶτον καὶ τὴν κόμην αὐτῆς ἐθαύμασεν ὅτι ξανθή, καὶ τοὺς ὀφθαλμοὺς ὅτι μεγάλοι καθάπερ βοός, καὶ τὸ πρόσωπον ὅτι λευκότερον ἀληθῶς καὶ τοῦ τῶν αἰγῶν γάλακτος, ὥσπερ τότε πρῶτον ὀφθαλμοὺς κτησάμενος, τὸν δὲ πρότερον χρόνον πεπηρωμένος.  Οὔτε οὖν τροφὴν προσεφέρετο πλὴν ὅσον ἀπογεύσασθαι· καὶ πότον, εἴ ποτε ἐβιάσθη, μέχρι τοῦ διαβρέξαι τὸ στόμα προσεφέρετο. Σιωπηλὸς ἦν ὁ πρότερον τῶν ἀκρίδων λαλίστερος, ἀργὸς ὁ περιττότερα τῶν αἰγῶν κινούμενος. Ἠμέλητο ἡ ἀγέλη· ἔρριπτο ἡ σύριγξ· χλωρότερον τὸ πρόσωπον ἦν πόας θερινῆς. Πρὸς μόνην Χλόην ἐγίνετο λάλος· καὶ εἴ ποτε ἀπʼ αὐτῆς ἐγένετο, τοιαῦτα πρὸς αὑτὸν ἀπελήρει
\pend


\pstart
1.18  “Τί ποτέ με Χλόης ἐργάζεται φίλημα; Χείλη μὲν ῥόδων ἁπαλώτερα καὶ στόμα κηρίων γλυκύτερον· τὸ δὲ φίλημα κέντρου μελίττης πικρότερον. Πολλάκις ἐφίλησα ἐρίφους, πολλάκις ἐφίλησα σκύλακας ἀρτιγεννήτους καὶ τὸν μόσχον, ὃν ὁ Δόρκων ἐδωρήσατο· ἀλλὰ τοῦτο φίλημα καινόν· ἐκπηδᾷ μου τὸ πνεῦμα, ἐξάλλεται ἡ καρδία, τήκεται ἡ ψυχή, καὶ ὅμως πάλιν φιλῆσαι θέλω.  Ὢ νίκης κακῆς· ὢ νόσου καινῆς, ἧς οὐδὲ εἰπεῖν οἶδα τοὔνομα. Ἆρα φαρμάκων ἐγεύσατο ἡ Χλόη μέλλουσά με φιλεῖν; Πῶς οὖν οὐκ ἀπέθανεν; Οἷον ᾅδουσιν αἱ ἀηδόνες, ἡ δὲ ἐμὴ σύριγξ σιωπᾷ· οἷον σκιρτῶσιν οἱ ἔριφοι, κἀγὼ κάθημαι· οἷον ἀκμάζει τὰ ἄνθη, κἀγὼ στεφάνους οὐ πλέκω, ἀλλὰ τὰ μὲν ἴα καὶ ὁ ὑάκινθος ἀνθεῖ, Δάφνις δὲ μαραίνεται.  Ἆρά μου καὶ Δόρκων εὐμορφότερος ὀφθήσεται;” Τοιαῦτα ὁ βέλτιστος Δάφνις ἔπασχε καὶ ἔλεγεν, οἷα πρῶτον γευόμενος τῶν ἔρωτος ἔργων καὶ λόγων.
\pend


\pstart
1.19  Ὁ δὲ Δόρκων ὁ βουκόλος ὁ τῆς Χλόης ἐραστὴς φυλάξας τὸν Δρύαντα φυτὸν κατορύττοντα κλήματος πρόσεισιν αὐτῷ μετὰ τυρίσκων τινῶν γεννικῶν, καὶ τοὺς μὲν δῶρον εἶναι δίδωσι, πάλαι φίλος ὤν, ἡνίκα αὐτὸς ἔνεμεν· ἐντεῦθεν δὲ ἀρξάμενος ἐνέβαλε λόγον περὶ τοῦ τῆς Χλόης γάμου·  καὶ εἰ λαμβάνοι γυναῖκα, δῶρα πολλὰ καὶ μεγάλα, ὡς βουκόλος, ἐπηγγέλλετο· ζεῦγος βοῶν ἀροτήρων, σμήνη τέτταρα μελιττῶν, φυτὰ μηλεῶν πεντήκοντα, δέρμα ταύρου τεμεῖν ὑποδήματα, μόσχον ἀνὰ πᾶν ἔτος, μηκέτι γάλακτος δεόμενον·  ὥστε σμικροῦ δεῖν ὁ Δρύας θελχθεὶς τοῖς δώροις ἐπένευσε τὸν γάμον. Ἐννοήσας δὲ ὡς κρείττονος ἡ παρθένος ἀξία νυμφίου, καὶ δείσας μήποτε κακοῖς ἀνηκέστοις περιπέσῃ, τόν τε γάμον ἀνένευσε καὶ συγγνώμην ἔχειν ᾐτήσατο καὶ τὰ ὀνομασθέντα δῶρα παρῃτήσατο.
\pend


\pstart
1.20  Δευτέρας δὴ διαμαρτὼν ὁ Δόρκων ἐλπίδος καὶ μάτην τυροὺς ἀγαθοὺς ἀπολέσας ἔγνω διὰ χειρῶν ἐπιθέσθαι τῇ Χλόῃ μόνῃ γενομένῃ· καὶ παραφυλάξας ὅτι παρʼ ἡμέραν ἐπὶ πότον ἄγουσι τὰς ἀγέλας ποτὲ μὲν ὁ Δάφνις ποτὲ δὲ ἡ παῖς, ἐπιτεχνᾶται τέχνην ποιμένι πρέπουσαν.  Λύκου μεγάλου δέρμα λαβών, ὅν ταῦρός ποτε πρὸ τῶν βοῶν μαχόμενος τοῖς κέρασι διέφθειρε, περιέτεινε τῷ σώματι, ποδῆρες κατανωτισάμενος, ὡς τούς τʼ ἐμπροσθίους πόδας ἐφηπλῶσθαι ταῖς χερσὶ καὶ τοὺς κατόπιν τοῖς σκέλεσιν ἄχρι πτέρνης καὶ τοῦ στόματος τὸ χάσμα σκέπειν τὴν κεφαλήν, ὥσπερ ἀνδρὸς ὁπλίτου κράνος·  ἐκθηριώσας δὲ αὑτὸν ὡς ἔνι μάλιστα, παραγίνεται πρὸς τὴν πηγήν, ἧς ἔπινον αἱ αἶγες καὶ τὰ πρόβατα μετὰ τὴν νομήν. Ἐν κοίλῃ δὲ πάνυ γῇ ἦν ἡ πηγή, καὶ περὶ αὐτὴν πᾶς ὁ τόπος ἀκάνθαις καὶ βάτοις καὶ ἀρκεύθῳ ταπεινῇ καὶ σκολύμοις ἠγρίωτο·  ῥᾳδίως ἂν ἐκεῖ καὶ λύκος ἀληθινὸς ἔλαθε λοχῶν. Ἐνταῦθα κρύψας ἑαυτὸν ἐπετήρει τοῦ πότου τὴν ὥραν ὁ Δόρκων, καὶ πολλὴν εἶχεν ἐλπίδα τῷ σχήματι φοβήσας λαβεῖν ταῖς χερσὶ τὴν χλόην.
\pend


\pstart
1.21  Χρόνος ὀλίγος διαγίνεται καὶ Χλόη κατήλαυνε τὰς ἀγέλας εἰς τὴν πηγήν, καταλιποῦσα τὸν Δάφνιν φυλλάδα χλωρὰν κόπτοντα τοῖς ἐρίφοις τροφὴν μετὰ τὴν νομήν.  Καὶ οἱ κύνες οἱ τῶν προβάτων ἐπὶ φυλακῇ καὶ τῶν αἰγῶν ἑπόμενοι, οἵα δὴ κυνῶν ἐν ῥινηλασίαις περιεργία, κινούμενον τὸν Δόρκωνα πρὸς τὴν ἐπίθεσιν τῆς κόρης φωράσαντες, πικρὸν μάλα ὑλακτήσαντες ὥρμησαν ὡς ἐπὶ λύκον καὶ περισχόντες, πρὶν ὅλως ἀναστῆναι διʼ ἔκπληξιν, ἔδακνον κατὰ τοῦ δέρματος.  Τέως μὲν οὖν τὸν ἔλεγχον αἰδούμενος καὶ ὑπὸ τοῦ δέρματος ἐπισκέποντος φρουρούμενος ἔκειτο σιωπῶν ἐν τῇ λόχμῃ· ἐπεὶ δὲ ἥ τε Χλόη πρὸς τὴν πρώτην θέαν διαταραχθεῖσα τὸν Δάφνιν ἐκάλει βοηθόν, οἵ τε κύνες περισπῶντες τὸ δέρμα τοῦ σώματος ἥπτοντο αὐτοῦ, μέγα οἰμώξας ἱκέτευε βοηθεῖν τὴν κόρην καὶ τὸν Δάφνιν ἤδη παρόντα.  Τοὺς μὲν δὴ κύνας ἀνακλήσει συνήθει ταχέως ἡμέρωσαν, τὸν δὲ Δόρκωνα κατά τε μηρῶν καὶ ὤμων δεδηγμένον ἀγαγόντες ἐπὶ τὴν πηγὴν ἀπένιψαν τὰ δήγματα καὶ διαμασησάμενοι φλοιὸν χλωρὸν πτελέας ἐπέπασαν·  ὑπό τε ἀπειρίας ἐρωτικῶν τολμημάτων ποιμενικὴν παιδιὰν νομίζοντες τὴν ἐπιβολὴν τοῦ δέρματος, οὐδὲν ὀργισθέντες, ἀλλὰ καὶ παραμυθησάμενοι καὶ μέχρι τινὸς χειραγωγήσαντες ἀπέπεμψαν.
\pend


\pstart
1.22  Καὶ ὁ μὲν κινδύνου παρὰ τοσοῦτον ἐλθὼν καὶ σωθεὶς ἐκ κυνός, οὐ λύκου στόματος, ἐθεράπευε τὸ σῶμα· ὁ δὲ Δάφνις καὶ ἡ Χλόη κάματον πολὺν ἔσχον μέχρι νυκτὸς τὰς αἶγας καὶ τὰς οἶς συλλέγοντες·  ὑπὸ γὰρ τοῦ δέρματος πτοηθεῖσαι καὶ ὑπὸ τῶν κυνῶν ὑλακτησάντων ταραχθεῖσαι αἱ μὲν εἰς πέτρας ἀνέδραμον, αἱ δὲ μέχρι τῆς θαλάττης αὐτῆς κατέδραμον. Καίτοιγε ἐπεπαίδευντο καὶ φωνῇ πείθεσθαι καὶ σύριγγι θέλγεσθαι καὶ χειρὸς πλαταγῇ συλλέγεσθαι·  ἀλλὰ τότε πάντων αὐταῖς ὁ φόβος λήθην ἐνέβαλε. Καὶ μόλις ὥσπερ λαγὼς ἐκ τῶν ἰχνῶν εὑρίσκοντες εἰς τὰς ἐπαύλεις ἤγαγον. Ἐκείνης μόνης τῆς νυκτὸς ἐκοιμήθησαν βαθὺν ὕπνον καὶ τῆς ἐρωτικῆς λύπης φάρμακον τὸν κάματον ἔσχον.  Αὖθις δὲ ἡμέρας ἐπελθούσης, πάλιν ἔπασχον παραπλήσια. Ἔχαιρον ἰδόντες, ἀπαλλαγέντες ἤλγουν· ἤθελόν τι, ἠγνόουν ὅ τι θέλουσι. Τοῦτο μόνον ᾔδεσαν ὅτι τὸν μὲν φίλημα, τὴν δὲ λουτρὸν ἀπώλεσεν.
\pend


\pstart
1.23  Ἐξέκαε δʼ αὐτοὺς καὶ ἡ ὥρα τοῦ ἔτους. Ἦρος ἦν ἤδη τέλος καὶ θέρους ἀρχή, καὶ πάντα ἐν ἀκμῇ· δένδρα ἐν καρποῖς, πεδία ἐν ληίοις. Ἡδεῖα μὲν τεττίγων ἠχή, γλυκεῖα δὲ ὀπώρας ὀδμή, τερπνὴ δὲ ποιμνίων βληχή.  Εἴκασεν ἄν τις καὶ τοὺς ποταμοὺς ᾅδειν ἠρέμα ῥέοντας καὶ τοὺς ἀνέμους συρίττειν ταῖς πίτυσιν ἐμπνέοντας καὶ τὰ μῆλα ἐρῶντα πίπτειν χαμαὶ καὶ τὸν ἥλιον φιλόκαλον ὄντα πάντας ἀποδύειν. Ὁ μὲν οὖν Δάφνις θαλπόμενος τούτοις ἅπασιν εἰς τοὺς ποταμοὺς ἐνέβαινε, καὶ ποτὲ μὲν ἐλούετο, ποτὲ δὲ τῶν ἰχθύων τοὺς ἐνδινεύοντας ἐθήρα· πολλάκις δὲ καὶ ἔπινεν, ὡς τὸ ἔνδοθεν καῦμα σβέσων.  Ἡ δὲ Χλόη μετὰ τὸ ἀμέλξαι τὰς οἶς καὶ τῶν αἰγῶν τὰς πολλὰς ἐπὶ πολὺν μὲν χρόνον εἶχε πηγνῦσα τὸ γάλα· δειναὶ γὰρ αἱ μυῖαι λυπῆσαι καὶ δακεῖν, εἰ διώκοιντο· τὸ δὲ ἐντεῦθεν ἀπολουσαμένη τὸ πρόσωπον πίτυος ἐστεφανοῦτο κλάδοις καὶ τῇ νεβρίδι ἐζώννυτο καὶ τὸν γαυλὸν ἀναπλήσασα οἴνου καὶ γάλακτος κοινὸν μετὰ τοῦ Δάφνιδος πότον εἶχε.
\pend


\pstart
1.24  Τῆς δὲ μεσημβρίας ἐπελθούσης ἐγίνετο ἤδη τῶν ὀφθαλμῶν ἅλωσις αὐτοῖς· ἡ μὲν γὰρ γυμνὸν ὁρῶσα τὸν Δάφνιν ἐς ἄθρουν ἐνέπιπτε τὸ κάλλος, καὶ ἐτήκετο μηδὲν αὐτοῦ μέρος μέμψασθαι δυναμένη· ὁ δὲ ἰδὼν ἐν νεβρίδι καὶ στεφάνῳ πίτυος ὀρέγουσαν τὸν γαυλὸν μίαν ᾤετο τῶν ἐκ τοῦ ἄντρου Νυμφῶν ὁρᾶν.  Ὁ μὲν οὖν τὴν πίτυν ἀπὸ τῆς κεφαλῆς ἁρπάζων αὐτὸς ἐστεφανοῦτο, πρότερον φιλήσας τὸν στέφανον· ἡ δὲ τὴν ἐσθῆτα αὐτοῦ λουομένου καὶ γυμνωθέντος ἐνεδύετο, πρότερον καὶ αὐτὴ φιλήσασα.  Ἤδη ποτὲ καὶ μήλοις ἀλλήλους ἔβαλον καὶ τὰς κεφαλὰς ἀλλήλων ἐκόσμησαν διακρίνοντες τὰς κόμας· καὶ ἡ μὲν εἴκασεν αὐτοῦ τὴν κόμην, ὅτι μέλαινα, μύρτοις, ὁ δὲ μήλῳ τὸ πρόσωπον αὐτῆς, ὅτι λευκὸν καὶ ἐνερευθὲς ἦν.  Ἐδίδασκεν αὐτὴν καὶ συρίττειν· καὶ ἀρξαμένης ἐμπνεῖν ἁρπάζων τὴν σύριγγα τοῖς χείλεσιν αὐτὸς τοὺς καλάμους ἐπέτρεχε· καὶ ἐδόκει μὲν διδάσκειν ἁμαρτάνουσαν, εὐπρεπῶς δὲ διὰ τῆς σύριγγος Χλόην κατεφίλει.
\pend


\pstart
1.25  Συρίττοντος δὲ αὐτοῦ τὸ μεσημβρινὸν καὶ τῶν ποιμνίων σκιαζομένων ἔλαθεν ἡ Χλόη κατανυστάξασα. Φωράσας τοῦτο ὁ Δάφνις καὶ καταθέμενος τὴν σύριγγα πᾶσαν αὐτὴν ἔβλεπεν ἀπλήστως, οἷα μηδὲν αἰδούμενος, καὶ ἅμα ἠρέμα ὑπεφθέγγετο “οἷον καθεύδουσιν ὀφθαλμοί, οἷον δὲ ἀποπνεῖ τὸ στόμα.  Οὐδὲ τὰ μῆλα τοιοῦτον, οὐδὲ αἱ ὄχναι. Ἀλλὰ φιλῆσαι μὲν δέδοικα· δάκνει τὸ φίλημα τὴν καρδίαν, καὶ ὥσπερ τὸ νέον μέλι μαίνεσθαι ποιεῖ· ὀκνῶ δὲ καὶ μὴ φιλήσας αὐτὴν ἀφυπνίσω.  Ὤ λάλων τεττίγων, οὐκ ἐάσουσιν αὐτὴν καθεύδειν μέγα ἠχοῦντες. Ἀλλὰ καὶ οἱ τράγοι τοῖς κέρασι παταγοῦσι μαχόμενοι. Ὤ λύκων ἀλωπέκων δειλοτέρων, οἳ τούτους οὐχ ἥρπασαν.”
\pend


\pstart
1.26  Ἐν τοιούτοις ὄντος αὐτοῦ λόγοις τέττιξ φεύγων χελιδόνα θηρᾶσαι θέλουσαν κατέπεσεν εἰς τὸν κόλπον τῆς Χλόης· καὶ ἡ χελιδὼν ἑπομένη τὸν μὲν οὐκ ἠδυνήθη λαβεῖν, ταῖς δὲ πτέρυξιν ἐγγὺς διὰ τὴν δίωξιν γενομένη τῶν παρειῶν αὐτῆς ἥψατο.  Ἡ δὲ οὐκ εἰδυῖα τὸ πραχθὲν μέγα βοήσασα τῶν ὕπνων ἐξέθορεν. Ἰδοῦσα δὲ καὶ τὴν χελιδόνα ἔτι πλησίον πετομένην καὶ τὸν Δάφνιν ἐπὶ τῷ δέει γελῶντα τοῦ φόβου μὲν ἐπαύσατο, τοὺς δὲ ὀφθαλμοὺς ἀπέματτεν ἔτι καθεύδειν θέλοντας.  Καὶ ὁ τέττιξ ἐκ τῶν κόλπων ἐπήχησεν ὅμοιον ἱκέτῃ χάριν ὁμολογοῦντι τῆς σωτηρίας. Πάλιν οὖν ἡ Χλόη μέγα ἐβόησεν, ὁ δὲ Δάφνις ἐγέλασε· καὶ προφάσεως λαβόμενος καθῆκεν αὐτῆς εἰς τὰ στέρνα τὰς χεῖρας καὶ ἐξάγει τὸν βέλτιστον τέττιγα, μηδὲ ἐν τῇ δεξιᾷ σιωπῶντα. Ἡ δὲ ἥδετο ἰδοῦσα καὶ ἐφίλησε λαβοῦσα καὶ αὖθις ἐνέβαλε τῷ κόλπῳ λαλοῦντα.
\pend


\pstart
1.27  Ἔτερψεν αὐτοὺς τότε φάττα βουκολικὸν ἐκ τῆς ὕλης φθεγξαμένη. Καὶ τῆς Χλόης ζητούσης μαθεῖν ὅ τι λέγει, διδάσκει αὐτὴν ὁ Δάφνις μυθολογῶν τὰ θρυλούμενα.  “Ἦν οὕτω, παρθένε, παρθένος καλὴ καὶ ἔνεμε βοῦς πολλὰς ἐν ὕλῃ· ἦν δὲ ἄρα καὶ ᾠδικὴ καὶ ἐτέρποντο αἱ βόες αὐτῆς τῇ μουσικῇ, καὶ ἔνεμεν οὔτε καλαύροπος πληγῇ οὔτε κέντρου προσβολῇ, ἀλλὰ καθίσασα ὑπὸ πίτυν καὶ στεφανωσαμένη πίτυϊ ᾖδε Πᾶνα καὶ τὴν Πίτυν, καὶ αἱ βόες τῇ φωνῇ παρέμενον.  Παῖς οὐ μακρὰν νέμων βοῦς, καὶ αὐτὸς καλὸς καὶ ᾠδικὸς ὡς ἡ παρθένος, φιλονεικήσας πρὸς τὴν μελῳδίαν, μείζονα ὡς ἀνήρ, ἡδίονα ὡς παῖς φωνὴν ἀντεπεδείξατο, καὶ τῶν βοῶν ὀκτὼ τὰς ἀρίστας ἐς τὴν ἰδίαν ἀγέλην θέλξας ἀπεβουκόλησεν.  Ἄχθεται ἡ παρθένος τῇ βλάβῃ τῆς ἀγέλης, τῇ ἥττῃ τῆς ᾠδῆς, καὶ εὔχεται τοῖς θεοῖς ὄρνις γενέσθαι πρὶν οἴκαδε ἀφικέσθαι. Πείθονται οἱ θεοὶ καὶ ποιοῦσι τήνδε τὴν ὄρνιν, ὄρειον ὡς παρθένον, μουσικὴν ὡς ἐκείνην. Καὶ ἔτι νῦν ᾅδουσα μηνύει τὴν συμφοράν, ὅτι βοῦς ζητεῖ πεπλανημένας.”
\pend


\pstart
1.28  Τοιάσδε τέρψεις αὐτοῖς τὸ θέρος παρεῖχε. Μετοπώρου δὲ ἀκμάζοντος καὶ τοῦ βότρυος Τύριοι λῃσταὶ Καρικὴν ἔχοντες ἡμιολίαν, ὡς ἂν μὴ δοκοῖεν βάρβαροι, προσέσχον τοῖς ἀγροῖς, καὶ ἐκβάντες σὺν μαχαίραις καὶ ἡμιθωρακίοις κατέσυρον πάντα τὰ εἰς χεῖρας ἐλθόντα, οἶνον ἀνθοσμίαν, πυρὸν ἄφθονον, μέλι ἐν κηρίοις· ἤλασάν τινας καὶ βοῦς ἐκ τῆς Δόρκωνος ἀγέλης.  Λαμβάνουσι καὶ τὸν Δάφνιν ἀλύοντα παρὰ τὴν θάλατταν· ἡ γὰρ Χλόη βραδύτερον ὡς κόρη τὰ πρόβατα ἐξῆγε τοῦ Δρύαντος, φόβῳ τῶν ἀγερώχων ποιμένων. Ἰδόντες δὲ μειράκιον μέγα καὶ καλὸν καὶ κρεῖττον τῆς ἐξ ἀγρῶν ἁρπαγῆς, μηκέτι μηδὲν μήτε ἐς τὰς αἶγας μήτε ἐς τοὺς ἄλλους ἀγροὺς περιεργασάμενοι κατῆγον αὐτὸν ἐπὶ τὴν ναῦν κλάοντα καὶ ἠπορημένον καὶ μέγα Χλόην καλοῦντα.  Καὶ οἱ μὲν τὸ πεῖσμα ἄρτι ἀπολύσαντες καὶ τὰς κώπας ταῖς χερσὶν ἐμβαλόντες ἀνέπλεον εἰς τὸ πέλαγος· Χλόη δὲ κατήλαυνε τὸ ποίμνιον, σύριγγα καινὴν τῷ Δάφνιδι δῶρον κομίζουσα. Ἰδοῦσα δὲ τὰς αἶγας τεταραγμένας καὶ ἀκούσασα τοῦ Δάφνιδος ἀεὶ μεῖζον αὐτὴν βοῶντος, προβάτων μὲν ἀμελεῖ καὶ τὴν σύριγγα ῥίπτει, δρόμῳ δὲ πρὸς τὸν Δόρκωνα παραγίνεται δεησομένη βοηθεῖν
\pend


\pstart
1.29  Ὁ δὲ ἔκειτο πληγαῖς νεανικαῖς συγκεκομμένος ὑπὸ τῶν λῃστῶν καὶ ὀλίγον ἐμπνέων, αἵματος πολλοῦ χεομένου. Ἰδὼν δὲ τὴν Χλόην καὶ ὀλίγον ἐκ τοῦ πρότερον ἔρωτος ἐμπύρευμα λαβὼν “ἐγὼ μὲν” εἶπε, “Χλόη, τεθνήξομαι μετʼ ὀλίγον· οἱ γάρ με ἀσεβεῖς λῃσταὶ πρὸ τῶν βοῶν μαχόμενον κατέκοψαν ὡς βοῦν.  Σὺ δὲ καὶ Δάφνιν σῶσον κἀμοὶ τιμώρησον κἀκείνους ἀπόλεσον. Ἐπαίδευσα τὰς βοῦς ἤχῳ σύριγγος ἀκολουθεῖν καὶ διώκειν τὸ μέλος αὐτῆς, κἂν νέμωνταί που μακράν. Ἴθι δή, λαβοῦσα τὴν σύριγγα ταύτην ἔμπνευσον αὐτῇ μέλος ἐκεῖνο, ὃ Δάφνιν μὲν ἐγώ ποτε ἐδιδαξάμην, Δάφνις δὲ σέ· τὸ δὲ ἐντεῦθεν τῇ σύριγγι μελήσει καὶ τῶν βοῶν ταῖς ἐκεῖ.  Χαρίζομαι δέ σοι καὶ τὴν σύριγγα αὐτήν, ᾗ πολλοὺς ἐρίζων καὶ βουκόλους ἐνίκησα καὶ αἰπόλους. Σὺ δὲ ἀντὶ τῶνδε καὶ ζῶντα ἔτι φίλησον καὶ ἀποθανόντα κλαῦσον· κἂν ἴδῃς ἄλλον νέμοντα τὰς βοῦς, ἐμοῦ μνημόνευσον.”
\pend


\pstart
1.30  Δόρκων μὲν τοσαῦτα εἰπὼν καὶ φίλημα φιλήσας ὕστατον ἀφῆκεν ἅμα τῷ φιλήματι τὴν ψυχήν· ἡ δὲ Χλόη λαβοῦσα τὴν σύριγγα καὶ ἐνθεῖσα τοῖς χείλεσιν ἐσύριττεν ὡς ἐδύνατο μέγιστον· καὶ αἱ βόες ἀκούουσι καὶ τὸ μέλος γνωρίζουσι καὶ ὁρμῇ μιᾷ μυκησάμεναι πηδῶσιν εἰς τὴν θάλατταν.  Βιαίου δὲ πηδήματος εἰς ἕνα τοῖχον τῆς νεὼς γενομένου κἀκ της ἐμπτώσεως τῶν βοῶν κοίλης τῆς θαλάττης διαστάσης στρέφεται μὲν ἡ ναῦς καὶ τοῦ κλύδωνος συνιόντος ἀπόλλυται, οἱ δὲ ἐκπίπτουσιν οὐχ ὁμοίαν ἔχοντες ἐλπίδα σωτηρίας.  Οἱ μὲν γὰρ λῃσταὶ τὰς μαχαίρας παρήρτηντο καὶ τὰ ἡμιθωράκια λεπιδωτὰ ἐνεδέδυντο καὶ κνημῖδας εἰς μέσην κνήμην ὑπεδέδεντο· ὁ δὲ Δάφνις ἀνυπόδητος, ὡς ἐν πεδίῳ νέμων, καὶ ἡμίγυμνος,  ὡς ἔτι τῆς ὥρας οὔσης καυματώδους. Ἐκείνους μὲν οὖν ἐπʼ ὀλίγον νηξαμένους τὰ ὅπλα κατήνεγκεν εἰς βυθόν· ὁ δὲ Δάφνις τὴν μὲν ἐσθῆτα ῥᾳδίως ἀπεδύσατο, περὶ δὲ τὴν νῆξιν ἔκαμνεν, οἷα πρότερον νηχόμενος ἐν ποταμοῖς μόνοις·  ὕστερον δὲ παρὰ τῆς ἀνάγκης τὸ πρακτέον διδαχθεὶς εἰς μέσας ὥρμησε τὰς βοῦς, καὶ βοῶν δύο κεράτων ταῖς δύο χερσὶ λαβόμενος ἐκομίζετο μέσος ἀλύπως καὶ ἀπόνως, ὥσπερ ἐλαύνων ἅμαξαν.  Νήχεται δὲ ἄρα βοῦς ὅσον οὐδὲ ἄνθρωπος· μόνων λείπεται τῶν ἐνύδρων ὀρνίθων καὶ αὐτῶν ἰχθύων· οὐδʼ ἂν ἀπόλοιτο βοῦς νηχόμενος, εἰ μὴ τῶν χηλῶν οἱ ὄνυχες περιπέσοιεν διάβροχοι γενόμενοι. Μαρτυροῦσι τῷ λόγῳ μέχρι νῦν πολλοὶ τόποι τῆς θαλάττης, βοὸς πόροι λεγόμενοι.
\pend


\pstart
1.31  Σώζεται μὲν δὴ τοῦτον τὸν τρόπον ὁ Δάφνις, δύο κινδύνους παρʼ ἐλπίδα πᾶσαν διαφυγών, λῃστηρίου καὶ ναυαγίας· ἐξελθὼν δὲ καὶ τὴν Χλόην ἐπὶ τῆς γῆς γελῶσαν ἅμα καὶ δακρύουσαν εὑρὼν ἐμπίπτει τε αὐτῆς τοῖς κόλποις καὶ ἐπυνθάνετο τί βουλομένη συρίσειεν·  ἡ δὲ αὐτῷ διηγεῖται πάντα· τὸν δρόμον τὸν ἐπὶ τὸν Δόρκωνα, τὸ παίδευμα τῶν βοῶν, πῶς κελευσθείη συρίσαι, καὶ ὅτι τέθνηκε Δόρκων· μόνον αἰδεσθεῖσα τὸ φίλημα οὐκ εἶπεν. Ἔδοξε δὴ τιμῆσαι τὸν εὐεργέτην, καὶ ἐλθόντες μετὰ τῶν προσηκόντων Δόρκωνα θάπτουσι τὸν ἄθλιον.  Γῆν μὲν οὖν πολλὴν ἐπένησαν, φυτὰ δὲ ἥμερα πολλὰ ἐφύτευσαν καὶ ἐξήρτησαν αὐτῶν τῶν ἔργων ἀπαρχάς· ἀλλὰ καὶ γάλα κατέσπεισαν καὶ βότρυς κατέθλιψαν καὶ σύριγγας πολλὰς κατέκλασαν.  Ἠκούσθη καὶ τῶν βοῶν ἐλεεινὰ μυκήματα καὶ δρόμοι τινὲς ὤφθησαν ἅμα τοῖς μυκήμασιν ἄτακτοι· καὶ ὡς ἐν ποιμέσιν εἰκάζετο καὶ αἰπόλοις, ταῦτα θρῆνος ἦν τῶν βοῶν ἐπὶ βουκόλῳ τετελευτηκότι.
\pend


\pstart
1.32  Μετὰ δὲ τὸν τοῦ Δόρκωνος τάφον λούει τὸν Δάφνιν ἡ Χλόη πρὸς τὰς Νύμφας ἀγαγοῦσα. Καὶ αὐτὴ τότε πρῶτον Δάφνιδος ὁρῶντος ἐλούσατο τὸ σῶμα λευκὸν καὶ καθαρὸν ὑπὸ κάλλους καὶ οὐδὲν λουτρῶν ἐς κάλλος δεόμενον·  καὶ ἄνθη συλλέξαντες, ὅσα τῆς ὥρας ἐκείνης, ἐστεφάνωσαν τὰ ἀγάλματα καὶ τὴν τοῦ Δόρκωνος σύριγγα τῆς πέτρας ἐξήρτησαν ἀνάθημα. Καὶ μετὰ τοῦτο ἐλθόντες ἐπεσκόπουν τὰς αἶγας καὶ τὰ πρόβατα.  Τὰ δὲ πάντα κατέκειτο μήτε νεμόμενα μήτε βληχώμενα, ἀλλʼ̓, οἶμαι, τὸν Δάφνιν καὶ τὴν Χλόην ἀφανεῖς ὄντας ποθοῦντα. Ἐπειδὴ οὖν ὀφθέντες καὶ ἐβόησαν τὸ σύνηθες καὶ ἐσύρισαν, τὰ μὲν ἀναστάντα ἐνέμετο, αἱ δὲ αἶγες ἐσκίρτων φριματτόμεναι, καθάπερ ἡδόμεναι σωτηρίᾳ συνήθους αἰπόλου.  Οὐ μὴν ὁ Δάφνις χαίρειν ἔπειθε τὴν ψυχήν, ἰδὼν τὴν Χλόην γυμνὴν καὶ τὸ πρότερον λανθάνον κάλλος ἐκκεκαλυμμένον. Ἤλγει τὴν καρδίαν ὡς ἐσθιομένην ὑπὸ φαρμάκων, καὶ αὐτῷ τὸ πνεῦμα ποτὲ μὲν λάβρον ἐξέπνει, καθάπερ τινὸς διώκοντος αὐτόν, ποτὲ δὲ ἐξέλειπε, καθάπερ ἐκδαπανηθὲν ἐν ταῖς πρότερον ἐπιδρομαῖς.  Ἐδόκει τὸ λουτρὸν εἶναι τῆς θαλάττης φοβερώτερον· ἐνόμιζε τὴν ψυχὴν ἔτι παρὰ τοῖς λῃσταῖς μένειν, οἷα νέος καὶ ἄγροικος καὶ ἔτι ἀγνοῶν τὸ ἔρωτος λῃστήριον.
\pend


\pstart
2.1  Ἤδη δὲ τῆς ὀπώρας ἀκμαζούσης καὶ ἐπείγοντος τοῦ τρυγητοῦ πᾶς ἦν κατὰ τοὺς ἀγροὺς ἐν ἔργῳ· ὁ μὲν ληνοὺς ἐπεσκεύαζεν, ὁ δὲ πίθους ἐξεκάθαιρεν, ὁ δὲ ἀρρίχους ἔπλεκεν·  ἔμελέ τινι δρεπάνης μικρᾶς ἐς βότρυος τομὴν καὶ ἑτέρῳ λίθου θλῖψαι τὰ ἔνοινα τῶν βοτρύων δυναμένου καὶ ἄλλῳ λύγου ξηρᾶς πληγαῖς κατεξασμένης, ὡς ἂν ὑπὸ φωτὶ νύκτωρ τὸ γλεῦκος φέροιτο.  Ἀμελήσαντες οὖν καὶ ὁ Δάφνις καὶ ἡ Χλόη τῶν αἰγῶν καὶ τῶν προβάτων, χειρὸς ὠφέλειαν ἄλλοις μετεδίδοσαν. Ὁ μὲν ἐβάσταζεν ἐν ἀρρίχοις βότρυς καὶ ἐπάτει ταῖς ληνοῖς ἐμβαλὼν καὶ εἰς τοὺς πίθους ἔφερε τὸν οἶνον· ἡ δὲ τροφὴν παρεσκεύαζε τοῖς τρυγῶσι καὶ ἐνέχει ποτὸν αὐτοῖς πρεσβύτερον οἶνον καὶ τῶν ἀμπέλων δὲ τὰς ταπεινοτέρας ἀπετρύγα.  Πᾶσα γὰρ κατὰ τὴν Λέσβον ἄμπελος ταπεινή, οὐ μετέωρος οὐδὲ ἀναδενδράς, ἀλλὰ κάτω τὰ κλήματα ἀποτείνουσα καὶ ὥσπερ κιττὸς νεμομένη· καὶ παῖς ἂν ἐφίκοιτο βότρυος ἄρτι τὰς χεῖρας ἐκ σπαργάνων λελυμένος.
\pend


\pstart
2.2  Οἷον οὖν εἰκὸς ἐν ἑορτῇ Διονύσου καὶ οἴνου γενέσεως αἱ μὲν γυναῖκες ἐκ τῶν πλησίον ἀγρῶν εἰς ἐπικουρίαν κεκλημέναι τῷ Δάφνιδι τοὺς ὀφθαλμοὺς ἐπέβαλλον καὶ ἐπῄνουν ὡς ὅμοιον τῷ Διονύσῳ τὸ κάλλος, καί τις τῶν θρασυτέρων καὶ ἐφίλησε καὶ τὸν Δάφνιν παρώξυνε, τὴν δὲ Χλόην ἐλύπησεν·  οἱ δὲ ἐν ταῖς ληνοῖς ποικίλας φωνὰς ἔρριπτον ἐπὶ τὴν Χλόην καὶ ὥσπερ ἐπί τινα Βάκχην Σάτυροι μανικώτερον ἐπήδων καὶ εὔχοντο γενέσθαι ποίμνια καὶ ὑπ’ ἐκείνης νέμεσθαι· ὥστε αὖ πάλιν ἡ μὲν ἥδετο, Δάφνις δὲ ἐλυπεῖτο.  Εὔχοντο δὴ ταχέως παύσασθαι τὸν τρυγητὸν καὶ λαβέσθαι τῶν συνήθων χωρίων καὶ ἀντὶ τῆς ἀμούσου βοῆς ἀκούειν σύριγγος ἢ τῶν ποιμνίων αὐτῶν βληχωμένων.  Καὶ ἐπεὶ διαγενομένων ὀλίγων ἡμερῶν αἱ μὲν ἄμπελοι ἐτετρύγηντο, πίθοι δὲ τὸ γλεῦκος εἶχον, ἔδει δὲ οὐκέτ’ οὐδὲν πολυχειρίας, κατήλαυνον τὰς ἀγέλας ἐς τὸ πεδίον καὶ μάλα χαίροντες τὰς Νύμφας προσεκύνουν, βότρυς αὐταῖς κομίζοντες ἐπὶ κλημάτων, ἀπαρχὰς τοῦ τρυγητοῦ.  Οὐδὲ τὸν πρότερον χρόνον ἀμελῶς ποτε παρῆλθον, ἀλλ’ ἀεὶ ἀρχόμενοι νομῆς προσήδρευον καὶ ἐκ νομῆς ἀνιόντες προσεκύνουν· καὶ πάντως τι ἐπέφερον, ἢ ἄνθος ἢ ὀπώραν ἢ φυλλάδα χλωρὰν ἢ γάλακτος σπονδήν. Καὶ τούτων μὲν ὕστερον ἀμοιβὰς ἐκομίσαντο παρὰ τῶν θεῶν·  τότε δὲ κύνες, φασίν, ἐκ δεσμῶν λυθέντες ἐσκίρτων, ἐσύριττον, ᾖδον, τοῖς τράγοις καὶ τοῖς προβάτοις συνεπάλαιον.
\pend


\pstart
2.3  Τερπομένοις δὲ αὐτοῖς ἐφίσταται πρεσβύτης σισύραν ἐνδεδυμένος, καρβατίνας ὑποδεδεμένος, πήραν ἐξηρτημένος, καὶ τὴν πήραν παλαιάν. Οὗτος πλησίον καθίσας αὐτῶν ὧδε εἶπε “Φιλητᾶς,  ὦ παῖδες, ὁ πρεσβύτης ἐγώ, ὃς πολλα μὲν ταῖσδε ταῖς Νύμφαις ᾖσα, πολλὰ δὲ τῷ Πανὶ ἐκείνῳ ἐσύρισα, βοῶν δὲ πολλῆς ἀγέλης ἡγησάμην μόνῃ μουσικῇ. Ἥκω δὲ ὑμῖν ὅσα εἶδον μηνύσων, ὅσα ἤκουσα ἀπαγγελῶν.  Κῆπός ἐστί μοι τῶν ἐμῶν χειρῶν, ὃν ἐξ οὗ νέμων διὰ γῆρας ἐπαυσάμην, ἐξεπονησάμην· ὅσα ὧραι φέρουσι, πάντα ἔχων ἐν αὑτῷ καθ’ ὥραν ἑκάστην.  Ἦρος ῥόδα κρίνα καὶ ὑάκινθος καὶ ἴα ἀμφότερα, θέρους μήκωνες καὶ ἀχράδες καὶ μῆλα πάντα, νῦν ἄμπελοι καὶ συκαῖ καὶ ῥοιαὶ καὶ μύρτα χλωρά.  Εἰς τοῦτον τὸν κῆπον ὀρνίθων ἀγέλαι συνέρχονται τὸ ἑωθινόν, τῶν μὲν ἐς τροφήν, τῶν δὲ ἐς ᾠδήν· συνηρεφὴς γὰρ καὶ κατάσκιος καὶ πηγαῖς τρισὶ κατάρρυτος· ἂν περιέλῃ τις τὴν αἱμασιάν, ἄλσος ὁρᾶν οἰήσεται.”
\pend


\pstart
2.4  “εἰσελθόντι δέ μοι τήμερον ἀμφὶ μέσην ἡμέραν ὑπὸ ταῖς ῥοιαῖς καὶ ταῖς μυρρίναις βλέπεται παῖς, μύρτα καὶ ῥοιὰς ἔχων, λευκὸς ὡς γάλα, ξανθὸς ὡς πῦρ, στιλπνὸς ὡς ἄρτι λελουμένος· γυμνὸς ἦν, μόνος ἦν·  ἔπαιζεν ὡς ἴδιον κῆπον τρυγῶν. Ἐγὼ μὲν οὖν ὥρμησα ἐπ’ αὐτὸν ὡς συλληψόμενος, δείσας μὴ ὑπ’ ἀγερωχίας τὰς μυρρίνας καὶ τὰς ῥοιὰς κατακλάσῃ· ὁ δέ με κούφως καὶ ῥᾳδίως ὑπέφευγε, ποτὲ μὲν ταῖς ῥοδωνιαῖς ὑποτρέχων, ποτὲ δὲ ταῖς μήκωσιν ὑποκρυπτόμενος,  ὥσπερ πέρδικος νεοττός. Καίτοι πολλάκις μὲν πράγματα ἔσχον ἐρίφους γαλαθηνοὺς διώκων, πολλάκις δὲ ἔκαμον μεταθέων μόσχους ἀρτιγεννήτους· ἀλλὰ τοῦτο ποικίλον τι χρῆμα ἦν καὶ ἀθήρατον. Καμὼν οὖν ὡς γέρων καὶ ἐπερεισάμενος τῇ βακτηρίᾳ καὶ ἅμα φυλάττων μὴ φύγοι, ἐπυνθανόμην τίνος ἐστὶ τῶν γειτόνων, καὶ τί βουλόμενος ἀλλότριον κῆπον τρυγᾷ.  Ὁ δὲ ἀπεκρίνατο μὲν οὐδέν, στὰς δὲ πλησίον ἐγέλα πάνυ ἁπαλὸν καὶ ἔβαλλέ με τοῖς μύρτοις καὶ οὐκ οἶδ’ ὅπως ἔθελγε μηκέτι θυμοῦσθαι. Ἐδεόμην οὖν εἰς χεῖρας ἐλθεῖν μηδὲν φοβούμενον ἔτι, καὶ ὤμνυον κατὰ τῶν μύρτων ἀφήσειν, ἐπιδοὺς μήλων καὶ ῥοιῶν, παρέξειν τε ἀεὶ τρυγᾶν τὰ φυτὰ καὶ δρέπειν τὰ ἄνθη, τυχὼν παρ’ αὐτοῦ φιλήματος ἑνός.”
\pend


\pstart
2.5  “Ἐνταῦθα πάνυ καπυρὸν γελάσας ἀφίησι φωνήν, οἵαν οὔτε χελιδὼν οὔτε ἀηδὼν οὔτε κύκνος, ὁμοίως ἐμοὶ γέρων γενόμενος. “Ἐμοὶ μέν, ὦ Φιλητᾶ, φιλῆσαί σε φθόνος οὐδείς· βούλομαι γὰρ φιλεῖσθαι μᾶλλον ἢ σὺ γενέσθαι νέος· ὅρα δὲ εἴ σοι καθ’ ἡλικίαν τὸ δῶρον·  οὐδὲν γάρ σε ὠφελήσει τὸ γῆρας πρὸς τὸ μὴ διώκειν ἐμὲ μετὰ τὸ ἓν φίλημα. Δυσθήρατος ἐγὼ καὶ ἱέρακι καὶ ἀετῷ καὶ εἴ τις ἄλλος τούτων ὠκύτερος ὄρνις. Οὔ τοι παῖς ἐγὼ καὶ εἰ δοκῶ παῖς, ἀλλὰ καὶ τοῦ Κρόνου πρεσβύτερος καὶ αὐτοῦ τοῦ παντός.  Καί σε οἶδα νέμοντα πρωθήβην ἐν ἐκείνῳ τῷ ὄρει τὸ πλατὺ βουκόλιον καὶ παρήμην σοι συρίττοντι πρὸς ταῖς φηγοῖς ἐκείναις, ἡνίκα ἤρας Ἀμαρυλλίδος· ἀλλά με οὐχ ἑώρας καίτοι πλησίον μάλα τῇ κόρῃ παρεστῶτα. Σοὶ μὲν οὖν ἐκείνην ἔδωκα· καὶ ἤδη σοι παῖδες, ἀγαθοὶ βουκόλοι καὶ γεωργοί·  νῦν δὲ Δάφνιν ποιμαίνω καὶ Χλόην· καὶ ἡνίκα ἂν αὐτοὺς εἰς ἓν συναγάγω τὸ ἑωθινόν, εἰς τὸν σὸν ἔρχομαι κῆπον καὶ τέρπομαι τοῖς ἄνθεσι καὶ τοῖς φυτοῖς κἀν ταῖς πηγαῖς ταύταις λούομαι. Διὰ τοῦτο καλὰ καὶ τὰ ἄνθη καὶ τὰ φυτά, τοῖς ἐμοῖς λουτροῖς ἀρδόμενα.  Ὅρα δὲ μή τί σοι τῶν φυτῶν κατακέκλασται, μή τις ὀπώρα τετρύγηται, μή τις ἄνθους ῥίζα πεπάτηται, μή τις πηγὴ τετάρακται, καὶ χαῖρε μόνος ἀνθρώπων ἐν γήρᾳ θεασάμενος τοῦτο τὸ παιδίον.ʼ” ”
\pend


\pstart
2.6  “ταῦτ’ εἰπὼν ἀνήλατο καθάπερ ἀηδόνος νεοττὸς ἐπὶ τὰς μυρρίνας, καὶ κλάδον ἀμείβων ἐκ κλάδου διὰ τῶν φύλλων ἀνεῖρπεν εἰς ἄκρον. Εἶδον αὐτοῦ καὶ πτέρυγας ἐκ τῶν ὤμων καὶ τοξάρια μεταξὺ τῶν πτερύγων,  καὶ οὐκέτι εἶδον οὔτε ταῦτα οὔτε αὐτόν. Εἰ δὲ μὴ μάτην ταύτας τὰς πολιὰς ἔφυσα μηδὲ γηράσας ματαιοτέρας τὰς φρένας ἐκτησάμην, Ἔρωτι, ὦ παῖδες, κατέσπεισθε καὶ Ἔρωτι ὑμῶν μέλει.”
\pend


\pstart
2.7  πάνυ ἐτέρφθησαν ὥσπερ μῦθον οὐ λόγον ἀκούοντες καὶ ἐπυνθάνοντο τί ἐστί ποτε ὁ Ἔρως, πότερα παῖς ἢ ὄρνις, καὶ τί δύναται. Πάλιν οὖν ὁ Φιλητᾶς ἔφη “θεός ἐστιν, ὦ παῖδες, ὁ Ἔρως, νέος καὶ καλὸς καὶ πετόμενος· διὰ τοῦτο καὶ νεότητι χαίρει καὶ κάλλος διώκει καὶ τὰς ψυχὰς ἀναπτεροῖ.  Δύναται δὲ τοσοῦτον ὅσον οὐδὲ ὁ Ζεύς. Κρατεῖ μὲν στοιχείων, κρατεῖ δὲ ἄστρων, κρατεῖ δὲ τῶν ὁμοίων θεῶν· οὐδὲ ὑμεῖς τοσοῦτον τῶν αἰγῶν καὶ τῶν προβάτων.  Τὰ ἄνθη πάντα Ἔρωτος ἔργα· τὰ φυτὰ πάντα τούτου ποιήματα· διὰ τοῦτον καὶ ποταμοὶ ῥέουσι καὶ ἄνεμοι πνέουσιν.  Ἔγνων δὲ ἐγὼ καὶ ταῦρον ἐρασθέντα, καὶ ὡς οἴστρῳ πληγεὶς ἐμυκᾶτο· καὶ τράγον φιλήσαντα αἶγα, καὶ ἠκολούθει πανταχοῦ. Αὐτὸς μὲν γὰρ ἤμην νέος καὶ ἠράσθην Ἀμαρυλλίδος· καὶ οὔτε τροφῆς ἐμεμνήμην οὔτε ποτὸν προσεφερόμην οὔτε ὕπνον ᾑρούμην.  Ἤλγουν τὴν ψυχήν, τὴν καρδίαν ἐπαλλόμην, τὸ σῶμα ἐψυχόμην· ἐβόων ὡς παιόμενος, ἐσιώπων ὡς νεκρούμενος, εἰς ποταμοὺς ἐνέβαινον ὡς καόμενος.  Ἐκάλουν τὸν Πᾶνα βοηθόν, ὡς καὶ αὐτὸν τῆς Πίτυος ἐρασθέντα· ἐπῄνουν τὴν Ἠχὼ τὸ Ἀμαρυλλίδος ὄνομα μετ’ ἐμὲ καλοῦσαν· κατέκλων τὰς σύριγγας, ὅτι μοι τὰς μὲν βοῦς ἔθελγον, Ἀμαρυλλίδα δὲ οὐκ ἦγον.  Ἔρωτος γὰρ οὐδὲν φάρμακον, οὐ πινόμενον, οὐκ ἐσθιόμενον, οὐκ ἐν ᾠδαῖς λαλούμενον, ὅτι μὴ φίλημα καὶ περιβολὴ καὶ συγκατακλινῆναι γυμνοῖς σώμασι.”
\pend


\pstart
2.8  Φιλητᾶς μὲν τοσαῦτα παιδεύσας αὐτοὺς ἀπαλλάττεται, τυρούς τινας παρ’ αὐτῶν καὶ ἔριφον ἤδη κεράστην λαβών· οἱ δὲ μόνοι καταλειφθέντες, τότε πρῶτον ἀκούσαντες τὸ Ἔρωτος ὄνομα τάς τε ψυχὰς συνεστάλησαν ὑπὸ λύπης, καὶ ἐπανελθόντες νύκτωρ εἰς τὰς ἐπαύλεις παρέβαλλον οἷς ἤκουσαν τὰ αὑτῶν.  “Ἀλγοῦσιν οἱ ἐρῶντες· καὶ ἡμεῖς ἀμελοῦσιν· ἠμελήκαμεν ὁμοίως. Καθεύδειν οὐ δύνανται· τοῦτο νῦν πάσχομεν καὶ ἡμεῖς. Κάεσθαι δοκοῦσι· καὶ παρ’ ἡμῖν τὸ πῦρ. Ἐπιθυμοῦσιν ἀλλήλους ὁρᾶν· διὰ τοῦτο θᾶττον εὐχόμεθα γενέσθαι τὴν ἡμέρα.  Σχεδὸν τοῦτ’ ἔστιν ὁ ἔρως, καὶ ἐρῶμεν ἀλλήλων οὐκ εἰδότες. Εἰ τοῦτο μή ἐστιν ὁ ἔρως ἐγὼ δὲ ὁ ἐρώμενος, τί οὖν ταῦτ’ ἀλγοῦμεν, τί δὲ ἀλλήλους ζητοῦμεν; ἀληθῆ πάντα εἶπεν ὁ Φιλητᾶς.  Τὸ ἐκ τοῦ κήπου παιδίον ὤφθη καὶ τοῖς πατράσιν ἡμῶν ὄναρ καὶ νέμειν ἡμᾶς τὰς ἀγέλας ἐκέλευσε. Πῶς ἄν τις αὐτὸ λάβοι; σμικρόν ἐστι καὶ φεύξεται. πῶς ἄν τις αὐτὸ φύγοι; πτερὰ ἔχει καὶ καταλήψεται.  Ἐπὶ τὰς Νύμφας δεῖ βοηθοὺς καταφεύγειν. Ἀλλ’ οὐδὲ Φιλητᾶν ὁ Πὰν ὠφέλησεν Ἀμαρυλλίδος ἐρῶντα. Ὅσα εἶπεν ἄρα φάρμακα, ταῦτα ζητητέον, φίλημα καὶ περιβολὴν καὶ κεῖσθαι γυμνοὺς χαμαί. Κρύος μέν, ἀλλὰ καρτερήσομεν δεύτεροι μετὰ Φιλητᾶν.”
\pend


\pstart
2.9  Τοῦτο αὐτοῖς γίνεται νυκτερινὸν παιδευτήριον. Καὶ ἀγαγόντες τῆς ἐπιούσης ἡμέρας τὰς ἀγέλας ἐς νομήν, ἐφίλησαν ἀλλήλους ἰδόντες καὶ (ὃ μήπω πρότερον ἐποίησαν) περιέβαλον τὰς χεῖρας ἐπαλλάξαντες· τὸ δὲ τρίτον ὤκνουν φάρμακον θρασύτερον γὰρ οὐ μόνον παρθένῳ ἀλλὰ καὶ νέῳ αἰπόλῳ.  Πάλιν οὖν νὺξ ἀγρυπνίαν ἔχουσα καὶ ἔννοιαν τῶν γεγενημένων καὶ κατάμεμψιν τῶν παραλελειμμένων. “Ἐφιλήσαμεν, καὶ οὐδὲν ὄφελος· περιεβάλομεν, καὶ οὐδὲν πλέον ἔσχομεν· τὸ οὖν συγκατακλινῆναι μόνον φάρμακον ἔρωτος. Πειρατέον καὶ τούτου· πάντως ἐν αὐτῷ τι κρεῖττόν ἐστι φιλήματος.”
\pend


\pstart
2.10  Ἐπὶ τούτοις τοῖς λογισμοῖς οἷον εἰκὸς καὶ ὀνείρατα ἑώρων ἐρωτικά, τὰ φιλήματα, τὰς περιβολάς· καὶ ὅσα μεθ’ ἡμέραν οὐκ ἔπραξαν, ταῦτα ὄναρ ἔπραξαν· γυμνοὶ μετ’ ἀλλήλων ἔκειντο.  Ἐνθεώτεροι δὴ κατὰ τὴν ἐπιοῦσαν ἡμέραν ἀνέστησαν καὶ ῥοίζῳ τὰς ἀγέλας κατήλαυνον ἐπειγόμενοι πρὸς τὰ φιλήματα· καὶ ἰδόντες ἀλλήλους ἅμα μειδιάματι προσέδραμον.  Τὰ μὲν οὖν φιλήματα ἐγένετο καὶ ἡ περιβολὴ τῶν χειρῶν ἠκολούθησε, τὸ δὲ τρίτον φάρμακον ἐβράδυνε, μήτε τοῦ Δάφνιδος τολμῶντος εἰπεῖν, μήτε τῆς Χλόης βουλομένης κατάρχεσθαι, ἔστε τύχῃ καὶ τοῦτο ἔπραξαν.
\pend


\pstart
2.11  Καθεζόμενοι ὑπὸ στελέχει δρυὸς πλησίον ἀλλήλων καὶ γευσάμενοι τῆς ἐν φιλήματι τέρψεως ἀπλήστως ἐνεφοροῦντο τῆς ἡδονῆς. Ἦσαν δὲ καὶ χειρῶν περιβολαὶ θλῖψιν τοῖς σώμασι παρέχουσαι.  Κατὰ τὴν τῶν χειρῶν περιβολὴν βιαιότερόν τι τοῦ Δάφνιδος ἐπισπασαμένου κλίνεταίπως ἐπὶ πλευρὰν ἡ Χλόη· κἀκεῖνος δὲ συγκατακλίνεται τῷ φιλήματι ἀκολουθῶν. Καὶ γνωρίσαντες τῶν ὀνείρων τὴν εἰκόνα κατέκειντο πολὺν χρόνον ὥσπερ συνδεδεμένοι.  Εἰδότες δὲ τῶν ἐντεῦθεν οὐδὲν καὶ νομίσαντες τοῦτο εἶναι πέρας ἐρωτικῆς ἀπολαύσεως, μάτην τὸ πλεῖστον τῆς ἡμέρας δαπανήσαντες διελύθησαν καὶ τὰς ἀγέλας ἀπήλαυνον, τὴν νύκτα μισοῦντες. Ἴσως δὲ ἄν τι καὶ τῶν ἀληθῶν ἔπραξαν, εἰ μὴ θόρυβος τοιόσδε πᾶσαν τὴν ἀγροικίαν ἐκείνην κατέλαβε.
\pend


\pstart
2.12  Νέοι Μηθυμναῖοι πλούσιοι διαθέσθαι τὸν τρυγητὸν ἐν ξενικῇ τέρψει θελήσαντες, ναῦν σμικρὰν καθελκύσαντες καὶ οἰκέτας προσκώπους καθίσαντες, τοὺς Μυτιληναίων ἀγροὺς παρέπλεον, ὅσοι θαλάττης πλησίον.  Εὐλίμενός τε γὰρ ἡ παραλία καὶ οἰκήσεσιν ἠσκημένη πολυτελῶς, καὶ λουτρὰ συνεχῆ, παράδεισοί τε καὶ ἄλση· τὰ μὲν φύσεως ἔργα, τὰ δ’ ἀνθρώπων τέχνη·  πάντα ἐνηβῆσαι καλά. Παραπλέοντες δὲ καὶ ἐνορμιζόμενοι κακὸν μὲν ἐποίουν οὐδέν, τέρψεις δὲ ποικίλας ἐτέρποντο, ποτὲ μὲν ἀγκίστροις καλάμων ἀπηρτημένοις ἐκ λίνου λεπτοῦ πετραίους ἰχθῦς ἁλιεύοντες ἐκ πέτρας ἁλιτενοῦς, ποτὲ δὲ κυσὶ καὶ δικτύοις λαγὼς φεύγοντας τὸν ἐν ταῖς ἀμπέλοις θόρυβον λαμβάνοντες·  ἤδη δὲ καὶ ὀρνίθων ἄγρας ἐμέλησεν αὐτοῖς, καὶ ἔλαβον βρόχοις χῆνας ἀγρίους καὶ νήττας καὶ ὠτίδας, ὥστε ἡ τέρψις αὐτοῖς καὶ τραπέζης ὠφέλειαν παρεῖχεν. Εἰ δέ τινος προσέδει, παρὰ τῶν ἐν τοῖς ἀγροῖς ἐλάμβανον, περιττοτέρους τῆς ἀξίας ὀβολοὺς καταβάλλοντες.  Ἔδει δὲ μόνον ἄρτου καὶ οἴνου καὶ στέγης· οὐ γὰρ ἀσφαλὲς ἐδόκει μετοπωρινῆς ὥρας ἐνεστώσης ἐνθαλαττεύειν· ὥστε καὶ τὴν ναῦν ἀνεῖλκον ἐπὶ τὴν γῆν νύκτα χειμέριον δεδοικότες.
\pend


\pstart
2.13  Τῶν δή τις ἀγροίκων ἐς ἀνολκὴν λίθου θλίβοντος τὰ πατηθέντα βοτρύδια χρῄζων σχοίνου, τῆς πρότερον ῥαγείσης, κρύφα ἐπὶ τὴν θάλατταν ἐλθών, ἀφρουρήτῳ τῇ νηὶ προσελθών, τὸ πεῖσμα ἐκλύσας,  οἴκαδε κομίσας ἐς ὅ τι ἔχρῃζεν ἐχρήσατο. Ἕωθεν οὖν οἱ Μηθυμναῖοι νεανίσκοι ζήτησιν ἐποιοῦντο τοῦ πείσματος καὶ (ὡμολόγει γὰρ οὐδεὶς τὴν κλοπὴν) ὀλίγα μεμψάμενοι τοὺς ξενοδόκους παρέπλεον· καὶ σταδίους τριάκοντα παρελάσαντες προσορμίζονται τοῖς ἀγροῖς, ἐν οἷς ᾤκουν ὁ Δάφνις καὶ ἡ Χλόη· ἐδόκει γὰρ αὐτοῖς καλὸν εἶναι τὸ πεδίον ἐς θήραν λαγῶν.  Σχοῖνον μὲν οὖν οὐκ εἶχον ὥστε ἐκδήσασθαι πεῖσμα· λύγον δὲ χλωρὰν μακρὰν στρέψαντες ὡς σχοῖνον, ταύτῃ τὴν ναῦν ἐκ τῆς πρύμνης ἄκρας εἰς τὴν γῆν ἔδησαν. Ἔπειτα τοὺς κύνας ἀφέντες ῥινηλατεῖν ἐν ταῖς εὐκαίροις φαινομέναις τῶν ὁδῶν ἐλινοστάτουν.  Οἱ μὲν δὴ κύνες ἅμα ὑλακῇ διαθέοντες ἐφόβησαν τὰς αἶγας, αἱ δὲ τὰ ὀρεινὰ καταλιποῦσαι μᾶλλόν τι πρὸς τὴν θάλατταν ὥρμησαν· ἔχουσαι δὲ οὐδὲν ἐν ψάμμῳ τρώξιμον, ἐλθοῦσαι πρὸς τὴν ναῦν αἱ θρασύτεραι αὐτῶν, τὴν λύγον τὴν χλωράν, ᾗ ἐδέδετο ἡ ναῦς, ἀπέφαγον.
\pend


\pstart
2.14  Ἦν δέ τι καὶ κλυδώνιον ἐν τῇ θαλάττῃ, κινηθέντος ἀπὸ τῶν ὀρῶν πνεύματος. Ταχὺ δὴ μάλα λυθεῖσαν αὐτὴν ὑπήνεγκεν ἡ παλίρροια τοῦ κύματος καὶ ἐς τὸ πέλαγος μετέωρον ἔφερεν. Αἰσθήσεως  δὲ τοῖς Μηθυμναίοις γενομένης οἱ μὲν ἐπὶ τὴν θάλατταν ἔθεον, οἱ δὲ τοὺς κύνας συνέλεγον· ἐβόων δὲ πάντες, ὡς πάντας τοὺς ἐκ τῶν πλησίον ἀγρῶν ἀκούσαντας συνελθεῖν. Ἀλλ’ ἦν οὐδὲν ὄφελος· τοῦ γὰρ πνεύματος ἀκμάζοντος ἀσχέτῳ τάχει κατὰ ῥοῦν ἡ ναῦς ἐφέρετο.  Οἱ οὖν Μηθυμναῖοι οὐκ ὀλίγων κτημάτων στερόμενοι ἐζήτουντὸν νέμοντα τὰς αἶγας· καὶ εὑρόντες τὸν Δάφνιν ἔπαιον, ἀπέδυον· εἷς δέ τις καὶ κυνόδεσμον ἀράμενος περιῆγε τὰς χεῖρας ὡς δήσων.  Ὁ δὲ ἐβόα τε παιόμενος καὶ ἱκέτευε τοὺς ἀγροίκους καὶ πρώτους τὸν Λάμωνα καὶ τὸν Δρύαντα βοηθοὺς ἐπεκαλεῖτο. Οἱ δὲ ἀντείχοντο σκληροὶ γέροντες καὶ χεῖρας ἐκ γεωργικῶν ἔργων ἰσχυρὰς ἔχοντες, καὶ ἠξίουν δικαιολογήσασθαι περὶ τῶν γεγενημένων.
\pend


\pstart
2.15  Ταὐτὰ δὲ καὶ τῶν ἄλλων ἀξιούντων δικαστὴν καθίζουσι Φιλητᾶν τὸν βουκόλον· πρεσβύτατός τε γὰρ ἦν τῶν παρόντων καὶ κλέος εἶχεν ἐν τοῖς κωμήταις δικαιοσύνης περιττῆς. Πρῶτοι δὲ κατηγόρουν οἱ Μηθυμναῖοι σαφῆ καὶ σύντομα, βουκόλον ἔχοντες δικαστήν.  “Ἤλθομεν εἰς τούτους τοὺς ἀγροὺς θηρᾶσαι θέλοντες. Τὴν μὲν οὖν ναῦν λύγῳ χλωρᾷ δήσαντες ἐπὶ τῆς ἀκτῆς κατελίπομεν, αὐτοὶ δὲ διὰ τῶν κυνῶν ζήτησιν ἐποιούμεθα θηρίων. Ἐν τούτῳ πρὸς τὴν θάλατταν αἱ αἶγες τούτου κατελθοῦσαι τήν τε λύγον κατεσθίουσι καὶ τὴν ναῦν ἀπολύουσιν.  Εἶδες αὐτὴν ἐν τῇ θαλάττῃ φερομένην, πόσων οἴει μεστὴν ἀγαθῶν; Οἵα μὲν ἐσθὴς ἀπόλωλεν, οἷος δὲ κόσμος σκευῶν, ὅσον δὲ ἀργύριον. Τοὺς ἀγροὺς ἄν τις τούτους ἐκεῖνα ἔχων ὠνήσαιτο. Ἀνθ’ ὧν ἀξιοῦμεν ἄγειν τοῦτον, πονηρὸν ὄντα αἰπόλον, ὃς ἐπὶ τῆς θαλάττης νέμει τὰς αἶγας ὡς ναύτης.”
\pend


\pstart
2.16  Τοσαῦτα οἱ Μηθυμναῖοι κατηγόρησαν· ὁ δὲ Δάφνις διέκειτο μὲν κακῶς ὑπὸ τῶν πληγῶν, Χλόην δὲ ὁρῶν παροῦσαν πάντων κατεφρόνει καὶ ὧδε εἶπεν “Ἐγὼ νέμω τὰς αἶγας καλῶς. Οὐδέποτε ᾐτιάσατο κωμήτης οὐδὲ εἷς ὡς ἢ κῆπόν τινος αἲξ ἐμὴ κατεβοσκήσατο ἢ ἄμπελον βλαστάνουσαν κατέκλασεν.  Οὗτοι δέ εἰσι κυνηγέται πονηροὶ καὶ κύνας ἔχουσι κακῶς πεπαιδευμένους, οἵτινες τρέχοντες πολλὰ καὶ ὑλακτοῦντες σκληρὰ κατεδίωξαν αὐτὰς ἐκ τῶν ὀρῶν καὶ τῶν πεδίων ἐπὶ τὴν θάλατταν, ὥσπερ λύκοι.  Ἀλλὰ ἀπέφαγον τὴν λύγον· οὐ γὰρ εἶχον ἐν ψάμμῳ πόαν ἢ κόμαρον ἢ θύμον. Ἀλλὰ ἀπώλετο ἡ ναῦς ὑπὸ τοῦ πνεύματος καὶ τῆς θαλάττης· ταῦτα χειμῶνος, οὐκ αἰγῶν ἐστὶν ἔργα. Ἀλλὰ ἐσθὴς ἐνέκειτο καὶ ἄργυρος· καὶ τίς πιστεύσει νοῦν ἔχων ὅτι τοσαῦτα φέρουσα ναῦς πεῖσμα εἶχε λύγον;”
\pend


\pstart
2.17  Τούτοις ἐπεδάκρυσεν ὁ Δάφνις καὶ εἰς οἶκτον ὑπηγάγετο τοὺς ἀγροίκους πολύν, ὥστε ὁ Φιλητᾶς, ὁ δικαστής, ὤμνυε Πᾶνα καὶ Νύμφας μηδὲν ἀδικεῖν Δάφνιν, ἀλλὰ μηδὲ τὰς αἶγας, τὴν δὲ θάλατταν καὶ τὸν ἄνεμον, ὧν ἄλλους εἶναι δικαστάς.  Ταῦτα λέγων οὐκ ἔπειθε Φιλητᾶς Μηθυμναίους, ἀλλ’ ὑπ’ ὀργῆς ὁρμήσαντες ἦγον πάλιν τὸν Δάφνιν καὶ συνδεῖν ἤθελον.  Ἐνταῦθα οἱ κωμῆται ταραχθέντες ἐπιπηδῶσιν αὐτοῖς ὡσεὶ ψᾶρες ἢ κολοιοί· καὶ ταχὺ μὲν ἀφαιροῦνται τὸν Δάφνιν ἤδη καὶ αὐτὸν μαχόμενον, ταχὺ δὲ ξύλοις παίοντες ἐκείνους εἰς φυγὴν ἐτρέψαντο· ἀπέστησαν δὲ οὐ πρότερον ἔστε τῶν ὅρων αὐτοὺς ἐξήλασαν ἐς ἄλλους ἀγρούς.
\pend


\pstart
2.18  Διωκόντων δὴ τοὺς Μηθυμναίους ἐκείνων ἡ Χλόη κατὰ πολλὴν ἡσυχίαν ἄγει πρὸς τὰς Νύμφας τὸν Δάφνιν καὶ ἀπονίπτει τε τὸ πρόσωπον ᾑμαγμένον ἐκ τῶν ῥινῶν ῥαγεισῶν ὑπὸ πληγῆς τινος, κἀκ τῆς πήρας προκομίσασα ζυμίτου μέρος καὶ τυροῦ τμῆμά τι, δίδωσι φαγεῖν· τό τε μάλιστα ἀνακτησόμενον αὐτόν, φίλημα ἐφίλησε μελιτῶδες ἁπαλοῖς τοῖς χείλεσι.
\pend


\pstart
2.19  Τότε μὲν δὴ παρὰ τοσοῦτον Δάφνις ἦλθε κακοῦ. Τὸ δὲ πρᾶγμα οὐ ταύτῃ ἐπέπαυτο, ἀλλ’ ἐλθόντες οἱ Μηθυμναῖοι μόλις εἰς τὴν ἑαυτῶν, ὁδοιπόροι μὲν ἀντὶ ναυτῶν τραυματίαι δὲ ἀντὶ τρυφώντων, ἐκκλησίαν τε συνήγαγον τῶν πολιτῶν καὶ ἱκετηρίας θέντες ἱκέτευον τιμωρίας ἀξιωθῆναι·  τῶν μὲν ἀληθῶν λέγοντες οὐδὲ ἕν, μὴ καὶ πρὸς καταγέλαστοι γένοιντο τοιαῦτα καὶ τοσαῦτα παθόντες ὑπὸ ποιμένων, κατηγοροῦντες δὲ Μυτιληναίων, ὡς τὴν ναῦν ἀφελομένων καὶ τὰ χρήματα διαρπασάντων πολέμου νόμῳ.  Οἱ δὲ πιστεύοντες διὰ τὰ τραύματα καὶ νεανίσκοις τῶν πρώτων παρ’ αὐτοῖς οἰκιῶν τιμωρῆσαι δίκαιον νομίζοντες Μυτιληναίοις μὲν πόλεμον ἀκήρυκτον ἐψηφίσαντο· τὸν δὲ στρατηγὸν ἐκέλευσαν δέκα ναῦς καθελκύσαντα κακουργεῖν αὐτῶν τὴν παραλίαν· πλησίον γὰρ χειμῶνος ὄντος οὐκ ἦν ἀσφαλὲς μείζονα στόλον πιστεύειν τῇ θαλάττῃ.
\pend


\pstart
2.20  Ὁ δὲ εὐθὺς τῆς ἐπιούσης ἀναγόμενος αὐτερέταις στρατιώταις ἐπέπλει τοῖς παραθαλαττίοις τῶν Μυτιληναίων ἀγροῖς. Καὶ πολλὰ μὲν ἥρπαζε ποίμνια, πολὺν δὲ σῖτον καὶ οἶνον, ἄρτι πεπαυμένου τοῦ τρυγητοῦ, καὶ ἀνθρώπους δὲ οὐκ ὀλίγους, ὅσοι τούτων ἐργάται.  Ἐπέπλευσε καὶ τοῖς τῆς Χλόης ἀγροῖς καὶ τοῦ Δάφνιδος, καὶ ἀπόβασιν ὀξεῖαν θέμενος λείαν ἤλαυνε τὰ ἐν ποσίν. Ὁ μὲν Δάφνις οὐκ ἔνεμε τὰς αἶγας ἀλλὰ ἐς τὴν ὕλην ἀνελθὼν φυλλάδα χλωρὰν ἔκοπτεν, ὡς ἔχοι τοῦ χειμῶνος παρέχειν τοῖς ἐρίφοις τροφήν· ὥστε ἄνωθεν θεασάμενος τὴν καταδρομὴν ἐνέκρυψεν ἑαυτὸν στελέχει κοίλῳ ξηρᾶς ὀξύης·  ἡ δὲ Χλόη παρῆν ταῖς ἀγέλαις καὶ διωκομένη καταφεύγει πρὸς τὰς Νύμφας ἱκέτις καὶ ἐδεῖτο φείσασθαι καὶ ὧν ἔνεμε καὶ αὐτῆς διὰ τὰς θεάς. Ἀλλ’ ἦν οὐδὲν ὄφελος· οἱ γὰρ Μηθυμναῖοι πολλὰ τῶν ἀγαλμάτων κατακερτομήσαντες καὶ τὰς ἀγέλας ἤλασαν κἀκείνην ἤγαγον ὥσπερ αἶγα ἢ πρόβατον παίοντες λύγοις.
\pend


\pstart
2.21  Ἔχοντες δὲ ἤδη τὰς ναῦς παντοδαπῆς ἁρπαγῆς μεστάς, οὐκέτ’ ἐγίνωσκον περαιτέρω πλεῖν, ἀλλὰ τὸν οἴκαδε πλοῦν ἐποιοῦντο καὶ τὸν χειμῶνα καὶ τοὺς πολεμίους δεδιότες. Οἱ μὲν οὖν ἀπέπλεον εἰρεσίᾳ προσταλαιπωροῦντες·  ἄνεμος γὰρ οὐκ ἦν· ὁ δὲ Δάφνις ἡσυχίας γενομένης ἐλθὼν εἰς τὸ πεδίον ἔνθα ἔνεμον, καὶ μήτε τὰς αἶγας ἰδὼν μήτε τὰ πρόβατα καταλαβὼν μήτε Χλόην εὑρὼν ἀλλ’ ἐρημίαν πολλὴν καὶ τὴν σύριγγα ἐρριμμένην,  ᾗ συνήθως ἐτέρπετο ἡ Χλόη, μέγα βοῶν καὶ ἐλεεινὸν κωκύων ποτὲ μὲν πρὸς τὴν φηγὸν ἔτρεχεν ἔνθα ἐκαθέζοντο, ποτὲ δὲ ἐπὶ τὴν θάλατταν ὡς ὀψόμενος αὐτήν, ποτὲ δὲ ἐπὶ τὰς Νύμφας, ἐφ’ ἃς ἑλκομένη κατέφυγεν. Ἐνταῦθα καὶ ἔρριψεν ἑαυτὸν χαμαὶ καὶ ταῖς Νύμφαις ὡς προδούσαις κατεμέμφετο.
\pend


\pstart
2.22  “Ἀφ’ ὑμῶν ἡρπάσθη Χλόη, καὶ τοῦτο ὑμεῖς ἰδεῖν ὑπεμείνατε; ἡ τοὺς στεφάνους ὑμῖν πλέκουσα, ἡ σπένδουσα τοῦ πρώτου γάλακτος, ἧς καὶ ἡ σύριγξ ἥδε ἀνάθημα;  Αἶγα μὲν οὐδὲ μίαν μοι λύκος ἥρπασε, πολέμιοι δὲ τὴν ἀγέλην καὶ τὴν συννέμουσαν. Καὶ τὰς μὲν αἶγας ἀποδεροῦσι καὶ τὰ πρόβατα καταθύσουσι,  Χλόη δὲ λοιπὸν πόλιν οἰκήσει. Ποίοις ποσὶν ἄπειμι παρὰ τὸν πατέρα καὶ τὴν μητέρα ἄνευ τῶν αἰγῶν, ἄνευ Χλόης, λιπερνήτης γενόμενος; ἔχω γὰρ νέμειν ἔτι οὐδέν.  Ἐνταῦθα περιμενῶ κείμενος ἢ θάνατον ἢ πόλεμον δεύτερον. Ἆρα καὶ σύ, Χλόη, τοιαῦτα πάσχεις; ἆρα μέμνησαι τοῦ πεδίου τοῦδε καὶ τῶν Νυμφῶν τῶνδε κἀμοῦ; ἢ παραμυθοῦνταί σε τὰ πρόβατα καὶ αἱ αἶγες αἰχμάλωτοι μετὰ σοῦ γενόμεναι;”
\pend


\pstart
2.23  Τοιαῦτα λέγοντα αὐτὸν ἐκ τῶν δακρύων καὶ τῆς λύπης ὕπνος βαθὺς καταλαμβάνει. Καὶ αὐτῷ αἱ τρεῖς ἐφίστανται Νύμφαι, μεγάλαι γυναῖκες καὶ καλαί, ἡμίγυμνοι καὶ ἀνυπόδητοι, τὰς κόμας λελυμέναι καὶ τοῖς ἀγάλμασιν ὅμοιαι.  Καὶ τὸ μὲν πρῶτον ἐῴκεσαν ἐλεοῦσαι τὸν Δάφνιν· ἔπειτα ἡ πρεσβυτάτη λέγει ἐπιρρωννύουσα “Μηδὲν ἡμᾶς μέμφου, Δάφνι· Χλόης γὰρ ἡμῖν μᾶλλον μέλει ἢ σοί. Ἡμεῖς τοι καὶ παιδίον οὖσαν αὐτὴν ἠλεήσαμεν καὶ ἐν τῷδε τῷ ἄντρῳ κειμένην αὐτὴν ἀνεθρέψαμεν.  Ἐκείνῃ πεδίοις κοινὸν οὐδέν. Καὶ νῦν δὲ ἡμῖν πεφρόντισται τὸ κατ’ ἐκείνην, ὡς μήτε εἰς τὴν Μήθυμναν κομισθεῖσα δουλεύοι μήτε μέρος γένοιτο λείας πολεμικῆς.  Τὸν Πᾶνα ἐκεῖνον τὸν ὑπὸ τῇ πίτυι ἱδρυμένον, ὃν ὑμεῖς οὐδέποτε οὐδὲ ἄνθεσιν ἐτιμήσατε, τούτου ἐδεήθημεν ἐπίκουρον γενέσθαι Χλόης· συνήθης γὰρ στρατοπέδοις μᾶλλον ἡμῶν καὶ πολλοὺς ἤδη πολέμους ἐπολέμησε τὴν ἀγροικίαν καταλιπών· καὶ ἔπεισι τοῖς Μηθυμναίοις οὐκ ἀγαθὸς πολέμιος.  Κάμνε δὲ μηδέν, ἀλλ’ ἀναστὰς ὄφθητι Λάμωνι καὶ Μυρτάλῃ, οἳ καὶ αὐτοὶ κεῖνται χαμαί, νομίζοντες καὶ σὲ μέρος γεγονέναι τῆς ἁρπαγῆς· Χλόη γάρ σοι τῆς ἐπιούσης ἀφίξεται μετὰ τῶν αἰγῶν, μετὰ τῶν προβάτων, καὶ νεμήσετε κοινῇ καὶ συρίσετε κοινῇ· τὰ δὲ ἄλλα μελήσει περὶ ὑμῶν Ἔρωτι.”
\pend


\pstart
2.24  Τοιαῦτα ἰδὼν καὶ ἀκούσας Δάφνις ἀναπηδήσας τῶν ὕπνων καὶ ὑφ’ ἡδονῆς καὶ λύπης μεστὸς δακρύων τὰ ἀγάλματα τῶν Νυμφῶν προσεκύνει καὶ ἐπηγγέλλετο σωθείσης Χλόης θύσειν τῶν αἰγῶν τὴν ἀρίστην.  Δραμὼν δὲ καὶ ἐπὶ τὴν πίτυν, ἔνθα τὸ τοῦ Πανὸς ἄγαλμα ἵδρυτο κερασφόρον, τραγοσκελές, τῇ μὲν σύριγγα τῇ δὲ τράγον πηδῶντα κατέχον, κἀκεῖνον προσεκύνει καὶ εὔχετο ὑπὲρ τῆς Χλόης καὶ τράγον θύσειν ἐπηγγέλλετο.  Καὶ μόλις ποτὲ περὶ ἡλίου καταφορὰς παυσάμενος δακρύων καὶ εὐχῶν, ἀράμενος τὰς φυλλάδας, ἃς ἔκοψεν, ἐπανῆλθεν εἰς τὴν ἔπαυλιν, καὶ τοὺς ἀμφὶ τὸν Λάμωνα πένθους ἀπαλλάξας,  εὐφροσύνης ἐμπλήσας, τροφῆς τε ἐγεύσατο καὶ ἐς ὕπνον ὥρμησεν οὐδὲ τοῦτον ἄδακρυν, ἀλλ’ εὐχόμενος μὲν αὖθις τὰς Νύμφας ὄναρ ἰδεῖν, εὐχόμενος δὲ τὴν ἡμέραν γενέσθαι ταχέως, ἐν ᾗ Χλόην ἐπηγγείλαντο αὐτῷ. Νυκτῶν πασῶν ἐκείνη ἔδοξε μακροτάτη γεγονέναι· ἐπράχθη δὲ ἐπ’ αὐτῆς τάδε.
\pend


\pstart
2.25  Ὁ στρατηγὸς ὁ τῶν Μηθυμναίων ὅσον δέκα σταδίους ἀπελάσας ἠθέλησε τοὺς στρατιώτας τῇ καταδρομῇ κεκμηκότας ἀναλαβεῖν. Ἄκρας οὖν ἐπεμβαινούσης τῷ πελάγει λαβόμενος ἐπεκτεινομένης μηνοειδῶς,  ἧς ἐντὸς θάλαττα γαληνότερον τῶν λιμένων ὅρμον εἰργάζετο, ἐνταῦθα τὰς ναῦς ἐπ’ ἀγκυρῶν μετεώρους διορμίσας, ὡς μηδὲ μίαν ἐκ τῆς γῆς τῶν ἀγροίκων τινὰ λυπῆσαι, ἀνῆκε τοὺς Μηθυμναίους εἰς τέρψιν εἰρηνικήν.  Οἱ δὲ ἔχοντες πάντων ἀφθονίαν ἐκ τῆς ἁρπαγῆς ἔπινον, ἔπαιζον, ἐπινίκιον ἑορτὴν ἐμιμοῦντο. Ἄρτι δὲ παυομένης ἡμέρας καὶ τῆς τέρψεως ἐς νύκτα ληγούσης αἰφνίδιον μὲν ἡ γῆ πᾶσα ἐδόκει λάμπεσθαι πυρί, κτύπος δὲ ἠκούετο ῥόθιος κωπῶν, ὡς ἐπιπλέοντος μεγάλου στόλου.  Ἐβόα τις ὁπλίζεσθαι τὸν στρατηγόν, ἄλλος ἄλλον ἐκάλει, καὶ τετρῶσθαί τις ἐδόκει, καὶ σχῆμά τις ἔκειτο νεκροῦ μιμούμενος. Εἴκασεν ἄν τις ὁρᾶν νυκτομαχίαν οὐ παρόντων πολεμίων.
\pend


\pstart
2.26  Τῆς δὲ νυκτὸς αὐτοῖς τοιαύτης γενομένης ἐπῆλθεν ἡ ἡμέρα πολὺ τῆς νυκτὸς φοβερωτέρα. Οἱ τράγοι μὲν οἱ τοῦ Δάφνιδος καὶ αἱ αἶγες κιττὸν ἐν τοῖς κέρασι κορυμβοφόρον εἶχον, οἱ δὲ κριοὶ καὶ αἱ οἶς τῆς Χλόης λύκων ὠρυγμὸν ὠρύοντο.  Ὤφθη δὲ καὶ αὐτὴ πίτυος ἐστεφανωμένη. Ἐγίνετο καὶ περὶ τὴν θάλατταν αὐτὴν πολλὰ παράδοξα. Αἵ τε γὰρ ἄγκυραι πειρωμένων ἀναφέρειν κατὰ βυθοῦ ἔμενον, αἵ τε κῶπαι καθιέντων εἰς εἰρεσίαν ἐθραύοντο· καὶ δελφῖνες πηδῶντες ἐξ ἁλὸς ταῖς οὐραῖς παίοντες τὰς ναῦς ἔλυον τὰ γομφώματα.  Ἠκούετό τις καὶ ἀπὸ τῆς ὀρθίου πέτρας τῆς ὑπὲρ τὴν ἄκραν σύριγγος ἦχος· ἀλλὰ οὐκ ἔτερπεν ὡς σύριγξ, ἐφόβει δὲ τοὺς ἀκούοντας ὡς σάλπιγξ.  Ἐταράττοντο οὖν καὶ ἐπὶ τὰ ὅπλα ἔθεον καὶ πολεμίους ἐκάλουν τοὺς οὐ βλεπομένους, ὥστε πάλιν εὔχοντο νύκτα ἐπελθεῖν, ὡς τευξόμενοι σπονδῶν ἐν αὐτῇ.  Συνετὰ μὲν οὖν πᾶσιν ἦν τὰ γινόμενα τοῖς φρονοῦσιν ὀρθῶς ὅτι ἐκ Πανὸς ἦν τὰ φαντάσματα καὶ ἀκούσματα μηνίοντός τι τοῖς ναύταις· οὐκ εἶχον δὲ τὴν αἰτίαν συμβαλεῖν (οὐδὲν γὰρ ἱερὸν ἐσεσύλητο Πανός), ἔστε ἀμφὶ μέσην ἡμέραν ἐς ὕπνον οὐκ ἀθεεὶ τοῦ στρατηγοῦ καταπεσόντος αὐτὸς ὁ Πὰν ὤφθη τοιάδε λέγων
\pend


\pstart
2.27  “Ὦ πάντων ἀνοσιώτατοι καὶ ἀσεβέστατοι, τί ταῦτα μαινομέναις φρεσὶν ἐτολμήσατε; Πολέμου μὲν τὴν ἀγροικίαν ἐνεπλήσατε τὴν ἐμοὶ φίλην, ἀγέλας δὲ βοῶν καὶ αἰγῶν καὶ ποιμνίων ἀπηλάσατε τὰς ἐμοὶ μελομένας·  ἀπεσπάσατε δὲ βωμῶν παρθένον, ἐξ ἧς Ἔρως μῦθον ποιῆσαι θέλει· καὶ οὔτε τὰς Νύμφας ᾐδέσθητε βλεπούσας οὔτε τὸν Πᾶνα ἐμέ. Οὔτ’ οὖν Μήθυμναν ὄψεσθε μετὰ τοιούτων λαφύρων πλέοντες, οὔτε τήνδε φεύξεσθε τὴν σύριγγα τὴν ὑμᾶς ταράξασαν·  ἀλλὰ ὑμᾶς βορὰν ἰχθύων θήσω καταδύσας, εἰ μὴ τὴν ταχίστην καὶ Χλόην ταῖς Νύμφαις ἀποδώσεις καὶ τὰς ἀγέλας Χλόης. Ἀνάστα δὴ καὶ ἐκβίβαζε τὴν κόρην μεθ’ ὧν εἶπον. Ἡγήσομαι δὲ ἐγὼ καὶ σοὶ τοῦ πλοῦ κἀκείνῃ τῆς ὁδοῦ.”
\pend


\pstart
2.28  Πάνυ οὖν τεθορυβημένος ὁ Βρύαξις (τοῦτο γὰρ ἐκαλεῖτο ὁ στρατηγὸς) ἀναπηδᾷ καὶ τῶν νεῶν καλέσας τοὺς ἡγεμόνας ἐκέλευσε τὴν ταχίστην ἐν τοῖς αἰχμαλώτοις ἀναζητεῖσθαι Χλόην.  Οἱ δὲ ταχέως καὶ ἀνεῦρον καὶ εἰς ὀφθαλμοὺς ἐκόμισαν· ἐκαθέζετο γὰρ τῆς πίτυος ἐστεφανωμένη. Σύμβολον δὴ καὶ τοῦτο τῆς ἐν τοῖς ὀνείροις ὄψεως ποιούμενος ἐπ’ αὐτῆς τῆς ναυαρχίδος εἰς τὴν γῆν αὐτὴν κομίζει.  Κἀκείνη ἄρτι ἀπεβεβήκει καὶ σύριγγος ἦχος ἀκούεται πάλιν ἐκ τῆς πέτρας, οὐκέτι πολεμικὸς καὶ φοβερός, ἀλλὰ ποιμενικὸς καὶ οἷος εἰς νομὴν ἡγεῖται ποιμνίων. Καὶ τά τε πρόβατα κατὰ τῆς ἀποβάθρας ἐξέτρεχεν ἐξολισθάνοντα τοῖς κέρασι τῶν χηλῶν, καὶ αἱ αἶγες πολὺ θρασύτερον, οἷα καὶ κρημνοβατεῖν εἰθισμέναι.
\pend


\pstart
2.29  Καὶ ταῦτα μὲν περιίσταται κύκλῳ τὴν Χλόην ὥσπερ χορός, σκιρτῶντα καὶ βληχώμενα καὶ ὅμοια χαίρουσιν· αἱ δὲ τῶν ἄλλων αἰπόλων αἶγες καὶ τὰ πρόβατα καὶ τὰ βουκόλια κατὰ χώραν ἔμενεν ἐν κοίλῃ νηί, καθάπερ αὐτὰ τοῦ μέλους μὴ καλοῦντος.  Θαύματι δὲ πάντων ἐχομένων καὶ τὸν Πᾶνα εὐφημούντων ὤφθη τούτων ἐν τοῖς στοιχείοις ἀμφοτέροις θαυμασιώτερα.  Τῶν μὲν Μηθυμναίων, πρὶν ἀνασπάσαι τὰς ἀγκύρας, ἔπλεον αἱ νῆες καὶ τῆς ναυαρχίδος ἡγεῖτο δελφὶν πηδῶν ἐξ ἁλός· τῶν δὲ αἰγῶν καὶ τῶν προβάτων ἡγεῖτο σύριγγος ἦχος ἥδιστος, καὶ τὸν συρίττοντα ἔβλεπεν οὐδείς, ὥστε τὰ ποίμνια καὶ αἱ αἶγες προῄεσαν ἅμα καὶ ἐνέμοντο τερπόμεναι τῷ μέλει.
\pend


\pstart
2.30  Δευτέρας που νομῆς καιρὸς ἦν, καὶ ὁ Δάφνις ἀπὸ σκοπῆς τινος μετεώρου θεασάμενος τὰς ἀγέλας καὶ τὴν Χλόην, μέγα βοήσας “ὦ Νύμφαι καὶ Πὰν” κατέδραμεν εἰς τὸ πεδίον καὶ περιπλακεὶς τῇ Χλόῃ λιποθυμήσας κατέπεσε.  Μόλις δὲ ἔμβιος ὑπὸ τῆς Χλόης φιλούσης καὶ ταῖς περιβολαῖς θαλπούσης γενόμενος ἐπὶ τὴν συνήθη φηγὸν ἔρχεται· καὶ ὑπὸ τῷ στελέχει καθίσας ἐπυνθάνετο πῶς ἀπέδρα τοσούτους πολεμίους.  Ἡ δὲ αὐτῷ κατέλεξε πάντα, τὸν τῶν αἰγῶν κιττόν, τὸν τῶν προβάτων ὠρυγμόν, τὴν ἐπανθήσασαν τῇ κεφαλῇ πίτυν, τὸ ἐν τῇ γῇ πῦρ, τὸν ἐν τῇ θαλάττῃ κτύπον, τὰ συρίγματα ἀμφότερα, τὸ πολεμικὸν καὶ τὸ εἰρηνικόν, τὴν νύκτα τὴν φοβεράν, ὅπως αὐτῇ τὴν ὁδὸν ἀγνοούσῃ καθηγήσατο τῆς ὁδοῦ μουσική.  Γνωρίσας οὖν ὁ Δάφνις τὰ τῶν Νυμφῶν ὀνείρατα καὶ τὰ τοῦ Πανὸς ἔργα διηγεῖται καὶ αὐτὸς ὅσα εἶδεν, ὅσα ἤκουσεν· ὅτι μέλλων ἀποθνήσκειν διὰ τὰς Νύμφας ἔζησε.  Καὶ τὴν μὲν ἀποπέμπει κομιοῦσαν τοὺς ἀμφὶ τὸν Δρύαντα καὶ Λάμωνα καὶ ὅσα πρέπει θυσίᾳ· αὐτὸς δὲ ἐν τούτῳ τῶν αἰγῶν τὴν ἀρίστην συλλαβὼν καὶ κιττῷ στεφανώσας, ὥσπερ ὤφθησαν τοῖς πολεμίοις, καὶ γάλα τῶν κεράτων κατασπείσας, ἔθυσέ τε ταῖς Νύμφαις καὶ κρεμάσας ἀπέδειρε καὶ τὸ δέρμα ἀνέθηκεν.
\pend


\pstart
2.31  Ἤδη δὲ παρόντων τῶν ἀμφὶ τὴν Χλόην, πῦρ ἀνακαύσας καὶ τὰ μὲν ἑψήσας τῶν κρεῶν τὰ δὲ ὀπτήσας ἀπήρξατό τε ταῖς Νύμφαις καὶ κρατῆρα μεστὸν γλεύκους ἐπέσπεισε· καὶ ἐκ φυλλάδος στιβάδας ὑποστορέσας ἐντεῦθεν ἐν τροφῇ ἦν καὶ πότῳ καὶ παιδιᾷ· καὶ ἅμα τὰς ἀγέλας ἐπεσκόπει, μὴ λύκος ἐμπεσὼν ἔργα ποιήσῃ πολεμίων.  ᾞσάν τινας καὶ ᾠδὰς εἰς τὰς Νύμφας, παλαιῶν ποιμένων ποιήματα. Νυκτὸς δὲ ἐπελθούσης αὐτοῦ κοιμηθέντες ἐν τῷ ἀγρῷ, τῆς ἐπιούσης τοῦ Πανὸς ἐμνημόνευσαν, καὶ τῶν τράγων τὸν ἀγελάρχην στεφανώσαντες πίτυος προσήγαγον τῇ πίτυι, καὶ ἀποσπείσαντες οἴνου καὶ εὐφημοῦντες τὸν θεὸν ἔθυσαν, ἐκρέμασαν, ἀπέδειραν·  καὶ τὰ μὲν κρέα ὀπτήσαντες καὶ ἑψήσαντες πλησίον ἔθηκαν ἐν τῷ λειμῶνι, ἐν τοῖς φύλλοις· τὸ δὲ δέρμα κέρασιν αὐτοῖς ἐνέπηξαν τῇ πίτυι πρὸς τῷ ἀγάλματι, ποιμενικὸν ἀνάθημα ποιμενικῷ θεῷ. Ἀπήρξαντο καὶ τῶν κρεῶν, ἀπέσπεισαν καὶ κρατῆρος μείζονος· ᾖσεν ἡ Χλόη, Δάφνις ἐσύρισεν.
\pend


\pstart
2.32  Ἐπὶ τούτοις κατακλινέντες ἤσθιον· καὶ αὐτοῖς ἐφίσταται Φιλητᾶς ὁ βουκόλος κατὰ τύχην στεφανίσκους τινὰς τῷ Πανὶ κομίζων καὶ βότρυς ἔτι ἐν φύλλοις καὶ κλήμασι· καὶ αὐτῷ τῶν παίδων ὁ νεώτατος εἵπετο Τίτυρος, πυρρὸν παιδίον καὶ γλαυκόν, λευκὸν παιδίον καὶ ἀγέρωχον· καὶ ἥλλετο κοῦφα, βαδίζων ὥσπερ ἔριφος.  Ἀναπηδήσαντες οὖν συνεστεφάνουν τὸν Πᾶνα καὶ τὰ κλήματα τῆς κόμης τῆς πίτυος ἐξήρτων, καὶ κατακλίναντες πλησίον αὑτῶν συμπότην ἐποιοῦντο.  Καὶ οἷα δὴ γέροντες ὑποβεβρεγμένοι πρὸς ἀλλήλους πολλὰ ἔλεγον· ὡς ἔνεμον ἡνίκα ἦσαν νέοι, ὡς πολλὰς λῃστῶν καταδρομὰς διέφυγον· ἐσεμνύνετό τις ὡς λύκον ἀποκτείνας, ἄλλος ὡς μόνου τοῦ Πανὸς δεύτερα συρίσας· τοῦτο τοῦ Φιλητᾶ τὸ σεμνολόγημα ἦν.
\pend


\pstart
2.33  Ὁ οὖν Δάφνις καὶ ἡ Χλόη πάσας δεήσεις προσέφερον μεταδοῦναι καὶ αὐτοῖς τῆς τέχνης, συρίσαι τε ἐν ἑορτῇ θεοῦ σύριγγι χαίροντος. Ἐπαγγέλλεται Φιλητᾶς, καίτοι τὸ γῆρας ὡς ἄπνουν μεμψάμενος, καὶ ἔλαβε σύριγγα τὴν τοῦ Δάφνιδος.  Ἡ δὲ ἦν μικρὰ πρὸς μεγάλην τέχνην, οἷα ἐν στόματι παιδὸς ἐμπνεομένη. Πέμπει οὖν Τίτυρον ἐπὶ τὴν ἑαυτοῦ σύριγγα,  τῆς ἐπαύλεως ἀπεχούσης σταδίους δέκα. Ὁ μὲν ῥίψας τὸ ἐγκόμβωμα γυμνὸς ὥρμησε τρέχειν, ὥσπερ νεβρός· ὁ δὲ Λάμων ἐπηγγείλατο αὐτοῖς τὸν περὶ τῆς σύριγγος ἀφηγήσεσθαι μῦθον, ὃν αὐτῷ Σικελὸς ωἰπόλος ᾖσεν ἐπὶ μισθῷ τράγῳ καὶ σύριγγι.
\pend


\pstart
2.34  “Αὕτη ἡ σύριγξ τὸ ὄργανον οὐκ ἦν ὄργανον ἀλλὰ παρθένος καλὴ καὶ τὴν φωνὴν μουσική. Αἶγας ἔνεμεν, Νύμφαις συνέπαιζεν, ᾖδεν οἷον νῦν. Πὰν ταύτης νεμούσης παιζούσης ᾀδούσης προσελθὼν ἔπειθεν ἐς ὅ τι ἔχρῃζε, καὶ ἐπηγγέλλετο τὰς αἶγας πάσας θήσειν διδυματόκους.  Ἡ δὲ ἐγέλα τὸν ἔρωτα αὐτοῦ, οὐδὲ ἐραστὴν ἔφη δέξεσθαι μήτε τράγον μήτε ἄνθρωπον ὁλόκληρον. Ὁρμᾷ διώκειν ὁ Πὰν πρὸς βίαν· ἡ Σύριγξ ἔφευγε καὶ τὸν Πᾶνα καὶ τὴν βίαν· φεύγουσα κάμνουσα ἐς δόνακας κρύπτεται, εἰς ἕλος ἀφανίζεται.  Πὰν τοὺς δόνακας ὀργῇ τεμών, τὴν κόρην οὐχ εὑρών, τὸ πάθος μαθὼν τὸ ὄργανον νοεῖ, τοὺς καλάμους κηρῷ συνδήσας ἀνίσους, καθ’ ὅ τι καὶ ὁ ἔρως ἄνισος αὐτοῖς· καὶ ἡ τότε παρθένος καλὴ νῦν ἐστι σύριγξ μουσική.”
\pend


\pstart
2.35  Ἄρτι ἐπέπαυτο τοῦ μυθολογήματος ὁ Λάμων, καὶ ἐπῄνει Φιλητᾶς αὐτὸν ὡς εἰπόντα μῦθον ᾠδῆς γλυκύτερον, καὶ ὁ Τίτυρος ἐφίσταται τὴν σύριγγα τῷ πατρὶ κομίζων, μέγα ὄργανον καὶ αὐλῶν μεγάλων,  καὶ ἵνα κεκήρωτο, χαλκῷ πεποίκιλτο. Εἴκασεν ἄν τις εἶναι ταύτην ἐκείνην, ἣν ὁ Πὰν πρώτην ἐπήξατο. Διεγερθεὶς οὖν ὁ Φιλητᾶς καὶ καθίσας ἐν καθέδρᾳ ὀρθὸς πρῶτον μὲν ἀπεπειράθη τῶν καλάμων εἰ εὖπνοι·  ἔπειτα μαθὼν ὡς ἀκώλυτον διατρέχει τὸ πνεῦμα, ἐνέπνει τὸ ἐντεῦθεν πολὺ καὶ νεανικόν. Αὐλῶν τις ἂν ᾠήθη συναυλούντων ἀκούειν· τοσοῦτον ἤχει τὸ σύριγμα. Κατ’ ὀλίγον δὲ τῆς βίας ἀφαιρῶν εἰς τὸ τερπνότερον μετέβαλλε τὸ μέλος.  Καὶ πᾶσαν τέχνην ἐπιδεικνύμενος εὐνομίας μουσικῆς ἐσύριττεν οἷον βοῶν ἀγέλῃ πρέπον, οἷον αἰπολίῳ πρόσφορον, οἷον ποίμναις φίλον. Τερπνὸν ἦν τὸ ποιμνίων, μέγα τὸ βοῶν, ὀξὺ τὸ αἰγῶν· ὅλως πάσας σύριγγας μία σύριγξ ἐμιμήσατο.
\pend


\pstart
2.36  Οἱ μὲν οὖν ἄλλοι σιωπῇ κατέκειντο τερπόμενοι· Δρύας δὲ ἀναστὰς καὶ κελεύσας συρίττειν Διονυσιακὸν μέλος, ἐπιλήνιον αὐτοῖς ὄρχησιν ὠρχήσατο· καὶ ἐῴκει ποτὲ μὲν τρυγῶντι, ποτὲ δὲ φέροντι ἀρρίχους,  εἶτα πατοῦντι τοὺς βότρυς, εἶτα πληροῦντι τοὺς πίθους, εἶτα πίνοντι τοῦ γλεύκους. Ταῦτα πάντα οὕτως εὐσχημόνως ὠρχήσατο Δρύας καὶ ἐναργῶς, ὥστε ἐδόκουν βλέπειν καὶ τὰς ἀμπέλους καὶ τὴν ληνὸν καὶ τοὺς πίθους καὶ ἀληθῶς Δρύαντα πίνοντα.
\pend


\pstart
2.37  Τρίτος δὴ γέρων οὗτος εὐδοκιμήσας ἐπ’ ὀρχήσει φιλεῖ Χλόην καὶ Δάφνιν· οἱ δὲ μάλα ταχέως ἀναστάντες ὠρχήσαντο τὸν μῦθον τοῦ Λάμωνος. Ὁ Δάφνις Πᾶνα ἐμιμεῖτο, τὴν Σύριγγα Χλόη· ὁ μὲν ἱκέτευε πείθων, ἡ δὲ ἀμελοῦσα ἐμειδία·  ὁ μὲν ἐδίωκε καὶ ἐπ’ ἄκρων τῶν ὀνύχων ἔτρεχε τὰς χηλὰς μιμούμενος, ἡ δὲ ἐνέφαινε τὴν κάμνουσαν ἐν τῇ φυγῇ. Ἔπειτα Χλόη μὲν εἰς τὴν ὕλην ὡς εἰς ἕλος κρύπτεται, Δάφνις  δὲ λαβὼν τὴν Φιλητᾶ σύριγγα τὴν μεγάλην ἐσύρισε γοερὸν ὡς ἐρῶν, ἐρωτικὸν ὡς πείθων, ἀνακλητικὸν ὡς ἐπιζητῶν, ὥστε ὁ Φιλητᾶς θαυμάσας φιλεῖ τε ἀναπηδήσας καὶ τὴν σύριγγα χαρίζεται φιλήσας καὶ εὔχεται καὶ Δάφνιν καταλιπεῖν αὐτὴν ὁμοίῳ διαδόχῳ.
\pend


\pstart
2.38  Ὁ δὲ τὴν ἰδίαν ἀναθεὶς τῷ Πανὶ τὴν σμικρὰν καὶ φιλήσας ὡς ἐκ φυγῆς ἀληθινῆς εὑρεθεῖσαν τὴν Χλόην ἀπήλαυνε τὴν ἀγέλην συρίττων νυκτὸς ἤδη ἐπιγινομένης· ἀπήλαυνε καὶ ἡ Χλόη τὴν ποίμνην τῷ μέλει τῆς σύριγγος συνᾴδουσα·  καὶ αἵ τε αἶγες πλησίον τῶν προβάτων ᾔεσαν, ὅ τε Δάφνις ἐβάδιζεν ἐγγὺς τῆς Χλόης, ὥστε ἐνέπλησαν ἕως νυκτὸς ἀλλήλους καὶ συνέθεντο θᾶττον τὰς ἀγέλας τῆς ἐπιούσης κατελάσαι·  καὶ οὕτως ἐποίησαν. Ἄρτι γοῦν ἀρχομένης ἡμέρας ἦλθον εἰς τὴν νομήν· καὶ τὰς Νύμφας προτέρας, εἶτα τὸν Πᾶνα προσαγορεύσαντες, τὸ ἐντεῦθεν ὑπὸ τῇ δρυὶ καθεσθέντες ἐσύριττον· εἶτα ἀλλήλους ἐφίλουν, περιέβαλλον, κατεκλίνοντο, καὶ οὐδὲν δράσαντες πλέον ἀνίσταντο. Ἐμέλησεν αὐτοῖς καὶ τροφῆς, καὶ ἔπιον οἶνον μίξαντες γάλα.
\pend


\pstart
2.39  Καὶ τούτοις ἅπασι θερμότεροι γενόμενοι καὶ θρασύτεροι, πρὸς ἀλλήλους ἤριζον ἔριν ἐρωτικήν, καὶ μετ’ ὀλίγον εἰς ὅρκων πίστιν προῆλθον. Ὁ μὲν δὴ Δάφνις τὸν Πᾶνα ὤμοσεν ἐλθὼν ἐπὶ τὴν πίτυν μὴ ζήσεσθαι μόνος ἄνευ Χλόης μηδὲ μιᾶς χρόνον ἡμέρας·  ἡ δὲ Χλόη τὰς Νύμφας εἰσελθοῦσα εἰς τὸ ἄντρον τὸν αὐτὸν Δάφνιδι ἕξειν καὶ θάνατον καὶ βίον. Τοσοῦτον δὲ ἄρα τῇ Χλόῃ τὸ ἀφελὲς προσῆν ὡς κόρῃ, ὥστε ἐξιοῦσα τοῦ ἄντρου καὶ δεύτερον ἠξίου λαβεῖν ὅρκον παρ’ αὐτοῦ “ὦ Δάφνι” λέγουσα “θεὸς ὁ Πὰν ἐρωτικός ἐστι καὶ ἄπιστος·  ἠράσθη μὲν Πίτυος, ἠράσθη δὲ Σύριγγος· παύεται δὲ οὐδέποτε Δρυάσιν ἐνοχλῶν καὶ Ἐπιμηλίσι Νύμφαις παρέχων πράγματα. Οὗτος μὲν οὖν ἀμεληθεὶς ἐν τοῖς ὅρκοις ἀμελήσει σε κολάσαι, κἂν ἐπὶ πλείονας ἔλθῃς γυναῖκας τῶν ἐν τῇ σύριγγι καλάμων·  σὺ δέ μοι τὸ αἰπόλιον τοῦτο ὄμοσον καὶ τὴν αἶγα ἐκείνην, ἥ σε ἀνέθρεψε, μὴ καταλιπεῖν Χλόην, ἔστ’ ἂν πιστή σοι μένῃ· ἄδικον δὲ εἰς σὲ καὶ τὰς Νύμφας γενομένην καὶ φεῦγε καὶ μίσει καὶ ἀπόκτεινον ὥσπερ λύκον.”  Ἥδετο ὁ Δάφνις ἀπιστούμενος καὶ στὰς εἰς μέσον τὸ αἰπόλιον καὶ τῇ μὲν τῶν χειρῶν αἰγὸς τῇ δὲ τράγου λαβόμενος ὤμνυε Χλόην φιλήσειν φιλοῦσαν· κἂν ἕτερον προκρίνῃ Δάφνιδος, ἀντ’ ἐκείνης αὑτὸν ἀποκτενεῖν.  Ἡ δὲ ἔχαιρε καὶ ἐπίστευεν ὡς κόρη καὶ νομίζουσα τὰς αἶγας καὶ τὰ πρόβατα ποιμένων καὶ αἰπόλων ἰδίους θεούς.
\pend


\pstart
3.1  Μυτιληναῖοι δὲ ὡς ᾔσθοντο τὸν ἐπίπλουν τῶν δέκα νεῶν καί τινες ἐμήνυσαν αὐτοῖς τὴν ἁρπαγὴν ἐλθόντες ἐκ τῶν ἀγρῶν, οὐκ ἀνασχετὸν νομίσαντες ταῦτα ἐκ Μηθυμναίων παθεῖν, ἔγνωσαν καὶ αὐτοὶ τὴν ταχίστην ἐπ’ αὐτοὺς ὅπλα κινεῖν·  καὶ καταλέξαντες ἀσπίδα τρισχιλίαν καὶ ἵππον πεντακοσίαν ἐξέπέμψαν κατὰ γῆν τὸν στρατηγὸν Ἵππασον, ὀκνοῦντες ἐν ὥρᾳ χειμῶνος τὴν θάλατταν.
\pend


\pstart
3.2  Ὁ δὲ ἐξορμηθεὶς ἀγροὺς μὲν οὐκ ἐλεηλάτει τῶν Μηθυμναίων οὐδὲ ἀγέλας καὶ κτήματα ἥρπαζε γεωργῶν καὶ ποιμένων, λῃστοῦ νομίζων ταῦτα ἔργα μᾶλλον ἢ στρατηγοῦ· ταχὺ δὲ ἐπὶ τὴν πόλιν αὐτὴν  ὡς ἐπιπεσούμενος ἀφρουρήτοις ταῖς πύλαις. Καὶ αὐτῷ σταδίους ὅσον ἑκατὸν ἀπέχοντι κῆρυξ ἀπαντᾷ σπονδὰς κομίζων.  Οἱ γὰρ Μηθυμναῖοι μαθόντες παρὰ τῶν ἑαλωκότων ὡς οὐδὲν ἴσασι Μυτιληναῖοι τῶν γεγενημένων, ἀλλὰ γεωργοὶ καὶ ποιμένες ὑβρίζοντας τοὺς νεανίσκους ταῦτα ἔδρασαν, μετεγίνωσκον μὲν ὀξύτερα τολμήσαντες εἰς γείτονα πόλιν ἢ σωφρονέστερα, σπουδὴν δὲ εἶχον ἀποδόντες πᾶσαν τὴν ἁρπαγὴν ἀδεῶς ἐπιμίγνυσθαι καὶ κατὰ γῆν καὶ κατὰ θάλατταν.  Τὸν μὲν οὖν κήρυκα τοῖς Μυτιληναίοις ὁ Ἵππασος ἀποστέλλει, καίτοιγε αὐτοκράτωρ στρατηγὸς κεχειροτονημένος· αὐτὸς δὲ τῆς Μηθύμνης ὅσον ἀπὸ δέκα σταδίων στρατόπεδον βαλόμενος τὰς ἐκ τῆς πόλεως ἐντολὰς ἀνέμενε.  Καὶ δύο διαγενομένων ἡμερῶν ἐλθὼν ὁ ἄγγελος τήν τε ἁρπαγὴν ἐκέλευσε κομίσασθαι καὶ ἀδικήσαντα μηδὲν ἀναχωρεῖν οἴκαδε· πολέμου γὰρ καὶ εἰρήνης ἐν αἱρέσει γενόμενοι, τὴν εἰρήνην εὕρισκον κερδαλεωτέραν.
\pend


\pstart
3.3  Ὁ μὲν δὴ Μηθυμναίων καὶ Μυτιληναίων πόλεμος ἀδόκητον λαβὼν ἀρχὴν καὶ τέλος οὕτω διελύθη. Γίνεται δὲ χειμὼν Δάφνιδι καὶ Χλόῃ τοῦ πολέμου πικρότερος· ἐξαίφνης γὰρ περιπεσοῦσα χιὼν πολλὴ πάσας μὲν ἀπέκλεισε τὰς ὁδούς, πάντας δὲ κατέκλεισε τοὺς γεωργούς.  Λάβροι μὲν οἱ χείμαρροι κατέρρεον, ἐπεπήγει δὲ κρύσταλλος· τὰ δένδρα ἐῴκει κατεσκελετευμένοις· ἡ γῆ πᾶσα ἀφανὴς ἦν ὅτι μὴ περὶ πηγάς που καὶ ῥεύματα.  Οὔτ’ οὖν ἀγέλην τις ἐς νομὴν ἦγεν οὔτε αὐτὸς προῄει τῶν θυρῶν, ἀλλὰ πῦρ καύσαντες μέγα περὶ ᾠδὰς ἀλεκτρυόνων οἱ μὲν λίνον ἔστρεφον, οἱ δὲ αἰγῶν τρίχας ἔπλεκον, οἱ δὲ πάγας ὀρνίθων ἐσοφίζοντο.  Τότε βοῶν ἐπὶ φάτναις φροντὶς ἦν ἄχυρον ἐσθιόντων, αἰγῶν καὶ προβάτων ἐν τοῖς σηκοῖς φυλλάδας, ὑῶν ἐν τοῖς συφεοῖς ἄκυλον καὶ βαλάνους.
\pend


\pstart
3.4  Ἀναγκαίας οὖν οἰκουρίας ἐπεχούσης ἅπαντας οἱ μὲν ἄλλοι γεωργοὶ καὶ νομεῖς ἔχαιρον πόνων τε ἀπηλλαγμένοι πρὸς ὀλίγον καὶ τροφὰς ἑωθινὰς ἐσθίοντες καὶ καθεύδοντες μακρὸν ὕπνον, ὥστε αὐτοῖς τὸν χειμῶνα δοκεῖν καὶ θέρους καὶ μετοπώρου καὶ ἦρος αὐτοῦ γλυκύτερον.  Χλόη δὲ καὶ Δάφνις ἐν μνήμῃ γενόμενοι τῶν καταλειφθέντων τερπνῶν, ὡς ἐφίλουν, ὡς περιέβαλλον, ὡς ἅμα τὴν τροφὴν προσεφέροντο, νύκτας τε ἀγρύπνους διῆγον καὶ λυπηρὰς καὶ τὴν ἠρινὴν ὥραν ἀνέμενον ἐκ θανάτου παλιγγενεσίαν.  Ἐλύπει δὲ αὐτοὺς ἢ πήρα τις ἐλθοῦσα εἰς χεῖρας, ἐξ ἧς συνήσθιον, ἢ γαυλὸς ὀφθείς, ἐξ οὗ συνέπιον, ἢ σύριγξ ἀμελῶς ἐρριμμένη, δῶρον ἐρωτικὸν γεγενημένη.  Εὔχοντο δὴ ταῖς Νύμφαις καὶ τῷ Πανὶ καὶ τούτων αὐτοὺς ἐκλύσασθαι τῶν κακῶν καὶ δεῖξαί ποτε αὐτοῖς καὶ ταῖς ἀγέλαις ἥλιον· ἅμα τε εὐχόμενοι τέχνην ἐζήτουν,  δι’ ἧς ἀλλήλους θεάσονται. Ἡ μὲν δὴ Χλόη δεινῶς ἄπορος ἦν καὶ ἀμήχανος· ἀεὶ γὰρ αὐτῇ συνῆν ἡ δοκοῦσα μήτηρ ἔριά τε ξαίνειν διδάσκουσα καὶ ἀτράκτους στρέφειν καὶ γάμου μνημονεύουσα· ὁ δὲ Δάφνις, οἷα σχολὴν ἄγων καὶ συνετώτερος κόρης, τοιόνδε σόφισμα εὗρεν ἐς θέαν τῆς Χλόης.
\pend


\pstart
3.5  Πρὸ τῆς αὐλῆς τοῦ Δρύαντος, ὑπ’ αὐτῇ τῇ αὐλῇ, μυρρίναι μεγάλαι δύο καὶ κιττὸς ἐπεφύκει· αἱ μυρρίναι πλησίον ἀλλήλων, ὁ κιττὸς ἀμφοτέρων μέσος, ὥστε ἐφ’ ἑκατέραν διαθεὶς τοὺς ἀκρεμόνας ὡς ἄμπελος ἄντρου σχῆμα διὰ τῶν φύλλων ἐπαλλαττόντων ἐποίει· καὶ ὁ κόρυμβος πολὺς καὶ μέγας ὅσος βότρυς κλημάτων ἐξεκρέματο.  Ἦν οὖν πολὺ πλῆθος περὶ αὐτὸν τῶν χειμερινῶν ὀρνίθων ἀπορίᾳ τῆς ἔξω τροφῆς· πολὺς μὲν κόψιχος, πολλὴ δὲ κίχλη, καὶ φάτται καὶ ψᾶρες καὶ ὅσον ἄλλο κιττοφάγον πτερόν.  Τούτων τῶν ὀρνίθων ἐπὶ προφάσει θήρας ἐξώρμησεν ὁ Δάφνις, ἐμπλήσας μὲν τὴν πήραν ὀψημάτων μεμελιτωμένων, κομίζων δὲ ἐς πίστιν ἰξὸν καὶ βρόχους.  Τὸ μὲν οὖν μεταξὺ σταδίων ἦν οὐ πλέον δέκα· οὔπω δὲ ἡ χιὼν λελυμένη πολὺν αὐτῷ κάματον παρέσχεν· ἔρωτι δὲ ἄρα πάντα βάσιμα καὶ πῦρ καὶ ὕδωρ καὶ Σκυθικὴ χιών.
\pend


\pstart
3.6  Δρόμῳ οὖν πρὸς τὴν αὐλὴν ἔρχεται καὶ ἀποσεισάμενος τῶν σκελῶν τὴν χιόνα τούς τε βρόχους ἔστησε καὶ τὸν ἰξὸν ῥάβδοις μακραῖς ἐπήλειψε· καὶ ἐκαθέζετο τὸ ἐντεῦθεν ὄρνιθας καὶ τὴν Χλόην μεριμνῶν.  Ἀλλ’ ὄρνιθες μὲν καὶ ἧκον πολλοὶ καὶ ἐλήφθησαν ἱκανοί, ὥστε πράγματα μυρία ἔσχε συλλέγων αὐτοὺς καὶ ἀποκτιννὺς καὶ ἀποδύων τὰ πτερά· τῆς δὲ αὐλῆς προῆλθεν οὐδείς, οὐκ ἀνήρ, οὐ γύναιον, οὐ κατοικίδιος ὄρνις, ἀλλὰ πάντες τῷ πυρὶ παραμένοντες ἔνδον κατεκέκλειντο, ὥστε πάνυ ἠπορεῖτο ὁ Δάφνις ὡς οὐκ αἰσίοις ὄρνισιν ἐλθών· καὶ ἐτόλμα πρόφασιν σκηψάμενος ὤσασθαι διὰ θυρῶν καὶ ἐζήτει πρὸς αὑτὸν ὅ τι λεχθῆναι πιθανώτατον.  “Πῦρ ἐναυσόμενος ἦλθον.ʼ ʽΜὴ γὰρ οὐκ ἦσαν ἀπὸ σταδίου γείτονες;ʼ ʽἌρτους αἰτησόμενος ἧκον.ʼ ʽἈλλ’ ἡ πήρα μεστὴ τροφῆς.ʼ ʽΟἴνου δέομαι.ʼ ʽΚαὶ μὴν χθὲς καὶ πρώην ἐτρύγησας.ʼ ʽΛύκος με ἐδίωκε.ʼ ʽΚαὶ ποῦ τὰ ἴχνη τοῦ λύκου;ʼ ʽΘηράσων ἀφικόμην τοὺς ὄρνιθας.ʼ ʽΤί οὖν  θηράσας οὐκ ἄπει;ʼ ʽΧλόην θεάσασθαι βούλομαι.” Πατρὶ δὲ τίς καὶ μητρὶ παρθένου τοῦτο ὁμολογεῖ; Πταίων δὴ πανταχοῦ “ἀλλ’ οὐδὲν” “τούτων ἁπάντων ἀνύποπτον. Ἄμεινον ἄρα σιγᾶν· Χλόην δὲ ἦρος ὄψομαι, ἐπεὶ μὴ εἵμαρτο, ὡς ἔοικε, χειμῶνός με  ταύτην ἰδεῖν.” Τοιαῦτα δή τινα διανοηθεὶς καὶ σιωπῇ τὰ θηραθέντα συλλαβὼν ὥρμητο ἀπιέναι· καὶ ὥσπερ αὐτὸν οἰκτείραντος τοῦ Ἔρωτος τάδε γίνεται.
\pend


\pstart
3.7  Περὶ τράπεζαν εἶχον οἱ ἀμφὶ τὸν Δρύαντα· κρέα διῃρεῖτο, ἄρτοι παρετίθεντο, κρατὴρ ἐκίρνατο. Εἷς δὴ κύων τῶν προβατευτικῶν ἀμέλειαν φυλάξας,  κρέας ἁρπάσας ἔφυγε διὰ θυρῶν. Ἀλγήσας ὁ Δρύας (καὶ γὰρ ἦν ἐκείνου μοῖρα) ξύλον ἀράμενος ἐδίωκε κατ’ ἴχνος ὥσπερ κύων· διώκων δὲ κατὰ τὸν κιττὸν γενόμενος ὁρᾷ τὸν Δάφνιν ἀνατεθειμένον ἐπὶ τοὺς ὤμους τὴν ἄγραν καὶ ἀποσοβεῖν ἐγνωκότα.  Κρέως μὲν οὖν καὶ κυνὸς αὐτίκα ἐπελάθετο, μέγα δὲ βοήσας “χαῖρε, ὦ παῖ” περιεπλέκετο καὶ κατεφίλει καὶ ἦγεν ἔσω λαβόμενος. Μικροῦ μὲν οὖν ἰδόντες ἀλλήλους εἰς τὴν γῆν κατερρύησαν· μεῖναι δὲ καρτερήσαντες ὀρθοὶ προσηγόρευσάν τε καὶ κατεφίλησαν· καὶ τοῦτο οἱονεὶ ἔρεισμα αὐτοῖς τοῦ μὴ πεσεῖν ἐγένετο.
\pend


\pstart
3.8  Τυχὼν οὖν ὁ Δάφνις παρ’ ἐλπίδας καὶ φιλήματος καὶ Χλόης τοῦ τε πυρὸς ἐκαθέσθη πλησίον καὶ ἐπὶ τὴν τράπεζαν ἀπὸ τῶν ὤμων τὰς φάττας ἀπεφορτίσατο καὶ τοὺς κοψίχους, καὶ διηγεῖτο πῶς ἀσχάλλων πρὸς τὴν οἰκουρίαν ὥρμησε πρὸς ἄγραν, καὶ ὅπως τὰ μὲν αὐτῶν βρόχοις τὰ δὲ ἰξῷ λάβοι τῶν μύρτων καὶ τοῦ κιττοῦ γλιχόμενα.  Οἱ δὲ ἐπῄνουν τὸ ἐνεργὸν καὶ ἐκέλευον ἐσθίειν ὧν ὁ κύων κατέλιπεν, ἐκέλευον δὲ καὶ τῇ Χλόῃ πιεῖν ἐγχέαι. Καὶ ἣ χαίρουσα τοῖς τε ἄλλοις ὤρεξε καὶ Δάφνιδι μετὰ τοὺς ἄλλους· ἐσκήπτετο γὰρ ὀργίζεσθαι, διότι ἐλθὼν ἔμελλεν ἀποτρέχειν οὐκ ἰδών· ὅμως μέντοι πρὶν προσενεγκεῖν ἀπέπιεν, εἶθ’ οὕτως ἔδωκεν. Ὁ δέ, καίτοι διψῶν, βραδέως ἔπινε, παρέχων ἑαυτῷ διὰ τῆς βραδυτῆτος μακροτέραν ἡδονήν.
\pend


\pstart
3.9  Ἡ μὲν δὴ τράπεζα ταχέως ἐγένετο κενὴ ἄρτων καὶ κρεῶν· καθήμενοι δὲ περὶ τῆς Μυρτάλης καὶ τοῦ Λάμωνος ἐπυνθάνοντο καὶ εὐδαιμόνιζον αὐτοὺς τοιούτου γηροτρόφου εὐτυχήσαντας.  Καὶ τοῖς ἐπαίνοις μὲν ἥδετο Χλόης ἀκροωμένης· ὅτε δὲ κατεῖχον αὐτόν, ὡς θύσοντες Διονύσῳ τῆς ἐπιούσης ἡμέρας, μικροῦ δεῖν ὑφ’ ἡδονῆς ἐκείνους ἀντὶ τοῦ Διονύσου προσεκύνησεν.  Αὐτίκα οὖν ἐκ τῆς πήρας προυκόμιζε μελιτώματα πολλὰ καὶ τοὺς θηραθέντας δὲ τῶν ὀρνίθων· καὶ τούτους ἐς τράπεζαν νυκτερινὴν ηὐτρέπιζον. Δεύτερος κρατὴρ ἵστατο καὶ δεύτερον πῦρ ἀνεκάετο.  Καὶ ταχὺ μάλα νυκτὸς γενομένης, δευτέρας τραπέζης ἐνεφοροῦντο, μεθ’ ἣν τὰ μὲν μυθολογήσαντες τὰ δὲ ᾅσαντες εἰς ὕπνον ἐχώρουν, Χλόη μετὰ τῆς μητρός, Δρύας ἅμα Δάφνιδι.  Χλόῃ μὲν οὖν οὐδὲν χρηστὸν ἦν ὅτι μὴ τῆς ἐπιούσης ἡμέρας ὀφθησόμενος ὁ Δάφνις· Δάφνις δὲ κενὴν τέρψιν ἐτέρπετο· τερπνὸν γὰρ ἐνόμιζε καὶ πατρὶ συγκοιμηθῆναι Χλόης· ὥστε καὶ περιέβαλλεν αὐτὸν καὶ κατεφίλει πολλάκις, ταῦτα πάντα ποιεῖν Χλόην ὀνειροπολούμενος.
\pend


\pstart
3.10  Ὡς δὲ ἐγένετο ἡμέρα, κρύος μὲν ἦν ἐξαίσιον καὶ αὖρα βόρειος ἀπέκαε πάντα. Οἱ δὲ ἀναστάντες θύουσι τῷ Διονύσῳ κριὸν ἐνιαύσιον καὶ πῦρ ἀνακαύσαντες μέγα παρεσκευάζοντο τροφήν.  Τῆς οὖν Νάπης ἀρτοποιούσης καὶ τοῦ Δρύαντος τὸν κριὸν ἕψοντος, σχολῆς ὁ Δάφνις καὶ ἡ Χλόη λαβόμενοι προῆλθον τῆς αὐλῆς ἵνα ὁ κιττός· καὶ πάλιν βρόχους στήσαντες καὶ ἰξὸν ἐπαλείψαντες ἐθήρων πλῆθος οὐκ ὀλίγον ὀρνίθων.  Ἦν δ’ αὐτοῖς καὶ φιλημάτων ἀπόλαυσις συνεχὴς καὶ λόγων ὁμιλία τερπνή. “Διὰ σὲ ἦλθον, Χλόη.” “Οἶδα, Δάφνι.” “Διὰ σὲ ἀπολλύω τοὺς ἀθλίους κοψίχους.” “Τί οὖν σοι γένωμαι;” “Μέμνησό μου.” “Μνημονεύω, νὴ τὰς Νύμφας, ἃς ὤμοσά ποτε εἰς ἐκεῖνο τὸ ἄντρον , εἰς ὃ ἥξομεν εὐθύς, ἂν ἡ χιὼν τακῇ.”  “Ἀλλὰ πολλή ἐστι, Χλόη, καὶ δέδοικα μὴ ἐγὼ πρὸ ταύτης τακῶ.” “Θάρρει, Δάφνι, θερμός ἐστιν ὁ ἥλιος.” “Εἰ γὰρ οὕτως γένοιτο. Χλόη, θερμός, ὡς τὸ κᾶον πῦρ τὴν καρδίαν τὴν ἐμήν.” “Παίζεις ἀπατῶν με.” “Οὐ μὰ τὰς αἶγας, ἃς σύ με ἐκέλευες ὀμνύειν.”
\pend


\pstart
3.11  Τοιαῦτα ἀντιφωνήσασα πρὸς τὸν Δάφνιν ἡ Χλόη καθάπερ ἠχώ, καλούντων αὐτοὺς τῶν περὶ τὴν Νάπην εἰσέδραμον, πολὺ περιττοτέραν τῆς χθιζῆς θήραν κομίζοντες· καὶ ἀπαρξάμενοι τῷ Διονύσῳ κρατῆρος ἤσθιον κιττῷ τὰς κεφαλὰς ἐστεφανωμένοι.  Καὶ ἐπεὶ καιρὸς ἦν, ἰακχάσαντες καὶ εὐάσαντες προύπεμπον τὸν Δάφνιν, πλήσαντες αὐτοῦ τὴν πήραν κρεῶν καὶ ἄρτων. Ἔδωκαν δὲ καὶ τὰς φάττας καὶ τὰς κίχλας Λάμωνι καὶ Μυρτάλῃ κομίζειν, ὡς αὐτοὶ θηράσοντες ἄλλας, ἔστ’ ἂν ὁ χειμὼν μένῃ καὶ ὁ κιττὸς μὴ λείπῃ.  Ὁ δὲ ἀπῄει, φιλήσας αὐτοὺς προτέρους Χλόης, ἵνα τὸ ἐκείνης φίλημα καθαρὸν μείνῃ. Καὶ ἄλλας δὲ πολλὰς ἦλθεν ὁδοὺς ἐπ’ ἄλλαις τέχναις, ὥστε μὴ παντάπασιν αὐτοῖς γενέσθαι τὸν χειμῶνα ἀνέραστον.
\pend


\pstart
3.12  Ἤδη δὲ ἦρος ἀρχομένου καὶ τῆς μὲν χιόνος λυομένης, τῆς δὲ γῆς γυμνουμένης καὶ τῆς πόας ὑπανθούσης οἵ τε ἄλλοι νομεῖς ἦγον τὰς ἀγέλας εἰς νομὴν καὶ πρὸ τῶν ἄλλων Χλόη καὶ Δάφνις, οἷα μείζονι δουλεύοντες ποιμένι.  Εὐθὺς οὖν δρόμος ἦν ἐπὶ τὰς Νύμφας καὶ τὸ ἄντρον, ἐντεῦθεν ἐπὶ τὸν Πᾶνα καὶ τὴν πίτυν, εἶτα ἐπὶ τὴν δρῦν, ὑφ’ ἣν καθίζοντες καὶ τὰς ἀγέλας ἔνεμον καὶ ἀλλήλους κατεφίλουν. Ἀνεζήτησαν δὲ καὶ ἄνθη στεφανῶσαι θέλοντες τοὺς θεούς· τὰ δὲ ἄρτι ὁ ζέφυρος τρέφων καὶ ὁ ἥλιος θερμαίνων ἐξῆγεν· ὅμως δὲ εὑρέθη καὶ ἴα καὶ νάρκισσος καὶ ἀναγαλλὶς καὶ ὅσα ἦρος πρωτοφορήματα.  Ἡ μὲν Χλόη καὶ ὁ Δάφνις ἀπὸ αἰγῶν καὶ ἀπὸ οἰῶν τινων γάλα νέον καὶ τοῦτο στεφανοῦντες τὰ ἀγάλματα κατέσπεισαν.  Ἀπήρξαντο καὶ σύριγγος, καθάπερ τὰς ἀηδόνας ἐς τὴν μουσικὴν ἐρεθίζοντες· αἱ δ’ ὑπεφθέγγοντο ἐν ταῖς λόχμαις καὶ τὸν Ἴτυν κατ’ ὀλίγον ἠκρίβουν, ὥσπερ ἀναμιμνησκόμεναι τῆς ᾠδῆς ἐκ μακρᾶς σιωπῆς.
\pend


\pstart
3.13  Ἐβληχήσατό που καὶ ποίμνιον· ἐσκίρτησάν που καὶ ἄρνες καὶ ταῖς μητράσιν ὑποκλάσαντες τὴν θηλὴν ἔσπασαν· τὰς δὲ μήπω τετοκυίας οἱ κριοὶ καταδιώκοντες καὶ κάτω στήσαντες ἔβαινον ἄλλος ἄλλην.  Ἐγίνοντο καὶ τράγων διώγματα καὶ ἐς τὰς αἶγας ἐρωτικώτερα πηδήματα, καὶ ἐμάχοντο περὶ τῶν αἰγῶν· καὶ ἕκαστος εἶχεν ἰδίας καὶ ἐφύλαττε μή τις αὐτὰς μοιχεύσῃ λαθών.  Καὶ γέροντας ὁρῶντας ἐξώρμησεν εἰς ἀφροδίτην τὰ τοιαῦτα θεάματα· οἱ δὲ καὶ νέοι καὶ σφριγῶντες καὶ πολὺν ἤδη χρόνον ἔρωτα ζητοῦντες ἐξεκάοντο πρὸς τὰ ἀκούσματα καὶ ἐτήκοντο πρὸς τὰ θεάματα καὶ ἐζήτουν καὶ αὐτοὶ περιττότερόν τι φιλήματος καὶ περιβολῆς, μάλιστα δὲ ὁ Δάφνις.  Οἷα γοῦν ἐνηβήσας τῇ κατὰ τὸν χειμῶνα οἰκουρίᾳ καὶ εὐσχολίᾳ πρός τε τὰ φιλήματα ὤργα καὶ πρὸς τὰς περιβολὰς ἐσκιτάλιζε καὶ ἦν ἐς πᾶν ἔργον περιεργότερος καὶ θρασύτερος.
\pend


\pstart
3.14  ᾜτει δὴ τὴν Χλόην χαρίσασθαί οἱ πᾶν ὅσον βούλεται καὶ γυμνὴν γυμνῷ συγκατακλινῆναι μακρότερον ἢ πρόσθεν εἰώθεσαν· τοῦτο γὰρ λείπειν τοῖς Φιλητᾶ παιδεύμασιν, ἵνα δὴ γένηται τὸ μόνον ἔρωτα παῦον φάρμακον.  Τῆς δὲ πυνθανομένης τί πλέον ἐστὶ φιλήματος καὶ περιβολῆς καὶ αὐτῆς κατακλίσεως καὶ τί ἔγνωκε δρᾶσαι γυμνὸς γυμνῇ συγκατακλινείς, “τοῦτο” εἶπεν “ὃ οἱ κριοὶ ποιοῦσι τὰς οἶς καὶ οἱ τράγοι τὰς αἶγας.  Ὁρᾷς ὡς μετὰ τοῦτο τὸ ἔργον οὔτε ἐκεῖναι φεύγουσιν ἔτι αὐτοὺς οὔτε ἐκεῖνοι κάμνουσι διώκοντες, ἀλλ’ ὥσπερ κοινῆς λοιπὸν ἀπολαύσαντες ἡδονῆς συννέμονται; Γλυκύ τι, ὡς ἔοικεν, ἐστὶ τὸ ἔργον καὶ νικᾷ τὸ ἔρωτος πικρόν.” “Εἶτα οὐχ ὁρᾷς,  ὦ Δάφνι, τὰς αἶγας καὶ τοὺς τράγους καὶ τοὺς κριοὺς καὶ τὰς οἶς ὡς ὀρθοὶ μὲν ἐκεῖνοι δρῶσιν, ὀρθαὶ δὲ ἐκεῖναι πάσχουσιν, οἱ μὲν ἐπιπηδήσαντες, αἱ δὲ κατανωτισάμεναι; Σὺ δέ με ἀξιοῖς συγκατακλινῆναι καὶ ταῦτα γυμνήν; Καίτοιγε ἐκεῖναι πόσον ἐκδεδυμένης ἐμοῦ  λασιώτεραι;” Πείθεται Δάφνις καὶ συγκατακλινεὶς αὐτῇ πολὺν χρόνον ἔκειτο καὶ οὐδὲν ὧν ἕνεκα ὤργα ποιεῖν ἐπιστάμενος ἀνίστησιν αὐτὴν καὶ κατόπιν περιεφύετο μιμούμενος τοὺς τράγους. Πολὺ δὲ μᾶλλον ἀπορηθείς, καθίσας ἔκλαεν εἰ καὶ κριῶν ἀμαθέστερος εἰς τὰ ἔρωτος ἔργα.
\pend


\pstart
3.15  Ἦν δέ τις αὐτῷ γείτων, γεωργὸς γῆς ἰδίας, Χρῶμις τοὔνομα, παρηβῶν ἤδη τὸ σῶμα. Τούτῳ γύναιον ἦν ἐπακτὸν ἐξ ἄστεος, νέον καὶ ὡραῖον καὶ ἀγροικίας ἁβρότερον·  τούτῳ Λυκαίνιον ὄνομα ἦν. Αὕτη ἡ Λυκαίνιον ὁρῶσα τὸν Δάφνιν καθ’ ἑκάστην ἡμέραν παρελαύνοντα τὰς αἶγας ἕωθεν εἰς νομήν, νύκτωρ ἐκ νομῆς, ἐπεθύμησεν ἐραστὴν κτήσασθαι δώροις δελεάσασα.  Καὶ δή ποτε λοχήσασα μόνον καὶ σύριγγα δῶρον ἔδωκε καὶ μέλι ἐν κηρίῳ καὶ πήραν ἐλάφου· εἰπεῖν δέ τι ὤκνει, τὸν Χλόης ἔρωτα καταμαντευομένη·  πάνυ γὰρ ἑώρα προσκείμενον αὐτὸν τῇ κόρῃ. Πρότερον μὲν οὖν ἐκ νευμάτων καὶ γέλωτος συνεβάλετο τοῦτο, τότε δὲ ἐξ ἑωθινοῦ σκηψαμένη πρὸς Χρῶμιν ὡς παρὰ τίκτουσαν ἄπεισι γείτονα κατόπιν τε αὐτοῖς παρηκολούθησε καὶ εἴς τινα λόχμην ἐγκρύψασα ἑαυτήν, ὡς μὴ βλέποιτο, πάντα ἤκουσεν ὅσα εἶπον, πάντα εἶδεν ὅσα ἔπραξαν· οὐκ ἔλαθεν αὐτὴν οὐδὲ κλαύσας ὁ Δάφνις.  Συναλγήσασα δὴ τοῖς ἀθλίοις καὶ καιρὸν ἥκειν νομίσασα διττόν, τὸν μὲν εἰς τὴν ἐκείνων σωτηρίαν, τὸν δὲ εἰς τὴν ἑαυτῆς ἐπιθυμίαν, ἐπιτεχνᾶταί τι τοιόνδε.
\pend


\pstart
3.16  Τῆς ἐπιούσης ὡς πάλιν παρὰ τὴν γυναῖκα τὴν τίκτουσαν ἀπιοῦσα φανερῶς ἐπὶ τὴν δρῦν, ἔνθα ἐκαθέζοντο Δάφνις καὶ Χλόη, παραγίνεται καὶ ἀκριβῶς  μιμησαμένη τὴν τεταραγμένην “σῶσόν με” εἶπε “Δάφνι, τὴν ἀθλίαν· ἐκ γάρ μοι τῶν χηνῶν τῶν εἴκοσιν ἕνα τὸν κάλλιστον ἀετὸς ἥρπασε καὶ οἷα μέγα φορτίον ἀράμενος οὐκ ἐδυνήθη μετέωρος ἐπὶ τὴν συνήθη τὴν ὑψηλὴν κομίσαι ἐκείνην πέτραν, ἀλλ’ εἰς τήνδε τὴν ὕλην τὴν ταπεινὴν ἔχων κατέπεσε.  Σὺ τοίνυν, πρὸς τῶν Νυμφῶν καὶ τοῦ Πανὸς ἐκείνου, συνεισελθὼν εἰς τὴν ὕλην (μόνη γὰρ δέδοικα) σῶσόν μοι τὸν χῆνα, μηδὲ περιίδῃς ἀτελῆ μοι τὸν ἀριθμὸν γενόμενον.  Τάχα δὲ καὶ αὐτὸν τὸν ἀετὸν ἀποκτενεῖς καὶ οὐκέτι πολλοὺς ὑμῶν ἄρνας καὶ ἐρίφους ἁρπάσει. Τὴν δὲ ἀγέλην τέως φρουρήσει Χλόη· πάντως αὐτὴν ἴσασιν αἱ αἶγες ἀεί σοι συννέμουσαν.”
\pend


\pstart
3.17  Οὐδὲν τῶν μελλόντων ὑποπτεύσας ὁ Δάφνις εὐθὺς ἀνίσταται καὶ ἀράμενος τὴν καλαύροπα κατόπιν ἠκολούθει τῇ Λυκαινίῳ· ἡ δὲ ἡγεῖτο ὡς μακροτάτω τῆς Χλόης. Καὶ ἐπειδὴ κατὰ τὸ πυκνότατον ἐγένοντο, πηγῆς πλησίον καθίσαι κελεύσασα αὐτὸν “ἐρᾷσ” εἶπε “Δάφνι, Χλόης· τοῦτο ἔμαθον ἐγὼ νύκτωρ παρὰ τῶν Νυμφῶν.  δι’ ὀνείρατος ἐμοὶ τὰ χθιζά σου διηγήσαντο δάκρυα καὶ ἐκέλευσάν σε σῶσαι διδαξαμένην τὰ ἔρωτος ἔργα. Τὰ δ’ ἐστὶν οὐ φιλήματα καὶ περιβολαὶ καὶ οἷα δρῶσι κριοὶ καὶ τράγοι· ἄλλα ταῦτα πηδήματα καὶ τῶν ἐκεῖ γλυκύτερα· πρόσεστι γὰρ αὐτοῖς χρόνος μακροτέρας ἡδονῆς.  Εἰ δή σοι φίλον ἀπηλλάχθαι κακῶν καὶ ἐν πείρᾳ γενέσθαι ζητουμένων τερπνῶν, ἴθι, παραδίδου μοι τερπνὸν σαυτὸν μαθητήν· ἐγὼ δὲ χαριζομένη ταῖς Νύμφαις ἐκεῖνα διδάξω.”
\pend


\pstart
3.18  Οὐκ ἐκαρτέρησεν ὁ Δάφνις ὑφ’ ἡδονῆς, ἀλλ’ ἅτε ἄγροικος καὶ αἰπόλος καὶ ἐρῶν καὶ νέος, πρὸ τῶν ποδῶν καταπεσὼν τὴν Λυκαίνιον ἱκέτευεν ὅτι τάχιστα διδάξαι τὴν τέχνην, δι’ ἧς ὃ βούλεται δράσει Χλόην·  καὶ ὥσπερ τι μέγα καὶ θεόπεμπτον ἀληθῶς μέλλων διδάσκεσθαι καὶ ἔριφον αὐτῇ σηκίτην δώσειν ἐπηγγείλατο καὶ τυροὺς ἁπαλοὺς πρωτορρύτου γάλακτος καὶ τὴν αἶγα αὐτήν.  Εὑροῦσα δὴ ἡ Λυκαίνιον αἰπολικὴν ἀφθονίαν, οἵαν οὐ προσεδόκησεν, ἤρχετο παιδεύειν τὸν Δάφνιν τοῦτον τὸν τρόπον. Ἐκέλευσεν αὐτὸν καθίσαι πλησίον αὐτῆς, ὡς εἶχε, καὶ φιλήματα φιλεῖν οἷα εἰώθει καὶ ὅσα, καὶ φιλοῦντα ἅμα περιβάλλειν καὶ κατακλίνεσθαι χαμαί.  Ὡς δὲ ἐκαθέσθη καὶ ἐφίλησε καὶ κατεκλίνη, μαθοῦσα ἐνεργεῖν δυνάμενον καὶ σφριγῶντα, ἀπὸ μὲν τῆς ἐπὶ πλευρὰν κατακλίσεως ἀνίστησιν, αὑτὴν δὲ ὑποστορέσασα ἐντέχνως ἐς τὴν τέως ζητουμένην ὁδὸν ἦγε. Τὸ δὲ ἐντεῦθεν οὐδὲν περιειργάζετο ξένον· αὐτὴ γὰρ ἡ φύσις λοιπὸν ἐπαίδευσε τὸ πρακτέον.
\pend


\pstart
3.19  Τελεσθείσης δὲ τῆς ἐρωτικῆς παιδαγωγίας ὁ μὲν Δάφνις ἔτι ποιμενικὴν γνώμην ἔχων ὥρμητο τρέχειν ἐπὶ τὴν Χλόην καὶ ὅσα ἐπεπαίδευτο δρᾶν αὐτίκα, καθάπερ δεδοικὼς μὴ βραδύνας ἐπιλάθοιτο· ἡ δὲ Λυκαίνιον κατασχοῦσα αὐτὸν ἔλεξεν ὧδε “ἔτι καὶ ταῦτά σε δεῖ μαθεῖν, Δάφνι.  Ἐγὼ γυνὴ τυγχάνουσα πέπονθα νῦν οὐδέν· πάλαι γάρ με ταῦτα ἀνὴρ ἄλλος ἐπαίδευσε, μισθὸν τὴν παρθενίαν λαβών· Χλόη δὲ συμπαλαίουσά σοι ταύτην τὴν πάλην καὶ οἰμώξεται καὶ κλαύσεται καὶ αἵματι ῥεύσεται πολλῷ καθάπερ πεφονευμένη.  Ἀλλὰ σὺ τὸ αἷμα μὴ φοβηθῇς, ἀλλ’ ἡνίκα ἂν πείσῃς αὐτήν σοι παρασχεῖν, ἄγαγε αὐτὴν εἰς τοῦτο τὸ χωρίον, ἵνα, κἂν βοήσῃ, μηδεὶς ἀκούσῃ, κἂν δακρύσῃ, μηδεὶς ἴδῃ, κἂν αἱμαχθῇ, λούσηται τῇ πηγῇ· καὶ μέμνησο ὅτι σε ἄνδρα ἐγὼ πρὸ Χλόης πεποίηκα.”
\pend


\pstart
3.20  Ἡ μὲν οὖν Λυκαίνιον τοσαῦτα ὑποθεμένη κατ’ ἄλλο μέρος τῆς ὕλης ἀπῆλθεν, ὡς ἔτι ζητοῦσα τὸν χῆνα· ὁ δὲ Δάφνις εἰς λογισμὸν ἄγων τὰ εἰρημένα, τῆς μὲν πρότερον ὁρμῆς ἀπήλλακτο, διοχλεῖν δὲ τῇ Χλόῃ περιττότερον ὤκνει φιλήματος καὶ περιβολῆς, μήτε βοῆσαι θέλων αὐτὴν ὡς πρὸς πολέμιον, μήτε δακρῦσαι ὡς ἀλγοῦσαν, μήτε αἱμαχθῆναι καθάπερ πεφονευμένην·  ἀρτιμαθὴς γὰρ ὢν ἐδεδοίκει τὸ αἷμα καὶ ἐνόμιζεν ὅτι ἄρα ἐκ μόνου τραύματος αἷμα γίνεται. Γνοὺς δὲ τὰ συνήθη τέρπεσθαι μετ’ αὐτῆς ἐξέβη τῆς ὕλης· καὶ ἐλθὼν ἵνα ἐκάθητο στεφανίσκον ἴων πλέκουσα, τόν τε χῆνα τῶν τοῦ ἀετοῦ ὀνύχων ἐψεύσατο ἐξαρπάσαι καὶ περιφὺς ἐφίλησεν, οἷον ἐν τῇ τέρψει Λυκαίνιον·  τοῦτο γὰρ ἐξῆν ὡς ἀκίνδυνον· ἡ δὲ τὸν στέφανον ἐφήρμοσεν αὐτοῦ τῇ κεφαλῇ καὶ τὴν κόμην ἐφίλησεν ὡς τῶν ἴων κρείττονα. Κἀκ τῆς πήρας προκομίσασα παλάθης μοῖραν καὶ ἄρτους τινὰς ἔδωκε φαγεῖν· καὶ ἐσθίοντος ἀπὸ τοῦ στόματος ἥρπαζε καὶ οὕτως ἤσθιεν ὥσπερ νεοττὸς ὄρνιθος.
\pend


\pstart
3.21  Ἐσθιόντων δὲ αὐτῶν καὶ περιττότερα φιλούντων ὧν ἤσθιον, ναῦς ἁλιέων ὤφθη παραπλέουσα. Ἄνεμος μὲν οὐκ ἦν, γαλήνη δὲ ἦν καὶ ἐρέττειν ἐδόκει. Καὶ ἤρεττον ἐρρωμένως· ἠπείγοντο γὰρ νεαλεῖς ἰχθῦς εἰς τὴν πόλιν διασώσασθαι τῶν τινι πλουσίων.  Οἷον οὖν εἰώθασι ναῦται δρᾶν ἐς καμάτων ἀμέλειαν, τοῦτο κἀκεῖνοι δρῶντες τὰς κώπας ἀνέφερον. Εἷς μὲν αὐτοῖς κελευστὴς ναυτικὰς ᾖδεν ᾠδάς, οἱ δὲ λοιποὶ καθάπερ χορὸς ὁμοφώνως κατὰ καιρὸν τῇ ἐκείνου φωνῇ ἐβόων.  Ἡνίκ’ οὖν ἐν ἀναπεπταμένῃ τῇ θαλάττῃ ταῦτα ἔπραττον, ἠφανίζετο ἡ βοὴ χεομένης τῆς φωνῆς εἰς πολὺν ἀέρα· ἐπεὶ δὲ ἄκρᾳ τινὶ ὑποδραμόντες εἰς κόλπον μηνοειδῆ καὶ κοῖλον εἰσήλασαν, μείζων μὲν ἠκούετο βοή, σαφῆ δὲ ἐξέπιπτεν εἰς τὴν γῆν τὰ τῶν κελευστῶν ᾅσματα.  Κοῖλος γὰρ τῷ πεδίῳ αὐλὼν ὑποκείμενος καὶ τὸν ἦχον εἰς αὑτὸν ὡς ὄργανον δεχόμενος πάντων τῶν φθεγγομένων μιμητὴν φωνὴν ἀπεδίδου, ἰδίᾳ μὲν τῶν κωπῶν τὸν ἦχον, ἰδίᾳ δὲ τὴν φωνὴν τῶν ναυτῶν· καὶ ἐγίνετο ἄκουσμα τερπνόν.  Φθανούσης γὰρ τῆς ἀπὸ τῆς θαλάττης φωνῆς, ἡ ἐκ τῆς γῆς φωνὴ τοσοῦτον ἐπαύετο βράδιον, ὅσον ἤρξατο.
\pend


\pstart
3.22  Ὁ μὲν οὖν Δάφνις εἰδὼς τὸ πραττόμενον μόνῃ τῇ θαλάττῃ προσεῖχε καὶ ἐτέρπετο τῇ νηὶ παρατρεχούσῃ τὸ πεδίον θᾶττον πτεροῦ καὶ ἐπειρᾶτό τινα διασώσασθαι τῶν ᾀσμάτων, ὡς γένοιτο τῆς σύριγγος μέλη·  ἡ δὲ Χλόη τότε πρῶτον πειρωμένη τῆς καλουμένης ἠχοῦς ποτὲ μὲν εἰς τὴν θάλατταν ἀπέβλεπε, τῶν ναυτῶν κελευόντων, ποτὲ δὲ εἰς τὴν ὕλην ἐπεστρέφετο,  ζητοῦσα τοὺς ἀντιφωνοῦντας. Καὶ ἐπεὶ παραπλευσάντων ἦν κἀν τῷ αὐλῶνι σιγή, ἐπυνθάνετο τοῦ Δάφνιδος, εἰ καὶ ὀπίσω τῆς ἄκρας ἐστὶ θάλαττα καὶ ναῦς ἄλλη παρέπλει καὶ ἄλλοι ναῦται τὰ αὐτὰ ᾖδον καὶ ἅμα πάντες σιωπῶσι.  Γελάσας οὖν ὁ Δάφνις ἡδὺ καὶ φιλήσας ἥδιον φίλημα καὶ τὸν τῶν ἴων στέφανον ἐκείνῃ περιθεὶς ἤρξατο αὐτῇ μυθολογεῖν τὸν μῦθον τῆς Ἠχοῦς, αἰτήσας, εἰ διδάξειε, μισθὸν παρ’ αὐτῆς ἄλλα φιλήματα δέκα.
\pend


\pstart
3.23  “Νυμφῶν, ὦ κόρη, πολὺ γένος, Μελίαι καὶ Δρυάδες καὶ Ἕλειοι· πᾶσαι καλαί, πᾶσαι μουσικαί. Καὶ μιᾶς τούτων θυγάτηρ Ἠχὼ γίνεται, θνητὴ μὲν ὡς ἐκ πατρὸς θνητοῦ, καλὴ δὲ ὡς ἐκ μητρὸς καλῆς.  Τρέφεται μὲν ὑπὸ Νυμφῶν, παιδεύεται δὲ ὑπὸ Μουσῶν συρίττειν, αὐλεῖν, τὰ πρὸς λύραν, τὰ πρὸς κιθάραν, πᾶσαν ᾠδήν, ὥστε καὶ παρθενίας εἰς ἄνθος ἀκμάσασα ταῖς Νύμφαις συνεχόρευε, ταῖς Μούσαις συνῇδεν· ἄρρενας δὲ ἔφευγε πάντας, καὶ ἀνθρώπους καὶ θεούς, φιλοῦσα τὴν παρθενίαν.  Ὁ Πὰν ὀργίζεται τῇ κόρῃ, τῆς μουσικῆς φθονῶν, τοῦ κάλλους μὴ τυχών, καὶ μανίαν ἐμβάλλει τοῖς ποιμέσι καὶ τοῖς αἰπόλοις. Οἱ δὲ ὥσπερ κύνες ἢ λύκοι διασπῶσιν αὐτὴν καὶ ῥίπτουσιν εἰς πᾶσαν γῆν ἔτι ᾅδοντα τὰ μέλη.  Καὶ τὰ μέλη Γῆ χαριζομένη Νύμφαις ἔκρυψε πάντα. Καὶ ἐτήρησε τὴν μουσικὴν καὶ γνώμῃ Μουσῶν ἀφίησι φωνὴν καὶ μιμεῖται πάντα, καθάπερ τότε ἡ κόρη, θεούς, ἀνθρώπους, ὄργανα, θηρία· μιμεῖται καὶ αὐτὸν συρίττοντα τὸν Πᾶνα.  Ὁ δὲ ἀκούσας ἀναπηδᾷ καὶ διώκει κατὰ τῶν ὀρῶν, οὐκ ἐρῶν τυχεῖν ἀλλ’ ἢ τοῦ μαθεῖν, τίς ἐστὶν ὁ λανθάνων μαθητής.” Ταῦτα μυθολογήσαντα τὸν Δάφνιν οὐ δέκα μόνον φιλήματα ἀλλὰ πάνυ πολλὰ κατεφίλησεν ἡ Χλόη· μικροῦ δὲ καὶ τὰ αὐτὰ εἶπεν ἡ Ἠχώ, καθάπερ μαρτυροῦσα ὅτι μηδὲν ἐψεύσατο.
\pend


\pstart
3.24  Θερμοτέρου δὲ καθ’ ἑκάστην ἡμέραν γινομένου τοῦ ἡλίου, οἷα τοῦ μὲν ἦρος παυομένου τοῦ δὲ θέρους ἀρχομένου, πάλιν αὐτοῖς ἐγίνοντο καιναὶ τέρψεις καὶ θέρειοι.  Ὁ μὲν γὰρ ἐνήχετο ἐν τοῖς ποταμοῖς, ἡ δὲ ἐν ταῖς πηγαῖς ἐλούετο· ὁ μὲν ἐσύριττεν ἁμιλλώμενος πρὸς τὰς πίτυς, ἡ δὲ ᾖδε ταῖς ἀηδόσιν ἐρίζουσα· ἐθήρων ἀκρίδας λάλους, ἐλάμβανον τέττιγας ἠχοῦντας, ἄνθη συνέλεγον, δένδρα ἔσειον, ὀπώραν ἤσθιον· ἤδη ποτὲ καὶ γυμνοὶ συγκατεκλίνησαν καὶ ἓν δέρμα αἰγὸς ἐπεσύραντο.  Καὶ ἐγένετο ἂν γυνὴ Χλόη ῥᾳδίως, εἰ μὴ Δάφνιν ἐτάραξε τὸ αἷμα. Ἀμέλει καὶ δεδοικὼς μὴ νικηθῇ τὸν λογισμόν ποτε, πολλὰ γυμνοῦσθαι τὴν Χλόην οὐκ ἐπέτρεπεν, ὥστε ἐθαύμαζε μὲν ἡ Χλόη, τὴν δὲ αἰτίαν ᾐδεῖτο πυθέσθαι.
\pend


\pstart
3.25  Ἐν τῷ θέρει τῷδε καὶ μνηστήρων πλῆθος ἦν περὶ τὴν Χλόην καὶ πολλοὶ πολλαχόθεν ἐφοίτων παρὰ τὸν Δρύαντα πρὸς γάμον αἰτοῦντες αὐτήν. Καὶ οἱ μέν τι δῶρον ἔφερον, οἱ δὲ ἐπηγγέλλοντο μεγάλα.  Ἡ μὲν οὖν Νάπη ταῖς ἐλπίσιν ἐπαιρομένη συνεβούλευεν ἐκδιδόναι τὴν Χλόην μηδὲ κατέχειν οἴκοι πρὸς πλέον τηλικαύτην κόρην, ἣ τάχα μικρὸν ὕστερον νέμουσα τὴν παρθενίαν ἀπολέσει καὶ ἄνδρα ποιήσεταί τινα τῶν ποιμένων ἐπὶ μήλοις ἢ ῥόδοις, ἀλλ’ ἐκείνην τε ποιῆσαι δέσποιναν οἰκίας καὶ αὐτοὺς πολλὰ λαβόντας ἰδίῳ φυλάττειν αὐτὰ καὶ γνησίῳ παιδίῳ (ἐγεγόνει γὰρ αὐτοῖς ἄρρεν παιδίον οὐ πρὸ πολλοῦ τινος)·  ὁ δὲ Δρύας ποτὲ μὲν ἐθέλγετο τοῖς λεγομένοις (μείζονα γὰρ ἢ κατὰ ποιμαίνουσαν κόρην δῶρα ὠνομάζετο παρ’ ἑκάστου), ποτὲ δὲ ὡς κρείττων ἐστὶν ἡ παρθένος μνηστήρων γεωργῶν καὶ ὡς, εἴ ποτε τοὺς ἀληθινοὺς γονέας εὕροι, μεγάλως αὐτοὺς εὐδαίμονας θήσει, ἀνεβάλλετο τὴν ἀπόκρισιν καὶ εἷλκε χρόνον ἐκ χρόνου καὶ ἐν τῷ τέως ἀπεκέρδαινεν οὐκ ὀλίγα δῶρα.  Ἡ μὲν δὴ μαθοῦσα λυπηρῶς πάνυ διῆγε καὶ τὸν Δάφνιν ἐλάνθανεν ἐπὶ πολύ, λυπεῖν οὐ θέλουσα· ὡς δὲ ἐλιπάρει καὶ ἐνέκειτο πυνθανόμενος καὶ ἐλυπεῖτο μᾶλλον μὴ μανθάνων ἢ ἔμελλε μαθών, πάντα αὐτῷ διηγεῖται, τοὺς μνηστευομένους ὡς πολλοὶ καὶ πλούσιοι, τοὺς λόγους οὓς ἡ Νάπη σπεύδουσα τὸν γάμον ἔλεγεν, ὡς οὐκ ἀπείπατο Δρύας, ἀλλ’ ὡς εἰς τὸν τρυγητὸν ἀναβέβληται.
\pend


\pstart
3.26  Ἔκφρων ἐπὶ τούτοις ὁ Δάφνις γίνεται καὶ ἐδάκρυσε καθήμενος, ἀποθανεῖσθαι μηκέτι συννεμούσης Χλόης λέγων· καὶ οὐκ αὐτὸς μόνος, ἀλλὰ καὶ τὰ πρόβατα μετὰ τοιοῦτον ποιμένα. Εἶτα ἀνενεγκὼν ἐθάρρει καὶ πείσειν ἐνενόει τὸν πατέρα καὶ ἕνα τῶν μνωμένων αὑτὸν ἠρίθμει καὶ πολὺ κρατήσειν ἤλπιζε τῶν ἄλλων.  Ἓν αὐτὸν ἐτάραττεν· οὐκ ἦν Λάμων πλούσιος· τοῦτο μόνον αὐτοῦ τὴν ἐλπίδα λεπτὴν εἰργάζετο· ὅμως δὲ ἐδόκει μνᾶσθαι, καὶ τῇ Χλόῃ συνεδόκει. Τῷ Λάμωνι μὲν οὖν οὐδὲν ἐτόλμησεν εἰπεῖν, τῇ Μυρτάλῃ δὲ θαρρήσας καὶ τὸν ἔρωτα ἐμήνυσε καὶ περὶ τοῦ γάμου λόγους προσήνεγκεν· ἡ δὲ τῷ Λάμωνι νύκτωρ ἐκοινώσατο.  Σκληρῶς δὲ ἐκείνου τὴν ἔντευξιν ἐνεγκόντος καὶ λοιδορήσαντος εἰ παιδὶ θυγάτριον ποιμένων προξενεῖ μεγάλην ἐν τοῖς γνωρίσμασιν ἐπαγγελλομένῳ τύχην, ὃς αὐτούς, εὑρὼν τοὺς οἰκείους, καὶ ἐλευθέρους θήσει καὶ δεσπότας ἀγρῶν μειζόνων, ἡ Μυρτάλη διὰ τὸν ἔρωτα φοβουμένη μὴ τελέως ἀπελπίσας ὁ Δάφνις τὸν γάμον τολμήσῃ τι θανατῶδες, ἄλλας αὐτῷ τῆς ἀντιρρήσεως αἰτίας ἀπήγγειλε.  “Πένητες ἐσμέν, ὦ παῖ, καὶ δεόμεθα νύμφης φερούσης τι μᾶλλον· οἱ δὲ πλούσιοι καὶ πλουσίων νυμφίων δεόμενοι. Ἴθι δή, πεῖσον Χλόην, ἡ δὲ τὸν πατέρα, μηδὲν αἰτεῖν μέγα· πάντως δή που κἀκείνη φιλεῖ σε καὶ βούλεται συγκαθεύδειν πένητι καλῷ μᾶλλον ἢ πιθήκῳ πλουσίῳ.”
\pend


\pstart
3.27  Μυρτάλη μὲν οὔποτε ἐλπίσασα Δρύαντα τούτοις συνθήσεσθαι μνηστῆρας ἔχοντα πλουσιωτέρους εὐπρεπῶς ᾤετο παρῃτῆσθαι τὸν γάμον· Δάφνις δὲ οὐκ εἶχε μέμφεσθαι τὰ λελεγμένα. Λειπόμενος δὲ πολὺ τῶν αἰτουμένων τὸ σύνηθες ἐρασταῖς πενομένοις ἔπραττεν· ἐδάκρυε καὶ τὰς Νύμφας αὖθις ἐκάλει βοηθούς.  Αἱ δὲ αὐτῷ καθεύδοντι νύκτωρ ἐν τοῖς αὐτοῖς ἐφίστανται σχήμασιν, ἐν οἷς καὶ πρότερον· ἔλεγε δὲ ἡ πρεσβυτάτη πάλιν “γάμου μὲν μέλει τῆς Χλόης ἄλλῳ θεῷ, δῶρα δέ σοι δώσομεν ἡμεῖς, ἃ θέλξει Δρύαντα.  Ἡ ναῦς ἡ τῶν Μηθυμναίων νεανίσκων, ἧς τὴν λύγον αἱ σαί ποτε αἶγες κατέφαγον, ἡμέρᾳ μὲν ἐκείνῃ μακρὰν τῆς γῆς ὑπηνέχθη πνεύματι· νυκτὸς δὲ πελαγίου ταράξαντος ἀνέμου τὴν θάλατταν εἰς τὴν γῆν εἰς τὰς τῆς ἄκρας πέτρας ἐξεβράσθη.  Αὐτὴ μὲν οὖν διεφθάρη καὶ πολλὰ τῶν ἐν αὐτῇ· βαλάντιον δὲ τρισχιλίων δραχμῶν ὑπὸ τοῦ κύματος ἀπεπτύσθη καὶ κεῖται φυκίοις κεκαλυμμένον πλησίον δελφῖνος νεκροῦ, δι’ ὃν οὐδεὶς προσῆλθεν ὁδοιπόρος, τὸ δυσῶδες τῆς σηπεδόνος παρατρέχων.  Ἀλλὰ σὺ πρόσελθε καὶ προσελθὼν ἀνελοῦ καὶ ἀνελόμενος δός. Ἱκανόν σοι νῦν μὲν δόξαι μὴ πένητι, χρόνῳ δὲ ὕστερον ἔσῃ καὶ πλούσιος.”
\pend


\pstart
3.28  Αἱ μὲν ταῦτα εἰποῦσαι τῇ νυκτὶ συναπῆλθον· γενομένης δὲ ἡμέρας ἀναπηδήσας ὁ Δάφνις περιχαρὴς ἤλαυνε ῥοίζῳ πολλῷ τὰς αἶγας εἰς τὴν νομήν· καὶ τὴν Χλόην φιλήσας καὶ τὰς Νύμφας προσκυνήσας κατῆλθεν ἐπὶ θάλατταν, ὡς περιρράνασθαι θέλων· καὶ ἐπὶ τῆς ψάμμου πλησίον τῆς κυματωγῆς ἐβάδιζε ζητῶν τὰς τρισχιλίας.  Ἔμελλε δὲ ἄρ’ οὐ πολὺν κάματον ἕξειν· ὁ γὰρ δελφὶς οὐκ ἀγαθὸν ὀδωδὼς αὐτῷ προσέπιπτεν ἐρριμμένος καὶ μυδῶν· οὗ τῇ σηπεδόνι καθάπερ ἡγεμόνι χρώμενος ὁδοῦ προσῆλθέ τε εὐθὺς καὶ τὰ φυκία ἀφελὼν εὑρίσκει τὸ βαλάντιον ἀργυρίου μεστόν.  Τοῦτο ἀνελόμενος καὶ εἰς τὴν πήραν ἐνθέμενος οὐ πρόσθεν ἀπῆλθε, πρὶν τὰς Νύμφας εὐφημῆσαι καὶ αὐτὴν τὴν θάλατταν· καίπερ γὰρ αἰπόλος ὤν, ἤδη καὶ τὴν θάλατταν ἐνόμιζε τῆς γῆς γλυκυτέραν, ὡς εἰς τὸν γάμον αὐτῷ τὸν Χλόης συλλαμβάνουσαν.
\pend


\pstart
3.29  Εἰλημμένος δὲ τῶν τρισχιλίων οὐκέτ’ ἔμελλεν, ἀλλ’ ὡς πάντων ἀνθρώπων πλουσιώτατος, οὐ μόνον τῶν ἐκεῖ γεωργῶν, αὐτίκα ἐλθὼν παρὰ τὴν Χλόην διηγεῖται τὸ ὄναρ, δείκνυσι τὸ βαλάντιον, κελεύει τὰς ἀγέλας φυλάττειν, ἔστ’ ἂν ἐπανέλθῃ, καὶ συντείνας σοβεῖ παρὰ τὸν Δρύαντα. Καὶ εὑρὼν πυρούς τινας ἁλωνοτριβοῦντα μετὰ τῆς Νάπης πάνυ θρασὺν ἐμβάλλει λόγον περὶ γάμου.  “Ἐμοὶ δὸς Χλόην γυναῖκα· ἐγὼ καὶ συρίττειν οἶδα καλῶς καὶ κλᾶν ἄμπελον καὶ φυτὰ κατορύττειν· οἶδα καὶ γῆν ἀροῦν καὶ λικμῆσαι πρὸς ἄνεμον. Ἀγέλην δὲ ὅπως νέμω μάρτυς Χλόη· πεντήκοντα αἶγας παραλαβὼν διπλασίονας πεποίηκα· ἔθρεψα καὶ τράγους μεγάλους καὶ καλούς· πρότερον δὲ ἀλλοτρίοις τὰς αἶγας ὑπεβάλλομεν.  Ἀλλὰ καὶ νέος εἰμὶ καὶ γείτων ὑμῖν ἄμεμπτος· καί με ἔθρεψεν αἴξ, ὡς Χλόην οἶς. Τοσοῦτον δὲ τῶν ἄλλων κρατῶν οὐδὲ δώροις ἡττηθήσομαι.  Ἐκεῖνοι δώσουσιν αἶγας καὶ πρόβατα καὶ ζεῦγος ψωραλέων βοῶν καὶ σῖτον μηδὲ ἀλεκτορίδας θρέψαι δυνάμενον· παρ’ ἐμοῦ δὲ αἵδε ὑμῖν τρισχίλιαι. Μόνον ἴστω τοῦτο μηδείς, μὴ Λάμων αὐτὸς οὑμὸς πατήρ.” Ἅμα τ’ ἐδίδου καὶ περιβαλὼν κατεφίλει.
\pend


\pstart
3.30  Οἱ δὲ παρ’ ἐλπίδα ἰδόντες τοσοῦτον ἀργύριον, αὐτίκα τε δώσειν ἐπηγγέλλοντο τὴν Χλόην καὶ πείσειν ὑπισχνοῦντο τὸν Λάμωνα.  Ἡ μὲν δὴ Νάπη μετὰ τοῦ Δάφνιδος αὐτοῦ μένουσα περιήλαυνε τὰς βοῦς καὶ τοῖς τριβόλοις κατειργάζετο τὸν στάχυν· ὁ δὲ Δρύας θησαυρίσας τὸ βαλάντιον ἔνθα ἀπέκειτο τὰ γνωρίσματα, ταχὺ παρὰ τὸν Λάμωνα καὶ τὴν Μυρτάλην ἐφέρετο, μέλλων παρ’ αὐτῶν, τὸ καινότατον,  μνᾶσθαι νυμφίον. Εὑρὼν δὲ κἀκείνους κριθία μετροῦντας οὐ πρὸ πολλοῦ λελικμημένα ἀθύμως τε ἔχοντας ὅτι μικροῦ δεῖν ὀλιγώτερα ἦν τῶν καταβληθέντων σπερμάτων, ἐπ’ ἐκείνοις μὲν παρεμυθήσατο,  κοινὴν ὁμολογήσας ἀφορίαν πανταχοῦ γεγονέναι, τὸν δὲ Δάφνιν ᾐτεῖτο Χλόῃ καὶ ἔλεγεν ὅτι πολλὰ ἄλλων διδόντων, οὐδὲν παρ’ αὐτῶν λήψεται, μᾶλλον δέ τι οἴκοθεν αὐτοῖς ἐπιδώσει· συντετράφθαι γὰρ ἀλλήλοις κἀν τῷ νέμειν συνῆφθαι φιλίᾳ ῥᾳδίως λυθῆναι μὴ δυναμένῃ· ἤδη δὲ καὶ ἡλικίαν ἔχειν ὡς καθεύδειν μετ’ ἀλλήλων.  Ὁ μὲν ταῦτα καὶ ἔτι πλείω ἔλεγεν, οἷα τοῦ πεῖσαι ἆθλον ἔχων τὰς τρισχιλίας· ὁ δὲ Λάμων μήτε πενίαν ἔτι προβάλλεσθαι δυνάμενος (αὐτοὶ γὰρ οὐχ ὑπερηφάνουν) μήτε ἡλικίαν Δάφνιδος (ἤδη γὰρ μειράκιον ἦν) τὸ μὲν ἀληθὲς οὐδ’ ὣς ἐξηγόρευσεν, ὅτι κρείττων ἐστὶ τοιούτου γάμου, χρόνον δὲ σιωπήσας ὀλίγον οὕτως ἀπεκρίνατο
\pend


\pstart
3.31  “Δίκαια ποιεῖτε τοὺς γείτονας προτιμῶντες τῶν ξένων καὶ πενίας ἀγαθῆς πλοῦτον μὴ νομίζοντες κρείττονα. Ὁ Πὰν ὑμᾶς καὶ αἱ Νύμφαι ἀντὶ τῶνδε φιλήσειαν.  Ἐγὼ δὲ σπεύδω μὲν καὶ αὐτὸς τὸν γάμον τοῦτον· καὶ γὰρ ἂν μαινοίμην, εἰ μὴ γέρων τε ὢν ἤδη καὶ χειρὸς εἰς τὰ ἔργα περιττοτέρας δεόμενος ᾤμην καὶ τὸν ὑμέτερον οἶκον φίλον προσλαβεῖν·  περισπούδαστος δὲ καὶ Χλόη, καλὴ καὶ ὡραία κόρη καὶ πάντα ἀγαθή· δοῦλος δὲ ὢν οὐδενός εἰμι τῶν ἐμῶν κύριος, ἀλλὰ δεῖ τὸν δεσπότην μανθάνοντα ταῦτα συγχωρεῖν. Φέρε οὖν ἀναβαλώμεθα τὸν γάμον εἰς τὸ μετόπωρον.  Ἀφίξεσθαι τότε λέγουσιν αὐτὸν οἱ παραγενόμενοι πρὸς ἡμᾶς ἐξ ἄστεος. Τότε ἔσονται ἀνὴρ καὶ γυνή· νῦν δὲ φιλείτωσαν ἀλλήλους ὡς ἀδελφοί. Ἴσθι μόνον, ὦ Δρύα, τοσοῦτον· σπεύδεις περὶ μειράκιον κρεῖττον ἡμῶν.”
\pend


\pstart
3.32  ὁ μὲν ταῦτα εἰπὼν ἐφίλησέ τε αὐτὸν καὶ ὤρεξε πότον ἤδη μεσημβρίας ἀκμαζούσης καὶ προύπεμψε μέχρι τινός, φιλοφρονούμενος πάντα· ὁ δὲ Δρύας οὐ παρέργως ἀκούσας τὸν ὕστερον λόγον τοῦ Λάμωνος ἐφρόντιζε βαδίζων καθ’ αὑτὸν ὅστις ὁ Δάφνις. “Ἐτράφη μὲν ὑπ’ αἰγὸς ὡς κηδομένων θεῶν· ἔστι δὲ καλὸς καὶ οὐδὲν ἐοικὼς σιμῷ γέροντι καὶ μαδώσῃ γυναικί· εὐπόρησε δὲ καὶ τρισχιλίων, ὅσον οὐδὲ ἀχράδων εἰκὸς ἔχειν αἰπόλον.  Ἆρα καὶ τοῦτον ἐξέθηκέ τις ὡς Χλόην; Ἆρα καὶ τοῦτον εὗρε Λάμων ὡς ἐκείνην ἐγώ; Ἆρα καὶ γνωρίσματα ὅμοια παρέκειτο τοῖς εὑρεθεῖσιν ὑπ’ ἐμοῦ; Ἐὰν ταῦτα οὕτως, ὦ δέσποτα Πὰν καὶ Νύμφαι φίλαι, τάχα οὗτος τοὺς ἰδίους εὑρὼν εὑρήσει τι καὶ τῶν Χλόης ἀπορρήτων.”  Τοιαῦτα μὲν πρὸς αὑτὸν ἐφρόντιζε καὶ ὠνειροπόλει μέχρι τῆς ἅλω· ἐλθὼν δὲ ἐκεῖ καὶ τὸν Δάφνιν μετέωρον πρὸς τὴν ἀκοὴν καταλαβὼν ἀνέρρωσέ τε γαμβρὸν προσαγορεύσας καὶ τῷ μετοπώρῳ τοὺς γάμους θύσειν ἐπαγγέλλεται, δεξιάν τε ἔδωκεν ὡς οὐδενὸς ἐσομένης ὅτι μὴ Δάφνιδος Χλόης.
\pend


\pstart
3.33  Θᾶττον οὖν νοήματος, μηδὲν πιὼν μηδὲ φαγὼν παρὰ τὴν Χλόην κατέδραμε· καὶ εὑρὼν αὐτὴν ἀμέλγουσαν καὶ τυροποιοῦσαν, τόν τε γάμον εὐηγγελίζετο καὶ ὡς γυναῖκα λοιπὸν μὴ λανθάνων κατεφίλει καὶ ἐκοινώνει τοῦ πόνου.  Ἤμελγε μὲν εἰς γαυλοὺς τὸ γάλα, ἐνεπήγνυ δὲ ταρσοῖς τοὺς τυρούς, προσέβαλλε ταῖς μητράσι τοὺς ἄρνας καὶ τοὺς ἐρίφους. Καλῶς δὲ ἐχόντων τούτων ἀπελούσαντο, ἐνέφαγον,  ἔπιον, περιῄεσαν ζητοῦντες ὀπώραν ἀκμάζουσαν. Ἦν δὲ ἀφθονία πολλὴ διὰ τὸ τῆς ὥρας πάμφορον· πολλαὶ μὲν ἀχράδες, πολλαὶ δὲ ὄχναι, πολλὰ δὲ μῆλα· τὰ μὲν ἤδη πεπτωκότα κάτω, τὰ δὲ ἔτι ἐπὶ τῶν φυτῶν· τὰ ἐπὶ τῆς γῆς εὐωδέστερα, τὰ ἐπὶ τῶν κλάδων εὐανθέστερα· τὰ μὲν οἷον οἶνος ἀπῶζε, τὰ δὲ οἷον χρυσὸς ἀπέλαμπε.  Μία μηλέα ἐτετρύγητο καὶ οὔτε καρπὸν εἶχεν οὔτε φύλλον· γυμνοὶ πάντες ἦσαν οἱ κλάδοι· καὶ ἓν μῆλον ἐλέλειπτο ἐν αὐτοῖς ἄκροις ἀκρότατον, μέγα καὶ καλὸν καὶ τῶν πολλῶν τὴν εὐωδίαν ἐνίκα μόνον· ἔδεισεν ὁ τρυγῶν ἀνελθεῖν, ἠμέλησε καθελεῖν· τάχα δὲ καὶ ἐφύλαττε τὸ καλὸν μῆλον ἐρωτικῷ ποιμένι.
\pend


\pstart
3.34  Τοῦτο τὸ μῆλον ὡς εἶδεν ὁ Δάφνις, ὥρμα τρυγᾶν ἀνελθὼν καὶ Χλόης κωλυούσης ἠμέλησεν· ἡ μὲν ἀμεληθεῖσα, ὀργισθεῖσα πρὸς τὰς ἀγέλας ἀπῆλθε· Δάφνις δὲ ἀναδραμὼν ἐφίκετο τρυγῆσαι καὶ ἐκόμισε δῶρον Χλόῃ καὶ λόγον τοιόνδ’ εἶπεν ὠργισμένῃ “ὦ παρθένε, τοῦτο τὸ μῆλον ἔφυσαν Ὧραι καλαὶ καὶ φυτὸν καλὸν ἔθρεψε πεπαίνοντος ἡλίου, καὶ ἐτήρησε Τύχη.  Καὶ οὐκ ἔμελλον αὐτὸ καταλιπεῖν ὀφθαλμοὺς ἔχων, ἵνα πέσῃ χαμαὶ καὶ ἢ ποίμνιον αὐτὸ πατήσῃ νεμόμενον ἢ ἑρπετὸν φαρμάξῃ συρόμενον ἢ χρόνος δαπανήσῃ κείμενον. Βλεπόμενον ἐπαινούμενον. Τοῦτο Ἀφροδίτη κάλλους ἔλαβεν ἆθλον·  τοῦτο ἐγὼ σοὶ δίδωμι νικητήριον. Ὁμοίως ἔχομεν τοὺς σοὺς μάρτυρας· ἐκεῖνος ἦν ποιμήν, αἰπόλος ἐγώ.” Ταῦτα εἰπὼν ἐντίθησι τοῖς κόλποις· ἡ δὲ ἐγγὺς γενόμενον κατεφίλησεν, ὥστε ὁ Δάφνις οὐ μετέγνω τολμήσας ἀνελθεῖν εἰς τοσοῦτον ὕψος· ἔλαβε γὰρ κρεῖττον καὶ χρυσοῦ μήλου φίλημα.
\pend


\pstart
4.1  Ἥκων δέ τις ἐκ τῆς Μυτιλήνης ὁμόδουλος τοῦ Λάμωνος ἤγγειλεν ὅτι ὀλίγον πρὸ τοῦ τρυγητοῦ ὁ δεσπότης ἀφίξεται μαθησόμενος μή τι τοὺς ἀγροὺς ὁ τῶν Μηθυμναίων ἐπίπλους ἐλυμήνατο.  Ἤδη οὖν τοῦ θέρους ἀπιόντος καὶ τοῦ μετοπώρου προσιόντος παρεσκεύαζεν αὐτῷ τὴν καταγωγὴν ὁ Λάμων εἰς πᾶσαν θέας ἡδονήν.  Πηγὰς ἐξεκάθαιρεν, ὡς τὸ ὕδωρ καθαρὸν ἔχοιεν· τὴν κόπρον ἐξεφόρει τῆς αὐλῆς, ὡς ἀπόζουσα μὴ διοχλοίη· τὸν παράδεισον ἐθεράπευεν, ὡς ὀφθείη καλός.
\pend


\pstart
4.2  Ἦν δὲ ὁ παράδεισος πάγκαλόν τι χρῆμα καὶ κατὰ τοὺς βασιλικούς. Ἐξετέτατο μὲν εἰς σταδίου μῆκος, ἐπέκειτο δὲ ἐν χώρῳ μετεώρῳ, τὸ εὖρος ἔχων πλέθρων τεττάρων.  Εἴκασεν ἄν τις αὐτὸν πεδίῳ μακρῷ. Εἶχε δὲ πάντα δένδρα, μηλέας, μυρρίνας, ὄχνας καὶ ῥοιὰς καὶ συκῆν καὶ ἐλαίας· ἑτέρωθι ἄμπελον ὑψηλήν· καὶ ἐπέκειτο ταῖς μηλέαις καὶ ταῖς ὄχναις περκάζουσα, καθάπερ περὶ τοῦ καρποῦ αὐταῖς προσερίζουσα.  Τοσαῦτα ἥμερα. Ἦσαν δὲ καὶ κυπάριττοι καὶ δάφναι καὶ πλάτανοι καὶ πίτυς. Ταύταις πάσαις ἀντὶ τῆς ἀμπέλου κιττὸς ἐπέκειτο· καὶ ὁ κόρυμβος αὐτοῦ μέγας ὢν καὶ μελαινόμενος βότρυν ἐμιμεῖτο.  Ἔνδον ἦν τὰ καρποφόρα φυτά, καθάπερ φρουρούμενα· ἔξωθεν περιειστήκει τὰ ἄκαρπα, καθάπερ θριγκὸς χειροποίητος· ταῦτα μέντοι λεπτῆς αἱμασιᾶς περιέθει περίβολος.  Ἐτέτμητο καὶ διακέκριτο πάντα καὶ στέλεχος στελέχους ἀφειστήκει, ἐν μετεώρῳ δὲ οἱ κλάδοι συνέπιπτον ἀλλήλοις καὶ ἐπήλλαττον τὰς κόμας· ἐδόκει μέντοι καὶ ἡ τούτων φύσις εἶναι τέχνη.  Ἦσαν καὶ ἀνθῶν πρασιαί, ὧν τὰ μὲν ἔφερεν ἡ γῆ, τὰ δὲ ἐποίει τέχνη· ῥοδωνιὰ καὶ ὑάκινθοι καὶ κρίνα χειρὸς ἔργα· ἰωνιὰς καὶ ναρκίσσους καὶ ἀναγαλλίδας ἔφερεν ἡ γῆ. Σκιά τε ἦν θέρους καὶ ἦρος ἄνθη καὶ μετοπώρου ὀπώρα καὶ κατὰ πᾶσαν ὥραν τρυφή.
\pend


\pstart
4.3  Ἐντεῦθεν εὔοπτον μὲν ἦν τὸ πεδίον καὶ ἦν ὁρᾶν τοὺς νέμοντας, εὔοπτος δ’ ἡ θάλαττα καὶ ἑωρῶντο οἱ παραπλέοντες· ὥστε καὶ ταῦτα μέρος ἐγίνετο τῆς ἐν τῷ παραδείσῳ τρυφῆς. Ἵνα τοῦ παραδείσου τὸ μεσαίτατον ἐπὶ μῆκος καὶ εὖρος ἦν, νεὼς Διονύσου καὶ βωμὸς ἦν· περιεῖχε τὸν μὲν βωμὸν κιττός, τὸν νεὼν δὲ κλήματα.  Εἶχε δὲ καὶ ἔνδοθεν ὁ νεὼς Διονυσιακὰς γραφάς· Σεμέλην τίκτουσαν, Ἀριάδνην καθεύδουσαν, Λυκοῦργον δεδεμένον, Πενθέα διαιρούμενον· ἐπῆσαν καὶ Ἰνδοὶ νικώμενοι καὶ Τυρρηνοὶ μεταμορφούμενοι· πανταχοῦ Σάτυροι πατοῦντες, πανταχοῦ Βάκχαι χορεύουσαι· οὐδὲ ὁ Πὰν ἠμέλητο· ἐκαθέζετο δὲ συρίττων ἐπὶ πέτρας, ὅμοιος ἐνδιδόντι κοινὸν μέλος καὶ τοῖς πατοῦσι καὶ ταῖς χορευούσαις.
\pend


\pstart
4.4  Τοιοῦτον ὄντα τὸν παράδεισον ὁ Λάμων ἐθεράπευε, τὰ ξηρὰ ἀποτέμνων, τὰ κλήματα ἀναλαμβάνων. Τὸν Διόνυσον ἐστεφάνωσε· τοῖς ἄνθεσιν ὕδωρ ἐπωχέτευσε. Πηγή τις ἦν, εὗρεν ἐς τὰ ἄνθη Δάφνις· ἐσχόλαζε μὲν τοῖς ἄνθεσιν ἡ πηγή, Δάφνιδος δὲ ὅμως ἐκαλεῖτο πηγή.  Παρεκελεύετο δὲ καὶ τῷ Δάφνιδι ὁ Λάμων πιαίνειν τὰς αἶγας ὡς δυνατὸν μάλιστα, πάντως που κἀκείνας λέγων ὄψεσθαι τὸν δεσπότην ἀφικόμενον διὰ μακροῦ.  Ὁ δὲ ἐθάρρει μὲν ὡς ἐπαινεθησόμενος ἐπ’ αὐταῖς· διπλασίονάς τε γὰρ ὧν ἔλαβεν ἐποίησε καὶ λύκος οὐδὲ μίαν ἥρπασε, καὶ ἦσαν πιότεραι τῶν οἰῶν· βουλόμενος δὲ προθυμότερον αὐτὸν γενέσθαι πρὸς τὸν γάμον πᾶσαν θεραπείαν καὶ προθυμίαν προσέφερεν, ἐξάγων τε αὐτὰς πάνυ ἕωθεν καὶ ἀπάγων τὸ δειλινόν.  Δὶς ἡγεῖτο ἐπὶ πότον· ἀνεζήτει τὰ εὐνομώτατα τῶν χωρίων· ἐμέλησεν αὐτῷ καὶ σκαφίδων καινῶν καὶ γαυλῶν πολλῶν καὶ ταρσῶν μειζόνων· τοσαύτη δὲ ἦν κηδεμονία, ὥστε καὶ τὰ κέρατα ἤλειφε καὶ τὰς τρίχας ἐθεράπευε.  Πανὸς ἄν τις ἱερὰν ἀγέλην ἔδοξεν ὁρᾶν. Ἐκοινώνει δὲ παντὸς εἰς αὐτὰς καμάτου καὶ ἡ Χλόη, καὶ τῆς ποίμνης παραμελοῦσα τὸ πλέον ἐκείναις ἐσχόλαζεν, ὥστε ἐνόμιζεν ὁ Δάφνις δι’ ἐκείνην αὐτὰς φαίνεσθαι καλάς.
\pend


\pstart
4.5  Ἐν τούτοις οὖσιν αὐτοῖς δεύτερος ἄγγελος ἐλθὼν ἐξ ἄστεος ἐκέλευεν ἀποτρυγᾶν τὰς ἀμπέλους ὅτι τάχιστα, καὶ αὐτὸς ἔφη παραμενεῖν ἔστ’ ἂν τοὺς βότρυς ποιήσωσι γλεῦκος, εἶτα οὕτως κατελθὼν εἰς τὴν πόλιν ἄξειν τὸν δεσπότην ἤδη τῆς μετοπωρινῆς τρύγης.  Τοῦτόν τε οὖν τὸν Εὔδρομον (οὕτω γὰρ ἐκαλεῖτο, ὅτι ἦν αὐτῷ ἔργον τρέχειν) ἐδεξιοῦντο πᾶσαν δεξίωσιν καὶ ἅμα τὰς ἀμπέλους ἀπετρύγων, τοὺς βότρυς ἐς τὰς ληνοὺς κομίζοντες, τὸ γλεῦκος εἰς τοὺς πίθους φέροντες, τῶν βοτρύων τοὺς ἡβῶντας ἐπὶ κλημάτων ἀφαιροῦντες, ὡς εἴη καὶ τοῖς ἐκ τῆς πόλεως ἐλθοῦσιν ἐν εἰκόνι καὶ ἡδονῇ γενέσθαι τρυγητοῦ.
\pend


\pstart
4.6  Μέλλοντος δὲ ἤδη σοβεῖν ἐς ἄστυ τοῦ Εὐδρόμου καὶ ἄλλα μὲν οὐκ ὀλίγα αὐτῷ Δάφνις ἔδωκεν, ἔδωκε δὲ καὶ ὅσα ἀπ’ αἰπολίου δῶρα, τυροὺς εὐπαγεῖς, ἔριφον ὀψίγονον, δέρμα αἰγὸς λευκὸν καὶ λάσιον, ὡς ἔχοι χειμῶνος ἐπιβάλλεσθαι τρέχων.  Ὁ δὲ ἥδετο καὶ ἐφίλει τὸν Δάφνιν καὶ ἀγαθόν τι ἐρεῖν περὶ αὐτοῦ πρὸς τὸν δεσπότην ἐπηγγέλλετο. Καὶ ὁ μὲν ἀπῄει φίλα φρονῶν, ὁ δὲ Δάφνις ἀγωνιῶν τῇ Χλόῃ συνένεμεν. Εἶχε δὲ κἀκείνην πολὺ δέος· μειράκιον γὰρ εἰωθὸς αἶγας βλέπειν καὶ οἶς καὶ γεωργοὺς καὶ Χλόην πρῶτον ἔμελλεν ὄψεσθαι δεσπότην, οὗ πρότερον μόνον ἤκουε τοὔνομα.  Ὑπέρ τε οὖν τοῦ Δάφνιδος ἐφρόντιζεν, ὅπως ἐντεύξεται τῷ δεσπότῃ καὶ περὶ τοῦ γάμου τὴν ψυχὴν ἐταράττετο, μὴ μάτην ὀνειροπολοῦσιν αὐτόν. Συνεχῆ μὲν οὖν τὰ φιλήματα καὶ ὥσπερ συμπεφυκότων αἱ περιβολαί· καὶ τὰ φιλήματα δειλὰ ἦν καὶ αἱ περιβολαὶ σκυθρωπαί, καθάπερ ἤδη παρόντα τὸν δεσπότην φοβουμένων ἢ λανθανόντων. Προσγίνεται δέ τις αὐτοῖς καὶ τοιόσδε τάραχος.
\pend


\pstart
4.7  Λάμπις τις ἦν ἀγέρωχος βουκόλος. Οὗτος καὶ αὐτὸς ἐμνᾶτο τὴν Χλόην παρὰ τοῦ Δρύαντος καὶ δῶρα ἤδη πολλὰ ἐδεδώκει σπεύδων τὸν γάμον.  Αἰσθόμενος οὖν ὡς, εἰ συγχωρηθείη παρὰ τοῦ δεσπότου, Δάφνις αὐτὴν ἄξεται, τέχνην ἐζήτει, δι’ ἧς τὸν δεσπότην αὐτοῖς ποιήσει πικρόν· καὶ εἰδὼς πάνυ αὐτὸν τῷ παραδείσῳ τερπόμενον, ἔγνω τοῦτον ὅσον οἷός τέ ἐστι διαφθεῖραι καὶ ἀποκοσμῆσαι.  Δένδρα μὲν οὖν τέμνων ἔμελλεν ἁλώσεσθαι διὰ τὸν κτύπον· ἐπεῖχε δὲ τοῖς ἄνθεσιν, ὥστε διαφθεῖραι αὐτά. Νύκτα δὴ φυλάξας καὶ ὑπερβὰς τὴν αἱμασιὰν τὰ μὲν ἀνώρυξε, τὰ δὲ κατέκλασε,  τὰ δὲ κατεπάτησεν ὥσπερ σῦς. Καὶ ὁ μὲν λαθὼν ἀπεληλύθει· Λάμων δὲ τῆς ἐπιούσης παρελθὼν εἰς τὸν κῆπον ἔμελλεν ὕδωρ αὐτοῖς ἐκ τῆς πηγῆς ἐπάξειν.  Ἰδὼν δὲ πᾶν τὸ χωρίον δεδῃωμένον καὶ ἔργον οἷον ἐχθρός, οὐ λῃστὴς ἐργάσαιτο, κατερρήξατο μὲν εὐθὺς τὸν χιτωνίσκον, βοῇ δὲ μεγάλῃ θεοὺς ἀνεκάλει, ὥστε καὶ ἡ Μυρτάλη τὰ ἐν χερσὶ καταλιποῦσα ἐξέδραμε καὶ ὁ Δάφνις ἐάσας τὰς αἶγας ἀνέδραμε· καὶ ἰδόντες ἐβόων καὶ βοῶντες ἐδάκρυον.
\pend


\pstart
4.8  Ἀλλ’ οἱ μὲν πτοούμενοι τὸν δεσπότην ἔκλαον· ἔκλαυσε δ’ ἄν τις καὶ ξένος ἐπιστάς· ἀπεκεκόσμητο γὰρ ὁ τόπος καὶ ἦν λοιπὸν γῆ πηλώδης. Τῶν δὲ εἴ τι διέφυγε τὴν ὕβριν, ὑπήνθει καὶ ἔλαμπε καὶ ἦν ἔτι καλὸν καὶ κείμενον.  Ἐπέκειντο δὲ αὐτοῖς καὶ μέλιτται συνεχὲς καὶ ἄπαυστον βομβοῦσαι καὶ θρηνούσαις ὅμοιον. Ὁ μὲν οὖν Λάμων ὑπ’ ἐκπλήξεως κἀκεῖνα ἔλεγε “Φεῦ τῆς ῥοδωνιᾶς, ὡς κατακέκλασται·  φεῦ τῆς ἰωνιᾶς, ὡς πεπάτηται· φεῦ τῶν ὑακίνθων καὶ τῶν ναρκίσσων, οὓς ἀνώρυξέ τις πονηρὸς ἄνθρωπος. Ἀφίξεται τὸ ἦρ, τὰ δὲ οὐκ ἀνθήσει· ἔσται τὸ θέρος, τὰ δὲ οὐκ ἀκμάσει· μετόπωρον, τὰ δὲ οὐδένα στεφανώσει.  Οὐδὲ σύ, δέσποτα Διόνυσε, τὰ ἄθλια ταῦτα ἠλέησας ἄνθη, οἷς παρῴκεις, ἃ ἔβλεπες, ἀφ’ ὧν ἐστεφάνωσά σε πολλάκις; Πῶς δείξω νῦν τὸν παράδεισον τῷ δεσπότῃ; Τίς ἐκεῖνος θεασάμενος ἔσται; Κρεμᾷ γέροντα ἄνθρωπον ἐκ μιᾶς πίτυος ὡς Μαρσύαν· τάχα δὲ καὶ Δάφνιν, ὡς τῶν αἰγῶν ταῦτα εἰργασμένων.”
\pend


\pstart
4.9  Δάκρυα ἦν ἐπὶ τούτοις θερμότερα, καὶ ἐθρήνουν τὰ ἄνθη λοιπόν, ἀλλὰ τὰς αὑτῶν συμφοράς. Ἐθρήνει καὶ Χλόη Δάφνιν εἰ κρεμήσεται καὶ εὔχετο μηκέτι ἐλθεῖν τὸν δεσπότην αὐτῶν καὶ ἡμέρας διήντλει μοχθηράς, ὡς ἤδη Δάφνιν βλέπουσα μαστιγούμενον.  Καὶ δὴ νυκτὸς ἀρχομένης ὁ Εὔδρομος αὐτοῖς ἀπήγγελλεν ὅτι ὁ μὲν πρεσβύτερος δεσπότης μεθ’ ἡμέρας ἀφίξεται τρεῖς, ὁ δὲ παῖς αὐτοῦ τῆς ἐπιούσης πρόσεισι.  Σκέψις οὖν ἦ περὶ τῶν συμβεβηκότων καὶ κοινωνὸν εἰς τὴν γνώμην τὸν Εὔδρομον παρελάμβανον· ὁ δὲ εὔνους ὢν τῷ Δάφνιδι παρῄνει τὸ συμβὰν ὁμολογῆσαι πρότερον τῷ νέῳ δεσπότῃ καὶ αὐτὸς συμπράξειν ἐπηγγέλλετο, τιμώμενος ὡς ὁμογάλακτος· καὶ ἡμέρας γενομένης οὕτως ἐποίησαν.
\pend


\pstart
4.10  Ἧκε μὲν ὁ Ἀστύλος ἐπὶ ἵππου καὶ παράσιτος αὐτοῦ, καὶ οὗτος ἐπὶ ἵππου· ὁ μὲν ἀρτιγένειος, ὁ δὲ Γνάθων (τουτὶ γὰρ ἐκαλεῖτο) τὸν πώγωνα ξυρώμενος πάλαι· ὁ δὲ Λάμων ἅμα τῇ Μυρτάλῃ καὶ τῷ Δάφνιδι πρὸ τῶν ποδῶν αὐτοῦ καταπεσὼν ἱκέτευεν οἰκτεῖραι γέροντα ἀτυχῆ καὶ πατρῴας ὀργῆς ἐξαρπάσαι τὸν οὐδὲν ἀδικήσαντα, ἅμα τε αὐτῷ καταλέγει πάντα.  Οἰκτείρει τὴν ἱκεσίαν ὁ Ἀστύλος καὶ ἐπὶ τὸν παράδεισον ἐλθὼν καὶ τὴν ἀπώλειαν τῶν ἀνθέων ἰδὼν αὐτὸς ἔφη παραιτήσεσθαι τὸν πατέρα καὶ κατηγορήσειν τῶν ἵππων, ὡς ἐκεῖ δεθέντες ἐξύβρισαν καὶ τὰ μὲν κατέκλασαν, τὰ δὲ κατεπάτησαν, τὰ δὲ ἀνώρυξαν λυθέντες.  Ἐπὶ τούτοις εὔχοντο μὲν αὐτῷ πάντα τὰ ἀγαθὰ Λάμων καὶ Μυρτάλη· Δάφνις δὲ δῶρα προσεκόμισεν ἐρίφους, τυρούς, ὄρνιθας καὶ τὰ ἔκγονα αὐτῶν, βότρυς ἐπὶ κλημάτων, μῆλα ἐπὶ κλάδων. Ἦν ἐν τοῖς δώροις καὶ ἀνθοσμίας οἶνος
\pend


\pstart
4.11  Ὁ μὲν δὴ Ἀστύλος ἐπῄνει ταῦτα καὶ περὶ θήραν εἶχε λαγῶν, οἷα πλούσιος νεανίσκος καὶ τρυφῶν ἀεὶ καὶ ἀφιγμένος εἰς τὸν ἀγρὸν εἰς ἀπόλαυσιν ξένης ἡδονῆς.  Ὁ δὲ Γνάθων, οἷα μαθὼν ἐσθίειν ἄνθρωπος καὶ πίνειν εἰς μέθην καὶ λαγνεύειν μετὰ τὴν μέθην καὶ οὐδὲν ἄλλο ὢν ἢ γνάθος καὶ γαστὴρ καὶ τὰ ὑπὸ γαστέρα, οὐ παρέργως εἶδε τὸν Δάφνιν τὰ δῶρα κομίσαντα, ἀλλὰ καὶ φύσει παιδεραστὴς ὢν καὶ κάλλος οἷον οὐδὲ ἐπὶ τῆς πόλεως εὑρών, ἐπιθέσθαι ἔγνω τῷ Δάφνιδι καὶ πείσειν ᾤετο ῥᾳδίως ὡς αἰπόλον.  Γνοὺς δὲ ταῦτα θήρας μὲν οὐκ ἐκοινώνει τῷ Ἀστύλῳ, κατιὼν δὲ ἵνα ἔνεμεν ὁ Δάφνις, λόγῳ μὲν τῶν αἰγῶν, τὸ δὲ ἀληθὲς Δάφνιδος ἐγίνετο θεατής· μαλθάξων δὲ αὐτὸν τάς τε αἶγας ἐπῄνει καὶ συρίσαι τὸ αἰπολικὸν ἠξίωσε καὶ ἔφη ταχέως ἐλεύθερον θήσειν τὸ πᾶν δυνάμενος.
\pend


\pstart
4.12  Ὡς δὲ εἶχε χειροήθη, νύκτωρ λοχήσας ἐκ τῆς νομῆς ἐλαύνοντα τὰς αἶγας πρῶτον μὲν ἐφίλησε προσδραμών, εἶτα ὄπισθεν παρασχεῖν τοιοῦτον οἷον αἱ αἶγες τοῖς τράγοις.  Τοῦ δὲ βραδέως νοήσαντος καὶ λέγοντος, ὡς αἶγας μὲν βαίνειν τράγους καλόν, τράγον δὲ οὐπώποτε εἶδέ τις βαίνοντα τράγον οὐδὲ κριὸν ἀντὶ τῶν οἰῶν κριὸν οὐδὲ ἀλεκτρυόνας ἀντὶ τῶν ἀλεκτορίδων ἀλεκτρυόνας, οἷος ἦν ὁ Γνάθων βιάζεσθαι τὰς χεῖρας προσφέρων·  ὁ δὲ μεθύοντα ἄνθρωπον καὶ ἑστῶτα μόλις παρωσάμενος ἔσφηλεν εἰς τὴν γῆν καὶ ὥσπερ σκύλαξ ἀποδραμὼν κείμενον κατέλιπεν ἀνδρός, οὐ παιδὸς πρὸς χειραγωγίαν δεόμενον· καὶ οὐκέτι προσίετο ὅλως, ἀλλ’ ἄλλοτε ἄλλῃ τὰς αἶγας ἔνεμεν, ἐκεῖνον μὲν φεύγων, Χλόην δὲ τηρῶν.  Οὐδὲ ὁ Γνάθων ἔτι περιειργάζετο, καταμαθὼν ὡς οὐ μόνον καλός, ἀλλὰ καὶ ἰσχυρός ἐστιν· ἐπετήρει δὲ καιρὸν διαλεχθῆναι περὶ αὐτοῦ τῷ Ἀστύλῳ καὶ ἤλπιζε δῶρον αὐτὸν ἕξειν παρὰ τοῦ νεανίσκου πολλὰ καὶ μεγάλα χαρίζεσθαι θέλοντος.
\pend


\pstart
4.13  Τότε μὲν οὖν οὐκ ἠδυνήθη, προσῄει γὰρ ὁ Διονυσοφάνης ἅμα τῇ Κλεαρίστῃ, καὶ ἦν θόρυβος πολὺς κτηνῶν, οἰκετῶν, ἀνδρῶν, γυναικῶν· μετὰ δὲ τοῦτο συνέταττε λόγον καὶ ἐρωτικὸν καὶ μακρόν.  Ἦν δὲ ὁ Διονυσοφάνης μεσαιπόλιος μὲν ἤδη, μέγας δὲ καὶ καλὸς καὶ μειρακίοις ἁμιλλᾶσθαι δυνάμενος· ἀλλὰ καὶ πλούσιος ἐν ὀλίγοις καὶ χρηστὸς ὡς οὐδεὶς ἕτερος.  Οὗτος ἐλθὼν τῇ πρώτῃ μὲν ἡμέρᾳ θεοῖς ἔθυσεν, ὅσοι προεστᾶσιν ἀγροικίας, Δήμητρι καὶ Διονύσῳ καὶ Πανὶ καὶ Νύμφαις, καὶ κοινὸν πᾶσι τοῖς παροῦσιν ἔστησε κρατῆρα· ταῖς δὲ ἄλλαις ἡμέραις ἐπεσκόπει τὰ τοῦ Λάμωνος ἔργα.  Καὶ ὁρῶν τὰ μὲν πεδία ἐν αὔλακι, τὰς δὲ ἀμπέλους ἐν κλήματι, τὸν δὲ παράδεισον ἐν κάλλει (περὶ γὰρ τῶν ἀνθέων Ἀστύλος τὴν αἰτίαν ἀνελάμβανεν) ἥδετο περιττῶς καὶ τὸν Λάμωνα ἐπῄνει καὶ ἐλεύθερον θήσειν ἐπηγγέλλετο.  Κατῆλθε μετὰ ταῦτα καὶ εἰς τὸ αἰπόλιον τάς τε αἶγας ὀψόμενος καὶ τὸν νέμοντα.
\pend


\pstart
4.14  Χλόη μὲν οὖν εἰς τὴν ὕλην ἔφυγεν, ὄχλον τοσοῦτον αἰδεσθεῖσα καὶ φοβηθεῖσα· ὁ δὲ Δάφνις εἱστήκει δέρμα λάσιον αἰγὸς ἐζωσμένος, πήραν νεορραφῆ κατὰ τῶν ὤμων ἐξηρτημένος, κρατῶν ταῖς χερσὶν ἀμφοτέραις τῇ μὲν ἀρτιπαγεῖς τυρούς, τῇ δὲ ἐρίφους γαλαθηνούς.  Εἴ ποτε Ἀπόλλων Λαομέδοντι θητεύων ἐβουκόλησε, τοιόσδε ἦν, οἷος τότε ὤφθη Δάφνις. Αὐτὸς μὲν οὖν εἶπεν οὐδέν, ἀλλὰ ἐρυθήματος πλησθεὶς ἔνευσε κάτω, προτείνας τὰ δῶρα· ὁ δὲ Λάμων “οὗτοσ”  εἶπε “σοί, δέσποτα, τῶν αἰγῶν αἰπόλος. Σὺ μὲν ἐμοὶ πεντήκοντα νέμειν δέδωκας καὶ δύο τράγους, οὗτος δέ σοι πεποίηκεν ἑκατὸν καὶ δέκα τράγους. Ὁρᾷς ὡς λιπαραὶ καὶ τὰς τρίχας λάσιαι καὶ τὰ κέρατα ἄθραυστοι. Πεποίηκε δὲ αὐτὰς καὶ μουσικάς· σύριγγος γοῦν ἀκούουσαι ποιοῦσι πάντα.”
\pend


\pstart
4.15  Παροῦσα δὲ τοῖς λεγομένοις ἡ Κλεαρίστη πεῖραν ἐπεθύμησε τοῦ λεχθέντος λαβεῖν καὶ κελεύει τὸν Δάφνιν ταῖς αἰξὶν οἷον εἴωθε συρίσαι καὶ ἐπαγγέλλεται συρίσαντι χαριεῖσθαι χιτῶνα καὶ χλαῖναν καὶ ὑποδήματα.  Ὁ δὲ καθίσας αὐτοὺς ὥσπερ θέατρον, στὰς ὑπὸ τῇ φηγῷ κἀκ τῆς πήρας τὴν σύριγγα προκομίσας πρῶτα μὲν ὀλίγον ἐνέπνευσε· καὶ αἱ αἶγες ἔστησαν τὰς κεφαλὰς ἀράμεναι· εἶτα ἐνέπνευσε τὸ νόμιον, καὶ αἱ αἶγες ἐνέμοντο νεύσασαι κάτω· αὖθις λιγυρὸν ἐνέδωκε,  καὶ ἀθρόαι κατεκλίνησαν· ἐσύρισέ τι καὶ ὀξὺ μέλος, αἱ δὲ ὥσπερ λύκου προσιόντος εἰς τὴν ὕλην κατέφυγον· μετ’ ὀλίγον ἀνακλητικὸν ἐφθέγξατο, καὶ ἐξελθοῦσαι τῆς ὕλης πλησίον αὐτοῦ τῶν ποδῶν συνέδραμον.  Οὐδὲ ἀνθρώπους οἰκέτας εἶδεν ἄν τις οὕτω πειθομένους προστάγματι δεσπότου. Οἵ τε οὖν ἄλλοι πάντες ἐθαύμαζον καὶ πρὸ πάντων ἡ Κλεαρίστη καὶ τὰ δῶρα ἀποδώσειν ὤμοσε καλῷ τε ὄντι αἰπόλῳ καὶ μουσικῷ· καὶ ἀνελθόντες εἰς τὴν ἔπαυλιν ἀμφὶ ἄριστον εἶχον καὶ τῷ Δάφνιδι ἀφ’ ὧν ἤσθιον ἔπεμψαν.
\pend


\pstart
4.16  Ὁ δὲ μετὰ τῆς Χλόης ἤσθιε καὶ ἥδετο γευόμενος ἀστικῆς ὀψαρτυσίας καὶ εὔελπις ἦν τεύξεσθαι τοῦ γάμου, πείσας τοὺς δεσπότας. Ὁ δὲ Γνάθων προσεκκαυθεὶς τοῖς κατὰ τὸ αἰπόλιον γεγενημένοις καὶ ἀβίωτον νομίζων τὸν βίον, εἰ μὴ τεύξεται Δάφνιδος, περιπατοῦντα τὸν Ἀστύλον ἐν τῷ παραδείσῳ φυλάξας καὶ ἀναγαγὼν εἰς τὸν τοῦ Διονύσου νεὼν πόδας καὶ χεῖρας κατεφίλει.  Τοῦ δὲ πυνθανομένου τίνος ἕνεκα ταῦτα δρᾷ, καὶ λέγειν κελεύοντος καὶ ὑπουργήσειν ὀμνύοντος, “οἴχεταί σοι Γνάθων” ἔφη “δέσποτα. Ὁ μέχρι νῦν μόνης τραπέζης τῆς σῆς ἐρῶν, ὁ πρότερον ὀμνὺς ὅτι μηδέν ἐστιν ὡραιότερον οἴνου γέροντος, ὁ κρείττους τῶν ἐφήβων τῶν ἐν Μυτιλήνῃ τοὺς σοὺς ὀψαρτυτὰς λέγων, μόνον λοιπὸν καλὸν εἶναι Δάφνιν νομίζω.  Καὶ τροφῆς μὲν τῆς πολυτελοῦς οὐ γεύομαι, καίτοι τοσούτων παρασκευαζομένων ἑκάστης ἡμέρας κρεῶν, ἰχθύων, μελιτωμάτων, ἡδέως δ’ ἂν αἲξ γενόμενος πόαν ἐσθίοιμι καὶ φύλλα, τῆς Δάφνιδος ἀκούων σύριγγος καὶ ὑπ’ ἐκείνου νεμόμενος. Σὺ δὲ σῶσον Γνάθωνα τὸν σὸν καὶ τὸν ἀήττητον ἔρωτα νίκησον.  Εἰ δὲ μή, σὲ ἐπόμνυμι, τὸν ἐμὸν θεόν, ξιφίδιον λαβὼν καὶ ἐμπλήσας τὴν γαστέρα τροφῆς ἐμαυτὸν ἀποκτενῶ πρὸ τῶν Δάφνιδος θυρῶν· σὺ δὲ οὐκέτι καλέσεις Γναθωνάριον, ὥσπερ εἰώθεις παίζων ἀεί.”
\pend


\pstart
4.17  Οὐκ ἀντέσχε κλάοντι καὶ αὖθις τοὺς πόδας καταφιλοῦντι νεανίσκος μεγαλόφρων καὶ οὐκ ἄπειρος ἐρωτικῆς λύπης, ἀλλ’ αἰτήσειν αὐτὸν παρὰ τοῦ πατρὸς ἐπηγγείλατο καὶ κομιεῖν εἰς τὴν πόλιν αὑτῷ μὲν δοῦλον, ἐκείνῳ δὲ ἐρώμενον.  Εἰς εὐθυμίαν δὲ αὐτὸν θέλων προαγαγεῖν ἐπυνθάνετο μειδιῶν εἰ οὐκ αἰσχύνεται Λάμωνος υἱὸν φιλῶν, ἀλλὰ καὶ σπουδάζει συγκατακλινῆναι νέμοντι αἶγας μειρακίῳ· καὶ ἅμα ὑπεκρίνετο τὴν τραγικὴν δυσωδίαν μυσάττεσθαι.  Ὁ δέ, οἷα πᾶσαν ἐρωτικὴν μυθολογίαν ἐν τοῖς τῶν ἀσώτων συμποσίοις πεπαιδευμένος, οὐκ ἀπὸ σκοποῦ καὶ ὑπὲρ αὑτοῦ καὶ ὑπὲρ τοῦ Δάφνιδος ἔλεγεν “οὐδεὶς ταῦτα, δέσποτα, ἐραστὴς πολυπραγμονεῖ· ἀλλ’ ἐν οἵῳ ποτε ἂν σώματι εὕρῃ τὸ κάλλος,  ἑάλωκε. Διὰ τοῦτο καὶ φυτοῦ τις ἠράσθη καὶ ποταμοῦ καὶ θηρίου. Καίτοι τίς οὐκ ἂν ἐραστὴν ἠλέησεν, ὃν ἔδει φοβεῖσθαι τὸν ἐρώμενον; Ἐγὼ δὲ σώματος μὲν ἐρῶ δούλου, κάλλους δὲ ἐλευθέρου.  Ὁρᾷς ὡς ὑακίνθῳ μὲν τὴν κόμην ὁμοίαν ἔχει, λάμπουσι δὲ ὑπὸ ταῖς ὀφρύσιν οἱ ὀφθαλμοὶ καθάπερ ἐν χρυσῇ σφενδόνῃ ψηφίς; Καὶ τὸ μὲν πρόσωπον ἐρυθήματος μεστόν, τὸ δὲ στόμα λευκῶν ὀδόντων ὥσπερ ἐλέφαντος.  Τίς ἐκεῖθεν οὐκ ἂν εὔξαιτο λαβεῖν ἐραστὴς γλυκέα φιλήματα; Εἰ δὲ νέμοντος ἠράσθην, θεοὺς ἐμιμησάμην. Βουκόλος ἦν Ἀγχίσης, καὶ ἔσχεν αὐτὸν Ἀφροδίτη· αἶγας ἔνεμε Βράγχος, καὶ Ἀπόλλων αὐτὸν ἐφίλησε· ποιμὴν ἦν Γανυμήδης, καὶ αὐτὸν ὁ Ζεὺς ἥρπασε.  Μὴ καταφρονῶμεν παιδός, ᾧ καὶ αἶγας ὡς ἐρώσας πειθομένας εἴδομεν· ἀλλὰ ὅτι μένειν ἐπὶ γῆς ἐπιτρέπουσι τοιοῦτον κάλλος χάριν ἔχωμεν τοῖς Διὸς ἀετοῖς.”
\pend


\pstart
4.18  Ἡδὺ γελάσας ὁ Ἀστύλος ἐπὶ τούτῳ μάλιστα τῷ λεχθέντι καὶ ὡς μεγάλους ὁ Ἔρως ποιεῖ σοφιστὰς εἰπὼν ἐπετήρει καιρόν, ἐν ᾧ τῷ πατρὶ περὶ Δάφνιδος διαλέξεται. Ἀκούσας δὲ τὰ λεχθέντα κρύφα πάντα ὁ Εὔδρομος καὶ τὰ μὲν τὸν Δάφνιν φιλῶν ὡς ἀγαθὸν νεανίσκον, τὰ δὲ ἀχθόμενος εἰ Γνάθωνος ἐμπαροίνημα γενήσεται τοιοῦτον κάλλος, αὐτίκα καταλέγει πάντα κἀκείνῳ καὶ Λάμωνι.  Ὁ μὲν οὖν Δάφνις ἐκ. πλαγεὶς ἐγίνωσκεν ἅμα τῇ Χλόῃ τολμῆσαι φυγεῖν ἢ ἀποθανεῖν, κοινωνὸν κἀκείνην λαβών· ὁ δὲ Λάμων προκαλεσάμενος ἔξω τῆς αὐλῆς τὴν Μυρτάλην “οἰχόμεθα” εἶπεν “ὦ γύναι. Ἥκει καιρὸς ἐκκαλύπτειν τὰ κρυπτά.  Ἔρρει μοι καὶ αἱ αἶγες καὶ τὰ λοιπὰ πάντα· ἀλλ’ οὐ μὰ τὸν Πᾶνα καὶ τὰς Νύμφας, οὐδ’ εἰ μέλλω βοῦς, φασίν, ἐν αὐλίῳ καταλείπεσθαι, τὴν Δάφνιδος τύχην ἥτις ἐστὶν οὐ σιωπήσομαι, ἀλλὰ καὶ ὅτι εὗρον ἐκκείμενον ἐρῶ καὶ ὅπως τρεφόμενον μηνύσω καὶ ὅσα εὗρον συνεκκείμενα δείξω. Μαθέτω Γνάθων ὁ μιαρὸς οἷος ὢν οἵων ἐρᾷ. Παρασκεύαζέ μοι μόνον εὐτρεπῆ τὰ γνωρίσματα.”
\pend


\pstart
4.19  οἱ μὲν ταῦτα συνθέμενοι ἀπῆλθον εἴσω πάλιν· ὁ δὲ Ἀστύλος σχολὴν ἄγοντι τῷ πατρὶ προσρυεὶς αἰτεῖ τὸν Δάφνιν εἰς τὴν πόλιν καταγαγεῖν ὡς καλόν τε ὄντα καὶ ἀγροικίας κρείττονα καὶ ταχέως ὑπὸ Γνάθωνος καὶ τὰ ἀστικὰ διδαχθῆναι δυνάμενον.  Χαίρων ὁ πατὴρ δίδωσι καὶ μεταπεμψάμενος τὸν Λάμωνα καὶ τὴν Μυρτάλην εὐηγγελίζετο μὲν αὐτοῖς ὅτι Ἀστύλον θεραπεύσει λοιπὸν ἀντὶ αἰγῶν καὶ τράγων Δάφνις, ἐπηγγέλλετο δὲ δύο ἀντ’ ἐκείνου δώσειν αὐτοῖς αἰπόλους.  Ἐνταῦθα ὁ Λάμων, πάντων ἤδη συνερρυηκότων καὶ ὅτι καλὸν ὁμόδουλον ἕξουσιν ἡδομένων, αἰτήσας λόγον ἤρξατο λέγειν “ἄκουσον, ὦ δέσποτα, παρ’ ἀνδρὸς γέροντος ἀληθῆ λόγον· ἐπόμνυμι δὲ τὸν Πᾶνα καὶ τὰς Νύμφας ὡς οὐδὲν ψεύσομαι.  Οὐκ εἰμὶ Δάφνιδος πατήρ, οὐδ’ εὐτύχησέ ποτε Μυρτάλη μήτηρ γενέσθαι. Ἄλλοι πατέρες ἐξέθηκαν τοῦτο τὸ παιδίον, ἴσως παιδίων πρεσβυτέρων ἅλις ἔχοντες· ἐγὼ δὲ εὗρον ἐκκείμενον καὶ ὑπὸ αἰγὸς ἐμῆς τρεφόμενον, ἣν καὶ ἀποθανοῦσαν ἔθαψα ἐν τῷ περικήπῳ φιλῶν ὅτι ἐποίησε μητρὸς ἔργα.  Εὗρον αὐτῷ καὶ γνωρίσματα συνεκκείμενα ὁμόλογ
\pend


\pstart
4.20  Ὁ μὲν Λάμων ταῦτα εἰπὼν ἐσιώπησε καὶ πολλὰ ἀφῆκε δάκρυα· τοῦ δὲ Γνάθωνος θρασυνομένου καὶ πληγὰς ἀπειλοῦντος, ὁ Διονυσοφάνης τοῖς εἰρημένοις ἐκπλαγεὶς τὸν μὲν Γνάθωνα σιωπᾶν ἐκέλευσε, σφόδρα τὴν ὀφρὺν εἰς αὐτὸν τοξοποιήσας, τὸν δὲ Λάμωνα πάλιν ἀνέκρινε καὶ παρεκελεύετο τἀληθῆ λέγειν μηδὲ ὅμοια πλάττειν μύθοις ἐπὶ τῷ κατέχειν τὸν υἱόν.  Ὡς δ’ ἀτενὴς ἦν καὶ κατὰ πάντων ὤμνυε θεῶν καὶ ἐδίδου βασανίζειν αὑτόν, εἴ τι ψεύδεται, παρακαθημένης τῆς Κλεαρίστης ἐβασάνιζε τὰ λελεγμένα. “Τί δ’ ἂν ἐψεύδετο Λάμων, μέλλων ἀνθ’ ἑνὸς δύο λαμβάνειν αἰπόλους; Πῶς δ’ ἂν καὶ ταῦτ’ ἔπλασσεν ἄγροικος; Οὐ γὰρ εὐθὺς ἦν ἄπιστον ἐκ τοιούτου γέροντος καὶ μητρὸς εὐτελοῦς υἱὸν καλὸν οὕτω γενέσθαι;”
\pend


\pstart
4.21  Ἐδόκει μὴ μαντεύεσθαι ἐπὶ πλέον, ἀλλὰ ἤδη τὰ γνωρίσματα σκοπεῖν εἰ λαμπρᾶς καὶ ἐνδοξοτέρας τύχης. Ἀπῄει μὲν Μυρτάλη κομιοῦσα πάντα φυλαττόμενα ἐν πήρᾳ παλαιᾷ·  κομισθέντα δὲ πρῶτος Διονυσοφάνης ἐπέβλεπε καὶ ἰδὼν χλαμύδιον ἁλουργές, πόρπην χρυσήλατον, ξιφίδιον ἐλεφαντόκωπον, μέγα βοήσας “ὦ Ζεῦ δέσποτα” καλεῖ τὴν γυναῖκα θεασομένην.  Ἡ δὲ ἰδοῦσα μέγα καὶ αὐτὴ βοᾷ “φίλαι Μοῖραι· οὐ ταῦτα ἡμεῖς συνεξεθήκαμεν ἰδίῳ παιδί; Οὐκ εἰς τούτους τοὺς ἀγροὺς κομιοῦσαν Σωφροσύνην ἀπεστείλαμεν; Οὐκ ἄλλα μὲν οὖν, ἀλλ’ αὐτὰ ταῦτα. Φίλε ἄνερ, ἡμέτερόν ἐστὶ τὸ παιδίον· σὸς υἱός ἐστι Δάφνις, καὶ πατρῴας ἔνεμεν αἶγας.”
\pend


\pstart
4.22  Ἔτι λεγούσης αὐτῆς καὶ τοῦ Διονυσοφάνους τὰ γνωρίσματα φιλοῦντος καὶ ὑπὸ περιττῆς ἡδονῆς δακρύοντος ὁ Ἀστύλος συνεὶς ὡς ἀδελφός ἐστι, ῥίψας θοιμάτιον ἔθει κατὰ τοῦ παραδείσου, πρῶτος τὸν Δάφνιν φιλῆσαι θέλων.  Ἰδὼν δὲ αὐτὸν ὁ Δάφνις θέοντα μετὰ πολλῶν καὶ βοῶντα “Δάφνι,” νομίσας ὅτι συλλαβεῖν αὐτὸν βουλόμενος τρέχει, ῥίψας τὴν πήραν καὶ τὴν σύριγγα πρὸς τὴν θάλατταν ἐφέρετο ῥίψων ἑαυτὸν ἀπὸ τῆς μεγάλης πέτρας.  Καὶ ἴσως ἄν, τὸ καινότατον, εὑρεθεὶς ἀπωλώλει Δάφνις, εἰ μὴ συνεὶς ὁ Ἀστύλος ἐβόα πάλιν “στῆθι, Δάφνι, μηδὲν φοβηθῇς· ἀδελφός εἰμί σου, καὶ γονεῖς οἱ μέχρι νῦν δεσπόται.  Νῦν ἡμῖν Λάμων τὴν αἶγα εἶπε καὶ τὰ γνωρίσματα ἔδειξεν· ὅρα δὲ ἐπιστραφεὶς πῶς ἐπίασι φαιδροὶ καὶ γελῶντες. Ἀλλ’ ἐμὲ πρῶτον φίλησον· ὄμνυμι δὲ τὰς Νύμφας ὡς οὐ ψεύδομαι.”
\pend


\pstart
4.23  Μόλις μετὰ τὸν ὅρκον ἔστη καὶ τὸν Ἀστύλον τρέχοντα περιέμεινε καὶ προσελθόντα κατεφίλησεν. Ἐν ᾧ δὲ ἐκεῖνον ἐφίλει, πλῆθος τὸ λοιπὸν ἐπιρρεῖ θεραπόντων, θεραπαινῶν, αὐτὸς ὁ πατήρ, ἡ μήτηρ μετ’ αὐτοῦ. Οὗτοι πάντες περιέβαλλον, κατεφίλουν, χαίροντες κλάοντες.  Ὁ δὲ τὸν πατέρα καὶ τὴν μητέρα πρὸ τῶν ἄλλων ἐφιλοφρονεῖτο, καὶ ὡς πάλαι εἰδὼς προσεστερνίζετο καὶ ἐξελθεῖν τῶν περιβολῶν οὐκ ἤθελεν· οὕτω φύσις ταχέως πιστοῦται. Ἐξελάθετο καὶ Χλόης πρὸς ὀλίγον· καὶ ἐλθὼν εἰς τὴν ἔπαυλιν ἐσθῆτά εἰς τὴν ἔπαυλιν ἐσθῆτά τε ἔλαβε πολυτελῆ καὶ παρὰ τὸν πατέρα τὸν ἴδιον καθεσθεὶς ἤκουεν αὐτοῦ λέγοντος οὕτως·
\pend


\pstart
4.24  “Ἔγημα, ὦ παῖδες, κομιδῇ νέος. Καὶ χρόνου διελθόντος ὀλίγου πατήρ, ὡς ᾤμην, εὐτυχὴς ἐγεγόνειν· ἐγένετο γάρ μοι πρῶτος υἱὸς καὶ δευτέρα θυγάτηρ καὶ τρίτος Ἄστυλος. ᾬμην ἱκανὸν εἶναι τὸ γένος, καὶ γενόμενον ἐπὶ πᾶσι τοῦτο τὸ παιδίον ἐξέθηκα, οὐ γνωρίσματα ταῦτα συνεκθείς, ἀλλʼ ἐντάφια.  Τὰ δὲ τῆς Τύχης ἄλλα βουλεύματα. Ὁ μὲν γὰρ πρεσβύτερος παῖς καὶ ἡ θυγάτηρ ὁμοίᾳ νόσῳ μιᾶς ἡμέρας ἀπώλοντο· σὺ δέ μοι προνοίᾳ θεῶν ἐσώθης, ἵνα πλείους ἔχωμεν χειραγωγούς.  Μήτʼ οὖν σύ μοι μνησικακήσῃς ποτὲ τῆς ἐκθέσεως— ἑκὼν γὰρ οὐκ ἐβουλευσάμην—, μήτε σὺ λυπηθῇς, Ἄστυλε, μέρος ληψόμενος ἀντὶ πάσης τῆς οὐσίας—κρεῖττον γὰρ τοῖς εὖ φρονοῦσιν ἀδελφοῦ κτῆμα οὐδέν—, ἀλλὰ φιλεῖτε ἀλλήλους καὶ χρημάτων ἕνεκα καὶ βασιλεῦσιν ἐρίζετε.  Πολλὴν μὲν γὰρ ἐγὼ ὑμῖν καταλείψω γῆν, πολλοὺς δὲ οἰκέτας δεξιούς, χρυσόν, ἄργυρον ὅσα ἄλλα εὐδαιμόνων κτήματα. Μόνον ἐξαίρετον τοῦτο Δάφνιδι τὸ χωρίον δίδωμι καὶ Λάμωνα καὶ Μυρτάλην καὶ τὰς αἶγας, ἃς αὐτὸς ἔνεμεν.”
\pend


\pstart
4.25  Ἔτι αὐτοῦ λέγοντος Δάφνις ἀναπηδήσας “καλῶς με” εἶπε “ταῦτα, πάτερ, ἀνέμνησας. Ἄπειμι τὰς αἶγας απἄξων ἐπὶ ποτόν, αἵ που νῦν διψῶσαι περιμένουσι τὴν σύριγγα τὴν ἐμήν· ἐγὼ δὲ ἐνταυθοῖ καθέζομαι.”  Ἡδὺ πάντες ἐξεγέλασαν ὅτι δεσπότης γεγενημένος ἔτι θέλει εἶναι αἰπόλος· κἀκείνας μὲν θεραπεύσων ἐπέμφθη τις ἄλλος, οἱ δὲ θύσαντες Διὶ Σωτῆρι συμπόσιον συνεκρότουν. Εἰς τοῦτο τὸ συμπόσιον μόνος οὐχ ἧκε Γνάθων, ἀλλὰ φοβούμενος ἐν τῷ νεῲ τοῦ Διονύσου καὶ τὴν ἡμέραν ἔμεινε καὶ τὴν νύκτα, ὥσπερ ἱκέτης.  Ταχείας δὲ φήμης εἰς πάντας ἐλθούσης ὅτι Διονυσοφάνης εὗρεν υἱόν, καὶ ὅτι Δάφνις ὁ αἰπόλος δεσπότης τῶν ἀγρῶν εὑρέθη, ἅμα ἕῳ συνέτρεχον ἄλλος ἀλλαχόθεν, τῷ μὲν μειρακίῳ συνηδόμενοι, τῷ δὲ πατρὶ αὐτοῦ δῶρα κομίζοντες· ἐν οἷς καὶ ὁ Δρύας πρῶτος ὁ τρέφων τὴν Χλόην. 6
\pend


\pstart
4.26  Ὁ δὲ Διονυσοφάνης κατεῖχε πάντας, κοινωνοὺς μετὰ τὴν εὐφροσύνην καὶ τῆς ἑορτῆς ἐσομένους. Παρεσκεύαστο δὲ πολὺς μὲν οἶνος, πολλὰ δὲ ἄλευρα, ὄρνιθες ἕλειοι, χοῖροι γαλαθηνοί, μελιτώματα ποικίλα· καὶ ἱερεῖα δὲ πολλὰ τοῖς ἐπιχωρίοις θεοῖς ἐθύετο.  Ἐνταῦθα ὁ Δάφνις συναθροίσας πάντα τὰ ποιμενικὰ κτήματα διένειμεν ἀναθήματα τοῖς θεοῖς. Τῷ Διονύσῳ μὲν ἀνέθηκε τὴν πήραν καὶ τὸ δέρμα, τῷ Πανὶ τὴν σύριγγα καὶ τὸν πλάγιον αὐλόν, τὴν καλαύροπα ταῖς Νύμφαις καὶ τοὺς γαυλοὺς οὓς αὐτὸς ἐτεκτήνατο.  Οὕτως δὲ ἄρα τὸ σύνηθες ξενιζούσης εὐδαιμονίας τερπνότερόν ἐστιν, ὥστε ἐδάκρυεν ἐφʼ ἑκάστῳ τούτων ἀπαλλαττόμενος· καὶ οὔτε τοὺς γαυλοὺς ἀνέθηκε πρὶν ἀμέλξαι, οὔτε τὸ δέρμα πρὶν ἐνδύσασθαι, οὔτε τὴν σύριγγα πρὶν συρίσαι·  ἀλλὰ καὶ ἐφίλησεν αὐτὰ πάντα καὶ τὰς αἶγας προσεῖπε καὶ τοὺς τράγους ἐκάλεσεν ὀνομαστί. Τῆς μὲν γὰρ πηγῆς καὶ ἔπιεν, ὅτι πολλάκις καὶ μετὰ Χλόης. Οὔπω δὲ ὡμολόγει τὸν ἔρωτα καιρὸν παραφυλάττων.
\pend


\pstart
4.27  Ἐν ᾧ δὲ Δάφνις ἐν θυσίαις ἦν, τάδε γίνεται περὶ τὴν Χλόην. Ἐκάθητο κλάουσα, λέγουσα, οἷα εἰκὸς ἦν· “Ἐξελάθετό μου Δάφνις. Ὀνειροπολεῖ γάμους πλουσίους.  Τί γὰρ αὐτὸν ὀμνύειν ἀντὶ τῶν Νυμφῶν τὰς αἶγας ἐκέλευον; Κατέλιπε ταύτας ὡς καὶ Χλόην. Οὐδὲ θύων ταῖς Νύμφαις καὶ τῷ Πανὶ ἐπεθύμησεν ἰδεῖν Χλόην. Εὗρεν ἴσως παρὰ τῇ μητρὶ θεραπαίνας ἐμοῦ κρείττονας. Χαιρέτω· ἐγὼ δὲ οὐ ζήσομαι.”
\pend


\pstart
4.28  Τοιαῦτα λέγουσαν, τοιαῦτα ἐννοοῦσαν ὁ Λάμπις ὁ βουκόλος μετὰ χειρὸς γεωργικῆς ἐπιστὰς ἥρπασεν αὐτήν, ὡς οὔτε Δάφνιδος ἔτι γαμήσοντος καὶ Δρύαντος ἐκεῖνον ἀγαπήσοντος. Ἡ μὲν οὖν ἐκομίζετο βοῶσα ἐλεεινόν, τῶν δέ τις ἰδόντων ἐμήνυσε τῇ Νάπῃ κἀκείνη τῷ Δρύαντι καὶ ὁ Δρύας τῷ Δάφνιδι.  Ὁ δὲ ἔξω τῶν φρενῶν γενόμενος οὔτε εἰπεῖν πρὸς τὸν πατέρα ἐτόλμα καὶ καρτερεῖν μὴ δυνάμενος εἰς τὸν περίκηπον εἰσελθὼν ὠδύρετο “ὢ πικρᾶς ἀνευρέσεωσ”  λέγων· “πόσον ἦν μοι κρεῖττον νέμειν; Πόσον ἤμην μακαριώτερος, δοῦλος ὤν; Τότε ἔβλεπον Χλόην, τότε , νῦν δὲ τὴν μὲν Λάμπις ἁρπάσας οἴχεται, νυκτὸς δὲ γενομένης συγκοιμήσεται. Ἐγὼ δὲ πίνω καὶ τρυφῶ καὶ μάτην τὸν Πᾶνα καὶ τὰς αἶγας καὶ τὰς Νύμφας ὤμοσα.”
\pend


\pstart
4.29  Ταῦτα τοῦ Δάφνιδος λέγοντος ἤκουσεν ὁ Γνάθων ἐν τῷ παραδείσῳ λανθάνων· καὶ καιρὸν ἥκειν διαλλαγῶν πρὸς αὐτὸν νομίζων τινὰς τῶν τοῦ Ἀστύλου νεανίσκων προσλαβὼν μεταδιώκει τὸν Δρύαντα.  Καὶ ἡγεῖσθαι κελεύσας ἐπὶ τὴν τοῦ Λάμπιδος ἔπαυλιν συνέτεινε δρόμον· καὶ καταλαβὼν ἄρτι εἰσάγοντα τὴν Χλόην, ἐκείνην τε ἀφαιρεῖται καὶ ἀνθρώπους γεωργοὺς συνηλόησε πληγαῖς.  Ἐσπούδαζε δὲ καὶ τὸν Λάμπιν δήσας ἄγειν ὡς αἰχμάλωτον ἐκ πολέμου τινός, εἰ μὴ φθάσας ἀπέδρα. Κατορθώσας δὲ τηλικοῦτον ἔργον νυκτὸς ἀρχομένης ἐπανέρχεται.  Καὶ τὸν μὲν Διονυσοφάνην εὑρίσκει καθεύδοντα, τὸν δὲ Δάφνιν ἀγρυπνοῦντα καὶ ἔτι ἐν τῷ περικήπῳ δακρύοντα. Προσάγει δὴ τὴν Χλόην αὐτῷ καὶ διδοὺς διηγεῖται πάντα· καὶ δεῖται μηδὲν ἔτι μνησικακοῦντα δοῦλον ἔχειν οὐκ ἄχρηστον, μηδὲ ἀφελέσθαι τραπέζης, μεθ’ ἣν τεθνήξεται λιμῷ.  Ὁ δὲ ἰδὼν καὶ ἔχων ἐν ταῖς χερσὶ τὴν Χλόην τῷ μὲν ὡς εὐεργέτῃ διηλλάττετο, τῇ δὲ ὑπὲρ τῆς ἀμελείας ἀπελογεῖτο.
\pend


\pstart
4.30  Βουλευομένοις δὲ αὐτοῖς ἐδόκει τὸν γάμον κρύπτειν, ἔχειν δὲ κρύφα τὴν Χλόην πρὸς μόνην ὁμολογήσαντα τὸν ἔρωτα τὴν μητέρα· ἀλλ’ οὐ συνεχώρει Δρύας, ἠξίου δὲ τῷ πατρὶ λέγειν καὶ πείσειν αὐτὸς ἐπηγγέλλετο.  Καὶ γενομένης ἡμέρας ἔχων ἐν τῇ πήρᾳ τὰ γνωρίσματα πρόσεισι τῷ Διονυσοφάνει καὶ τῇ Κλεαρίστῃ καθημένοις ἐν τῷ παραδείσῳ (παρῆν δὲ καὶ ὁ Ἀστύλος καὶ αὐτὸς ὁ Δάφνις) καὶ σιωπῆς γενομένης ἤρξατο λέγειν  “Ὁμοία με ἀνάγκη Λάμωνι τὰ μέχρι νῦν ἄρρητα ἐκέλευσε λέγειν. Χλόην ταύτην οὔτε ἐγέννησα οὔτε ἀνέθρεψα, ἀλλὰ ἐγέννησαν μὲν ἄλλοι, κειμένην δὲ ἐν ἄντρῳ Νυμφῶν ἀνέθρεψεν οἶς.  Εἶδον τοῦτο αὐτὸς καὶ ἰδὼν ἐθαύμασα θαυμάσας ἔθρεψα. Μαρτυρεῖ μὲν καὶ τὸ κάλλος, ἔοικε γὰρ οὐδὲν ἡμῖν· μαρτυρεῖ δὲ καὶ τὰ γνωρίσματα, πλουσιώτερα γὰρ ἢ κατὰ ποιμένα. Ἴδετε ταῦτα καὶ τοὺς προσήκοντας τῇ κόρῃ ζητήσατε, ἵν’ ἀξία ποτὲ Δάφνιδος φανῇ.”
\pend


\pstart
4.31  Τοῦτο οὔτε Δρύας ἀσκόπως ἔρριψεν οὔτε Διονυσοφάνης ἀμελῶς ἤκουσεν, ἀλλ’ ἰδὼν εἰς τὸν Δάφνιν καὶ ὁρῶν αὐτὸν χλωριῶντα καὶ κρύφα δακρύοντα ταχέως ἐφώρασε τὸν ἔρωτα· καὶ ὡς ὑπὲρ παιδὸς ἰδίου μᾶλλον ἢ κόρης ἀλλοτρίας δεδοικὼς διὰ πάσης ἀκριβείας ἤλεγχε τοὺς λόγους τοῦ Δρύαντος.  Ἐπεὶ δὲ καὶ τὰ γνωρίσματα εἶδε κομισθέντα, ὑποδήματα κατάχρυσα, τὰς περισκελίδας, τὴν μίτραν, προσκαλεσάμενος τὴν Χλόην παρεκελεύετο θαρρεῖν, ὡς ἄνδρα μὲν ἔχουσαν ἤδη, ταχέως δὲ εὑρήσουσαν καὶ τὸν πατέρα καὶ τὴν μητέρα.  Καὶ τὴν μὲν ἡ Κλεαρίστη παραλαβοῦσα ἐκόσμει λοιπὸν ὡς υἱοῦ γυναῖκα, τὸν δὲ Δάφνιν ὁ Διονυσοφάνης ἀναστήσας μόνον ἀνέκρινεν εἰ παρθένος ἐστί· τοῦ δὲ ὀμόσαντος μηδὲν γεγονέναι φιλήματος καὶ ὅρκων πλέον, ἡσθεὶς ἐπὶ τῷ συνωμοσίῳ κατέκλινεν αὐτούς.
\pend


\pstart
4.32  Ἦν οὖν μαθεῖν οἷόν ἐστι τὸ κάλλος, ὅταν κόσμον προσλάβῃ. Ἐνδυθεῖσα γὰρ ἡ Χλόη καὶ ἀναπλεξαμένη τὴν κόμην καὶ ἀπολούσασα τὸ πρόσωπον εὐμορφοτέρα τοσοῦτον ἐφάνη πᾶσιν, ὥστε καὶ Δάφνις αὐτὴν μόλις ἐγνώρισεν.  Ὤμοσεν ἄν τις καὶ ἄνευ τῶν γνωρισμάτων ὅτι τοιαύτης κόρης οὐκ ἦν Δρύας πατήρ. Ὅμως μέντοι παρῆν καὶ αὐτὸς καὶ συνειστιᾶτο μετὰ τῆς Νάπης, συμπότας ἔχων ἐπὶ κλίνης ἰδίας τὸν Λάμωνα καὶ τὴν Μυρτάλην.  Πάλιν οὖν ταῖς ἑξῆς ἡμέραις ἐθύετο ἱερεῖα καὶ κρατῆρες ἵσταντο καὶ ἀνετίθει καὶ Χλόη τὰ ἑαυτῆς, τὴν σύριγγα, τὴν πήραν, τὸ δέρμα, τοὺς γαυλούς· ἐκέρασε δὲ καὶ τὴν πηγὴν οἴνῳ τὴν ἐν τῷ ἄντρῳ, ὅτι καὶ ἐτράφη παρ’ αὐτῇ, καὶ ἐλούσατο πολλάκις ἐν αὐτῇ·  ἐστεφάνωσε καὶ τὸν τάφον τῆς οἰός, δείξαντος Δρύαντος, καὶ ἐσύρισέ τι καὶ αὐτὴ τῇ ποίμνῃ, καὶ ταῖς θεαῖς συρίσασα εὔξατο τοὺς ἐκθέντας εὑρεῖν ἀξίους τῶν Δάφνιδος γάμων.
\pend


\pstart
4.33  Ἐπεὶ δὲ ἅλις ἦν τῶν κατ’ ἀγρὸν ἑορτῶν, ἔδοξε βαδίζειν εἰς τὴν πόλιν καὶ τούς τε τῆς Χλόης πατέρας ἀναζητεῖν καὶ περὶ τὸν γάμον αὐτῶν μηκέτι βραδύνειν.  Ἕωθεν οὖν συσκευασάμενοι τῷ Δρύαντι μὲν ἔδωκαν ἄλλας τρισχιλίας, τῷ Λάμωνι δὲ τὴν ἡμίσειαν μοῖραν τῶν ἀγρῶν θερίζειν καὶ τρυγᾶν καὶ τὰς αἶγας ἅμα τοῖς αἰπόλοις καὶ ζεύγη βοῶν τέτταρα καὶ ἐσθῆτας χειμερινὰς καὶ ἐλευθέραν τὴν γυναῖκα· καὶ μετὰ τοῦτο ἤλαυνον ἐπὶ Μυτιλήνην ἵπποις καὶ ζεύγεσι καὶ τρυφῇ πολλῇ.  Τότε μὲν οὖν ἔλαθον τοὺς πολίτας, νυκτὸς κατελθόντες· τῆς δὲ ἐπιούσης ὄχλος ἠθροίσθη περὶ τὰς θύρας ἀνδρῶν, γυναικῶν. Οἱ μὲν τῷ Διονυσοφάνει συνήδοντο παῖδα εὑρόντι, καὶ μᾶλλον ὁρῶντες τὸ κάλλος τοῦ Δάφνιδος· αἱ δὲ τῇ Κλεαρίστῃ συνέχαιρον ἅμα κομιζούσῃ καὶ παῖδα καὶ νύμφην.  Ἐξέπλησσε γὰρ κἀκείνας ἡ Χλόη κάλλος ἐκφαίνουσα παρευδοκιμηθῆναι μὴ δυνάμενον· ὅλη γὰρ ἐκίττα ἡ πόλις ἐπὶ τῷ μειρακίῳ καὶ τῇ παρθένῳ· καὶ εὐδαιμόνιζον μὲν ἤδη τὸν γάμον, εὔχοντο δὲ καὶ τὸ γένος ἄξιον τῆς μορφῆς εὑρεθῆναι τῆς κόρης· καὶ γυναῖκες πολλαὶ τῶν μέγα πλουσίων ἠράσαντο θεοῖς αὐταὶ πιστευθῆναι μητέρες θυγατρὸς οὕτω καλῆς.
\pend


\pstart
4.34  Ὄναρ δὲ Διονυσοφάνει μετὰ φροντίδα πολλὴν εἰς βαθὺν ὕπνον κατενεχθέντι τοιόνδε γίνεται. Ἐδόκει τὰς Νύμφας δεῖσθαι τοῦ Ἔρωτος ἤδη ποτε αὐτοῖς κατανεῦσαι τὸν γάμον· τὸν δὲ ἐκλύσαντα τὸ τοξάριον καὶ ἀποθέμενον τὴν φαρέτραν κελεῦσαι τῷ Διονυσοφάνει πάντας τοὺς ἀρίστους Μυτιληναίων θέμενον συμπότας, ἡνίκα ἂν τὸν ὕστατον πλήσῃ κρατῆρα, τότε δεικνύειν ἑκάστῳ τὰ γνωρίσματα, τὸ δὲ ἐντεῦθεν ᾅδειν τὸν ὑμέναιον.  Ταῦτα ἰδὼν καὶ ἀκούσας ἕωθεν ἀνίσταται καὶ κελεύσας λαμπρὰν ἑστίασιν παρασκευασθῆναι τῶν ἀπὸ γῆς τῶν ἀπὸ θαλάττης καὶ εἴ τι ἐν λίμναις καὶ εἴ τι ἐν ποταμοῖς, πάντας τοὺς ἀρίστους Μυτιληναίων ποιεῖται συμπότας.  Ὡς δὲ ἤδη νὺξ ἦν καὶ ἐπέπληστο κρατὴρ ἐξ οὗ σπένδουσιν Ἑρμῇ, εἰσκομίζει τις ἐπὶ σκεύους ἀργυροῦ θεράπων τὰ γνωρίσματα καὶ περιφέρων ἐνδέξια πᾶσιν ἐδείκνυε.
\pend


\pstart
4.35  Τῶν μὲν οὖν ἄλλων ἐγνώρισεν οὐδείς, Μεγακλῆς δέ τις διὰ γῆρας ὕστατος κατακείμενος ὡς εἶδε, γνωρίσας πάνυ μέγα καὶ νεανικὸν ἐβόα “τίνα ὁρῶ ταῦτα; Τί γέγονάς μοι θυγάτριον; Ἆρα σὺ ζῇς ἢ ταῦτά τις ἐβάστασε μόνα ποιμὴν ἐντυχών;  Δέομαι, Διονυσόφανες, εἰπέ μοι· πόθεν ἔχεις ἐμοῦ παιδίου γνωρίσματα; Μὴ φθονήσῃς μετὰ Δάφνιν εὑρεῖν τι κἀμέ.” Κελεύσαντος δὲ τοῦ Διονυσοφάνους πρότερον ἐκεῖνον λέγειν τὴν ἔκθεσιν, ὁ Μεγακλῆς οὐδὲν ὑφελὼν τοῦ  τόνου τῆς φωνῆς ἔφη “ἦν ὀλίγος μοι βίος τὸν πρότερον χρόνον· ὃν γὰρ εἶχον εἰς χορηγίας καὶ τριηραρχίας ἐξεδαπάνησα. Ὅτε ταῦτα ἦν, γίνεταί μοι θυγάτριον. Τοῦτο τρέφειν ὀκνήσας ἐν πενίᾳ, τούτοις τοῖς γνωρίσμασι κοσμήσας ἐξέθηκα, εἰδὼς ὅτι πολλοὶ καὶ οὕτω σπουδάζουσι πατέρες γενέσθαι.  Καὶ τὸ μὲν ἐξέκειτο ἐν ἄντρῳ Νυμφῶν πιστευθὲν ταῖς θεαῖς, ἐμοὶ δὲ πλοῦτος ἐπέρρει καθ’ ἑκάστην ἡμέραν κληρονόμον οὐκ ἔχοντι.  Οὐκέτι γοῦν οὐδὲ θυγατρίου γενέσθαι πατὴρ εὐτύχησα, ἀλλ’ οἱ θεοὶ ὥσπερ γέλωτά με ποιούμενοι νύκτωρ ὀνείρους μοι ἐπιπέμπουσι, δηλοῦντες ὅτι με πατέρα ποιήσει ποίμνιον.”
\pend


\pstart
4.36  Ἀνεβόησεν ὁ Διονυσοφάνης μεῖζον τοῦ Μεγακλέους καὶ ἀναπηδήσας εἰσάγει Χλόην πάνυ καλῶς κεκοσμημένην καὶ λέγει “τοῦτο τὸ παιδίον ἐξέθηκας. Ταύτην σοὶ τὴν παρθένον οἶς προνοίᾳ θεῶν ἐξέθρεψεν,  ὡς αἲξ Δάφνιν ἐμοί. Λαβὲ τὰ γνωρίσματα καὶ τὴν θυγατέρα, λαβὼν δὲ ἀπόδος Δάφνιδι νύμφην. Ἀμφοτέρους ἐξεθήκαμεν, ἀμφοτέρους εὑρήκαμεν, ἀμφοτέρων ἐμέλησε Πανὶ καὶ Νύμφαις καὶ Ἔρωτι.”  Ἐπῄνει τὰ λεγόμενα ὁ Μεγακλῆς καὶ τὴν γυναῖκα Ῥόδην μετεπέμπετο καὶ τὴν Χλόην ἐν τοῖς κόλποις εἶχε. Καὶ ὕπνον αὐτοῦ μένοντες εἵλοντο· Δάφνις γὰρ οὐδενὶ διώμνυτο προήσεσθαι τὴν Χλόην, οὐδὲ αὐτῷ τῷ πατρί.
\pend


\pstart
4.37  Ἡμέρας δὲ γενομένης συνθέμενοι πάλιν εἰς τὸν ἀγρὸν ἤλαυνον· ἐδεήθησαν γὰρ τοῦτο Δάφνις καὶ Χλόη, μὴ φέροντες τὴν ἐν ἄστει διατριβήν. Ἐδόκει δὲ κἀκείνοις ποιμενικούς τινας αὐτοῖς ποιῆσαι τοὺς γάμους.  Ἐλθόντες οὖν παρὰ τὸν Λάμωνα τόν τε Δρύαντα τῷ· Μεγακλεῖ προσήγαγον καὶ τῇ Ῥόδῃ τὴν Νάπην συνέστησαν καὶ τὰ πρὸς τὴν ἑορτὴν παρεσκευάζοντο λαμπρῶς. Παρέδωκε μὲν οὖν ἐπὶ ταῖς Νύμφαις τὴν Χλόην ὁ πατὴρ καὶ μετ’ ἄλλων πολλῶν ἐποίησεν ἀναθήματα τὰ γνωρίσματα καὶ Δρύαντι τὰς λειπούσας εἰς τὰς μυρίας ἐπλήρωσεν.
\pend


\pstart
4.38  Ὁ δὲ Διονυσοφάνης, εὐημερίας οὔσης, αὐτοῦ πρὸ τοῦ ἄντρου στιβάδας ὑπεστόρεσεν ἐκ χλωρᾶς φυλλάδος καὶ πάντας τοὺς κωμήτας κατακλίνας εἱστία πολυτελῶς.  Παρῆσαν δὲ Λάμων καὶ Μυρτάλη, Δρύας καὶ Νάπη, οἱ Δόρκωνι προσήκοντες,  Ἦν οὖν ὡς ἐν τοιοῖσδε συμπόταις πάντα γεωργικὰ καὶ ἄγροικα· ὁ μὲν ᾖδεν οἷα ᾅδουσι θερίζοντες, ὁ δὲ ἔσκωπτε τὰ ἐπὶ ληνοῖς σκώμματα· Φιλητᾶς ἐσύρισε, Λάμπις ηὔλησε, Δρύας καὶ Λάμων ὠρχήσαντο, Χλόη καὶ Δάφνις ἀλλήλους κατεφίλουν.  Ἐνέμοντο δὲ καὶ αἱ αἶγες πλησίον, ὥσπερ καὶ αὐταὶ κοινωνοῦσαι τῆς ἑορτῆς. Τοῦτο τοῖς μὲν ἀστικοῖς οὐ πάνυ τερπνὸν ἦν· ὁ δὲ Δάφνις καὶ ἐκάλεσέ τινας αὐτῶν ὀνομαστὶ καὶ φυλλάδα χλωρὰν ἔδωκε καὶ κρατήσας ἐκ τῶν κεράτων κατεφίλησε.
\pend


\pstart
4.39  Κατὰ ταυτὰ οὐ τότε μόνον ἀλλ’ ἔστε ἔζων τὸν πλεῖστον χρόνον ποιμενικῶς διῆγον, θεοὺς σέβοντες Νύμφας καὶ Πᾶνα καὶ Ἔρωτα, ἀγέλας δὲ προβάτων καὶ αἰγῶν πλείστας κτησάμενοι, ἡδίστην δὲ τροφὴν νομίζοντες ὀπώραν καὶ γάλα.  Ἀλλὰ καὶ ἄρρεν παιδίον ὑπέθηκαν καὶ θυγάτριον γενόμενον δεύτερον οἰὸς ἑλκύσαι θηλὴν ἐποίησαν, καὶ ἐκάλεσαν τὸν μὲν Φιλοποίμενα τὴν δὲ Ἀγέλην, καὶ τὸ ἄντρον ἐκόσμησαν καὶ εἰκόνας ἀνέθεσαν καὶ βωμὸν εἵσαντο Ποιμένος Ἔρωτος· καὶ τῷ Πανὶ δὲ ἔδοσαν ἀντὶ τῆς πίτυος νεὼν Πανὸς Στρατιώτου ὀνομάσαντες.
\pend


\pstart
4.40  Ἀλλὰ ταῦτα μὲν ὕστερον καὶ ὠνόμασαν καὶ ἔπραξαν· τότε δὲ νυκτὸς γενομένης πάντες αὐτοὺς παρέπεμπον εἰς τὸν θάλαμον, οἱ μὲν συρίττοντες, οἱ δὲ αὐλοῦντες, οἱ δὲ δᾷδας μεγάλας ἀνίσχοντες.  Καὶ ἐπεὶ πλησίον ἦσαν τῶν θυρῶν, ᾖδον σκληρᾷ καὶ ἀπηνεῖ τῇ φωνῇ, καθάπερ τριαίναις γῆν ἀναρρηγνύντες, οὐχ ὑμέναιον ᾅδοντες.  Δάφνις δὲ καὶ Χλόη γυμνοὶ συγκατακλινέντες περιέβαλλον ἀλλήλους καὶ κατεφίλουν, ἀγρυπνήσαντες τῆς νυκτὸς ὅσον οὐδὲ γλαῦκες· καὶ ἔδρασέ τι Δάφνις ὧν αὐτὸν ἐπαίδευσε Λυκαίνιον, καὶ τότε Χλόη πρῶτον ἔμαθεν ὅτι τὰ ἐπὶ τῆς ὕλης γενόμενα ἦν ποιμένων παίγνια.
\pend
\endnumbering
\end{greek}
\end{Leftside}

\end{pairs}
\Columns
\end{document}
